\chapter{Лексика}

\section{Структурные типы тибетских слов}

Все слова тибетского языка по составу и соотношению морфем можно разделить на три структурных типа: непроизводные, производные и сложные.

1. \emph{Непроизводные слова}. Подавляющее большинство непроизводных слов представляет односложную знаменательную морфему, например:
\begin{prfsample}
	\item \prfC{མི་}{mi}{<<человек>>},
	\item \prfC{རི་}{ri}{<<гора>>},
	\item \prfC{ང་}{nga}{<<я>>},
	\item \prfC{གསུམ་}{\ul{g}sum}{<<три>>},
	\item \prfC{ལེན་}{len}{<<брать>>},
	\item \prfC{འགྲོ་}{'gro}{<<идти>>}.	
\end{prfsample}
Все подобные словоморфемы относятся к наиболее древней, устойчивой части словарного фонда тибетского языка.

2. \emph{Производные слова} образуются:

1) сочетанием знаменательной морфемы со словообразующей служебной морфемой, например:
\begin{prfsample}
	\item \prfC{རྟ་\selA{པ་}}{\ul{r}ta-\selA{pa}}{<<всадник>>},
	\item \prfC{རྐང་\selA{པ་}}{\ul{r}kang-\selA{pa}}{<<нога>>},
	\item \prfC{ལག་\selA{པ་}}{lag-\selA{pa}}{<<рука>>};
\end{prfsample}

2) сочетанием знаменательной морфемы с определённой знаменательной морфемой, выступающей в служебной функции и сохраняющей лишь общее лексическое значение. К последним относятся, например, знаменательные морфемы \prfC{ཁང}{khang}{<<дом, здание>>} и \prfC{རྭ}{r\ul{w}a}{<<ограда>>}, которые образуют ряд слов типа:
\begin{prfsample}
	\item \prfC{སྨན་\selA{ཁང་}}{\ul{s}man-\selA{khang}}{<<аптека>>} (\prfC{སྨན་}{\ul{s}man}{<<лекарство>>});
	\item \prfC{ཟ་\selA{ཁང་}}{za-\selA{khang}}{<<столовая>>} (\prfC{ཟ་}{za}{<<есть>>});
	\item \prfC{ཆང་\selA{རྭ་}}{chang-\selA{r\ul{w}a}}{<<трактир>>} (\prfC{ཆང་}{chang}{<<вино>>});
	\item \prfC{ཁྲིམས་\selA{རྭ་}}{khrim\ul{s}-\selA{r\ul{w}a}}{<<суд>> (помещение)} (\prfC{ཁྲིམས་}{khrim\ul{s}}{<<закон>>}).
\end{prfsample}

3. \emph{Сложные слова}. Основная масса тибетских слов представлена сложными словами, которые образуются:

1) соединением двух (значительно реже трёх) знаменательных морфем, связь между которыми вскрывается лишь этимологически, например:
\begin{prfsample}
	\item \prfC{རྩྭ་ཐང་}{\ul{r}ts\ul{w}a-thang}{<<пастбище>>} (букв. <<трава-равнина>>),
	\item \prfC{ཚོགས་འདུ་}{tshog\ul{s}-'du}{<<собрание, митинг>>} (букв. <<собираться-собираться>>),
	\item \prfC{ས་ཡོམ་}{sa-yom}{<<землетрясение>>} (букв. <<земля-колебаться>>);
\end{prfsample}

2) лексикализацией синтаксических образований типа:

а) свободных словосочетаний, построенных по схеме <<определение -- определяемое>>, например: словосочетание
\prfC{ཁང་\selA{པ་}བཟང་\selA{པོ་}}{khang-\selA{pa} \ul{b}zang-\selA{po}}{<<хороший дом>>}, утрачивая служебные морфемы \prfB{པ}{pa} и \prfB{པོ}{po} превращается в слово \prfC{ཁང་བཟང་}{khang-\ul{b}zang}{<<особняк, хоромы>>};

б) определений к сказуемому, построенных по схеме: <<существительное или прилагательное + косвенная падежная частица>>, например: \prfC{སྟེང་ན་}{\ul{s}teng-na}{<<наверху>>}, \prfC{རྒྱབ་ཏུ་}{\ul{r}gyab-tu}{<<сзади>>}, \prfC{ཡག་པོར་}{yag-por}{<<хорошо>>} соответственно образованы \prfC{སྟེང་}{\ul{s}teng}{<<верх>>}, \prfC{རྒྱབ་}{\ul{r}gyab}{<<спина>>}, \prfC{ཡག་པོ་}{yag-po}{<<хороший>>} и косвенных падежных частиц \prfB{ན}{na}, \prfB{ཏུ}{tu}, \prfB{ར}{ra}.

Разные классы слов тяготеют к определённым структурным типам, как это показано на таблице (таб. \ref{tab:11}).

\begin{tabularx}{\textwidth}{p{0.3\textwidth}ccc}
	\caption{Классы и структурные типы слов}\label{tab:11}\\
	\toprule
	\multirow[m]{2}{*}{\small\centering Класс слов} & \multicolumn{3}{c}{\small Структурный тип слова}\\
	\cmidrule{2-4}
	 & \parbox[m]{0.2\textwidth}{\small\centering Непро\-из\-вод\-ное} & \parbox[m]{0.2\textwidth}{\small\centering Производное} & \parbox[m]{0.2\textwidth}{\small\centering Сложное}\\
	\midrule
	\endhead
	Существительное & + & + & + \\
	\addlinespace
	Прилагательное & -- & + &  \parbox[t]{0.2\textwidth}{\centering +\\(не типично)} \\
	\addlinespace
	Числительное & & & \\
	\quad а) количественное & + & -- & + \\
	\quad б) порядковое & -- & + & + \\
	\addlinespace
	Местоимение & + & + & \parbox[t]{0.2\textwidth}{\centering +\\ (новый письм. язык)} \\
	\addlinespace
	Наречие & + & + & -- \\
	\addlinespace
	Глагол & + & -- & \parbox[t]{0.2\textwidth}{\centering +\\ (новый письм. язык)} \\
	\addlinespace
	Служебные слова & + & + & -- \\
	\bottomrule
\end{tabularx}

\section[Гонорифические средства]{Гонорифические средства\protect\prfnote{Гоноратив --- <<форма вежливости>>. Грамматическая категория, передающая отношение говорящего к лицу, о котором идёт речь.}}

К особенностям тибетской лексики следует отнести также наличие двух больших групп слов, обозначающих одни и те же понятия, но слова одной группы употребляются в так называемой вежливой речи, а другой --- в обыкновенной речи.

Все слова вежливой речи можно разбить на две неодинаковые по своей величине группы.

Первую, меньшую, группу составляют непроизводные слова, полностью отличающиеся от слов, употребляемых речи обычной, например:

\begin{tabularx}{\textwidth}{*{3}{p{0.3\textwidth}}}
	\toprule
	Обыкновенное слово & Вежливое слово & Значение\\
	\midrule
	\endhead
	\prfB{མགོ་}{\ul{m}go} & \prfB{དབུ་}{\ul{d}bu} & <<голова>>\\
	\prfB{ཁ་}{kha} & \prfB{ཞལ་}{zha\ul{l}} & <<рот>>; <<лицо>>\\
	\prfB{ལག་པ་}{lag-pa} & \prfB{ཕྱག་}{phyag} & <<рука>>\\
	\prfB{བྱེད་}{bye\ul{d}} & \prfB{གནང་}{\ul{g}nang} & <<делать>>\\
	\prfB{འགྲོ་}{'gro} & \prfB{ཕེབས་}{pheb\ul{s}} & <<идти>>\\
	\prfB{ལབ་}{lab} & \prfB{གསུང་}{\ul{g}sung} & <<говорить>>\\
	\bottomrule
\end{tabularx}

Основная группа слов, употребляемых в вежливой, речи, образуется путём прибавления к знаменательному слову (морфеме) обычной речи соответствующего (по функции, принадлежности и т.п.) слова вежливой речи, которое в составе нового производного слова вежливой речи теряет свое лексическое значение и указывает лишь на то, что данное новое слово принадлежит к категории слов вежливой речи. Так, например, если к обычным знаменательным словам (морфемам)
\begin{prfsample}
	\item \prfC{ཁབ་}{khab}{<<игла>>},
	\item \prfC{དམ་}{dam}{<<печать>>},
	\item \prfC{ལྡེ་}{\ul{l}de}{<<ключ (замка)>>},
	\item \prfC{དཔེ་}{\ul{d}pe}{<<книга>>},
	\item \prfC{ཤོག་}{shog}{<<бумага>>}	
\end{prfsample}
прибавить слово вежливой речи \prfC{ཕྱག་}{phyag}{<<рука>>},
то получим соответствующие слова вежливой речи:
\begin{prfsample}
	\item \prfC{\selA{ཕྱག་}ཁབ་}{\selA{phyag}-khab}{<<игла>>},
	\item \prfC{\selA{ཕྱག་}དམ་}{\selA{phyag}-dam}{<<печать>>},
	\item \prfC{\selA{ཕྱག་}ལྡེ་}{\selA{phyag}-\ul{l}de}{<<ключ (замка)>>},
	\item \prfC{\selA{ཕྱག་}དཔེ་}{\selA{phyag}-\ul{d}pe}{<<книга>>},
	\item \prfC{\selA{ཕྱག་}ཤོག་}{\selA{phyag}-shog}{<<бумага>>}.
\end{prfsample}

\section{Иноязычные заимствования}

Словарный состав тибетского языка пополнялся и обогащался за счёт заимствования из языков соседних народов.

Наиболее многочисленны и значимы \emph{заимствования из санскрита}. Санскритские заимствования в классическом тибетском языке --- это мощный слой лексики, обслуживавшей целые области духовной и культурной жизни тибетского народа. Так, тщательно разработанная философская терминология буддизма, многочисленные названия растений, входящих в тибетскую фармакопею, многие названия животных и минералов, ряд бытовых терминов представляют собой заимствования из санскрита.

Для удобства передачи санскритских заимствований средствами тибетской графики в тибетском алфавите имеется пять так называемых перевернутых графем и шесть сложных графем, специально воспроизводящих соответствующие согласные санскрита:

\begin{tabularx}{\textwidth}{p{0.12\textwidth}*{11}{>{\centering\arraybackslash}p{0.05\textwidth}}}
	\toprule
	Санскрит & \textit{\d{t}a} & \textit{\d{t}ha} & \textit{\d{d}a} & \textit{\d{n}a} & \textit{\d{s}a} & \textit{k\d{s}a} & \textit{gha} & \textit{dsha} & \textit{\d{d}ha} & \textit{dha} & \textit{bha}\\
	Тибетский & \prfA{ཊ} & \prfA{ཋ} & \prfA{ཌ} & \prfA{ཎ} & \prfA{ཥ} & \prfA{ཀྵ} & \prfA{གྷ} & \prfA{ཛྷ} & \prfA{ཌྷ} & \prfA{དྷ} & \prfA{བྷ}\\
	\bottomrule
\end{tabularx}

В тибетский алфавит специально введено также несколько сложных графем, воспроизводящих следующие гласные и двугласные санскрита:

\begin{tabularx}{\linewidth}{p{0.12\textwidth}*{8}{>{\centering\arraybackslash}p{0.06\textwidth}}}
	\toprule
	Санскрит & \textit{\d{r}} & \textit{\={\d{r}}} & \textit{\d{l}} & \textit{\={\d{l}}} & \textit{ai} & \textit{au} & \textit{a\d{m}} & \textit{a\d{h}}\\
	Тибетский & \prfA{རྀ} &  \prfA{རཱྀ} &  \prfA{ལྀ} &  \prfA{ལཱྀ} &  \prfA{ཨཻ} &  \prfA{ཨཽ} &  \prfA{ཨཾ} &  \prfA{ཨཿ}\\
	\bottomrule
\end{tabularx}

Для передачи долготы гласного в санскритских заимствованиях внизу справа от читаемой (корневой) графемы подписывается \prfA{འ}, например: \prfB{ཀཱ}{ka:}, \prfB{ལཱ}{la:}.

При фонетической записи санскритских слог слоги часто конструируются иначе, чем это допускается правилами орфографии для тибетских слогов, например: \prfB{ཀརྨ་}{karma}, \prfB{དྷརྨ་}{dharma}, \prfB{ཀ་ནཡཱ་}{kany\={a}}\prfnote{Слово, обозначающее созвездие и знак зодиака Дева. В книге \prfA{ན} написана с подписной \prfA{ཡ} }.

Санскритские слова входили в тибетский язык не только путём фонетического заимствования, но и путём буквального перевода частей сложного слова (калька). Этот способ очень характерен для тибетской традиции при переводе на тибетский иноязычных текстов. Так, большинство сложных слов тибетской философской терминологии являются кальками с санскритских сложных слов, вошедших в тибетский язык при переводе санскритских текстов.

Под влиянием санскрита в тибетском языке обраэовалась большая группа сложных слов и устойчивых словосочетаний, функционирующих как слово. Это так называемые образные слова и выражения, которые наряду со словами, обозначающими названия того или иного предмета, явления, описательно передают эти же названия и употребляются главным образом в языке художественных произведений.
Ср., например:
\begin{prfsample}
	\item \prfC{ཆུ་སྐྱར་གནས་}{chu-\ul{s}kyar-gna\ul{s}}{<<обитель чаек>>} --- \prfC{སྤྲིན་}{\ul{s}prin}{<<облако>>};
	\item \prfC{ཆུ་ལས་བྱུང་བ་}{chu-la\ul{s}-byung-ba}{<<из воды возникший>>} --- \prfC{བྱེ་མ་}{bye-ma}{<<песок>>};
	\item \prfC{ཆུའི་སྲེ་མ་}{chu'i-sre-ma}{<<водяная куница>>} --- \prfC{སྲམ་}{sram}{<<выдра>>}	
\end{prfsample}
 и т.п.

Ряд образных выражений имеет несколько значений. Ср., например: \prfC{ཆུ་སྐྱོབ་}{chu-\ul{s}kyob}{<<защита от воды>>} --- 1) \prfC{རི་}{ri}{<<гора>>}, 2) \prfC{ཞྭ་མོ་}{zh\ul{w}a-mo}{<<шапка>>}, 3) \prfC{ཁང་ཐོག་}{khang-thog}{<<крыша>>}.

Для многих предметов имеется несколько образных выражений, так, для солнца их 71, например: \prfC{བཅུ་གཉིས་བདག་པོ་}{\ul{b}cu-\ul{g}nyi\ul{s}-\ul{b}dag-po}{<<владыка двенадцати [знаков зодиака]>>} и т.п.

К образным выражениям примыкают эпитеты и имена различных божеств, духов, мифических животных, например: \prfC{ཆགས་འཇོམས་}{chag\ul{s}-'jom\ul{s}}{<<Победивший страсти>>} --- один из многочисленных эпитетов Будды Гаутамы; \prfC{བཅུ་གཅིག་ཞལ་}{\ul{b}cu-\ul{g}cig-zhal}{<<Одиннадцатиликий>>} --- одно из имен бодхисаттвы Авалокитешвары. В тибетском толковом словаре\footnote[18]{\prfA{དགེ་བཤེས་ཆོས་ཀྱི་གྲགས་པས་བརྩམས་པའི་བརྡ་དག་མིང་ཚིག་གསལ་བ་བཞུགས་སོ༎པེ་ཅིན་}, 1957.} такого рода образные выражения и эпитеты составляют около 10\% всех словарных статей.

Ряд \emph{заимствований из китайского языка} также относится ещё к древнему периоду (VI--VII вв. н.э.) и в настоящее время воспринимается как слова тибетского корня, например: \prfC{ཤག་}{\mfa{sja}\toneG}{<<дом>>} восходит к китайскому {\chinfont ??} \textit{ся}. Китайские заимствования, бытующие в тибетском языке, в основном представляют социально-бытовые и земледельческие термины, например: \prfB{པིར་}{pi:} (кит. {\chinfont 筆} \textit{би}) <<кисть для письма>>, \prfB{ཁོ་ཙེ་}{\mfa{k'otse}} (кит. {\chinfont 筷子} \textit{куайцзы}) <<палочки для еды>>, \prfB{ཚལ་}{\mfa{ts'e:}} (кит. {\chinfont 菜} \textit{цай}) <<овощи>>.

В процессе формирования нового письменного языка тибетский язык пополнился обширной новой лексикой для обозначения самых разнообразных понятий, относящихся к различным сферам общественной жизни. Вся эта новая лексика входит в тибетский язык из китайского. Для ввода её используются следующие способы:

1. Фонетическое заимствование. Этим путём в тибетском языке передаётся преимущественно административная, научно-техническая терминология, частично политическая терминология и географические названия. Для удобства заимствования китайских слов в тибетский язык введена новая графема \prfB{ཧྥ}{фа}\prfnote{Хапхатафа}. Приведем примеры фонетического заимствования:

\begin{tabularx}{\textwidth}{*{3}{p{0.3\textwidth}}}
	\toprule
	\parbox[m]{0.3\textwidth}{\small Китайское слово} & \parbox[m]{0.3\textwidth}{\small Тибетское заимствование} & \parbox[m]{0.3\textwidth}{\small Значение}\\
	\midrule
	{\chinfont 部長} \textit{бучжан} & \prfB{པུའུ་ཀྲང་}{бучжанг} & <<министр>>\\
	{\chinfont 共產黨} \textit{гунчаньдан} & \prfB{གུང་ཁྲན་ཏང་}{гунгчендан} & <<компартия>>\\
	{\chinfont 方程式} \textit{фанчэнши} & \prfB{ཧྥང་ཁྲེང་ཧྲི་}{фангченгши} & <<формула>> (математическая)\\
	{\chinfont 電子} \textit{дяньцзы} & \prfB{དེན་ཙི་}{денцзи} & <<электрон>>\\
	{\chinfont 帆布} \textit{фаньбу} & \prfB{ཧྥན་ཕུའུ་}{фэнпу} & <<парусина>>\\
	\bottomrule
\end{tabularx}

2. Сочетание фонетического заимствования с переводом, например: {\chinfont 人民公社} \textit{женьмин гуншэ} передаётся как \prfB{མི་དམངས་ཀུང་ཧྲི་}{миманг гунгше}, где \textit{миманг} --- <<народ>>, а \textit{гунгше} --- фонетическое заимствование китайского слова \textit{гуншэ} <<коммуна>>; {\chinfont 非洲} \textit{фэйчжоу} <<Африка>> передаётся как \prfB{ཧྥེ་གླིང་}{фэлинг}, где \textit{фэ} --- фонетическое заимствование китайского слова \textit{фэй}, а \textit{линг} <<материк>> --- перевод китайского {\chinfont 洲} \textit{чжоу}.

3. Калька, т.е. перевод слова по частям, например: слово {\chinfont 汽船} \textit{цичуань} <<пароход>>, букв, <<пар-лодка>>, передаётся по-тибетски словом \prfB{རླངས་གྲུ་}{лангчжу}, где \textit{ланг} --- <<пар>>, а \textit{чжу} --- <<лодка>>.

4. Передача нового понятия при помощи тибетской лексики, например: {\chinfont 考古}  <<археология>> передаётся как \prfA{གནའ་རྫས་ལ་རྟོག་ཞིབ་} букв. <<исследование предметов старины, древности>>.

\emph{Монгольские заимствования} в тибетском языке также играют значительную роль. Они стали проникать в тибетский язык начиная с XIII--XIV вв., когда в царствование Хубилай-хана был установлен монгольский протекторат над Тибетом. Вторым периодом проникновения монгольских заимствований был конец XVI в. -- середина XVIII в. --- время правления преемников Гуши-хана Хошутского (1582-1654). И наконец к третьему периоду можно отвести XVIII-XIX вв., когда в буддийских монастырях Тибета появляются многочисленные выходцы из Монголии.

К монгольским заимствованиям относятся такие слова, как \prfC{ནོ་ཡོན་}{ноёон}{<<князь, начальник>>}, \prfC{དམ་ཁ་}{тамка}{<<печать>>}, \prfC{ཨ་ཅོར་}{ачор}{<<салфетка, полотенце>>}, \prfC{ཐར་བག་}{таваг}{<<тарелка>>}, \prfC{ཨེམ་ཆི་}{эмцы}{<<врач>>} и т.п.

В тибетском языке встречаются также заимствования из персидского (проникшие в Тибет как через Индию, так и непосредственно из Ирана) и тюркских языков.

С усилением экономических связей с Индией после Синхайской революции в тибетский язык проникли заимствования из хинди и урду, например, разговорное \prfC{ཏཱར་}{\mfa{ta:r}}{<<телеграмма>>} восходит к хинди \textit{tar} <<проволока, телеграмма>>.
