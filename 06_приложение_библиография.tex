\chapter{Приложение}

Предлагаемое приложение состоит из четырёх текстов (трех отрывков текста и сказки).

Первый отрывок представляет собой две шлоки из широко известного сочинения <<Сокровищница драгоценных назидательных речений>> (\prfB{ལེགས་པར་བཤད་པོ་རིང་པོ་ཆེའི་གཏེར་ཞེས་བྱ་བ་བསྟན་བཅོས་བཞུགས་སོ་}{leg\ul{s}-par \ul{b}sha\ul{d}-po ring-po-che'i \ul{g}ter zhe\ul{s}-bya-ba \ul{bs}tan-\ul{b}co\ul{s} \ul{b}zhug\ul{s}-so}), относящегося к началу XIII в.

Второй отрывок является началом популярного среди тибетцев рассказа <<История про птиц и обезьян>> (\prfB{བྱ་སྤྲེལ་གཏམ་རྒྱུད་}{bya \ul{s}pre\ul{l} \ul{g}tam-\ul{r}gyu\ul{d}}), относящегося к первой половине XIX в. Рассказ в эзоповской форме повествует о вторжении непальских турков в Тибет в конце XVIII в.

Третий отрывок взят из статьи <<Ежедневной газеты Тибетского района>> (\prfB{བོད་ལྗོང་ཉིན་རེ་གསར་འགྱུར་}{bo\ul{d}-\ul{l}jong nyin-re \ul{g}sar-'gyur}) от 20.VI.1960 г. Этот отрывок отражает новый письменный язык.

Предлагаемая сказка представляет собой образец старого письменного языка, находящегося под сильным влиянием разговорной речи.

Чтобы дать возможность читателю сопоставить русский перевод с транслитерацией, мы при транслитерации текстов отошли от общепринятой традиции. Так, начальное слово фразы даётся с прописной буквы, черточке в тибетском тексте соответствует запятая в транслитерации. Там, где в русском переводе фраза начинается с абзаца, в транслитерации она также даётся с абзаца, а знаки препинания в транслитерации (кроме запятой) соответствуют знакам препинания русского перевода.
\pagebreak

\section{Текст 1}

\begin{verse}
	\prfA{བློ་གྲོས་ཞན་པའི་འདོད་གཏམ་དང་།}\\
	\prfA{ཅང་ཤེས་མིན་པའི་རྟ་མཆོག་དང་།}\\
	\prfA{གཡུལ་ངོར་ལྷུང་བའི་རལ་གྲི་རྣམས།}\\
	\prfA{སུ་ཡི་གྲོགས་འགྱུར་ངེས་པ་མེད།}

	* * *

	\prfA{ཤེས་རབ་མེད་པའི་བླུན་པོ་རྣམས།}\\
	\prfA{མང་ཡང་དགྲ་ཡི་དབང་དུ་འགྲོ།}\\
	\prfA{གླང་ཆེན་སྟོབས་ལྡན་ཁྱུར་ལོངས་པ།}\\
	\prfA{རི་བོང་བློ་ལྡན་གཅིག་གིས་བཏུལ།}
\end{verse}

\begin{center}Транслитерация\end{center}

\begin{verse}
	\emph{\ul{B}lo-gro\ul{s} zhan-pa'i 'do\ul{d}-gtam dang,}\\
	\emph{cang-she\ul{s} min-pa'i \ul{r}ta-\ul{m}chog dang,}\\
	\emph{\ul{g}.yul-ngor lhung-ba'i ra\ul{l}-gri \ul{r}nams, ---}\\
	\emph{su-yi grog\ul{s} 'gyur nge\ul{s}-pa me\ul{d}.}

	* * *
	
	\emph{She\ul{s}-rab me\ul{d}-pa'i \ul{b}lun-po \ul{r}nam\ul{s}, ---}\\
	\emph{mang yang \ul{d}gra-yi \ul{d}bang-du 'gro,}\\
	\emph{\ul{g}lang-chen \ul{s}tob{s}-\ul{l}dan khyur long\ul{s}-pa}\\
	\emph{ri-bong \ul{b}lo-\ul{l}dan \ul{g}cig-gi\ul{s} \ul{b}tul.}
\end{verse}

\begin{center}Перевод\end{center}

\begin{verse}
	Безотчётные слова малоумного,\\
	породистая, но глупая лошадь,\\
	мечи, выроненные на поле боя, ---\\
	неизвестно кому сослужат службу.
	
	* * *

	Глупцов пусть будет и много, ---\\
	под власть врагов подпадут они;\\
	так стадо могучих слонов\\
	один умный заяц в покорность привел.
\end{verse}

\pagebreak

\section{Текст 2}

\prfA{སྔོན་ཡུལ་བཀྲ་ཤིས་མཛེས་དགའི་ཡུལ་གྱི་རི་ཀུན་བཟང་ཟེས་བྱ་བ་རྩ་བ་ནགས་འདབ་དང་།\quad{}   སྐེད་པ་སྤང་རི།\quad{}རྩེ་མོ་གངས་རི་ཡིན་པ་ཞིག་ཡོད་པར།\quad{}དེའི་རྩེ་མོ་ལ་སེང་གེ་དཀར་མོ་ཡི་ཁྱུ།\quad{}སྤང་རི་ལ་ལྷ་བྱ་གོང་མོ་གཙོས་བྱ་སྣ་ཚོགས་དང་།\quad{}རྩ་བ་ལ་སྟག་གཟིག་དོམ་དྲེད་ལ་སོགས་པའི་གཅན་གཟན་སྣ་ཚོགས་དང་།\quad{}ནགས་ཚལ་དེའི་ཁུལ་ཞིག་ན་སྤྲེའུ་སྤྲེན་གྱི་སྒྲ་གཅན་སོགས་སྤྲེའུ་མང་ཙམ་གནས་སོ༎}

\begin{center}Транслитерация\end{center}

\emph{\ul{S}ngon yul \ul{b}kra-shis-\ul{m}dze\ul{s}-dga'i yul-gyi ri kun-\ul{b}zang ze\ul{s}-bya-ba. \ul{R}tsa-ba nag\ul{s}-'dab dang, \ul{s}ke\ul{d}-pa \ul{s}pang-ri, \ul{r}tse-mo gang\ul{s}-ri yin-pa zhig yod-par.
De'i \ul{r}tse-mo-la seng-ge \ul{d}kar-mo-yi khyu, \ul{s}pang-ri-la lha-bya gong-mo \ul{g}tsos bya \ul{s}na-tshog\ul{s} dang, \ul{r}tsa-ba-la \ul{s}tag \ul{g}zig dom dre\ul{d} la-sog\ul{s}-pa'i \ul{g}can-\ul{g}zan \ul{s}na-tshog\ul{s} dang.
Nag\ul{s}-tsha\ul{l} de'i khul zhig-na \ul{s}pre'u \ul{s}pren-gyi \ul{s}gra-\ul{g}can sog\ul{s} \ul{s}pre'u mang-tsam gna\ul{s}-so.
}

\begin{center}Перевод\end{center}

В стародавние времена в местности Чжасицзега была гора, которую называли Кюнсанг (Прекрасная для всех). У подножия этой горы раскинулся дремучий лес, её склоны поросли зелёными лугами, а вершину покрывали вечные снега. На вершине горы жила стая белых львов, на склонах, покрытых зелёными лугами, гнездились всевозможные птицы, предводимые Небесной птицей --- белой куропаткой, у подножия горы в густом лесу обитали различные звери: тигры, леопарды, медведи. В одном из районов этого леса жило множество обезьян.

\section{Текст 3}

\prfA{ཞིང་ཁ་ཆེ་ཆུང་ཚང་མར་བདག་གཉེར་ཡག་པོ་བྱེད་དགོས།

ཟ་འབྲུ་འབོར་ཆེན་ཐོན་སྤེལ་བྱ་རྒྱུ་ནི་རང་རེའི་ལྗོངས་ཀྱི་ཞིང་ལས་ཐོན་སྐྱེད་ནང་གི་དོན་གནད་མངོན་ཅན་ཞིག་རེད།
ཟ་འབྲུའི་ཐོན་སྐྱེད་ཡར་རྒྱས་ཕྱིན་ན་གཞི་ནས་མི་དམངས་ཚོའི་འཚོ་བར་འགན་སྲུང་དང་མཐོ་རུ་གཏོང་ཐུབ་བ་མ་ཟད།
བཟོ་ལས་དང་དེ་མིན་གྱི་འཛུགས་སྐྲུན་ལས་དོན་ཡར་རྒྱས་གཏོང་རྒྱུར་ཡང་གཞི་ནས་འགན་སྲུང་བྱེད་ཐུབ་ཀྱི་རེད།
དེར་བརྟེན་ནག་ཚོས་ངེས་པར་དུ་ཟ་འབྲུ་ཐོན་སྤེལ་བྱེད་དགོས་པ་དེ་རང་རེའི་ལྗོང་ཀྱི་ཞིང་ལས་ཐོན་སྐྱེད་ཀྱི་གནས་དང་པོར་བཞག་སྟབས་ཟ་འབྲུ་ཐོན་སྤེལ་བྱ་རྒྱུར་མཐོང་ཆུང་བྱེད་པའི་བསམ་བློ་ཇི་འདྲ་ཡིན་ཡང་མི་འགྲིགས་པ་རེད།
འོན་ཀྱང་ཟ་འབྲུ་མ་ཡིན་པའི་དེ་མིན་ཞིང་ལས་ཐོན་སྐྱེད་ལ་མཐོང་ཆུང་བྱས་ན་འགྲིགས་ཀྱི་མ་རེད་ཅེ་ན།
འགྲིགས་ཀྱི་མ་རེད།
གང་ལེགས་ཤེ་ན།
ཟ་འབྲུའི་དོན་གནད་ཐག་མ་ཆོད་གོང་དུ་མི་ཚོས་ཟ་འབྲུའི་དོན་གནད་ཐག་གཅོག་བྱེད་དགོས་པའི་རེ་འདུན་འདོན་ངེས་དང་།
ཟ་འབྲུའི་དོན་གནད་ཐག་གཅོབ་ཟིན་རྗེས་ད་དུང་དོན་གནད་གཞན་དག་ཐག་གཅོད་བྱེད་དགོས་པའི་རེ་འདུན་འདོན་རྒྱུ་ཡིན་པ་དཔེར་ན།
སྣུམ་དང་ཤ་རིགས་ཀྱི་དོན་གནད་ལ་སོགས་པ་རེད༎
}

\begin{center}Транслитерация\end{center}

{\large \emph{Zhing-kha che-chung tshang-mar \ul{b}dag-\ul{g}nyer yag-po bye\ul{d} dgo\ul{s}}}

\emph{
Za-'bru 'bor-chen thon-\ul{s}pel-bya-\ul{r}gyu ni rang-re'i \ul{l}jongs-kyi zhing-la\ul{s} thon-\ul{s}kye\ul{d} nang-gi don-gna\ul{d} \ul{m}ngon-can zhig re\ul{d}.
Za-'bru'i thon-\ul{s}kye\ul{d} yar-\ul{r}gya\ul{s}-phyin-na \ul{g}zhi-na\ul{s} mi-\ul{d}mang\ul{s}-tsho'i 'tsho-bar 'gan-\ul{s}rung dang \ul{m}tho-ru-\ul{g}tong thub-ba ma-za\ul{d},
\ul{b}zo-la\ul{s} dang de-min-gyi 'dzug\ul{s}-\ul{s}krun la\ul{s}-don yar-\ul{r}gya\ul{s}-\ul{g}tong-\ul{r}gyur yang \ul{g}zhi-na\ul{s} 'gan-\ul{s}rung-bye\ul{d} thub-kyi-re\ul{d}.
Der-\ul{br}ten nag-tsho\ul{s} nge\ul{s}-par-du za-'bru thon-\ul{s}pel-bye\ul{d} \ul{d}go\ul{s}-pa de rang-re'i \ul{l}jong-kyi zhing-la\ul{s} thon-\ul{s}kye\ul{d}-kyi \ul{g}na\ul{s} dang-por \ul{b}zhag \ul{s}tab\ul{s} za-'bru thon-\ul{s}pel-bya-\ul{r}gyur \ul{m}thong-chung-bye\ul{d}-pa'i \ul{b}sam-\ul{b}lo ji-'dra yin yang mi 'grig\ul{s}-pa-re\ul{d}.
'On-kyang za-'bru ma yin-pa'i de-min zhing-la\ul{s} thon-\ul{s}kye\ul{d}-la \ul{m}thong-chung-bya\ul{s}-na 'grig\ul{s}-kyi-ma-red ce-na?
'Grig\ul{s}-kyi-ma-re\ul{d}.
Gang-leg\ul{s} she-na?
Za-'bru'i don-\ul{g}na\ul{d} thag ma cho\ul{d} gong du mi tsho\ul{s} za-'bru'i don-\ul{g}na\ul{d} thag-\ul{g}cog-bye\ul{d} \ul{d}go\ul{s}-pa'i re-'dun-'don nge\ul{s} dang,
za-'bru'i don-\ul{g}na\ul{d} thag-\ul{g}cob zin \ul{r}jes da-dung don-\ul{g}na\ul{d} \ul{g}zhan-dag thag-\ul{g}co\ul{d}-bye\ul{d} \ul{d}go\ul{s}-pa'i re-'dun-'don \ul{r}gyu-yin-pa \ul{d}per-na \ul{s}num dang sha-rig\ul{s}-kyi don-\ul{g}na\ul{d}-la sog\ul{s}-pa-re\ul{d}.
}

\begin{center}Перевод\end{center}

Необходимо хорошо обработать все поля

Увеличение производства зерновых культур является актуальной задачей сельского хозяйства нашего уезда. Увеличив производство зерновых, можно не только обеспечить прожиточный минимум населения и повысить его, но и обеспечить развитие промышленности и нужды строительства. Поэтому мы должны поставить во главу угла сельскохозяйственного производства нашего уезда задачу увеличения зерновых, и, следовательно, ни в какой мере недопустимо пренебрежительное отношение к увеличению производства зерновых культур. Но можно ли пренебрегать другими, не зерновыми, отраслями сельскохозяйственного производства? Нет, нельзя. Как же быть? Пока не будет решена проблема зерновых, люди непременно будут требовать решения этой проблемы. Когда же проблема зерновых будет разрешена, то встанет вопрос о разрешении других проблем, как, например, проблемы производства масла, мяса и т.д.

\section{Текст 4}

\prfA{རི་བོང་གིས་སེང་གེ་བསད་པ།

༄༅༎བློ་དང་ལྡན་ན་ཉམ་ཆུང་ཡང་།
སྟོབས་ལྡན་དགྲ་བོ་ཅི་བྱར་ཡོད།
རི་དྭགས་རྒྱལ་པོ་སྟོབས་ལྡན་ཡང་།
རི་བོང་བློ་དང་ལྡན་པས་བསད།
ཅེས་སེང་གེ་དང་རི་བོང་གཉིས་རྒྱུན་འགྲོགས་ཤིང་ཡུལ་གཅིག་ན་གནས་པ་ན།
སེང་གེ་རང་བཞིན་གྱིས་ང་རྒྱལ་ཆེ་ལ།
རི་བོང་སྟོབས་ཆུང་བས་དེ་ལ་སེང་གེས་ལན་གྲངས་མང་དུ་བརྙས་ཤིང་བཙེས་པ་ན།
མཐར་ཡིད་ཕུན་ནས་སྨྲས་པ་ཇོ་བོ་རང་ཉིད་དང་འདྲ་བའི་སེམས་ཅན་ཞིག་གིས་འདི་སྐད་ཟེར།
ང་དང་སྟོབས་འགྲན་ནུས་བ་ཡོད་ན་སྟོབས་འགྲན།
དེ་ལྟར་མ་ཡིན་ན་ངའི་བྲན་ཡིན་ནོ་ཟེར་མཁན་གཅིག་མཆིས་སོ།
ཞེས་སྒྲས་པ་དང་།
སེང་གེ་ང་རྒྱལ་ཤིན་ཏུ་ཆེ་བས་དེ་ལྟ་བུ་དེ་གང་ན་ཡོད།
དེ་བཞིན་སྟོབས་འགྲན་པར་བྱའོ་ཞེས་པ་དང་།
རི་བོང་གིས་ཁྲིད་ནས་སོང་སྟེ།
ཁྲོན་པ་ཤིན་ཏུ་གཏིང་རིང་བ་ཞིག་བསྟན་ནས་ཕ་གིའི་ནང་ན་གདའོ།
ཞེས་བསྟན་ཏོ།
སེང་གེ་དེའི་ཁར་སོང་བ་ན་གཟུགས་བརྙན་ལ་དགྲ་བོར་བསམ་སྟེ།
དྲིས་པ་ན་ཁོས་ཀྱང་འདྲི་བར་སྣང་ངོ་།
ཁོས་འཁྲབ་པ་ན་གཟུགས་བརྙན་དེས་ཀྱང་འཁྲབ་ཀྱིན་འདུག་གོ།
དེས་མཆེ་བ་གཙིགས་པ་ན་ཁོས་ཀྱང་མཆེ་བ་གཙིགས་པར་སྣང་ངོ་།
དེ་དག་ལ་སོགས་པ་ལུས་ཀྱི་རྣམ་འགྱུར་སྣ་ཚོགས་བསྟན་པས་དེས་ཀྱང་དེ་ལྟར་བྱེད་ཅིང་འདུག་པ་ན།
སེང་གེ་དེའི་སྟེང་དུ་མཆོངས་ནས་ཤིའོ།
བློ་གསལ་རི་བོང་དེ་གཡུལ་ལས་རྣམ་པར་རྒྱལ་བར་གྱུར་ཏོ༎
}


\begin{center}Транслитерация\end{center}

\emph{Ri-bong-gi\ul{s} seng-ge \ul{b}sa\ul{d}-pa}

\epigraph{\ul{B}lo-dang-\ul{l}dan-na nyam-chung yang, ---\\
\ul{s}tob\ul{s}-\ul{l}dan \ul{d}gra-bo ci byar yo\ul{d};\\
ri-d\ul{w}ag\ul{s} \ul{r}gya\ul{l}-po \ul{s}tob\ul{s}-\ul{l}dan yang,\\
ri-bong \ul{b}lo-dang-\ul{l}dan-pa\ul{s} \ul{b}sa\ul{d}, ce\ul{s}.}

\emph{Seng-ge dang ri-bong \ul{g}nyi\ul{s} \ul{r}gyun 'grog\ul{s}-shing yul \ul{g}cig-na \ul{g}na\ul{s}-pa-na.
Seng-ge rang-\ul{b}zhin-gyi\ul{s} nga-\ul{r}gya\ul{l} che la,
ri-bong \ul{s}tob\ul{s} chung-ba\ul{s} de-la seng-ges lan-grangs mang-du-\ul{br}nya\ul{s}-shing \ul{b}tse\ul{s}-pa-na.
\ul{M}thar yi\ul{d}-phun-na\ul{s} \ul{s}mra\ul{s}-pa: <<Jo-bo! Rang-nyi\ul{d} dang-'dra-ba'i sems-can zhig-gi\ul{s} 'di \ul{s}ka\ul{d} zer:
"Nga dang \ul{s}tob\ul{s} 'gran nu\ul{s}-ba yo\ul{d}-na \ul{s}tob\ul{s}-'gran.
De-\ul{l}tar ma yin-na nga'i bran yin no" zer-\ul{m}khan \ul{g}cig \ul{m}chi\ul{s} so>>, ---
zhe\ul{s} \ul{s}gra\ul{s}-pa dang.}

\emph{Seng-ge nga-\ul{r}gya\ul{l} shin-tu che-ba\ul{s}: <<De-\ul{l}ta-bu de gang-na yo\ul{d}?
De-\ul{b}zhin \ul{s}tob\ul{s} 'gran-par-bya'o>>, --- zhe\ul{s}-pa dang.
Ri-bong-gi\ul{s} khri\ul{d}-na\ul{s} song-\ul{s}te,
khron-pa shin-tu \ul{g}ting-ring-ba zhig \ul{bs}tan-na\ul{s}: <<Pha-gi'i nang-na \ul{g}da'o>>, ---
zhe\ul{s} \ul{bs}tan to.
Seng-ge de'i khar song-ba-na gzug\ul{s}-\ul{br}nyan-la \ul{d}gra-bor \ul{b}sam-\ul{s}te.
Dri\ul{s}-pa-na kho\ul{s} kyang 'dri-bar-\ul{s}nang ngo.
Kho\ul{s} 'khrab-pa-na \ul{g}zug\ul{s}-\ul{br}nyan de\ul{s} kyang 'khrab-kyin-'dug go.
De\ul{s} \ul{m}che-ba \ul{g}tsig\ul{s}-pa-na kho\ul{s} kyang \ul{m}che-ba \ul{g}tsig\ul{s}-par-\ul{s}nang ngo.
De-dag-la sog\ul{s}-pa lu\ul{s}-kyi \ul{r}nam-'gyur \ul{s}na-tshog\ul{s} \ul{bs}tan-pa\ul{s} de\ul{s} kyang de-\ul{l}tar bye\ul{d}-cing 'dug-pa-na.
Seng-ge de'i \ul{s}teng-du \ul{m}chong\ul{s}-na\ul{s} shi'o.
\ul{B}lo-\ul{g}sa\ul{l} ri-bong de g.yu\ul{l} la\ul{s} \ul{r}nam-par \ul{r}gya\ul{l}-bar-gyur to.
}

\begin{center}Перевод\end{center}

Как кролик льва погубил

\epigraph{Умный пусть будет и слаб, ---\\
с ним бессилен сильный враг;\\
царя зверей в расцвете сил\\
умный кролик погубил.}{}

Жили в одной местности лев и кролик. У льва был очень заносчивый характер и он постоянно обижал и оскорблял слабенького кролика. В конце концов пораскинул кролик мозгами и сказал льву: <<О господин! Появилось какое-то существо, похожее на тебя и сказало: "Если [лев] может потягаться со мной силой, то пусть потягается. Если же нет, то станет моим рабом"{}>>.

Характер у льва был очень надменный, и сказал он: <<Где это [существо]? Я померяюсь с ним силой>>. Кролик повел [льва], привел [его] к очень глубокому колодцу и,
указав на колодец, сказал: <<Там, внутри>>. Лев подошёл к краю [колодца] и принял [свое] отражение [в колодце] за врага. Обратился [к нему] с вопросом, оно (отражение) ответило тем же вопросом. Замахнулся [на него] лев, отражение тоже замахнулось. Лев выпустил когти, отражение также выпустило когти. Лев принимал различные [устрашающие] позы, то же самое делало и отражение. Тогда лев прыгнул на [свое] отражение и погиб в колодце. Так умный кролик вышел победителем.

\section{Словарь}

Словарь включает все слова и выражения, встречающиеся в текстах 1--4. Словарные статьи расположены в алфавитном порядке читаемых (основных) графем. Словарные статьи в пределах одной читаемой графемы располагаются по степени усложнения слога: сначала идут слоги с приписными графемами в алфавитном порядке этих графем, затем с огласовками, подписными, префиксальными и, наконец, с надписными графемами.

\begin{center}\prfB{ཀ}{ka}\end{center}
\begin{description}
	\item \prfC{ཀུན་}{kun}{<<все; весь, целый>>}
	\item \prfC{དཀར་པོ་}{\ul{d}kar-po}{<<белый>>}
	\item \prfC{སྐད་}{\ul{s}ka\ul{d}}{<<слово; речь>>}
	\item \prfC{སྐེད་པ་}{\ul{s}ke\ul{d}-pa}{<<поясница; талия>>}
\end{description}

\begin{center}\prfB{ཁ}{kha}\end{center}
\begin{description}
	\item \prfC{ཁ་}{kha}{<<рот; отверстие>>}
	\item \prfC{ཁུལ་}{khu\ul{l}}{<<район, область>>}
	\item \prfC{ཁྱུ་}{khyu}{<<стая; стадо>>}
	\item \prfB{ཁྲིད་}{khri\ul{d}} --- прош. вр. от \prfC{འཁྲིད་}{'khri\ul{d}}{<<вести>>}
	\item \prfC{ཁྲོན་པ་}{khron-pa}{<<колодец>>}
	\item \prfC{འཁྲབ་}{'khrab}{<<замахиваться>>}
\end{description}

\begin{center}\prfB{ག}{ga}\end{center}
\begin{description}
	\item \prfC{གང་ན་}{gang-na}{<<где>>}
	\item \prfC{གང་ལེགས་}{gang-leg\ul{s}}{<<как лучше?>>}
	\item \prfC{གངས་}{gang\ul{s}}{<<снег>>}
	\item \prfB{གིས་}{gi\ul{s}} см. \prfB{གྱིས་}{gyi\ul{s}}
	\item \prfB{གྱིས་}{gyi\ul{s}} --- вариант орудной падежной частицы
	\item \prfC{གྲོགས་}{grog\ul{s}}{<<друг>>}
	\item \prfC{གླང་ཆེན་}{\ul{g}lang-chen}{<<слон>>}
	\item \prfC{དགོས་}{\ul{d}go\ul{s}}{<<быть необходимым, долженствовать>>}
	\item \prfC{དགྲ་བོ་}{\ul{d}gra-bo}{<<враг>>}
	\item \prfC{འགན་སྲུང་}{'gan-s\ul{r}ung}{<<гарантия, обеспечение>>}
	\item \prfC{འགན་སྲུང་གཏོང་}{'gan-s\ul{r}ung-\ul{g}tong}{<<гарантировать, обеспечивать>>}
	\item \prfB{འགྱུར་}{'gyur} (прош. вр. \prfB{གྱུར་}{gyur}) <<становиться, превращаться>>
	\item \prfC{འགྲན་}{'gran}{<<соревноваться, тягаться>>}
	\item \prfC{འགྲན་པར་བྱ་}{'gran-par-bya}{<<буду соревноваться, тягаться>>}
	\item \prfB{འགྲིགས་}{'grig\ul{s}} --- прош.вр. от \prfC{འགྲིག་}{'grig}{<<быть допустимым>>}
	\item \prfB{འགྲོ་}{'gro} (прош.вр. \prfB{ཕྱིན་}{phyin}) <<идти>>
	\item \prfB{འགྲོགས་}{'grog\ul{s}} --- прош.вр. от \prfC{འགྲོག་}{'grog}{<<водить компанию, дружить>>}
	\item \prfC{རྒྱུན་}{\ul{r}gyun}{<<постоянно>>}
\end{description}
	
\begin{center}\prfB{ང}{nga}\end{center}
\begin{description}
	\item \prfC{ང་}{nga}{<<я>>} (местоимение 1-го л. ед.числа)
	\item \prfC{ང་རྒྱལ་}{nga-rgya\ul{l}}{<<высокомерие, надменность>>}
	\item \prfC{ངེས་པ་}{nge\ul{s}-pa}{<<определённый; обязательный>>}
	\item \prfC{ངེས་པར་དུ་}{nge\ul{s}-par-du}{<<обязательно, непременно>>}
	\item \prfC{མངོན་ཅན་}{\ul{m}ngon-can}{<<актуальный, первостепенный>>}
	\item \prfC{སྔོན་}{\ul{s}ngon}{<<давно; ранее, прежде>>}
\end{description}
	
\begin{center}\prfB{ཅ}{ca}\end{center}
\begin{description}
	\item \prfC{ཅང་ཤེས་}{cang-she\ul{s}}{<<всезнание>>}
	\item \prfC{ཅི་བྱར་ཡོད་}{ci-byar-yo\ul{d}}{<<что можно поделать>> или <<ничего не сделаешь>>}
	\item \prfC{ཅེས་}{ce\ul{s}}{<<говорить; говориться>>}
	\item \prfC{གཅན་གཟན་}{\ul{g}can-\ul{g}zan}{<<крупный хищник>>}
	\item \prfC{གཅིག་}{\ul{g}cig}{<<один>>}
\end{description}
	
\begin{center}\prfB{ཆ}{cha}\end{center}
\begin{description}
	\item \prfC{ཆུང་}{chung}{<<быть маленьким>>}
	\item \prfC{ཆེ་}{che}{<<быть большим>>}
	\item \prfC{ཆེ་ཆུང་}{che-chung}{<<величина, размер>>}
	\item \prfB{མཆིས་}{\ul{m}chi\ul{s}} --- прош. вр. от \prfC{མཆི་}{\ul{m}chi}{<<появляться, прибывать>>}
	\item \prfC{མཆེ་བ་}{\ul{m}che-ba}{<<клыки; зубы>>}
	\item \prfB{མཆོངས་}{\ul{m}chong\ul{s}} --- прош. вр. от \prfC{མཆོང་}{\ul{m}chong}{<<прыгать>>}
\end{description}

\begin{center}\prfB{ཇ}{ja}\end{center}
\begin{description}
	\item \prfC{ཇི་འདྲ་}{ji-'dra}{<<какого рода?>>}
	\item \prfC{ཇོ་བོ་}{jo-bo}{<<владыка, господин>>}
	\item \prfC{ལྗོངས་}{\ul{l}jongs}{<<уезд>>}
\end{description}

\begin{center}\prfB{ཉ}{nya}\end{center}
\begin{description}
	\item \prfC{ཉམ་ཆུང་}{nyam-chung}{<<скромный; слабый; немощный>>}
	\item \prfC{གཉིས་}{\ul{g}nyi\ul{s}}{<<два>>}
	\item \prfC{བརྙས་}{\ul{br}nya\ul{s}}{<<обижать; оскорблять>>}
\end{description}

\begin{center}\prfB{ཏ}{ta}\end{center}
\begin{description}
	\item \prfB{ཏོ་}{to} см. \prfB{སོ་}{so}
	\item \prfC{གཏམ་}{\ul{g}tam}{<<слова, речь>>}
	\item \prfC{གཏམ་རྒྱུད་}{\ul{g}tam-\ul{r}gyu\ul{d}}{<<рассказ, история>>}
	\item \prfC{གཏོང་རིང་བ་}{\ul{g}tong-ring-ba}{<<глубокий>>}
	\item \prfC{བཏུལ་}{\ul{b}tu\ul{l}}{<<покорять, подчинять; усмирять>>}
	\item \prfC{རྟ་མཆོག་}{\ul{r}ta-\ul{m}chog}{<<породистая лошадь>>}
	\item \prfC{ལྟ་བུ་}{\ul{l}ta-bu}{<<подобный>>}
	\item \prfC{སྟག་}{\ul{s}tag}{<<тигр>>}
	\item \prfC{སྟབས་}{\ul{s}tab\ul{s}}{<<так как, из-за>> (союз)}
	\item \prfB{སྟེ་}{\ul{s}te} --- частица, образующая незавершенные формы глагола
	\item \prfC{སྟེང་}{\ul{s}teng}{<<верх>>}
	\item \prfC{སྟེང་ཏུ་}{\ul{s}teng-tu}{<<на>> (предлог)}
	\item \prfC{སྟོབས་}{\ul{s}tob\ul{s}}{<<сила, мощь>>}
	\item \prfC{སྟོབས་ལྡན་}{\ul{s}tob\ul{s}-\ul{l}dan}{<<сильный, мощный>>}
	\item \prfC{བསྟན་}{\ul{bs}tan} прош. вр. от \prfC{སྟོན་}{\ul{s}ton}{<<показывать, указывать>>}
\end{description}
	
\begin{center}\prfB{ཐ}{tha}\end{center}
\begin{description}
	\item \prfC{ཐག་གཅོད་}{thag-\ul{g}co\ul{d}}{<<решать, разрешать>>}
	\item \prfC{ཐག་ཆོད་}{thag-cho\ul{d}}{<<решаться, разрешаться>>}
	\item \prfC{ཐུབ་}{thub}{<<мочь, быть в состоянии>>}
	\item \prfC{ཐོན་སྐྱེད་}{thon-\ul{s}kye\ul{d}}{<<производство>>}
	\item \prfC{ཐོན་སྤེལ་}{thon-\ul{s}pe\ul{l}}{<<увеличение производства>>}
	\item \prfC{ཐོན་སྤེལ་བྱ་རྒྱུ་}{thon-\ul{s}pe\ul{l}-bya-\ul{r}gyu}{<<увеличение производства>>}
	\item \prfC{ཐོན་སྤེལ་བྱེད་}{thon-\ul{s}pe\ul{l}-bye\ul{d}}{<<увеличивать производство>>}
	\item \prfC{མཐོ་རུ་གཏོང་}{\ul{m}tho-ru-\ul{g}tong}{<<повышать>>}
	\item \prfC{མཐོང་ཆུང་བྱེད་}{\ul{m}tho-chung-bye\ul{d}}{<<пренебрегать>>}
\end{description}

\begin{center}\prfB{ད}{da}\end{center}
\begin{description}
	\item \prfC{དང་}{dang}{<<и, с>> (соединительный союз)}
	\item \prfC{དང་པོ་}{dang-po}{<<первый>>}
	\item \prfC{དམ་}{dam}{<<или>>}
	\item \prfC{དེ་}{de}{<<тот, та, то>>}
	\item \prfC{དེ་ལྡར་}{de-\ul{l}tar}{<<так, подобно тому>>}
	\item \prfC{དེ་མིན་}{de-min}{<<кроме того>>}
	\item \prfC{དེར་བརྟེན་}{der-\ul{br}ten}{<<поэтому>>}
	\item \prfC{དོན་གནད་}{don-\ul{g}na\ul{d}}{<<проблема, задача; дело>>}
	\item \prfC{དོམ་}{dom}{<<медведь>>}
	\item \prfB{དྲིས་}{dri\ul{s}} --- прош. вр. от \prfC{འདྲི་}{'dri}{<<спрашивать>>}
	\item \prfC{དྲེད་}{dre\ul{d}}{<<желтый медведь>>}
	\item \prfC{གདའ་}{\ul{g}da}{<<иметься, быть>>}
	\item \prfC{བདག་གཉེར་བྱེད་}{\ul{b}dag-\ul{g}nyer-bye\ul{d}}{<<хозяйствовать, управлять, заведовать>>}
	\item \prfC{འདི་}{'di}{<<этот, эта, это>>}
	\item \prfC{འདོད་}{'do\ul{d}}{<<желать, хотеть>>}
	\item \prfC{འདྲ་བ་}{'dra-ba}{<<подобный, похожий>>}
	\item \prfC{འདྲི་བ་}{'dri-ba}{<<вопрос>>}
\end{description}
	
\begin{center}\prfB{ན}{na}\end{center}
\begin{description}
	\item \prfC{ན་}{na I}{<<если>>}
	\item \prfB{ན་}{na II} --- вариант косвенной падежной частицы
	\item \prfC{ནགས་འདབ་}{nag\ul{s}-'dab}{<<густой, дремучий лес>>}
	\item \prfC{ནགས་ཚུལ་}{nag\ul{s}-tshul}{<<лес>>}
	\item \prfC{ནང་}{nang}{<<внутренняя часть; внутренний>>}
	\item \prfC{ནང་ན་}{nang-na}{<<внутри>>}
	\item \prfB{ནས་}{na\ul{s}} --- частица, образующая незавершенные формы глагола
	\item \prfC{ནུས་}{nu\ul{s}}{<<мочь, быть в состоянии>>}
	\item \prfC{གནས་}{\ul{g}na\ul{s}}{1. <<жить, пребывать>>; 2. <<место>>}
	\item \prfC{རྣམ་འགྱུར་}{\ul{r}nam-'gyur}{<<вид, внешний облик>>}
	\item \prfC{རྣམ་པར་རྒྱལ་བར་འགྱུར་}{\ul{r}nam-par-\ul{r}gya\ul{l}-bar-'gyur}{<<одержать полную победу>>}
	\item \prfC{སྣ་ཚོགས་}{\ul{s}na-tshog\ul{s}}{<<различный, разнообразный>>}
	\item \prfC{སྣང་}{\ul{s}nang}{<<быть видимым; ощущаться; восприниматься>>}
	\item \prfC{སྣུམ་}{\ul{s}num}{<<масло>>}
\end{description}
   
\begin{center}\prfB{པ}{pa}\end{center}
\begin{description}
	\item \prfC{སྤར་རི་}{\ul{s}par-ri}{<<горный, альпийский луг>>}
	\item \prfC{སྤྲེའུ་}{\ul{s}pre'u}{<<обезьяна>>}
	\item \prfC{སྤྲེལ་}{\ul{s}pre\ul{l}}{<<обезьяна>>}
\end{description}
	
\begin{center}\prfB{ཕ}{pha}\end{center}
\begin{description}
	\item \prfC{ཕ་གི་}{pha-gi}{<<там>>}
\end{description}

\begin{center}\prfB{བ}{ba}\end{center}
\begin{description}
	\item \prfC{བྱ་}{bya}{<<птица>>}
	\item \prfC{བྲན་}{bran}{<<раб>>}
	\item \prfC{བླུན་པོ་}{\ul{b}lun-po}{<<глупец>>}
	\item \prfC{བློ་}{\ul{b}lo}{<<разум>>}
	\item \prfC{བློ་དང་ལྡན་}{\ul{b}lo-dang-\ul{l}dan}{<<обладать умом, быть умным>>}
	\item \prfC{བློ་གྲོས་}{\ul{b}lo-gro\ul{s}}{<<острый ум>>}
	\item \prfC{བློ་གསལ་}{\ul{b}lo-\ul{g}sa\ul{l}}{<<светлый ум>>}
	\item \prfC{དབང་}{\ul{d}bang}{<<власть; влияние>>}
	\item \prfC{འབོར་ཆེན་}{'bor-chen}{<<большое количество>>}
\end{description}

\begin{center}\prfB{མ}{ma}\end{center}
\begin{description}
	\item \prfC{མ་ཟད་}{ma-za\ul{d}}{<<не только\ldots{} но>>}
	\item \prfC{མང་}{mang}{<<много>>}
	\item \prfC{མི་དམངས་}{mi-\ul{d}mang\ul{s}}{<<народ, люди>>}
	\item \prfC{མེད་པ་}{me\ul{d}-pa}{<<не имеющий>>}
	\item \prfB{སྨྲས་}{\ul{s}m\ul{r}a\ul{s}} --- прош. вр. от \prfC{སྨྲ་}{\ul{s}m\ul{r}a}{<<говорить>>}
\end{description}

\begin{center}\prfB{ཙ}{tsa}\end{center}
\begin{description}
	\item \prfB{གཙིགས་}{\ul{g}tsig\ul{s}} --- прош. вр. от \prfC{གཙིག་}{\ul{g}tsig}{<<скалить [зубы]>>}
	\item \prfC{གཙོས་}{\ul{g}tso\ul{s}}{<<быть главным, стоять во главе>>}
	\item \prfB{བཙེས་}{\ul{b}tse\ul{s}}(правильнее \prfB{གཙེས་}{\ul{g}tse\ul{s}}) --- прош. вр. от \prfC{འཚེ་}{'tshe}{<<вредить; мучить>>}
	\item \prfC{རྩ་བ་}{\ul{r}tsa-ba}{<<корень, основа>>}
	\item \prfC{རྩེ་མོ་}{\ul{r}tse-mo}{<<вершина>>}
\end{description}

\begin{center}\prfB{ཚ}{tsha}\end{center}
\begin{description}
	\item \prfC{ཚང་མ་}{tshang-ma}{<<все; всё>>}
	\item \prfC{འཚོ་བ་}{'tsho-ba}{<<жизнь>>}
\end{description}
	
\begin{center}\prfB{ཛ}{dza}\end{center}
\begin{description}
	\item \prfC{འཛུགས་སྐྲུན་}{'dzug\ul{s}-\ul{s}krun}{<<строительство>>}
\end{description}
	
\begin{center}\prfB{ཞ}{zha}\end{center}
\begin{description}
	\item \prfC{ཞན་པ་}{zhan-pa}{<<неважный, низкопробный, плохой>>}
	\item \prfC{ཞིག་}{zhig}{<<некий>> (частица неопределённости)}
	\item \prfC{ཞིང་ཁ་}{zhing-kha}{<<обработанная земля, поле>>}
	\item \prfC{ཞིང་ལས་}{zhing-la\ul{s}}{<<сельское хозяйство>>}
	\item \prfB{ཞེས་}{zhe\ul{s}} см. \prfB{ཅེས་}{ce\ul{s}}
	\item \prfC{ཞེས་བྱ་བ་}{zhe\ul{s}-bya-ba}{<<называемый, именуемый>>}
	\item \prfC{གཞི་ནས་}{gzhi-na\ul{s}}{<<только тогда>>}
	\item \prfC{བཞག་}{\ul{b}zhag}{<<ставить; класть>>}
\end{description}
	
\begin{center}\prfB{ཟ}{za}\end{center}
\begin{description}
	\item \prfC{ཟ་འབྲུ་}{za-'bru}{<<зерновые культуры>>}
	\item \prfC{ཟེར་}{zer}{<<говорить>>}
	\item \prfC{ཟེར་མཁན་}{zer-\ul{m}khan}{<<говорящий>>}
	\item \prfC{གཟིག་}{\ul{g}zig}{<<леопард>>}
	\item \prfC{གཟུགས་བརྙན་}{\ul{g}zug\ul{s}-\ul{br}nyan}{<<отражение>>}
	\item \prfC{བཟང་}{\ul{b}zang}{<<быть хорошим>>}
	\item \prfC{བཟོ་ལས་}{\ul{b}zo-la\ul{s}}{<<промышленность>>}
\end{description}
   
\begin{center}\prfB{འ}{'a}\end{center}
\begin{description}
	\item \prfC{འོང་ཀྱང་}{'ong-kyang}{<<однако>>}
\end{description}

\begin{center}\prfB{ཡ}{ya}\end{center}
\begin{description}
	\item \prfC{ཡག་པོ་}{yag-po}{<<хороший>>}
	\item \prfC{ཡར་རྒྱས་}{yar-\ul{r}gya\ul{s}}{<<развитие>>}
	\item \prfC{ཡར་རྒྱས་འགྲོ་}{yar-\ul{r}gya\ul{s}-'gro}{<<развиваться>>}
	\item \prfC{ཡར་རྒྱས་གཏོང་}{yar-\ul{r}gya\ul{s}-{g}tong}{<<развивать>>}
	\item \prfC{ཡིད་ཕུན་}{yid-phun}{<<раскидывать умом>>}
	\item \prfC{ཡིན་}{yin}{<<есть>> (глагол-связка)}
	\item \prfC{ཡོད་}{yo\ul{d}}{<<быть, иметься; находиться>>}
	\item \prfC{ཡུལ་}{yu\ul{l}}{<<страна; местность; район>>}
	\item \prfC{གཡུལ་}{\ul{g}.yu\ul{l}}{<<битва, сражение>>}
	\item \prfC{གཡུལ་ངོ་}{\ul{g}.yu\ul{l}-ngo}{<<поле боя>>}
\end{description}

\begin{center}\prfB{ར}{ra}\end{center}
\begin{description}
	\item \prfC{རང་ཉིད་}{rang-nyid}{<<сам>>}
	\item \prfC{རང་བཞིན་}{rang-{b}zhin}{<<характер, натура>>}
	\item \prfC{རང་རེ་}{rang-re}{<<каждый>>}
	\item \prfC{རལ་གྲི་}{ra\ul{l}-gri}{<<меч; кинжал>>}
	\item \prfC{རི་}{ri}{<<гора>>}
	\item \prfC{རི་དྭགས་}{ri-d\ul{w}ag\ul{s}}{<<красная дичь>> (олень, антилопа и т.п.)}
	\item \prfC{རི་དྭགས་རྒྱལ་པོ་}{ri-d\ul{w}ag\ul{s}-\ul{r}gya\ul{l}-po}{<<царь зверей, лев>>}
	\item \prfC{རི་བོང་}{ri-bong}{<<кролик; заяц>>}
	\item \prfC{རེ་འདུན་}{re-'dun}{<<требования; чаяния>>}
	\item \prfC{རེ་འདུན་འདོན་}{re-'dun-'don}{<<выставлять требования>>}
	\item \prfC{རེད་}{re\ul{d}}{<<есть, быть>> (глагол-связка)}
\end{description}

\begin{center}\prfB{ལ}{la}\end{center}
\begin{description}
	\item \prfC{ལན་གྲངས་}{lan-grang\ul{s}}{<<неоднократно>>}
	\item \prfB{ལས་}{la\ul{s}} --- исходная падежная частица
	\item \prfC{ལས་དོན་}{la\ul{s}-don}{<<дело>>}
	\item \prfC{ལུས་}{lu\ul{s}}{<<тело>>}
\end{description}

\begin{center}\prfB{ཤ}{sha}\end{center}
\begin{description}
	\item \prfC{ཤི་}{shi}{<<умирать>>}
	\item \prfB{ཤིང་}{shing} --- частица, образующая незавершенные формы глагола
	\item \prfC{ཤིན་ཏུ་}{shin-tu}{<<очень, весьма, крайне>>}
	\item \prfC{ཤེས་རབ་}{she\ul{s}-rab}{<<мудрость>>}
\end{description}

\begin{center}\prfB{ས}{sa}\end{center}
\begin{description}
	\item \prfC{སུ་}{su}{<<кто>>}
	\item \prfC{སེང་གེ་}{seng-ge}{<<лев>>}
	\item \prfC{སེམས་ཅན་}{sem\ul{s}-can}{<<живое существо>>}
	\item \prfB{སོ་}{so} --- конечная частица
	\item \prfB{སོང་}{song} --- прош. вр. от \prfC{འགྲོ་}{'gro}{<<идти>>}
	\item \prfB{བསད་}{\ul{b}sa\ul{d}} --- прош. вр. от \prfC{གསོད་}{\ul{g}so\ul{d}}{<<убивать>>}
	\item \prfB{བསམ་}{\ul{b}sam} (правильнее \prfB{བསམས་}{\ul{b}sam\ul{s}}) --- прош. вр. от \prfC{སེམས་}{sem\ul{s}}{<<думать>>}
\end{description}

\chapter{Краткая библиография}

\begin{enumerate}
	\item \emph{Рерих Ю.Н.}, Тибетский язык, М.: УРСС, 2001.
	\item \emph{Семичов Б.В.}, \emph{Парфионович Ю.М.}, \emph{Дандарон Б.Д.}, Тибетско-русский словарь, под ред. Ю.М. Парфионовича, М., 1963.
	\item \emph{Цыбиков Г.}, Пособие к изучению тибетского языка, Владивосток, 1908.
	\item \emph{Шмидт Я.}, Грамматика тибетского языка, СПб., 1839.
	\item \emph{Шмидт Я.}, Тибетско-русский словарь, СПб., 1843.
	\item <<Alphabetum Tangutanum Sive Tibetanum>>, Roma, 1773.
	\item \emph{Amundsen E.}, Primer of standard Tibetan, Darjeeling, [б.г.].
	\item \emph{Вacot J.}, Grammaire du tibetain litteraire, Paris, 1946.
	\item \emph{Вacot J.}, Index morfologique, Paris, 1948.
	\item \emph{Bacot J.}, Les slokas grammaticaux de Thonmi Sambhota, 2, Paris, 1928.
	\item \emph{Bacot J.}, La structure du Tibetain, Paris, 1954.
	\item \emph{Bacot J.}, L'Ecriture cursive tibetaine, Paris, 1912.
	\item \emph{Bell C.A.}, Grammar of Colloquial Tibetan, Calcutta, 1919.
	\item \emph{Bell C.A.}, English-Tibetan Colloquial Dictionary, Calcutta, 1920.
	\item \emph{Bell C.A.}, Manual of Colloquial Tibetan, Calcutta, 1905.
	\item \emph{Bhattacharya V.}, Bhota-Prakasa (A Tibetan chrestomathy), Calcutta, 1939.
	\item \emph{Chandra L.}, Tibetan-Sanskrit Dictionary, vol. 1-12, 1959-1960.
	\item \emph{Das S.C.}, A Tibetan-English Dictionary, Calcutta, 1902.
	\item \emph{Das S.C.}, An Introduction to the Grammar of the Tibetan Language, Darjeeling, 1915.
	\item \emph{DesgodinsA.}, Essai de grammaire tibetaine pour la language parle avec alphabet et prononciation, Paris, 1899.
	\item <<Dictionnaire thibetain-latin-franqais>>, Hongkong, 1899.
	\item \emph{Durr J.A.}, Morphologie du verbe Tibetain, Heidelberg, Winter, 1950.
	\item \emph{Foucaux P.E.}, Grammaire de la langue tibaine, Paris, 1858.
	\item \emph{Giorgi A.A.}, Alphabetum Thibetanum, Roma, 1761.
	\item \emph{Giraudeau P.}, Grammatica Latino-Thibetana, Hongkong, 1909.
	\item \emph{Giraudeau S.E.} et \emph{Core F.}, Dictionnaire francais-tibetain, Paris, 1956.
	\item \emph{Gould} and \emph{Richardson}, Tibetan Word Book, Oxford, 1943.
	\item \emph{Gould} and \emph{Richardson}, Tibetan Syllables, London, 1943.
	\item \emph{Hannah H.E.}, A Grammar of the Tibetan Language, Calcutta, 1912.
	\item \emph{Henderson V.C.}, Tibetan Manual, Calcutta, 1903.
	\item \emph{Jaeschke H.A.}, Handworterbuch der tibetisqhen Sprache, Gnadau, 1871.
	\item \emph{Jaeschke H.A.}, Tibetan Grammar, Berlin, 1929.
	\item \emph{Jaeschke H.A.}, A Tibetan-English Dictionary, London, 1881.
	\item \emph{Koerber H.W.}, Morphology of the Tibetan Language, Los Angelos-San Francisko, 1935.
	\item \emph{Körös A.Cs.}, A Grammar of the Tibetan Language, Calcutta, 1834.
	\item \emph{Körös A.Cs.}, Essai towards a Dictionary Tibetan and English, Calcutta, 1834.
	\item \emph{Kun Chang} and \emph{Shefts B.}, A Manual of Spoken Tibetan, Seatle, 1964.
	\item \emph{Lalou M.}, Manuel elementaire de Tibetain classique, Paris, 1950.
	\item \emph{Lewin T.H.}, A Manuel of Tibetan, Calcutta, 1879.
	\item \emph{Reorich G.}, The Tibetan Dialect of Lachul, Calcutta, 1933.
	\item \emph{Reorich G.}, Textbook of Colloqial Tibetan, Calcutta, 1957.
	\item \emph{Reorich G.}, Le parler de l Amdo, Roma, 1958.
	\item \emph{Richter E.}, Grundlagen der Phonetik des Lhasa-Dialektes, Berlin, 1964.
	\item \emph{Richter E.}, Tibetisch-Deutsches Wötrterbuch, Leipzig, 1966.
	\item \emph{Simonsson N.}, Indo-Tibetishe Studion, Uppsala, 1957.
	\item \emph{Uray G.}, The Glassification of the Dialects of the Eastern Tibet, Budapest, 1949.
	\item \emph{Walleser M.}, Zur Aussprache des Sanskrit und Tibetischen, Leipzig, 1926.
	\item \emph{Wangdi T.}, Tibetan-English-Hindi Guide, Calcutta, 1909.
	\item \emph{Wolfenden S.N.}, Outlines of Tibeto-Burman Linguistic Morphology, London, 1929.
	\item \emph{Wylie T.}, A Standard System Tibetan Transcription, Cambridge, 1959.
	\item \emph{Zla-ba-Bsam-'grub K.}, An English-Tibetan Dictionary, Calcutta, 1919.
	\item {\chinfont 金朋烏, 藏語拉蔭曰客則昌都詁的比較研究, 北京, 1958年}
	\item \prfA{སྐད་ལྡ་ཤན་སྦྱར་བའི་མའི་སྐད་གསལ་བའི་མེ་ལོར༎}
	\item \prfA{སྐད་བའི་ཤན་སྦྱར་བའི་མེ་ལོང་གི་ཡི་གེ༎}
	\item \prfA{ཀརྨ་སི་ཏུའི་སུམ་རྟག་འགྲེལ་ཆེན༎}
	\item \prfA{དགེ་བཤེས་ཆོས་ཀྱི་གྲགས་པས་བརྩམས་པའི་བརྡ་དག་མེང་ཚིག་གསལ་བ་བཞུགས་སོ, པེ་ཅིན, སྤྱི་ལོ 1957༎}
	\item \prfA{བོད་ཀྱི་བརྡའ་སྤྲོད་པ་སུམ་ཅུ་པ་དང་རྟགས་ཀྱི་འཇུག་པའི་མཆན་འགྲེལ་མདོར་བསྡུས་ཏེ་བཆཱོད་པ་རོ་མཚར་འཕྲུལ་གྱི་སྡེ་མིག་ཅེས་བྱ་བ་བཞུགས་སོ, པེ་ཅིན, སྤྱི་ལོ་ 1957༎}
\end{enumerate}
