\section{Имя существительное}

Имя существительное --- часть речи с общим значением предметности. Морфологически существительное в тибетском языке характеризуется: а) способом выражения отвлеченных и конкретных имен; б) своими средствами словообразования; в) наличием категорий множественности и пола.

Синтаксически существительное характеризуется тем, что: а) выступает в предложении подлежащим, дополнением, определением, обстоятельством и именной частью сказуемого; б) сочетается с притяжательной падежной частицей (её вариантами: \prfB{གི་}{gi}, \prfB{གྱི་}{gyi}, \prfB{ཀྱི་}{kyi}, \prfB{ཡི་}{yi}, \prfB{འི་}{'i}), которая соединяет препозиционное определение с определяемым; в) непосредственно принимает после себя определения, выраженные прилагательным или числительным; г) как правило, не сочетается с отрицательными частицами \prfB{མ་}{ma} и \prfB{མི་}{mi}.

\subsection{Имена отвлеченные и конкретные}

\emph{Имена отвлеченные} по своей структуре могут быть трёхсложными или двусложными. По способу образования они делятся на производные (I) и сложные (II).

I. Производные имена отвлеченные образуются от именных форм глагола прибавлением служебной морфемы \prfB{ཉིད་}{nyi\ul{d}}. Приведем примеры.

\begin{tabularx}{\textwidth}{*{3}{p{0.3\textwidth}}}
    \toprule
    \parbox[m]{0.3\textwidth}{Глагол} & \parbox[m]{0.3\textwidth}{\centering Именная форма глагола} & \parbox[m]{0.3\textwidth}{\centering Отвлеченное имя}\\
    \midrule
    \endhead
    \prfC[-p]{འཐུག་}{'thug}{<<быть густым>>} & \prfB{འཐུག་པ་}{'thug-pa} & \prfC[-p]{འཐུག་པ་ཉིད་}{'thug-pa-nyi\ul{d}}{<<густота>>}\\
    \prfC[-p]{ལྡམ་}{\ul{l}dam}{<<быть ленивым>>} & \prfB{ལྡམ་པ་}{\ul{l}dam-pa} & \prfC[-p]{ལྡམ་པ་ཉིད་}{\ul{l}dam-pa-nyi\ul{d}}{<<леность>>}\\
    \prfC[-p]{འབོལ་}{'bol}{<<быть добрым>>} & \prfB{འབོལ་བ་}{'bol-ba} & \prfC[-p]{འབོལ་བ་ཉིད་}{'bol-ba-nyi\ul{d}}{<<доброта>>}\\
    \prfC[-p]{སྒྲེ་}{\ul{s}gre}{<<быть нагим>>} & \prfB{སྒྲེ་བ་}{\ul{s}gre-ba} & \prfC[-p]{སྒྲེ་བ་ཉིད་}{\ul{s}gre-ba-nyi\ul{d}}{<<нагота>>}\\
    \prfC[-p]{སྟོང་}{\ul{s}tong}{<<быть пустым>>} & \prfB{སྟོང་བ་}{\ul{s}tong-ba} & \prfC[-p]{སྟོང་བ་ཉིད་}{\ul{s}tong-ba-nyi\ul{d}}{<<пустота>>}\\
    \bottomrule
\end{tabularx}

II. Сложные имена отвлеченные образуются путём соединения двух знаменательных морфем с противоположным значением, например:
\begin{prfsample}
    \item \prfC{ཆེ་}{che}{<<быть большим>>} и \prfC{ཆུང་}{chung}{<<быть маленьким>>} образуют сложное имя \prfC{ཆེ་ཆུང་}{che-chung}{<<величина, размер>>};
    \item \prfC{སྐམ་}{\ul{s}kam}{<<быть сухим>>} и \prfC{རློན་}{\ul{r}lon}{<<быть сырым>>} образуют сложное имя \prfC{སྐམ་རློན་}{\ul{s}kam-\ul{r}lon}{<<сырость, влажность>>}.    
\end{prfsample}

Имена конкретные по своей структуре могут быть односложными, двусложными и многосложными, однако для тибетского существительного двусложная структура наиболее характерна. По способу образования конкретные имена существительные делятся на простые (I), производные (II) и сложные (III).

I. Простые имена существительные состоят из одной, чаще односложной, свободной морфемы, например:
\begin{prfsample}
    \item \prfC{མི་}{mi}{<<человек>>},
    \item \prfC{རྟ་}{\ul{r}ta}{<<лошадь>>},
    \item \prfC{མིག་}{mig}{<<глаз>>},
    \item \prfC{གཟིག་}{\ul{g}zig}{<<леопард>>}.    
\end{prfsample}
Значительно реже простые имена существительные состоят из одной двусложной морфемы, например:
\begin{prfsample}
    \item \prfC{ཡི་གེ་}{yi-ge}{<<буква>>},
    \item \prfC{སྲོག་ལེ་}{s\ul{r}og-le}{<<пила>>}.   
\end{prfsample}

II. Производные имена существительные подразделяются на два типа.

Производимо имена \emph{первого типа}, а они охватывают большую часть производных имен существительных, состоят из одной знаменательной (свободной или связанной) морфемы и служебной словообразующей морфемы.

В тибетском языке можно выделить следующие служебные словообразующие морфемы:

\begin{enumerate}
    \item \prfB{པ་}{pa}, \prfB{པོ་}{po}, \prfB{བ་}{ba}, \prfB{བོ་}{bo}, \prfB{མ་}{ma}, \prfB{མོ་}{mo};
    \item \prfB{ཀ་}{ka}, \prfB{ཁ་}{kha}, \prfB{ག་}{ga}
    \footnote[25]{\prfB{ཀ་}{ka}, \prfB{ཁ་}{kha}, \prfB{ག་}{ga} --- три варианта одной словообразующей морфемы. Их употребление регламентируется следующим правилом тибетской орфографии: \prfB{ཀ་}{ka} употребляется тогда, когда знаменательная морфема имеет приписные графемы \prfB{ག}{ga}, \prfB{ད}{da}, \prfB{བ}{ba}, \prfB{ས}{sa}; \prfB{ཁ་}{kha} употребляется после приписных графем \prfB{ན}{na}, \prfB{ར}{ra}, \prfB{ལ}{la}; \prfB{ག་}{ga} --- после приписных графем \prfB{ང}{nga}, \prfB{མ}{ma}, \prfB{འ}{'a} или при отсутствии их. Однако эти правила не всегда строго соблюдаются.},
    \prfB{ཆ་}{cha};
    \item \prfB{གུ་}{gu}, \prfB{བུ་}{bu}.
\end{enumerate}

В зависимости от характера знаменательной морфемы указанные словообразующие морфемы могут выполнять различные функции.

1. В том случае, когда знаменательная морфема является связанной, т.е, неспособной выступать в качестве самостоятельного слова, словообразующие морфемы не
вносят какого-либо дополнительного лексического значения, а лишь образуют отдельно употребляющиеся слова. Так, например, от связанных знаменательных морфем
\begin{prfsample}
    \item \prfC{ཟམ་}{zam}{<<мост>>},
    \item \prfC{མཐོ་}{\ul{m}tho}{<<молоток>>},
    \item \prfC{དབྱར་}{\ul{d}byar}{<<лето>>},
    \item \prfC{གླ་}{\ul{g}la}{<<жалованье>>},
    \item \prfC{འབྲས་}{'bras}{<<плод>>} 
\end{prfsample}
при помощи служебных морфем образуются самостоятельные слова двусложной формы того же значения:
\begin{prfsample}
    \item \prfB{ཟམ་བ་}{zam-ba},
    \item \prfB{མཐོ་བ་}{\ul{m}tho-ba},
    \item \prfB{དབྱར་ཀ་}{\ul{d}byar-ka},
    \item \prfB{གླ་ཆ་}{\ul{g}la-cha},
    \item \prfB{འབྲས་བུ་}{'bras-bu}.
\end{prfsample}
При наличии знаменательных морфем-омонимов значение слова может различаться при помощи разных словообразующих морфем, например, от связанных морфем \prfC{གྲུ་}{gru}{(<<лодка>>)} и \prfC{གྲུ་}{gru}{(<<угол>>)} образуются разные двусложные слова: \prfC{གྲུ་བོ་}{gru-bo}{<<лодка>>} и \prfC{གྲུ་མོ་}{gru-mo}{<<угол>>}.

2. В том случае, когда знаменательная морфема является свободной, т.е. может выступать как самостоятельное слово, словообразующие морфемы, присоединяясь к морфеме-слову, образуют новые слова нескольких видов:

а) словообразующие морфемы \prfB{པ་}{pa}, \prfB{པོ་}{po}, \prfB{བ་}{ba}, \prfB{བོ་}{bo}, \prfB{མ་}{ma}, \prfB{མོ་}{mo} образуют новое слово со значением деятеля, профессии, национальности, принадлежности к той или иной местности, например:
\begin{prfsample}
    \item \prfC{རྟ་}{\ul{r}ta}{<<лошадь>>} --- \prfC{རྟ་པ་}{\ul{r}ta-pa}{<<всадник>>};
    \item \prfC{དགར་}{\ul{d}gar}{<<кузнечное дело>>} --- \prfC{དགར་བ་}{\ul{d}gar-ba}{<<кузнец>>};
    \item \prfC{ཀ་གནམ་}{ka-\ul{g}nam}{<<Канам>> (местность)} --- \prfC{ཀ་གནམ་པ་}{ka-\ul{g}nam-pa}{<<уроженец Канама, канамец>>}.    
\end{prfsample}

Одновременно с этим морфема \prfB{པོ་}{po} (реже \prfB{པ་}{pa}, \prfB{བ་}{ba}) может указывать на мужской пол деятеля, а морфема \prfB{མོ་}{mo} (реже \prfB{མ་}{ma}) --- на женский пол, например:
\begin{prfsample}
    \item \prfC{བོད་}{bod}{<<Тибет>>} --- \prfC{བོད་པ་}{bod-pa}{<<тибетец>>}, \prfC{བོད་མོ་}{bod-mo}{<<тибетянка>>};
\end{prfsample}


б) морфемы \prfB{ཀ་}{ka}, \prfB{ཁ་}{kha}, \prfB{ག་}{ga} (реже \prfB{ཆ་}{cha}, \prfB{བུ་}{bu}), присоединяясь к именной морфеме, образуют новые слова, семантически связанные со значением исходного слова, например:
\begin{prfsample}
    \item \prfC{གྲི་}{gri}{<<нож>>} --- \prfC{གྲི་ཁ་}{gri-kha}{<<лезвие>>};
    \item \prfC{ས་}{sa}{<<земля>>} --- \prfC{ས་ཆ་}{sa-cha}{<<местность>>};
    \item \prfC{ནོར་}{nor}{<<богатство>>} --- \prfC{ནོར་བུ་}{nor-bu}{<<драгоценность>>}.
\end{prfsample}
Эти же служебные морфемы, присоединяясь к предикативной морфеме, образуют имя со значением момента совершения действия, например:
\begin{prfsample}
    \item \prfC{འཆི་}{'chi}{<<умирать>>} --- \prfC{འཆི་ཀ་}{'chi-ka}{<<момент смерти, агония>>};
\end{prfsample}

в) морфемы \prfB{གུ་}{gu} и \prfB{བུ་}{bu} образуют уменьшительные имена, например:
\begin{prfsample}
    \item \prfC{ལུག་}{lug}{<<овца>>} --- \prfC{ལུག་གུ་}{lug-gu}{<<овечка, ягненок>>};
    \item \prfC{ཁྱི་}{khyi}{<<собака>>} --- \prfC{ཁྱི་གུ་}{khyi-gu}{<<собачонка, щенок>>};
    \item \prfC{ཕུད་}{phud}{<<мешок>>} --- \prfC{ཕུད་བུ་}{phud-bu}{<<мешочек>>}.
\end{prfsample}
Часто уменьшительные имена образуются при помощи суффикса \prfB{འུ་}{'u}, например:
\begin{prfsample}
    \item \prfC{མི་}{mi}{<<человек>>} --- \prfC{མིའུ་}{mi'u}{<<человечек, карлик>>};
    \item \prfC{སྟེ་}{\ul{s}te}{<<топор>>} --- \prfC{སྟེའུ་}{\ul{s}te'u}{<<топорик>>}.
\end{prfsample}

Производные имена \emph{второго типа} состоят из двух знаменательных морфем, одна из которых выступает в служебной функции. Эти имена можно подразделить на две группы:

1. Имена, где знаменательная морфема в служебной функции сохраняет свое общее лексическое значение. К знаменательным морфемам, выступающим в служебной функции, относятся:
\prfC{ཁང་}{khang}{<<дом>>}, \prfC{རྭ་}{r\ul{w}a}{<<ограда>>}, \prfC{སྟབས་}{\ul{s}tab\ul{s}}{<<способ, метод>>}, \prfC{མཁན་}{\ul{m}khan}{<<знающий, умелый>>} и некоторые другие.

Сочетаясь со знаменательными морфемами, эти морфемы образуют слова с однородным значением. Так, морфема \prfB{ཁང་}{khang}, сочетаясь со знаменательными морфемами, образует ряд слов со значением <<помещения>>, например:
\begin{prfsample}
    \item \prfC{ཚོང་ཁང་}{tshong-khang}{<<магазин>>}, где \prfC{ཚོང་}{tshong}{<<торговля>>};
    \item \prfC{སྨན་ཁང་}{\ul{s}man-khang}{<<аптека>>}, где \prfC{སྨན་}{\ul{s}man}{<<лекарство>>};
    \item \prfC{ཟ་ཁང་}{za-khang}{<<столовая>>}, где \prfC{ཟ་}{za}{<<есть, кушать>>};
\end{prfsample}
морфема \prfB{མཁན་}{\ul{m}khan} сочетаясь со знаменательными морфемами, образует ряд слов со значением лица, сведущего в чем-либо, умеющего что-либо делать, например:
\begin{prfsample}
    \item \prfC{ཡིག་མཁན་}{yig-\ul{m}khan}{<<писец, грамотей>>}, где \prfC{ཡིག་}{yig}{<<буква>>};
    \item \prfC{ཤིང་མཁན་}{shing-\ul{m}khan}{<<плотник, столяр>>}, где \prfC{ཤིང་}{shing}{<<дерево>>}.
\end{prfsample}

В новом письменном языке для образования ряда терминов используется слово \prfC{རིང་ལུགས་}{ring-lug\ul{s}}{<<учение, школа, секта>>}:
\begin{prfsample}
    \item \prfC{སྤྱི་ཙོགས་རིང་ལུགས་}{\ul{s}pyi-tsog\ul{s}-ring-lug\ul{s}}{<<социализм>>} (\prfC{སྤྱི་ཙོགས་}{\ul{s}pyi-tsog\ul{s}}{<<общество>>}),
    \item \prfC{བཙན་རྒྱལ་རིང་ལུགས་}{\ul{b}tsan-\ul{r}gya\ul{l}-ring-lug\ul{s}}{<<империализм>>} (\prfC{བཙན་རྒྱལ་}{\ul{b}tsan-\ul{r}gya\ul{l}}{<<империя>>}).
\end{prfsample}

2. Имена, где знаменательная морфема в служебной функции утратила свое лексическое значение. К таким именам относятся производные существительные, употребляемые в вежливой речи. Они образуются присоединением морфемы вежливой речи к обычной знаменательной морфеме. В этом случае лексическое значение производного слова полностью вытекает из лексического значения обычной морфемы, а знаменательная морфема вежливости указывает лишь на то, что производное слово принадлежит к вежливой речи. Например, \prfC{ཕྱག་}{phyag}{<<рука>>} образует группу слов вежливой речи:
\begin{prfsample}
    \item \prfC{ཕྱག་དཔེ་}{phyag-\ul{d}pe}{<<книга>>},
    \item \prfC{ཕྱག་ཤོག་}{phyag-shog}{<<бумага>>},
    \item \prfC{ཕྱག་ཁབ་}{phyag-khab}{<<игла>>},
\end{prfsample}
от слов обычной речи соответственно:
\begin{prfsample}
    \item \prfC{དཔེ་}{\ul{d}pe}{<<книга>>},
    \item \prfC{ཤོག་}{shog}{<<бумага>>},
    \item \prfC{ཁབ་}{khab}{<<игла>>}.
\end{prfsample}

III. Сложные имена составляют самую большую группу имен существительных. Они образуются:

1) путём сложения двух знаменательных морфем (связанных или свободных), связь между которыми вскрывается лишь этимологически. Следующая таблица (табл. \ref{tab:12}) показывает типы связей между знаменательными морфемами, образующими сложное слово, а также характер этих морфем.

2) путём лексикализации синтаксических образований двух типов:

а) <<определение --- определяемое>>, например:
\begin{prfsample}
    \item \prfC{མགོན་པ་མེད་པ་}{\ul{m}gon-pa me\ul{d}-pa}{<<неимеющий защитника>>}  --- 
\prfC{མགོན་མེད་}{\ul{m}gon-me\ul{d}}{<<беззащитный>>};
\end{prfsample}

б) <<обстоятельство образа действия --- сказуемое>>, например:
\begin{prfsample}
    \item \prfC{སྔོན་དུ་རྩིས་པ་}{\ul{s}ngon-du \ul{r}tsi\ul{s}-pa}{<<заранее подсчитать>>} --- \prfC{སྔོན་རྩིས་}{\ul{s}ngon-\ul{r}tsi\ul{s}}{<<смета, бюджет>>}.
\end{prfsample}

\begin{tabularx}{\textwidth}{p{0.2\textwidth}p{0.35\textwidth}p{0.35\textwidth}} %1,2,3
    \caption{Тип связи и характер морфем сложных имен существительных}\label{tab:12}\\
    \toprule
    \parbox{0.2\textwidth}{\centering Тип связи} & \parbox{0.35\textwidth}{\centering Характер морфем} & \parbox{0.35\textwidth}{\centering Примеры}\\
    \midrule
    \endhead
    Ко\-пу\-ля\-тив\-ная & предметная + предметная & \prfC{རྐང་ལག་}{\ul{r}kang-lag}{<<конечности>> (букв. <<рука-нога>>)}\\
    & предикативная + предикативная & \prfC{ཚོགས་འདུ་}{tshog\ul{s}-'du}{<<собрание, митинг>> (букв. <<собираться-собираться вместе>>)}\\
    \midrule
    Атри\-бу\-тив\-ная & предметная + предметная & \prfC{རྩྭ་ཐང་}{\ul{r}ts\ul{w}a-thang}{<<пастбище>> (букв. <<трава-равнина>>)}\\
    & предметная + качественная & \prfC{ལག་དམར་}{lag-\ul{d}mar}{<<палач>> (букв. <<рука-красный>>)} \\
    & вербальная + предметная & \prfC{རྒྱུགས་ཆུ་}{\ul{r}gyug\ul{s}-chu}{<<поток>> (букв. <<бежать-вода>>)}\\
    \midrule
    Субъектно-пре\-ди\-ка\-тив\-ная & предметная + вербальная & \prfC{ས་ཡོམ་}{sa-yom}{<<землетрясение>> (букв. <<земля-колебаться>>)}\\
    \midrule
    Объектно-пре\-ди\-ка\-тив\-ная & предметная + вербальная & \prfC{ཤ་འཛིན་}{sha-'dzin}{<<вилка>> (букв. <<мясо-держать>>)}\\
    \bottomrule
\end{tabularx}

\subsection{Грамматические категории существительных}

1. \emph{Категория множественности}. В тибетском языке отсутствует грамматическая категория числа: существительное само по себе не выражает ни единичности, ни множественности. Так, например,
\begin{prfsample}
    \item слово \prfB{ཡལ་ག་}{yal-ga} может означать и <<ветвь>> и <<ветки>>,
    \item слово \prfB{མེ་ཏོག་}{me-tog} может означать и <<цветок>> и <<цветы>>.    
\end{prfsample}
Однако существуют специальные служебные морфемы для передачи понятия множественности.

Наиболее употребительными служебными морфемами множественности являются \prfB{རྣམས་}{\ul{r}nam\ul{s}} и \prfB{ཚོ་}{tsho}. Разница между ними состоит только в том, что морфема \prfB{རྣམས་}{\ul{r}nam\ul{s}} употребляется как в старом, так и в новом письменном языке, а морфема \prfB{ཚོ་}{tsho} употребляется в новом письменном языке, куда она вошла из разговорного языка.

Менее употребительна служебная морфема \prfB{དག་}{dag}.
Эта морфема служила для выражения двойственного числа при переводах с санскрита на тибетский. В языке же оригинальной литературы она употребляется как обычный показатель множественности, например:
\begin{prfsample}
    \item \prfC{དཀའ་ངལ་དེ་དག་}{\ul{d}ka'-ngal de-dag}{<<эти трудности>>}.
\end{prfsample}

Морфемы множественности следуют за существительным или группой существительного, множественность которого они выражают. При наличии однородных членов морфема множественности ставится только после последнего слова, например:
\begin{prfsample}
    \item \prfC{སློབ་པ་དང་སློབ་དཔོན་རྣམས་}{\ul{s}lob-pa dang \ul{s}lob-\ul{d}pon-\ul{r}nam\ul{s}}{<<ученики и преподаватели>>}.
\end{prfsample}

Идея множественности может выражаться не только грамматически (посредством служебных морфем), но и при помощи различных знаменательных слов. В последнем случае служебные морфемы множественности употреблены быть не могут, например:
\begin{prfsample}
    \item \prfC{ཁང་པ་ལྔ་}{khang-pa \ul{l}nga}{<<пять домов>>} (\prfC{ཁང་པ་}{khang-pa}{<<дом>>});
    \item \prfB{མི་ཙང་མ་}{mi tsang-ma} или \prfC{མི་ཐམས་ཅད་}{mi tham\ul{s}-ca\ul{d}}{<<все люди>>} (\prfC{མི་}{mi}{<<человек>>});
    \item \prfC{ར་ལུག་མང་པོ་}{ra-lug mang-po}{<<много баранов>>} (\prfC{ར་ལུག་}{ra-lug}{<<баран>>});
    \item \prfC{དཔེ་ཆ་ཁ་ཤས་}{\ul{d}pe-cha kha-sha\ul{s}}{<<несколько книг>>} (\prfC{དཔེ་ཆ་}{\ul{d}pe-cha}{<<книга>>}).
\end{prfsample}

В новом письменном языке для выражения множественности используются знаменательные морфемы \prfC{རྣམ་པ་}{\ul{r}nam-pa}{<<отдельный, каждый>>} и \prfC{ཁག་}{khag}{<<каждый, все>>}.
Первая морфема употребляется в обращениях, например:
\begin{prfsample}
    \item \prfC{བློ་མཐུན་རྣམ་པ་}{\ul{b}lo-\ul{m}thun-\ul{r}nam-pa}{<<товарищи!>>}.    
\end{prfsample}
Вторая морфема характерна тем, что может сочетаться с числительными (это не свойственно служебным морфемам множественности), например:
\begin{prfsample}
    \item \prfC{གྲོང་སྡེ་ཁག་ཉི་ཤུ་རྩ་བརྒྱད་}{grong-\ul{s}de-khag nyi-shu-\ul{r}tsa-\ul{br}gya\ul{d}}{<<28 населённых пунктов>>}.
\end{prfsample}

2. \emph{Категория рода} существует только у одушевленных имен.

Род одушевленных имен--лиц выражается:

а) оппозицией служебных морфем \prfB{པོ་}{po} (реже \prfB{པ་}{pa} или \prfB{བོ་}{bo}) и \prfB{མོ་}{mo} (реже \prfB{མ་}{ma}), например:
\begin{prfsample}
    \item \prfC{རྒྱལ་པོ་}{\ul{r}gyal-po}{<<царь>>} --- \prfC{རྒྱལ་མོ་}{\ul{r}gyal-mo}{<<царица>>},
    \item \prfC{དགུར་པོ་}{\ul{d}gur-po}{<<горбун>>} --- \prfC{དགུར་མོ་}{\ul{d}gur-mo}{<<горбунья>>},
    \item \prfC{གླ་པོ་}{\ul{g}la-po}{<<поденщик>>} --- \prfC{གླ་མོ་}{\ul{g}la-mo}{<<поденщица>>},
    \item \prfC{ཁྲམ་པ་}{khram-pa}{<<лгун>>} --- \prfC{ཁྲམ་མ་}{khram-ma}{<<лгунья>>},
    \item \prfC{ནུ་བོ་}{nu-bo}{<<младший брат>>} --- \prfC{ནུ་མོ་}{nu-mo}{<<младшая сестра>>};
\end{prfsample}

6) присоединением служебных морфем \prfB{མ་}{ma} или \prfB{མོ་}{mo} (для женского рода), например:
\begin{prfsample}
    \item \prfC{གར་མཁན་}{gar-\ul{m}khan}{<<танцор>>} --- \prfC{གར་མཁན་མ་}{gar-\ul{m}khan-ma}{<<танцовщица>>},
    \item \prfC{བུ་}{bu}{<<сын, мальчик>>} --- \prfC{བུ་མོ་}{bu-mo}{<<дочь, девочка>>}.
\end{prfsample}

Род одушевленных имен--животных выражается оппозицией знаменательных морфем \prfC{ཕོ་}{pho}{<<самец>>} и \prfC{མོ་}{mo}{<<самка>>}, например:
\begin{prfsample}
    \item \prfC{ཕག་ཕོ་}{phag-pho}{<<боров>>} --- \prfC{ཕག་མོ་}{phag-mo}{<<свинья>>}
    \item \prfC{བྱ་ཕོ་}{bya-pho}{<<петух>>} --- \prfC{བྱ་མོ་}{bya-mo}{<<курица>>}
    \item \prfC{ཁྱི་ཕོ་}{khyi-pho}{<<кобель>>} --- \prfC{ཁྱི་མོ་}{khyi-mo}{<<сука>>}.
\end{prfsample}

\label{page71}Тибетское существительное \emph{не имеет грамматической категории падежа}. Правда, национальные грамматики тибетского языка, следуя грамматической системе санскрита, и европейские грамматики тибетского языка, исходя из грамматических систем индоевропейских языков, выделяют в тибетском языке от шести до восьми падежей. Однако с этим нельзя согласиться. В тибетском языке существуют специальные служебные морфемы, которые служат для выражения отношений между словами, соответствующих в известной мере тому, что в других языках передаётся падежными аффиксами. В данной работе они условно называются падежными частицами. Эти служебные морфемы соотносятся с существительным не на морфологическом, а на синтаксическом уровне, т.е. соотносятся с ним не как с частями речи, а только как с членами предложения, причём они примыкают не к существительному, а ко всей группе слов, выступающих как тот или иной член предложения, например:
\begin{prfsample}
    \item \prfC[-p]{[དཔུང་སྡེའི་འཐབ་འཛིན་པ་ཚོ་དང་ལས་མི་ཚོ]ས་}{[\ul{d}pung-\ul{s}de'i 'thab-'dzin-pa-tsho dang la\ul{s}-mi-tsho]\ul{s}}{<<бойцами воинских подразделений и трудящимися>>}.
\end{prfsample}
Тут в качестве падежной частицы выступает
служебная морфема \prfB{ས་}{sa}. То, что заключено в квадратные скобки, буквально значит <<бойцы воинских подразделений и трудящиеся>>.

\subsection{Синтаксические характеристики существительных}

1. Существительное может выступать в предложении:

а) \emph{подлежащим}, которое чаще всего стоит в начале предложения, обычно допуская перед собой только обстоятельство места или времени, а также инверсированное дополнение. Если сказуемое предложения выражено переходным глаголом, то подлежащее оформляется одним из вариантов орудной падежной частицы, а при непереходном глаголе подлежащее выступает без оформления --- не присоединяет служебных слов, например:
\begin{prfsample}
    \item \prfC{བུས་ག་རེ་གནང་}{bu\ul{s} ga-re gnang}{<<что делает сын>>} (здесь \prfC{བུ་}{bu}{<<сын>>} принимает орудную падежную частицу \prfB{ས་}{sa});
    \item \prfC{ཉི་མ་ཤར་བྱུང་}{nyi-ma shar-byung}{<<солнце взошло>>} (здесь \prfC{ཉི་མ་}{nyi-ma}{<<солнце>>} выступает без оформления служебным словом);
\end{prfsample}

б) \emph{прямым или косвенным дополнением}. Прямое дополнение стоит перед сказуемым, косвенное дополнение предшествует прямому. Косвенное дополнение обязательно оформляется косвенной падежной частицей, прямое дополнение выступает без оформления, например:
\begin{prfsample}
    \item \prfC{ངས་ཅོ་ཅོ་ལ་ཡི་གེ་འབྲི་}{nga\ul{s} co-co la yi-ge 'bri}{<<я пишу письмо брату>>} (\prfC{ཅོ་ཅོ་}{co-co}{<<брат>>}, \prfB{ལ་}{la} --- косвенная падежная частица, \prfC{ཡི་གེ་}{yi-ge}{<<письмо>>}).
\end{prfsample}
Прямое дополнение может быть инверсировано при помощи указательного местоимения \prfB{དེ་}{de} например:
\begin{prfsample}
    \item \prfC{ཡི་གེ་དེ་ངས་འབྲི་}{yi-ge de nga\ul{s} 'bri}{<<письмо я пишу>>};
\end{prfsample}

в) \emph{обстоятельством места или времени}, оформляясь при этом косвенной падежной частицей, например:
\begin{prfsample}
    \item \prfC{བོད་ལ་འབྲོག་པ་མང་པོ་ཡོད་}{bo\ul{d} la 'brog-pa mang-po yo\ul{d}}{<<в Тибете много скотоводов>>} (\prfC{བོད་}{bo\ul{d}}{<<Тибет>>}, \prfB{ལ་}{la} --- косвенная падежная частица);
    \item \prfC{ཉི་མ་རེ་རེ་ལ་ཚོགས་འདུ་གཉིས་རེ་ཚོགས་གི་འདུག་}{nyi-ma re-re la tshog\ul{s}-'du \ul{g}nyis-re tshogs-gi-'dug}{<<каждый день бывает по два собрания>>}
    (\prfC{ཉི་མ་}{nyi-ma}{<<день>>}, \prfC{རེ་རེ་}{re-re}{<<каждый>>}, \prfB{ལ་}{la} --- падежная частица);
\end{prfsample}

г) \emph{именной частью сказуемого}, например:
\begin{prfsample}
    \item \prfC{ང་ཚོ་དགེ་ཕྲུག་ཡིན་}{nga-tsho \ul{d}ge-phrug yin}{<<мы --- учащиеся>>}
    (\prfC{དགེ་ཕྲུག་}{\ul{d}ge-phrug}{<<учащийся>>}
    \footnote[26]{
    Выступая именным сказуемым, тибетское существительное не принимает морфем множественности; идея множественности передаётся только через подлежащее, например: \prfC{འདི་དཔེ་ཆ་རེད་}{'di \ul{d}pe-cha re\ul{d}}{<<это книга>>}, но
    \prfC{འདི་ཚོ་དཔེ་ཆ་རེད་}{'di-tsho \ul{d}pe-cha re\ul{d}}{<<это книги>>}
    (букв. <<эти книга есть>>), где \prfB{ཚོ་}{tsho} --- морфема множественности.
    },
    \prfB{ཡིན་}{yin} --- глагольная связка);
\end{prfsample}

д) \emph{определением}, например:
\begin{prfsample}
    \item \prfC{ཡུ་རོབ་ཀྱི་མ་རྩ་རིང་ལུགས་}{yu-rob-kyi ma-\ul{r}tsa-ring-lug\ul{s}}{<<европейский капитализм>>};
    \item \prfC{མི་དམངས་ཀྱི་ཕན་བདེ་}{mi-\ul{d}mang\ul{s}-kyi phan-\ul{b}de}{<<народное благо>>}
\end{prfsample}
(\prfB{ཀྱི་}{kyi} --- притяжательная падежная частица, \prfC{ཡུ་རོབ་}{yu-rob}{<<Европа>>},
\prfC{མི་དམངས་}{mi-\ul{d}mang\ul{s}}{<<народ>>}).

2. Существительное сочетается с притяжательной падежной частицей.

3. Существительное непосредственно после себя принимает определения, выраженные прилагательными, числительными или местоимениями, например:
\begin{prfsample}
    \item \prfC{དཔེ་ཆ་ཡག་པོ་གསུམ་}{\ul{d}pe-cha yag-po \ul{g}sum}{<<три хорошие книги>>} (букв. <<книга хороший три>>);
    \item \prfC{མི་ཚང་མ་}{mi-tshang-ma}{<<все люди>>} (букв. <<человек все>>).
\end{prfsample}

4. Существительное, как правило, не сочетается с частицами отрицания \prfB{མི་}{mi} и \prfB{མ་}{ma}, т.е. сочетания типа <<не дом>>, <<не человек>> образованы быть не могут. Крайне редко встречаются сочетания типа
\begin{prfsample}
    \item \prfC{མི་ཤིང་མི་རྡོ་}{mi shing mi \ul{r}do}{<<ни дерево, ни камень>>}.    
\end{prfsample}
