\chapter{Введение}

Тибетцы составляют коренное население Тибетского нагорья. Основная их масса, а именно около 2 млн. 800 тыс. человек, проживает на территории, входящей в настоящее время в состав КНР (Тибетский автономный район "--- 1 млн. 274 тыс. человек; автономные округа и районы провинции Сычуань "--- 712 тыс. человек, провинции Цинхай "--- 516 тыс. человек, провинции Ганьсу "--- 204 тыс. человек, провинции Юньнань "--- 66 тыс. человек)\footnote[1]{См. <<Восточно-Азиатский этнографический сборник>>, II, М., 1961, стр. 80-82.}. Свыше 300 тыс. тибетцев проживают в Непале, Сиккаме, Бутане, Ладаке, Балтистане.

Общее самоназвание тибетцев <<пёба>> (\prfA{བོད་པ་}), но более распространены самоназвания по областям. Так, жители района Амдо называют себя <<амдова>>, у жителей района Цзан существует самоназвание <<цзаяба>>, у жителей района Сикан "--- <<камба>> и т.д.

Помимо названий по областям у тибетцев существуют названия в зависимости от характера их хозяйственной деятельности. Кочевники-скотоводы известны под именем <<чжокба>> (\prfC{འབྲོག་}{чжок}{<<горное пастбище>>}), а земледельцы называют себя <<ронгва>> (\prfC{རོང་}{ронг}{<<горная долина>>}).

Тибетцы говорят на диалектах (наречиях) тибетского языка, который входит в тибето-бирманскую ветвь китайско-тибетской семьи языков. Крупнейший русский востоковед и тибетолог Ю.Н. Рерих (1902--1960) выделял следующие группы наречий тибетского языка: группа наречий Центрального Тибета, включающая лхасский диалект, группа южно-тибетских наречий, группа западно-тибетских наречий, группа северо-тибетских наречий, группа наречий северо-востока, группа наречий крайнего востока и группа наречий юго-востока. В свою очередь эти наречия распадаются на ряд отдельных говоров.

Общим для всех областей Тибетского нагорья является письменный язык, который значительно отличается от языка разговорного. Эти различия лежат главным образом в области синтаксиса простого и сложного предложений и проявляются прежде всего в употреблении разных служебных слов.

Эти различия существуют и в области лексики. Так, в письменном языке имеется большая группа слов и устойчивых словосочетаний, которые описательно передают названия того или иного предмета, явления. Так, например, вместо слова \prfC{ཤིན་}{шинг}{<<дерево>>} в письменном языке может быть употреблено слово \prfB{ཀང་མཐུང་}{кангтхунг} (букв. <<ногами пьющий>>), вместо слова \prfC{བྱེ་མ་}{цема}{<<песок>>}  "--- \prfB{ཆུ་ལས་བྱུང་བ་}{цюле цюнгва} (букв. <<из воды возникший>>) и т.д.

Тибетское письмо было выработано на основе брахми "--- одной из наиболее распространённых разновидностей древнеиндийского письма. Письменный тибетский язык сформировался в течение второй половины VII в. и всего VIII в.

На тибетском языке существует богатая и разнообразная литература. Первые памятники тибетской письменности относятся к VIII--X вв. Среди них имеются записи мифов, афоризмов, исторические хроники, найденные в Дуньхуане, а также эпиграфические памятники, как например надпись в монастыре Самъе \prfA{(བསམ་ཡས་)}. В этот период появляются первые переводы буддийских текстов, составившие знаменитый буддийский канон, который в IX в. подвергся первой кодификации\footnote[2]{Окончательный вид буддийский канон на тибетском языке получил в XIV в.}. Он состоит из двух обширнейших разделов: <<Канчжур>> (\prfA{བཀའ་འགྱུར་}), включающих переводы с санскрита основных буддийских трактатов, и <<Танчжур>> (\prfA{བསྟན་འགྱུར་}), содержащий комментарии на эти трактаты. В каноне имеются сочинения по философии, логике, ритуалам, а также грамматике и поэтике. Ценность канона определяется также и тем, что санскритские оригиналы многих текстов, вошедших в него, в настоящее время утеряны.

На тибетском языке имеется также очень богатая апокрифическая литература, или <<сокровенные книги>>.

Среди поэтических произведений можно выделить творчество поэта-от\-шель\-ни\-ка Миларепы (1040--1123), дидактические сочинения Потапы (XI в.), Сакья-пандиты (XIII в.), лирику VI Далай-ламы и т.д.

В Тибете распространены переводы художественных произведений индийской литературы, а также оригинальные тибетские произведения, которые написаны на сюжеты этой литературы, например: <<Двадцать пять рассказов Веталы>>, <<Паячатантра>> и т.д.

На тибетском языке существует и разнообразная историческая литература, которую можно разделить на три жанра: сочинения по истории религии (\prfA{ཆོས་བྱུང་}), сочинения по истории династий (\prfA{རྒྱལ་རབ་}), хронологические таблицы (\prfA{རེའུ་མིག་}).

Богата и тибетская грамматическая литература. Первая тибетская грамматика была составлена в 653 г. Ее автором является создатель тибетской азбуки Тонми Самбхота. Первоначально эта грамматика состояла из восьми частей, но в XI в. "--- в период правления тибетского царя Ландармы (\prfA{གླང་དར་མ་}) "--- когда началось гонение на буддизм и уничтожались буддийские книги, грамматика Тонми погибла. Впоследствии, как сообщают авторы более поздних грамматических сочинений, были обнаружены две части из грамматики Тонми: первая и шестая. Появилось множество толкований двух этих частей и комментариев к ним, что и составило обширную тибетскую грамматическую литературу. К наиболее известным тибетским грамматикам можно отнести грамматики Ренчендондуба\footnote[3]{Рукопись библиотеки Восточного ф-та ЛГУ, Мс., ин.№ 1931/3а}, V Далай-ламы (закончена в 1664 г.), Кармы Ситу (род. в 1699 г.), Джандахутухты (1717--1786) и т.п. Следует отметить, что тибетская грамматическая литература складывалась под влиянием индийской, причём многое из индийской грамматической литературы было заимствовано чисто механически (например, схема системы склонения и т.п.).

Проф. Ю.Н.Рерих давал следующую периодизацию развития тибетского письменного языка\footnote[4]{См. Ю.Н.Рерих, Основные проблемы тибетского языкознания, --- «Советское востоковедение», 1958, № 4.} : 1) вторая половина VII в. -- середина IX в. --- период древних переводов с санскрита; 2) вторая половина IX в. -- конец XV в. --- время дальнейшего оформления письменного языка, замена заимствованных санскритских слов переводами; 3) XIV в. -- XVIII в. --- окончательное создание языка философских трактатов. Последний период связан с именем Цзонхавы (1357--1419), чьи произведения считаются непревзойдёнными по стилю и ясности изложения. Этот письменный тибетский язык можно назвать \emph{старым письменным языком}. С середины XIX в. под влиянием эпистолярного стиля и разговорной стихии старый письменный язык начинает воспринимать некоторые элементы разговорного языка, в какой-то степени приближаясь к нему.

Процесс приближения старого письменного языка к языку разговорному принимает организованный, целенаправленный характер после 1951 г., когда Тибет снова вошёл в состав Китая. В связи с этим возникла настоятельная необходимость издания на тибетском языке литературы, призванной воздействовать на самый широкий круг читателей и, следовательно, быть максимально понятной каждому мало-мальски грамотному человеку. Грамматическими и лексическими средствами существовавшего письменного языка выполнить эту задачу не представлялось возможным. Поэтому встал вопрос о реформе письменного языка с тем, чтобы сделать его более понятным простому читателю, и о вводе новой лексики, о помощью которой можно было бы описывать изменения, происходящие в социально-экономической, политической и административной жизни тибетского общества, а также знакомить тибетцев с основами новых знаний в области науки и техники\footnote[5]{Первые попытки приблизить письменный тибетский язык к языку разговорному имели место ещё в начале XX в. В это время в Дарджилинге и Калимпонге, где функционировали школы для тибетцев, стали издаваться учебники и хрестоматии, написанные на языке, приближавшемся к разговорному, начала выходить небольшая газета}.

Реформа старого письменного языка и формирование нового письменного языка происходит в специфических условиях: 1) в КНР осуществляется централизованный контроль над всей литературой, которая выходит на тибетском языке после 1951 г.; 2) издаваемая литература (в основном политическая, социально-экономическая, научно-популярная и, реже, художественная в виде небольших рассказов на современные темы) представляет собой почти на 100\% переводы с китайского языка (исключение составляет небольшое число памятников, изданных до 1960 г. "--- главным образом Цинхайским издательством). Многочисленных авторов с их индивидуальным языковым стилем заменили коллективы редакторов и переводчиков, сосредоточенных в нескольких издательствах. Все это создало условия для централизованной выработки новой терминологии, а также некоторых общих правил синтаксиса. В тибетской литературной традиции нечто похожее уже имело место, когда в первой половине IX в. была создана особая коллегия по делам переводов буддийских сочинений, составившая правила перевода с древнеиндийского языка на тибетский.

В ходе широкого обсуждения проблем перевода современной литературы на тибетский язык\footnote[6]{Первые такие обсуждения в Пекине и на местах состоялись в 1953г. в связи с предстоящим переводом на тибетский язык конституции КНР (принятой 20 сентября 1954 г.).}, в котором принимали участие китайские тибетологи, высокообразованные ламы и работники издательств (редакторы и переводчики), было решено приблизить письменный язык к разговорному, сохраняя при этом основные синтаксические нормы старого письменного языка. Так начал формироваться язык, который мы называем \emph{новым письменным языком}.

Однако отсутствие единого разговорного языка препятствовало и препятствует формированию единого нового письменного языка. Стремясь сделать печатный текст доступнее для понимания, местные издательства\footnote[7]{Кроме Пекинского издательства литературы на языках нацменьшинств литературу на тибетском языке издают местные издательства в Ланьчжоу, Чэнду, Синане.} вынуждены учитывать при переводах некоторые синтаксические особенности, присущие тому или иному диалекту, и употреблять диалектизмы. Это находит отражение в языке газет, выходящих в том или ином тибетском автономном округе\footnote[8]{Например: <<Ганнаньская газета>> (\prfB{ཀན་ལྷོའི་གསར་འགྱུར་}{генхлё саргюр}) отражает черты северо-восточного диалекта, <<Ежедневная газета округа Ганьцзы>> (\prfB{དཀར་མ ཛེས་ཉིན་རེའི་གསར་འགྱུར་}{гарцзы нинре саргюр}) "--- восточного диалекта, <<Цинхайская газета>> (\prfB{མཚོ་སྔོན་བོད་ཡིག་གསར་འགྱུར་}{цон\ul{ё}н п\ul{ё}их саргюр}) "--- северного диалекта и т.д.} . (Издательство литературы на языках нацменьшинств в Пекине, которое издаёт наибольшее количество литературы на тибетском языке, ориентируется на лхасский диалект.)

Существует известный разнобой и в создании новой лексики тибетского языка. Как правило, местные издательства должны использовать слова и термины, принятые Пекинским издательством литературы на языках нацменьшинств, однако это не всегда соблюдалось. Одно и то же слово некоторые издательства переводили средствами тибетского языка, другие прибегали к фонетическому заимствованию из китайского; одни и те же фонетические заимствования разные издательства сплошь и рядом передавали в разной транскрипции.

Хотя в тибетской литературе и установилась традиция, согласно которой новые для тибетского языка слова (если это не названия растений, минералов, лекарств и т.п.) не заимствовались в исходном фонетическом облике из другого языка, а передавались средствами тибетского языка путём перевода или калькировались, однако в настоящее время при переводах с китайского на тибетский эту традицию нередко игнорируют. Многие новые слова заимствуются из китайского в их фонетическом облике, хотя перевод этих слов средствами тибетского языка и возможен и более понятен. Зачастую непонятное фонетическое заимствование в тексте какой-либо брошюры поясняется тут же в скобках более понятной калькой или переводом по смыслу, например:

\begin{tabularx}{\textwidth}{XXX}

\makecell[c]{Китайское слово} &
\makecell[c]{Тибетское\\заимствование} &
\makecell[c]{Тибетский перевод,\\ заключаемый в скобки}\\[5mm]
\cline{1-3}
\makecell[l]{{\chinfont 火车}\\\textit{хочэ}\\<<поезд>>} &
\prfB{ཧོ་ཁྱེ་}{хочэ} &
\makecell[l]{\prfA{མེ་འཁོར་}\\\textit{мэнкхор}\\<<огненное колесо>>}\\
\cline{1-3}
\makecell[l]{{\chinfont 播种机}\\\textit{бочжунцзи}\\<<сеялка>>} &
\prfB{བོའོ་ཀྱུང་ཟི་}{боочжунгцзи} &
\makecell[l]{\prfA{སོན་འདེབས་འཕྱུལ་འཁོར་}\\\textit{сонтэп чулькхор}\\<<машина, которая сеет>>}\\
\cline{1-3}
\makecell[l]{{\chinfont 办公室}\\\textit{баньгунши}\\<<канцелярия>>} &
\prfB{བན་ཀུན་ཧི་}{бэнкунгши} &
\makecell[l]{\prfA{གཤུང་ལས་ཁང་}\\\textit{шунглэкханг}\\<<дом делопроизводства>>}
\end{tabularx}
\bigskip

В ряде случаев уже бытовавшие в тибетском языке названия, особенно географические, заменены фонетическими заимствованиями из китайского, например: \prfC{ཕ་རན་སི་}{пхарэнси}{<<Франция>>} "--- на \prfB{ཧྥའ་གོ་}{фаго} (кит. {\unifont 法国} \textit{фаго}), \prfC{རྒྱལ་སེར་}{гьясер}{<<Россия>>} "--- на \prfB{ཨོ་གོ་}{ого} (кит. {\unifont 俄国} \textit{эго}).

На новом письменном языке появилось много изданий. Это различная со\-циаль\-но-эко\-но\-ми\-чес\-кая, политическая, научно-популярная литература. Значительно меньше издаётся художественной литературы. В основном это переводы с китайского небольших рассказов. Некоторые брошюры снабжаются параллельным текстом на китайском языке.

\begin{center}
* * *
\end{center}

Предлагаемый очерк посвящён описанию тибетского языка в его старой и новой письменных формах. Ниже, при употреблении выражения <<тибетский язык>>, мы всегда имеем в виду как старый, так и новый письменный язык, независимо от того, существуют ли описываемые явления в разговорном языке.

Если речь идет о явлениях, присущих только старому письменному языку, то употребляется наименование <<старый письменный язык>>, а наименование <<новый письменный язык>> употребляется тогда, когда речь идет только о явлениях, имеющих место в новом письменном языке, но отсутствующих в старом письменном языке.

Иногда для сопоставления мы упоминаем о явлениях, присущих языку разговорному. Тогда употребляется наименование <<разговорный язык>>.

