\section{Имя-атрибутив}

В тибетском языке имеется большая группа имен, обозначающих предмет по его качеству. Такого рода имена мы называем имена-атрибутивы. Эти имена образуются путем присоединения к существительному служебных морфем \prfB{ཅན་}{can} и \prfB{བཅས་}{\ul{b}ca\ul{s}} или глаголов \prfB{ལྡན་}{\ul{l}dan}, \prfB{ཡོད་}{yo\ul{d}}, \prfB{མངའ་}{\ul{m}nga'} выступающих в служебной функции в роли словообразовательных морфем, с общим исходным значением
как для морфем, так и для глаголов <<иметь, обладать, наличествовать>>.

Очень продуктивным словообразующим элементом имен этой категории является служебная морфема	\prfB{ཅན་}{can}. При помощи этой морфемы образуется большое количество имен типа \prfC{ནོར་ཅན་}{nor-can}{'богатый'}, \prfC{ཧུར་བརྩེན་ཅན་}{hur-\ul{b}\ul{r}tsen-can}{'активный'}, \prfC{ཡོན་ཏན་ཅན་}{yon-tan-can}{'достойный'}.

Менее продуктивна служебная морфема. \prfB{བཅས་}{\ul{b}ca\ul{s}}, образующая имена того же типа, например: \prfC{ཆོས་བཅས་}{cho\ul{s}-\ul{b}ca\ul{s}}{'благочестивый'}.

Не менее продуктивен, чем служебная морфема \prfB{ཅན་}{can} знаменательный глагол \prfB{ལྡན་}{\ul{l}dan} выступающий в служебной функции. Этот глагол полный аналог морфеме \prfB{ཅན་}{can}. Так, \prfB{ནོར་ཅན་}{nor-can} = \prfB{ནོར་ལྡན་}{nor-\ul{l}dan};
\prfB{ཡོན་ཏན་ཅན་}{yon-tan-can} = \prfB{ཡོན་ཏན་ལྡན་}{yon-tan-\ul{l}dan} и т.д.

Имена такого же порядка могут быть образованы и от именной формы глагола \prfB{ལྡན་}{\ul{l}dan} --- \prfC{ལྡན་པ་}{\ul{l}dan-pa}{'имеющий, обладающий'},
при этом употребляется соединительный союз \prfB{དང་}{dang}, например:
\prfC{ནོར་དང་ལྡན་པ་}{nor-dang-\ul{l}dan-pa}{'богатый'}, \prfC{ཧུར་བརྩེན་དང་ལྡན་པ་}{hur-\ul{b}\ul{r}tsen-dang-\ul{l}dan-pa}{'активный'}\footnote[31]{
    По этой же схеме могут образовываться и имена-атрибутивы и от морфемы \prfB{བཅས་}{\ul{b}ca\ul{s}}, например:
    \prfB{ཆོས་བཅས་}{cho\ul{s}-\ul{b}ca\ul{s}} = \prfC{ཆོས་དང་བཅས་པ་}{cho\ul{s}-dang-\ul{b}ca\ul{s}-pa}{'благочестивый'}.
}

Аналогичные имена могут быть образованы таким же образом от глаголов \prfB{ཡོད་}{yo\ul{d}} и \prfB{མངའ་}{\ul{m}nga'} или их именных форм \prfB{ཡོད་པ་}{yo\ul{d}-pa}, \prfB{མངའ་བ་}{\ul{m}nga'-ba} 'имеющий, обладающий'. Только в последнем случае не требуется соеди-
нительного союза \prfB{དང་}{dang}, например:
\prfC{ཡོན་ཏན་མངའ་བ་}{yon-tan-\ul{m}nga'-ba}{'достойный'}, \prfC{བཀྲ་ཤིས་ཡོད་པ་}{\ul{b}kra-shi\ul{s}-yo\ul{d}-pa}{'счастливый'}.

Те же имена, но с негативным значением могут быть образованы от глагола \prfC{མེད་}{me\ul{d}}{'не иметь, не иметься'} (негативная форма глагола \prfB{ཡོད་}{yo\ul{d}}) или от его именной формы \prfC{མེད་པ་}{me\ul{d}-pa}{'не имеющий, не имеющийся'}, например: \prfC{ཡོན་ཏེན་མེད་}{yon-ten-me\ul{d}}{'недостойный'}.

Европейские грамматики тибетского языка относят рассматриваемую группу имен к категории имен прилагательных. На самом деле имена-атрибутивы обладают признаками как имен прилагательных, так и имен существительных (см. таб. \ref{tab:13}).

\begin{tabularx}{\textwidth}{|L{10}|C{3}|C{3}|C{4}|C{4}|C{4}|C{4}|}
    \caption{Сравнение свойств имен существительных, прилагательных и атрибутивов}\label{tab:13}\\
    \hline
    \multirow{2}{*}{Имена} & \multicolumn{2}{c|}{\makecell{Выступает\\определением}} & \multirow{2}{*}{\vertcell{4cm}{Наличие категории множественности}}&\multirow{2}{*}{\vertcell{4cm}{Наличие категории рода}} &\multirow{2}{*}{ \vertcell{4cm}{Наличие категории степеней сравнения}} &\multirow{2}{*}{\vertcell{4cm}{Со\-че\-тае\-мость с частицами отрицания}}\\
    \cline{2-3}
    & в препозиции & в постпозиции & & & & \\
    \hline
    Имя существительное & + & -- & + & + & -- & --\\
    Имя прилагательное\hyperref[tab:13:spec1]{$^*$} & -- & + & -- & -- & + & +\\
    Имя-атрибутив & + & + & + & -- & -- & --\\
    \hline
\end{tabularx}
{\footnotesize{\label{tab:13:spec1}* У имен прилагательных отсутствует грамматическая категория рода. Однако в редких случаях род может выражаться оппозицией словообразующих морфем \prfB{པོ་}{po} и \prfB{མོ་}{mo}, например: \prfC{ཡག་པོ་}{yag-po}{'хороший'} --- \prfC{ཡག་མོ་}{yag-mo}{'хорошая'}.}}

Из таб. \ref{tab:13} видно, что имя-атрибутив может выступать определением как в препозиции, так и в постпозиции, принимая в первом случае притяжательную падежную частицу; ср., например:	\prfB{ནོར་ཅན་གྱི་མི་}{nor-can-gyi mi} и \prfC{མི་ནོར་ཅན་}{mi nor-can}{'богатый человек'}.

Имя-атрибутив может принимать частицы множественности \prfB{རྣམས་}{\ul{r}nam\ul{s}} и \prfB{ཚོ་}{tsho}, например:
\prfC{ཡོན་ཏན་མེད་རྣམས་}{yon-tan-me\ul{d}-\ul{r}nam\ul{s}}{'недостойные'}.

В определении имена-атрибутивы выступают обычно подлежащим, прямым или косвенным дополнением, определением.

Для этих имен характерно также то, что в языке художественных произведений они выступают как средство образной речи, например:
\prfC{མངར་ལྡན་}{\ul{m}ngar-\ul{l}dan}{'мед' (букв. 'сладкий')}, \prfC{འཆི་མེད་}{'chi-me\ul{d}}{'ворон' (букв. 'бессмертный')}.
