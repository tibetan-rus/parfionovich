\section{Имя-атрибутив}

В тибетском языке имеется большая группа имен, обозначающих предмет по его качеству. Такого рода имена мы называем имена-атрибутивы. Эти имена образуются путем присоединения к существительному служебных морфем \prfB{ཅན་}{can} и \prfB{བཅས་}{\ul{b}ca\ul{s}} или глаголов \prfB{ལྡན་}{\ul{l}dan}, \prfB{ཡོད་}{yo\ul{d}}, \prfB{མངའ་}{\ul{m}nga'} выступающих в служебной функции в роли словообразовательных морфем, с общим исходным значением
как для морфем, так и для глаголов <<иметь, обладать, наличествовать>>.

Очень продуктивным словообразующим элементом имен этой категории является служебная морфема	\prfB{ཅན་}{can}. При помощи этой морфемы образуется большое количество имен типа \prfC{ནོར་ཅན་}{nor-can}{'богатый'}, \prfC{ཧུར་བརྩེན་ཅན་}{hur-\ul{b}\ul{r}tsen-can}{'активный'}, \prfC{ཡོན་ཏན་ཅན་}{yon-tan-can}{'достойный'}.

Менее продуктивна служебная морфема. \prfB{བཅས་}{\ul{b}ca\ul{s}}, образующая имена того же типа, например: \prfC{ཆོས་བཅས་}{cho\ul{s}-\ul{b}ca\ul{s}}{'благочестивый'}.

Не менее продуктивен, чем служебная морфема \prfB{ཅན་}{can} знаменательный глагол \prfB{ལྡན་}{\ul{l}dan} выступающий в служебной функции. Этот глагол полный аналог морфеме \prfB{ཅན་}{can}. Так, \prfB{ནོར་ཅན་}{nor-can} = \prfB{ནོར་ལྡན་}{nor-\ul{l}dan};
\prfB{ཡོན་ཏན་ཅན་}{yon-tan-can} = \prfB{ཡོན་ཏན་ལྡན་}{yon-tan-\ul{l}dan} и т.д.

Имена такого же порядка могут быть образованы и от именной формы глагола \prfB{ལྡན་}{\ul{l}dan} --- \prfC{ལྡན་པ་}{\ul{l}dan-pa}{'имеющий, обладающий'},
при этом употребляется соединительный союз \prfB{དང་}{dang}, например:
\prfC{ནོར་དང་ལྡན་པ་}{nor-dang-\ul{l}dan-pa}{'богатый'}, \prfC{ཧུར་བརྩེན་དང་ལྡན་པ་}{hur-\ul{b}\ul{r}tsen-dang-\ul{l}dan-pa}{'активный'}\footnote[31]{
    По этой же схеме могут образовываться и имена-атрибутивы и от морфемы \prfB{བཅས་}{\ul{b}ca\ul{s}}, например:
    \prfB{ཆོས་བཅས་}{cho\ul{s}-\ul{b}ca\ul{s}} = \prfC{ཆོས་དང་བཅས་པ་}{cho\ul{s}-dang-\ul{b}ca\ul{s}-pa}{'благочестивый'}.
}

Аналогичные имена могут быть образованы таким же образом от глаголов \prfB{ཡོད་}{yo\ul{d}} и \prfB{མངའ་}{\ul{m}nga'} или их именных форм \prfB{ཡོད་པ་}{yo\ul{d}-pa}, \prfB{མངའ་བ་}{\ul{m}nga'-ba} 'имеющий, обладающий'. Только в последнем случае не требуется соеди-
нительного союза \prfB{དང་}{dang}, например:
\prfC{ཡོན་ཏན་མངའ་བ་}{yon-tan-\ul{m}nga'-ba}{'достойный'}, \prfC{བཀྲ་ཤིས་ཡོད་པ་}{\ul{b}kra-shi\ul{s}-yo\ul{d}-pa}{'счастливый'}.

Те же имена, но с негативным значением могут быть образованы от глагола \prfC{མེད་}{me\ul{d}}{'не иметь, не иметься'} (негативная форма глагола \prfB{ཡོད་}{yo\ul{d}}) или от его именной формы \prfC{མེད་པ་}{me\ul{d}-pa}{'не имеющий, не имеющийся'}, например: \prfC{ཡོན་ཏེན་མེད་}{yon-ten-me\ul{d}}{'недостойный'}.

Европейские грамматики тибетского языка относят рассматриваемую группу имен к категории имен прилагательных. На самом деле имена-атрибутивы обладают признаками как имен прилагательных, так и имен существительных (см. таб. \ref{tab:13}).

\begin{tabularx}{\textwidth}{|L{10}|C{3}|C{3}|C{4}|C{4}|C{4}|C{4}|}
    \caption{Сравнение свойств имен существительных, прилагательных и атрибутивов}\label{tab:13}\\
    \hline
    \multirow{2}{*}{Имена} & \multicolumn{2}{c|}{\makecell{Выступает\\определением}} & \multirow{2}{*}{\vertcell{4cm}{Наличие категории множественности}}&\multirow{2}{*}{\vertcell{4cm}{Наличие категории рода}} &\multirow{2}{*}{ \vertcell{4cm}{Наличие категории степеней сравнения}} &\multirow{2}{*}{\vertcell{4cm}{Со\-че\-тае\-мость с частицами отрицания}}\\
    \cline{2-3}
    & в препозиции & в постпозиции & & & & \\
    \hline
    Имя существительное & + & -- & + & + & -- & --\\
    Имя прилагательное\hyperref[tab:13:spec1]{$^*$} & -- & + & -- & -- & + & +\\
    Имя-атрибутив & + & + & + & -- & -- & --\\
    \hline
\end{tabularx}
{\footnotesize{\label{tab:13:spec1}* У имен прилагательных отсутствует грамматическая категория рода. Однако в редких случаях род может выражаться оппозицией словообразующих морфем \prfB{པོ་}{po} и \prfB{མོ་}{mo}, например: \prfC{ཡག་པོ་}{yag-po}{'хороший'} --- \prfC{ཡག་མོ་}{yag-mo}{'хорошая'}.}}

Из таб. \ref{tab:13} видно, что имя-атрибутив может выступать определением как в препозиции, так и в постпозиции, принимая в первом случае притяжательную падежную частицу; ср., например:	\prfB{ནོར་ཅན་གྱི་མི་}{nor-can-gyi mi} и \prfC{མི་ནོར་ཅན་}{mi nor-can}{'богатый человек'}.

Имя-атрибутив может принимать частицы множественности \prfB{རྣམས་}{\ul{r}nam\ul{s}} и \prfB{ཚོ་}{tsho}, например:
\prfC{ཡོན་ཏན་མེད་རྣམས་}{yon-tan-me\ul{d}-\ul{r}nam\ul{s}}{'недостойные'}.

В определении имена-атрибутивы выступают обычно подлежащим, прямым или косвенным дополнением, определением.

Для этих имен характерно также то, что в языке художественных произведений они выступают как средство образной речи, например:
\prfC{མངར་ལྡན་}{\ul{m}ngar-\ul{l}dan}{'мед' (букв. 'сладкий')}, \prfC{འཆི་མེད་}{'chi-me\ul{d}}{'ворон' (букв. 'бессмертный')}.


\section{Местоимение}

В тибетском языке можно выделить следующие виды местоимений: личные, указательные, определительные, неопределенные и вопросительные. Им присущи следующие синтаксические функции:

а) личные, вопросительные и указательные местоимения выступают в функции подлежащего, дополнения, именной части сказуемого и определения, т.е. в тех функциях, что и существительное;

б) определительные и неопределенные местоимения выступают в функции определения в правой позиции, т.е. они соотносимы с прилагательными и числительными.

1. \emph{Личные местоимения}, их родовая отнесенность и сфера употребления показаны в  таб. \ref{tab:14}.

\begin{tabularx}{\textwidth}{|L{3}|L{5}|L{5}|L{18}|}
    \caption{Личные местоимения}\label{tab:14}\\
    \hline
    Лицо & Место\-име\-ния & Род & Примечания\\
    \hline
    \multirow{6}{*}{1-е} & \prfB{ང་}{nga} & общий & употребляется и в разговорном языке\\
    & \prfB{བདག་}{\ul{b}dag} & общий & употребляется и в разговорном языке (западный диалект)\\
    & \prfB{ཁོ་བོ་}{kho-bo} & мужской & \\
    & \prfB{ཁོ་མོ་}{kho-mo} & женский & \\
    & \prfB{ཁོ་མོ་}{nge\ul{d}} & мужской & вежливая форма; употребляется также в диалекте Чамдо\\
    & \prfB{ཕྲན་}{phran} & общий & эпистолярный стиль (уничижительно)\\
    \hline
    \multirow{2}{*}{2-е} & \prfB{ཁྱོད་}{khyo\ul{d}} & общий & употребляется и в разговорном языке\\
    & \prfB{ཁྱེད་}{khye\ul{d}} & общий & вежливая форма; употребляется и в разговорном языке\\
    \hline
    \multirow{6}{*}{3-е} & \prfB{ཁོ་}{kho} & общий & употребляется и в разговорном языке\\
    & \prfB{ཁོ་པ་}{kho-pa} & общий & употребляется и в разговорном языке\\
    & \prfB{ཁོ་མ་}{kho-ma} & женский & употребляется и в разговорном языке\\
    & \prfB{ཁོང་}{khong} & общий & вежливая форма; употребляется и в разговорном языке\\
    & \prfB{ཕོ་}{pho} & мужской & новый письменный язык; употребляется и в разговорном языке\\
    & \prfB{མོ་}{mo} & женский & новый письменный язык; употребляется и в разговорном языке\\
    \hline
\end{tabularx}

В новом письменном языке в роли местоимения 1-го и 1-го лица также употребляется (как и в разговорном языке) местоименная морфема \prfC{རང་}{rang}{'сам'}.

Морфема	\prfB{རང་}{rang} может присоединяться к местоимениям 1-го лица \prfB{ང་}{nga} и \prfB{ཁོ་མོ་}{nge\ul{d}}, 2-го лица	\prfB{ཁྱོད་}{khyo\ul{d}} и \prfB{ཁྱེད་}{khye\ul{d}}  3-го лица \prfB{ཁོ་}{kho}, образуя своего рода сложные местоимения \prfB{ང་རང་}{nga-rang} , \prfB{ངེད་རང་}{nge\ul{d}-rang}, \prfB{}{khyo\ul{d}-rang}, \prfB{}{khye\ul{d}-rang},
\prfB{}{kho-rang} со значением исходного слова. Сама же морфема	\prfB{རང་}{rang} полностью утрачивает свое лексическое значение. Чтобы выразить лексическое значение <<сам>>, данную морфему удваивают, например: \prfC{ང་རང་རང་}{nga-rang-rang}{'я сам'}.

Множественное число личных местоимений образуется прибавлением служебных морфем \prfB{ཚོ་}{tsho}, \prfB{རྣམས་}{\ul{r}nams}, \prfB{ཅག་}{cag} или \prfB{ཅག་རྣམས་}{cag-\ul{r}nams}.

В старом письменном языке 1-е лицо множественного числа обозначается также морфемой	\prfB{འུ་ཅག་}{'u-cag} (или \prfB{འུ་བུ་ཅག་}{'u-bu-cag}) 'мы', причем иной раз в сочетании с морфемой множественности \prfB{རྣམས་}{\ul{r}nams}, т.е.	\prfB{འུ་ཅག་རྣམས་}{'u-cag-\ul{r}nams}.

Личные местоимения могут принимать числительное \prfC{གཉིས་}{\ul{g}nyi\ul{s}}{'два'}, например:
\prfC{ཁོ་གཉིས་}{kho-\ul{g}nyi\ul{s}}{'они оба, они вдвоем'};
\prfB{ངེད་རང་གཉིས་}{nge\ul{d}-rang-\ul{g}nyi\ul{s}} или \prfC{རང་གཉིས་}{rang-\ul{g}nyi\ul{s}}{'я и ты, мы вдвоем'}.

2. \emph{Указательные местоимения}:	\prfC{འདི་}{'di}{'этот, эта, это'} и \prfC{དེ་}{de}{'тот, та, то'}. Множественное лицо от этих местоимений образуется при помощи служебных морфем \prfB{དག་}{dag}, \prfB{རྣམས་}{\ul{r}nams}, \prfB{ཚོ་}{tsho}. Указательные местоимения, присоединяя знаменательную морфему \prfC{འདྲ་}{'dra}{'подобный, похожий'}, образуют местоименные слова \prfC{འདི་འདྲ་}{'di-'dra}{'подобный этому, похожий на это, этакий'} и \prfC{དེ་འདྲ་}{de-'dra}{'такой'}. Эти слова выражают как единичность, так и множественность.

3. \emph{Определительные местоимения}: \prfC{ཀུན་}{kun}{'все'}; \prfB{ཐམས་ཅད་}{tham\ul{s}-ca\ul{d}} и \prfC{ཚང་མ་}{tshang-ma}{'все, всё'}; \prfB{སོ་སོ་}{so-so}, \prfC{རེ་རེ་}{re-re}{'каждый'}. Эти местоимения констатируют, что охват лиц или предметов является полным.

4. \emph{Неопределенные местоимения} указывают на неполноту, частичность или отсутствие определенности. К ним относятся местоимения \prfB{ཁ་ཤས་}{kha-sha\ul{s}} и \prfC{འགའ་ཞིག་}{'ga'-zhig}{'некоторые, несколько'}; \prfB{གཞན་}{\ul{g}zhan} или \prfC{གཞན་པ་}{\ul{g}zhan-pa}{'другой, иной'}; \prfC{ལ་ལ་}{la-la}{'некоторые'}.

5. \emph{Вопросительные местоимения}:
\prfC{གང་}{gang}{'кто, который, где'}, \prfC{ཅི་}{ci}{'что, что за'}, \prfC{སུ་}{su}{'кто'} и \prfC{ག་རེ་}{ga-re}{'что'}.

Вопросительные местоимения \prfB{གང་}{gang} и \prfB{ཅི་}{ci}, сочетаясь с определенными служебными и знаменательными морфемами, образуют большое число различных вопросительных слов, как-то:
\prfC{གང་དུ་}{gang-du}{'куда'},
\prfC{གང་ནས་}{gang-na\ul{s}}{'откуда'},
\prfC{གང་ཞིག་}{gang-zhig}{'который'},
\prfC{ཅི་ཙམ་}{ci-tsam}{'сколько'},
\prfC{ཅི་ལྟར་}{ci-\ul{l}tar}{'как, каким образом'} и т.д.

Вопросительное местоимение \prfC{ག་རེ་}{ga-re}{'что'} употребляется только в новом письменном языке, куда оно вошло из разговорного языка.

\emph{Притяжательные местоимения} как таковые в тибетском языке \emph{отсутствуют}. Притяжательность передается посредством местоимений, выступающих как определения, например:
\prfC{ཁོང་གི་རྟ་}{khong-gi \ul{r}ta}{'его лошадь'},
\prfC{ངའི་དཔེ་ཆ་}{nga'i \ul{d}pe-cha}{'моя книга'},
\prfC{སུའི་པོ་ལོ་}{su'i po-lo}{'чей мяч?'}.

Особняком стоят местоимения	\prfB{ཉིད་}{nyi\ul{d}}, \prfB{རང་ཉིད་}{rang-nyi\ul{d}}, \prfC{རང་རང་}{rang-rang}{'сам'}, которые подчеркивают тождество лиц и ограничивают их от других лиц, например: \prfC{ཁྱེད་རང་ཉིད་}{khyed rang-nyi\ul{d}}{'Вы сам'}.

\section{Имя числительное}

Имя числительное --- часть речи, обозначающая количество или порядок предметов при счете. Синтаксически числительные вступают в сочетания с существительными, обозначающими предметы, количество которых выражается числительными. Они следуют за существительным, а если при существительном имеется прилагательное, то числительные ставятся после них.

Числительные тибетского языка можно разбить на следующие разряды: количественные, дробные, неопределенные, собирательные и порядковые.

1. \emph{Количественные числительные}. Имеется 18 следующих простых числительных, с помощью которых образуются все числа:

\begin{multicols}{2}
    \prfC{གཅིག་}{\ul{g}cig}{'один'}\\
    \prfC{གཉིས་}{\ul{g}nyi\ul{s}}{'два'}\\
    \prfC{གསུམ་}{\ul{g}sum}{'три'}\\
    \prfC{བཞི་}{\ul{b}zhi}{'четыре'}\\
    \prfC{ལྔ་}{\ul{l}nga}{'пять'}\\
    \prfC{དྲུག་}{drug}{'шесть'}\\
    \prfC{བདུན་}{\ul{b}dun}{'семь'}\\
    \prfC{བརྒྱད་}{\ul{br}gya\ul{d}}{'восемь'}\\
    \prfC{དགུ་}{\ul{d}gu}{'девять'}\\
    \prfC{བཅུ་}{\ul{b}cu}{'десять'}\\
    \prfC{བརྒྱ་}{\ul{br}gya}{'сто'}\\
    \prfC{སྟོང་}{\ul{s}tong}{'тысяча'}\\
    \prfC{ཁྲི་}{khri}{'десять тысяч'}\\
    \prfC{འབུམ་}{'bum}{'сто тысяч'}\\
    \prfC{བྱ་བ་}{bya-ba}{'миллион'}\\
    \prfC{ས་ཡ་}{sa-ya}{'десять миллионов'}\\
    \prfC{དུང་ཕུར་}{dung-phur}{'сто миллионов'}\\
    \prfC{ཐེར་འབུམ་}{ther-'bum}{'тысяча миллионов'}\\
\end{multicols}

Сложные числительные образуются по определенным правилам, числа от 11 до 19 сложением десяти с соответствующим числительным, обозначающим единицы, например:
\prfC{བཅུ་བཞི་}{\ul{b}cu-\ul{b}zhi}{'четырнадцать'}
(следует иметь в виду, что 15 и 18 пишутся соответственно \prfB{བཅོ་ལྔ་}{\ul{b}co-\ul{l}nga} и \prfB{བཅོ་བརྒྱད་}{\ul{b}co-\ul{br}gya\ul{d}}, а не \prfB{བཅུ་ལྔ་}{\ul{b}cu-\ul{l}nga} и \prfB{བཅུ་བརྒྱད་}{\ul{b}cu-\ul{br}gya\ul{d}})

Числа от 21 до 99 образуются по схеме:

\fbox{единицы} --- \fbox{\prfA{བཅུ་}'десять'} --- \fbox{служебная морфема} -- \fbox{единицы}

Каждый десяток (за исключением первого) принимает свою служебную морфему:
\begin{tabularx}{\textwidth}{X|X}
    21-29 --- \prfB{རྩ་}{\ul{r}tsa} & 61-69  --- \prfB{རེ་}{re} \\
    31-39 --- \prfB{སོ་}{so} & 71-79  --- \prfB{དོན་}{don} \\
    41-49 --- \prfB{ཞེ་}{zhe} & 81-89 --- \prfB{གྱ་}{gya} \\
    51-59 --- \prfB{ང་}{nga} & 91-99 --- \prfB{གོ་}{go} \\
\end{tabularx}

Например: \prfC{}{\ul{b}zhi-\ul{b}cu-zhe-\ul{l}nga}{'сорок пять'}.

Необходимо обратить внимание на следующие два момента:

1. Служебная морфема второго десятка может употребляться также и со всеми остальными десятками.

2. При образовании чисел от 20 до 99 меняется орфография некоторых числительных:
\begin{description}
    \item 20 пишется \prfB{ཉི་ཤུ་}{nyi-shu}, а не \prfB{གཉིས་བཅུ་}{\ul{g}nyi\ul{s}-\ul{b}cu}, 21 --- \prfB{ཉི་ཤུ་རྩ་གཅིག་}{nyi-shu-\ul{r}tsa-\ul{g}cig} и т.д.;
    \item \prfC{གསུམ་}{\ul{g}sum}{'три'} перед десятком (равно как и перед следующими разрядами чисел) теряет приписную букву и пишется \prfB{སུམ་}{sum}, например: \prfC{སུམ་ཅུ་}{sum-cu}{'тридцать'};
    \item \prfC{བཅུ་}{\ul{b}cu}{'десять'} после чисел 3, 6, 7, 8 теряет префикс и пишется \prfB{ཅུ་}{cu}, например: \prfC{བདུན་ཅུ་}{\ul{b}dun-cu}{'семьдесят'}, но \prfC{བཞི་བཅུ་}{\ul{b}zhi-\ul{b}cu}{'сорок'};
    \item с круглыми десятками пишется слово \prfC{ཐམ་པ་}{tham-pa}{'полный, весь'} (о числах), например: \prfC{ལྔ་བཅུ་ཐམ་པ་}{\ul{l}nga-\ul{b}cu tham-pa}{'пятьдесят'}.
\end{description}
		 
Перейдем к числительным разрядам больше ста.

Простые числа, показывающие количество тысяч, десятков и сотен тысяч, могут стоять как перед, так и после этих разрядов. Если простые числа стоят перед разрядами, то \prfB{གཅིག་}{\ul{g}cig}, \prfB{གཉིས་}{\ul{g}nyi\ul{s}}, \prfB{གསུམ་}{\ul{g}sum} соответственно меняют свое написание на \prfB{ཆིག་}{chig}, \prfB{ཉིས་}{nyi\ul{s}}, \prfB{སུམ་}{sum}, например: \prfC{ཆིག་སྟོང་}{chig-\ul{s}tong}{'одна тысяча'}, \prfC{ཉིས་ཁྲི་}{nyi\ul{s}-khri}{'двадцать тысяч'}. Если же простые числа стоят после указанных разрядов, то они сохраняют обычное написание, причем когда простое число стоит после разряда тысяча, то между ним и разрядом следует морфема \prfB{ཕྲག་}{phrag}, например: \prfC{སྟོང་ཕྲག་གསུམ་}{\ul{s}tong-phrag-\ul{g}sum}{'три тысячи'}. Морфема \prfB{ཕྲག་}{phrag} употребляется также с сотнями, тысячами и десятками тысяч, когда они выступают как круглые числа, например: \prfC{བརྒྱ་ཕྲག་}{\ul{br}gya-phrag}{'сотня'}, \prfC{ཁྲི་ཕྲག་}{khri-phrag}{'десяток тысяч'}. С круглыми числами следующих разрядов в этой функции обычно употребляет
ся морфема \prfB{ཚོ་}{tsho}, например: \prfC{འབུམ་ཚོ་}{'bum-tsho}{'сотня тысяч'}.

В числе, состоящем из трех и более разрядов, при отсутствии промежуточного разряда вместо него ставится соединительный союз \prfB{}{}dang, например:
\prfC{བརྒྱ་དང་བདུན་}{\ul{br}gya dang \ul{b}dun}{'сто семь' (букв. 'сто и семь')};
\prfC{བཞི་སྟོང་དང་ལྔ་}{\ul{b}zhi-\ul{s}tong dang \ul{l}nga}{'четыре тысячи пять' (букв. 'четыре тысячи и пять')}.

Числительные могут также обозначаться и цифрами, имеющими следующие начертания:
\begin{tabularx}{\textwidth}{XXXXXXXXXX}
    \prfA{༡} & \prfA{༢} & \prfA{༣} & \prfA{༤} & \prfA{༥} & \prfA{༦} & \prfA{༧} & \prfA{༨} & \prfA{༩} & \prfA{༠}\\
    1 & 2 & 3 & 4 & 5 & 6 & 7 & 8 & 9 & 0\\
\end{tabularx}

Прочие числа (от десяти и более) при данном способе их обозначения записываются так же, как и в языках, пользующихся арабскими цифрами, только разряды не разделяются точками; ср., например: 1.250.832 --- \prfA{༡༢༥༠༨༣༢}.

2. \emph{Дробные числительные} образуются при помощи морфемы \prfC{ཆ་}{cha}{'часть, доля'}. Число, которое пишется слева от \prfB{ཆ་}{cha} --- знаменатель, справа --- числитель, например: \prfC{}{\ul{l}nga-cha-\ul{b}zhi}{'четыре пятых'},
\prfC{}{\ul{g}sum-cha-\ul{g}nyi\ul{s}}{'две третьих'}. Если знаменатель единица, то он не пишется, например: \prfC{གསུམ་ཆ་}{\ul{g}sum-cha}{'одна треть'}, \prfC{བདུན་ཆ་}{\ul{b}dun-cha}{'одна седьмая'}.

Целое число плюс дробь в письменном языке можно выразить двояко:
\begin{description}
    \item а) дробь + соединительный союз \prfB{དང་}{dang} + целое число, например:
    \prfC{ཕྱེད་དང་དྲུག་}{phyed dang drug}{'шесть с половиной'}, где	\prfC{ཕྱེད་}{phyed}{'половина'}, \prfC{དྲུག་}{drug}{'шесть'} (это построение представляет собой кальку с санскрита);
    \item б) целое число + соединительный союз + дробь, например: \prfC{དྲུག་དང་ཕྱེད་ཀ་}{drug dang phye\ul{d}-ka}{'шесть с половиной'} (эта форма употребляется и в разговорной речи).
\end{description}

3. \emph{Неопределенные числительные} обозначают неопределенное количество:
\prfC{མང་པོ་}{mang-po}{'много'},
\prfC{ཉུང་བ་}{nyung-ba}{'мало'},
\prfC{ཁ་ཤས་}{kha-sha\ul{s}}{'несколько'},
\prfB{འགའ་}{'ga}, \prfC{འགའ་ཞིག་}{'ga-zhig}{'некоторые, несколько'}.

4. \emph{Собирательные числительные} образуются присоединением к количественным числительным служебной морфемы \prfB{ཀ་}{ka}, например: \prfC{གཉིས་ཀ་}{\ul{g}nyi\ul{s}-ka}{'оба, двое'}, \prfC{བདུན་ཀ་}{\ul{b}dun-ka}{'семеро'}.

Сюда же можно отнести и числительные со служебной морфемой \prfB{པོ་}{po}:
\prfC{གསུམ་པོ་}{\ul{g}sum-po}{'из трех состоящий, трое'},
\prfC{བཞི་པོ་}{\ul{b}zhi-po}{'из четырех состоящий, четверо'} и т.д., например:
\prfC{བུ་དྲུག་པོ་}{bu drug-po}{'шестеро сыновей'} (букв. 'сыновья из шести состоящие').

5. \emph{Порядковые числительные} образуются путем присоединения к количественным числительным морфемы \prfB{པ་}{pa}. например: \prfC{བཅུ་གཅིག་པ་}{\ul{b}cu-\ul{g}sig-pa}{'одиннадцатый'}. Исключение составляет порядковое числительное <<первый>>, которое имеет особое написание, а именно: \prfB{དང་པོ་}{dang-po}.

В книгах, изданных ксилографическим способом, при нумерации страниц, частей, томов, а также в индексах и т.п. использовались знаки алфавита, причем номера от 1 до 30 соответственно обозначались знаками тибетского алфавита от \prfB{ཀ}{ka} до \prfB{ཨ}{a}. Номера свыше тридцати обозначались следующим образом:
\begin{description}
    \item 31 до 60 --- от \prfB{ཀི}{ki} до \prfB{ཨི}{i}
    \item 61 до 90 --- от \prfB{ཀུ}{ku} до \prfB{ཨུ}{u}
    \item 91 до 120 --- от \prfB{ཀེ}{ke} до \prfB{ཨེ}{e}
    \item 121 до 150 --- от \prfB{ཀོ}{ko} до \prfB{ཨོ}{o}
    \item 151 до 180 --- от	\prfB{ཀཱ}{kaa} до \prfB{ཨཱ}{aa}
    \item 181 до 210 --- от	\prfB{ཀཱི}{kii} до \prfB{ཨཱི}{ii}
    \item 211 до 240 --- от \prfB{ཀཱུ}{kuu} до \prfB{ཨཱུ}{uu}
    \item 241 до 270 --- от \prfB{ཀཱེ}{kee} до \prfB{ཨཱེ}{ee}
    \item 271 до 300 --- от \prfB{ཀཱོ}{koo} до \prfB{ཨཱོ}{oo}
\end{description}
