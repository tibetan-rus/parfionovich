\section{Имя-атрибутив}

В тибетском языке имеется большая группа имен, обозначающих предмет по его качеству. Такого рода имена мы называем имена-атрибутивы. Эти имена образуются путём присоединения к существительному служебных морфем \prfB{ཅན་}{can} и \prfB{བཅས་}{\ul{b}ca\ul{s}} или глаголов \prfB{ལྡན་}{\ul{l}dan}, \prfB{ཡོད་}{yo\ul{d}}, \prfB{མངའ་}{\ul{m}nga'} выступающих в служебной функции в роли словообразовательных морфем, с общим исходным значением как для морфем, так и для глаголов <<иметь, обладать, наличествовать>>.

Очень продуктивным словообразующим элементом имен этой категории является служебная морфема	\prfB{ཅན་}{can}. При помощи этой морфемы образуется большое количество имен типа
\begin{prfsample}
    \item \prfC{ནོར་ཅན་}{nor-can}{<<богатый>>},
    \item \prfC{ཧུར་བརྩེན་ཅན་}{hur-\ul{b}\ul{r}tsen-can}{<<активный>>},
    \item \prfC{ཡོན་ཏན་ཅན་}{yon-tan-can}{<<достойный>>}.
\end{prfsample}

Менее продуктивна служебная морфема. \prfB{བཅས་}{\ul{b}ca\ul{s}}, образующая имена того же типа, например:
\begin{prfsample}
    \item \prfC{ཆོས་བཅས་}{cho\ul{s}-\ul{b}ca\ul{s}}{<<благочестивый>>}.
\end{prfsample}

Не менее продуктивен, чем служебная морфема \prfB{ཅན་}{can} знаменательный глагол \prfB{ལྡན་}{\ul{l}dan} выступающий в служебной функции. Этот глагол полный аналог морфеме \prfB{ཅན་}{can}. Так,
\begin{prfsample}
    \item \prfB{ནོར་ཅན་}{nor-can} = \prfB{ནོར་ལྡན་}{nor-\ul{l}dan};
    \item \prfB{ཡོན་ཏན་ཅན་}{yon-tan-can} = \prfB{ཡོན་ཏན་ལྡན་}{yon-tan-\ul{l}dan}
\end{prfsample}
и т.д.

Имена такого же порядка могут быть образованы и от именной формы глагола \prfB{ལྡན་}{\ul{l}dan} --- \prfC{ལྡན་པ་}{\ul{l}dan-pa}{<<имеющий, обладающий>>}, при этом употребляется соединительный союз \prfB{དང་}{dang}, например:
\begin{prfsample}
    \item \prfC{ནོར་དང་ལྡན་པ་}{nor-dang-\ul{l}dan-pa}{<<богатый>>},
    \item \prfC{ཧུར་བརྩེན་དང་ལྡན་པ་}{hur-\ul{b}\ul{r}tsen-dang-\ul{l}dan-pa}{<<активный>>}\footnote[31]{По этой же схеме могут образовываться и имена-атрибутивы и от морфемы \prfB{བཅས་}{\ul{b}ca\ul{s}}, например:
    \begin{prfsample}
        \item \prfB{ཆོས་བཅས་}{cho\ul{s}-\ul{b}ca\ul{s}} = \prfC{ཆོས་དང་བཅས་པ་}{cho\ul{s}-dang-\ul{b}ca\ul{s}-pa}{<<благочестивый>>}.
    \end{prfsample}
    }
\end{prfsample}

Аналогичные имена могут быть образованы таким же образом от глаголов \prfB{ཡོད་}{yo\ul{d}} и \prfB{མངའ་}{\ul{m}nga'} или их именных форм \prfB{ཡོད་པ་}{yo\ul{d}-pa}, \prfB{མངའ་བ་}{\ul{m}nga'-ba} <<имеющий, обладающий>>. Только в последнем случае не требуется соединительного союза \prfB{དང་}{dang}, например:
\begin{prfsample}
    \item \prfC{ཡོན་ཏན་མངའ་བ་}{yon-tan-\ul{m}nga'-ba}{<<достойный>>},
    \item \prfC{བཀྲ་ཤིས་ཡོད་པ་}{\ul{b}kra-shi\ul{s}-yo\ul{d}-pa}{<<счастливый>>}.
\end{prfsample}

Те же имена, но с негативным значением могут быть образованы от глагола \prfC{མེད་}{me\ul{d}}{<<не иметь, не иметься>>} (негативная форма глагола \prfB{ཡོད་}{yo\ul{d}}) или от его именной формы \prfC{མེད་པ་}{me\ul{d}-pa}{<<не имеющий, не имеющийся>>}, например: 
\begin{prfsample}
    \item \prfC{ཡོན་ཏེན་མེད་}{yon-ten-me\ul{d}}{<<недостойный>>}.
\end{prfsample}

Европейские грамматики тибетского языка относят рассматриваемую группу имен к категории имен прилагательных. На самом деле имена-атрибутивы обладают признаками как имен прилагательных, так и имен существительных (см. таб. \ref{tab:13}).

\begin{tabularx}{\textwidth}{p{0.3\textwidth}*{6}{c@{\hspace{1em}}}}
    \caption{Сравнение свойств имен существительных, прилагательных и атрибутивов}\label{tab:13}\\
    \toprule
    \parbox[m]{0.3\textwidth}{\small\centering Имена} &
    \vertcell{0.2\textwidth}{\small\raggedright Выступает определением в препозиции} &
    \vertcell{0.2\textwidth}{\small\raggedright Выступает определением в постпозиции} &
    \vertcell{0.2\textwidth}{\small\raggedright Наличие категории множественности} &
    \vertcell{0.2\textwidth}{\small\raggedright Наличие категории рода} &
    \vertcell{0.2\textwidth}{\small\raggedright Наличие категории степеней сравнения} &
    \vertcell{0.2\textwidth}{\small\raggedright Со\-че\-тае\-мость с частицами отрицания}\\
    \midrule
    \endhead
    Имя существительное & + & -- & + & + & -- & --\\
    Имя прилагательное\hyperref[tab:13:spec1]{$^*$} & -- & + & -- & -- & + & +\\
    Имя-атрибутив & + & + & + & -- & -- & --\\
    \bottomrule
\end{tabularx}
{\footnotesize{\label{tab:13:spec1}* У имен прилагательных отсутствует грамматическая категория рода. Однако в редких случаях род может выражаться оппозицией словообразующих морфем \prfB{པོ་}{po} и \prfB{མོ་}{mo}, например:
\begin{prfsample}
    \item \prfC{ཡག་པོ་}{yag-po}{<<хороший>>} --- \prfC{ཡག་མོ་}{yag-mo}{<<хорошая>>}.    
\end{prfsample}
}}

Из таб. \ref{tab:13} видно, что имя-атрибутив может выступать определением как в препозиции, так и в постпозиции, принимая в первом случае притяжательную падежную частицу; ср., например:
\begin{prfsample}
    \item \prfB{ནོར་ཅན་གྱི་མི་}{nor-can-gyi mi} и \prfC{མི་ནོར་ཅན་}{mi nor-can}{<<богатый человек>>}.
\end{prfsample}

Имя-атрибутив может принимать частицы множественности \prfB{རྣམས་}{\ul{r}nam\ul{s}} и \prfB{ཚོ་}{tsho}, например:
\begin{prfsample}
    \item \prfC{ཡོན་ཏན་མེད་རྣམས་}{yon-tan-me\ul{d}-\ul{r}nam\ul{s}}{<<недостойные>>}.
\end{prfsample}

В определении имена-атрибутивы выступают обычно подлежащим, прямым или косвенным дополнением, определением.

Для этих имен характерно также то, что в языке художественных произведений они выступают как средство образной речи, например:
\begin{prfsample}
    \item \prfC{མངར་ལྡན་}{\ul{m}ngar-\ul{l}dan}{<<мед>> (букв. <<сладкий>>)},
    \item \prfC{འཆི་མེད་}{'chi-me\ul{d}}{<<ворон>> (букв. <<бессмертный>>)}.
\end{prfsample}

\section{Местоимение}

В тибетском языке можно выделить следующие виды местоимений: личные, указательные, определительные, неопределённые и вопросительные. Им присущи следующие синтаксические функции:

а) личные, вопросительные и указательные местоимения выступают в функции подлежащего, дополнения, именной части сказуемого и определения, т.е. в тех функциях, что и существительное;

б) определительные и неопределённые местоимения выступают в функции определения в правой позиции, т.е. они соотносимы с прилагательными и числительными.

1. \emph{Личные местоимения}, их родовая отнесённость и сфера употребления показаны в  таб. \ref{tab:14}.

\begin{tabularx}{\textwidth}{cp{0.15\textwidth}cp{0.5\textwidth}}
    \caption{Личные местоимения}\label{tab:14}\\
    \toprule
    \parbox[c]{0.1\textwidth}{\centering Лицо} &
    \parbox[c]{0.15\textwidth}{\centering Место\-име\-ния} &
    \parbox[c]{0.15\textwidth}{\centering Род} &
    \parbox[c]{0.5\textwidth}{\centering Примечания}\\
    \midrule
    \endhead
    \multirow[t]{6}{*}{1-е} & \prfB{ང་}{nga} & общий & употребляется и в разговорном языке\\
    & \prfB{བདག་}{\ul{b}dag} & общий & употребляется и в разговорном языке (западный диалект)\\
    & \prfB{ཁོ་བོ་}{kho-bo} & мужской & \\
    & \prfB{ཁོ་མོ་}{kho-mo} & женский & \\
    & \prfB{ཁོ་མོ་}{nge\ul{d}} & мужской & вежливая форма; употребляется также в диалекте Чамдо\\
    & \prfB{ཕྲན་}{phran} & общий & эпистолярный стиль (уничижительно)\\
    \midrule
    \multirow[t]{2}{*}{2-е} & \prfB{ཁྱོད་}{khyo\ul{d}} & общий & употребляется и в разговорном языке\\
    & \prfB{ཁྱེད་}{khye\ul{d}} & общий & вежливая форма; употребляется и в разговорном языке\\
    \midrule
    \multirow[t]{6}{*}{3-е} & \prfB{ཁོ་}{kho} & общий & употребляется и в разговорном языке\\
    & \prfB{ཁོ་པ་}{kho-pa} & общий & употребляется и в разговорном языке\\
    & \prfB{ཁོ་མ་}{kho-ma} & женский & употребляется и в разговорном языке\\
    & \prfB{ཁོང་}{khong} & общий & вежливая форма; употребляется и в разговорном языке\\
    & \prfB{ཕོ་}{pho} & мужской & новый письменный язык; употребляется и в разговорном языке\\
    & \prfB{མོ་}{mo} & женский & новый письменный язык; употребляется и в разговорном языке\\
    \bottomrule
\end{tabularx}

В новом письменном языке в роли местоимения 1-го и 1-го лица также употребляется (как и в разговорном языке) местоименная морфема \prfC{རང་}{rang}{<<сам>>}.

Морфема	\prfB{རང་}{rang} может присоединяться к местоимениям 1-го лица \prfB{ང་}{nga} и \prfB{ཁོ་མོ་}{nge\ul{d}}, 2-го лица	\prfB{ཁྱོད་}{khyo\ul{d}} и \prfB{ཁྱེད་}{khye\ul{d}}, 3-го лица \prfB{ཁོ་}{kho}, образуя своего рода сложные местоимения \prfB{ང་རང་}{nga-rang}, \prfB{ངེད་རང་}{nge\ul{d}-rang}, \prfB{ཁྱོད་རང་}{khyo\ul{d}-rang}, \prfB{ཁྱེད་རང་}{khye\ul{d}-rang}, \prfB{ཁོ་རང་}{kho-rang} со значением исходного слова. Сама же морфема \prfB{རང་}{rang} полностью утрачивает свое лексическое значение. Чтобы выразить лексическое значение <<сам>>, данную морфему удваивают, например:
\begin{prfsample}
    \item \prfC{ང་རང་རང་}{nga-rang-rang}{<<я сам>>}.
\end{prfsample}

Множественное число личных местоимений образуется прибавлением служебных морфем \prfB{ཚོ་}{tsho}, \prfB{རྣམས་}{\ul{r}nams}, \prfB{ཅག་}{cag} или \prfB{ཅག་རྣམས་}{cag-\ul{r}nams}.

В старом письменном языке 1-е лицо множественного числа обозначается также морфемой	\prfB{འུ་ཅག་}{'u-cag} (или \prfB{འུ་བུ་ཅག་}{'u-bu-cag}) <<мы>>, причём иной раз в сочетании с морфемой множественности \prfB{རྣམས་}{\ul{r}nams}, т.е. \prfB{འུ་ཅག་རྣམས་}{'u-cag-\ul{r}nams}.

Личные местоимения могут принимать числительное \prfC{གཉིས་}{\ul{g}nyi\ul{s}}{<<два>>}, например:
\begin{prfsample}
    \item \prfC{ཁོ་གཉིས་}{kho-\ul{g}nyi\ul{s}}{<<они оба, они вдвоем>>};
    \item \prfB{ངེད་རང་གཉིས་}{nge\ul{d}-rang-\ul{g}nyi\ul{s}} или \prfC{རང་གཉིས་}{rang-\ul{g}nyi\ul{s}}{<<я и ты, мы вдвоем>>}.
\end{prfsample}

2. \emph{Указательные местоимения}:	\prfC{འདི་}{'di}{<<этот, эта, это>>} и \prfC{དེ་}{de}{<<тот, та, то>>}. Множественное лицо от этих местоимений образуется при помощи служебных морфем \prfB{དག་}{dag}, \prfB{རྣམས་}{\ul{r}nams}, \prfB{ཚོ་}{tsho}. Указательные местоимения, присоединяя знаменательную морфему \prfC{འདྲ་}{'dra}{<<подобный, похожий>>}, образуют местоименные слова \prfC{འདི་འདྲ་}{'di-'dra}{<<подобный этому, похожий на это, этакий>>} и \prfC{དེ་འདྲ་}{de-'dra}{<<такой>>}. Эти слова выражают как единичность, так и множественность.

3. \emph{Определительные местоимения}: \prfC{ཀུན་}{kun}{<<все>>}; \prfB{ཐམས་ཅད་}{tham\ul{s}-ca\ul{d}} и \prfC{ཚང་མ་}{tshang-ma}{<<все, всё>>}; \prfB{སོ་སོ་}{so-so}, \prfC{རེ་རེ་}{re-re}{<<каждый>>}. Эти местоимения констатируют, что охват лиц или предметов является полным.

4. \emph{Неопределённые местоимения} указывают на неполноту, частичность или отсутствие определённости. К ним относятся местоимения \prfB{ཁ་ཤས་}{kha-sha\ul{s}} и \prfC{འགའ་ཞིག་}{'ga'-zhig}{<<некоторые, несколько>>}; \prfB{གཞན་}{\ul{g}zhan} или \prfC{གཞན་པ་}{\ul{g}zhan-pa}{<<другой, иной>>}; \prfC{ལ་ལ་}{la-la}{<<некоторые>>}.

5. \emph{Вопросительные местоимения}:
\prfC{གང་}{gang}{<<кто, который, где>>}, \prfC{ཅི་}{ci}{<<что, что за>>}, \prfC{སུ་}{su}{<<кто>>} и \prfC{ག་རེ་}{ga-re}{<<что>>}.

Вопросительные местоимения \prfB{གང་}{gang} и \prfB{ཅི་}{ci}, сочетаясь с определёнными служебными и знаменательными морфемами, образуют большое число различных вопросительных слов, как-то:
\begin{prfsample}
    \item \prfC{གང་དུ་}{gang-du}{<<куда>>},
    \item \prfC{གང་ནས་}{gang-na\ul{s}}{<<откуда>>},
    \item \prfC{གང་ཞིག་}{gang-zhig}{<<который>>},
    \item \prfC{ཅི་ཙམ་}{ci-tsam}{<<сколько>>},
    \item \prfC{ཅི་ལྟར་}{ci-\ul{l}tar}{<<как, каким образом>>}
\end{prfsample}
и т.д.

Вопросительное местоимение \prfC{ག་རེ་}{ga-re}{<<что>>} употребляется только в новом письменном языке, куда оно вошло из разговорного языка.

\emph{Притяжательные местоимения} как таковые в тибетском языке \emph{отсутствуют}. Притяжательность передаётся посредством местоимений, выступающих как определения, например:
\begin{prfsample}
    \item \prfC{ཁོང་གི་རྟ་}{khong-gi \ul{r}ta}{<<его лошадь>>},
    \item \prfC{ངའི་དཔེ་ཆ་}{nga'i \ul{d}pe-cha}{<<моя книга>>},
    \item \prfC{སུའི་པོ་ལོ་}{su'i po-lo}{<<чей мяч?>>}.
\end{prfsample}

Особняком стоят местоимения	\prfB{ཉིད་}{nyi\ul{d}}, \prfB{རང་ཉིད་}{rang-nyi\ul{d}}, \prfC{རང་རང་}{rang-rang}{<<сам>>}, которые подчеркивают тождество лиц и ограничивают их от других лиц, например:
\begin{prfsample}
    \item \prfC{ཁྱེད་རང་ཉིད་}{khyed rang-nyi\ul{d}}{<<Вы сам>>}.    
\end{prfsample}

\section{Имя числительное}

Имя числительное --- часть речи, обозначающая количество или порядок предметов при счёте. Синтаксически числительные вступают в сочетания с существительными, обозначающими предметы, количество которых выражается числительными. Они следуют за существительным, а если при существительном имеется прилагательное, то числительные ставятся после них.

Числительные тибетского языка можно разбить на следующие разряды: количественные, дробные, неопределённые, собирательные и порядковые.

1. \emph{Количественные числительные}. Имеется 18 следующих простых числительных, с помощью которых образуются все числа:

\begin{tabularx}{\textwidth}{l@{\hspace{3em}}l}
    \prfC{གཅིག་}{\ul{g}cig}{<<один>>} & \prfC{བཅུ་}{\ul{b}cu}{<<десять>>}\\
    \prfC{གཉིས་}{\ul{g}nyi\ul{s}}{<<два>>} & \prfC{བརྒྱ་}{\ul{br}gya}{<<сто>>}\\
    \prfC{གསུམ་}{\ul{g}sum}{<<три>>} & \prfC{སྟོང་}{\ul{s}tong}{<<тысяча>>}\\
    \prfC{བཞི་}{\ul{b}zhi}{<<четыре>>} & \prfC{ཁྲི་}{khri}{<<десять тысяч>>}\\
    \prfC{ལྔ་}{\ul{l}nga}{<<пять>>} & \prfC{འབུམ་}{'bum}{<<сто тысяч>>}\\
    \prfC{དྲུག་}{drug}{<<шесть>>} & \prfC{བྱ་བ་}{bya-ba}{<<миллион>>}\\
    \prfC{བདུན་}{\ul{b}dun}{<<семь>>} & \prfC{ས་ཡ་}{sa-ya}{<<десять миллионов>>}\\
    \prfC{བརྒྱད་}{\ul{br}gya\ul{d}}{<<восемь>>} & \prfC{དུང་ཕུར་}{dung-phur}{<<сто миллионов>>}\\
    \prfC{དགུ་}{\ul{d}gu}{<<девять>>} & \prfC{ཐེར་འབུམ་}{ther-'bum}{<<тысяча миллионов>>}\\
\end{tabularx}

Сложные числительные образуются по определённым правилам, числа от 11 до 19 сложением десяти с соответствующим числительным, обозначающим единицы, например:
\begin{prfsample}
    \item \prfC{བཅུ་བཞི་}{\ul{b}cu-\ul{b}zhi}{<<четырнадцать>>}
\end{prfsample}
(следует иметь в виду, что 15 и 18 пишутся соответственно \prfB{བཅོ་ལྔ་}{\ul{b}co-\ul{l}nga} и \prfB{བཅོ་བརྒྱད་}{\ul{b}co-\ul{br}gya\ul{d}}, а не \prfB{བཅུ་ལྔ་}{\ul{b}cu-\ul{l}nga} и \prfB{བཅུ་བརྒྱད་}{\ul{b}cu-\ul{br}gya\ul{d}})

Числа от 21 до 99 образуются по схеме:

\fbox{\parbox[c][2em][c]{10ex}{\centering единицы}} ---
\fbox{\parbox[c][2em][c]{15ex}{\centering \prfA{བཅུ་} <<десять>>}} ---
\fbox{\parbox[c][2em][c]{15ex}{\centering служебная морфема}} ---
\fbox{\parbox[c][2em][c]{10ex}{\centering единицы}}

Каждый десяток (за исключением первого) принимает свою служебную морфему:
\begin{tabularx}{0.5\textwidth}{l@{\hspace{3em}}l}
    21-29 --- \prfB{རྩ་}{\ul{r}tsa} & 61-69  --- \prfB{རེ་}{re} \\
    31-39 --- \prfB{སོ་}{so} & 71-79  --- \prfB{དོན་}{don} \\
    41-49 --- \prfB{ཞེ་}{zhe} & 81-89 --- \prfB{གྱ་}{gya} \\
    51-59 --- \prfB{ང་}{nga} & 91-99 --- \prfB{གོ་}{go} \\
\end{tabularx}

Например:
\begin{prfsample}
    \item \prfC{བཞི་བཅུ་ཞེ་ལྔ་}{\ul{b}zhi-\ul{b}cu-zhe-\ul{l}nga}{<<сорок пять>>}.
\end{prfsample}

Необходимо обратить внимание на следующие два момента:

1. Служебная морфема второго десятка может употребляться также и со всеми остальными десятками.

2. При образовании чисел от 20 до 99 меняется орфография некоторых числительных:
\begin{description}
    \item 20 пишется \prfB{ཉི་ཤུ་}{nyi-shu}, а не \prfB{གཉིས་བཅུ་}{\ul{g}nyi\ul{s}-\ul{b}cu}, 21 --- \prfB{ཉི་ཤུ་རྩ་གཅིག་}{nyi-shu-\ul{r}tsa-\ul{g}cig} и т.д.;
    \item \prfC{གསུམ་}{\ul{g}sum}{<<три>>} перед десятком (равно как и перед следующими разрядами чисел) теряет приписную букву и пишется \prfB{སུམ་}{sum}, например: \prfC{སུམ་ཅུ་}{sum-cu}{<<тридцать>>};
    \item \prfC{བཅུ་}{\ul{b}cu}{<<десять>>} после чисел 3, 6, 7, 8 теряет префикс и пишется \prfB{ཅུ་}{cu}, например: \prfC{བདུན་ཅུ་}{\ul{b}dun-cu}{<<семьдесят>>}, но \prfC{བཞི་བཅུ་}{\ul{b}zhi-\ul{b}cu}{<<сорок>>};
    \item с круглыми десятками пишется слово \prfC{ཐམ་པ་}{tham-pa}{<<полный, весь>>} (о числах), например: \prfC{ལྔ་བཅུ་ཐམ་པ་}{\ul{l}nga-\ul{b}cu tham-pa}{<<пятьдесят>>}.
\end{description}
		 
Перейдем к числительным разрядам больше ста.

Простые числа, показывающие количество тысяч, десятков и сотен тысяч, могут стоять как перед, так и после этих разрядов. Если простые числа стоят перед разрядами, то \prfB{གཅིག་}{\ul{g}cig}, \prfB{གཉིས་}{\ul{g}nyi\ul{s}}, \prfB{གསུམ་}{\ul{g}sum} соответственно меняют свое написание на \prfB{ཆིག་}{chig}, \prfB{ཉིས་}{nyi\ul{s}}, \prfB{སུམ་}{sum}, например:
\begin{prfsample}
    \item \prfC{ཆིག་སྟོང་}{chig-\ul{s}tong}{<<одна тысяча>>},
    \item \prfC{ཉིས་ཁྲི་}{nyi\ul{s}-khri}{<<двадцать тысяч>>}.
\end{prfsample}
Если же простые числа стоят после указанных разрядов, то они сохраняют обычное написание, причём когда простое число стоит после разряда тысяча, то между ним и разрядом следует морфема \prfB{ཕྲག་}{phrag}, например:
\begin{prfsample}
    \item \prfC{སྟོང་ཕྲག་གསུམ་}{\ul{s}tong-phrag-\ul{g}sum}{<<три тысячи>>}.
\end{prfsample}
Морфема \prfB{ཕྲག་}{phrag} употребляется также с сотнями, тысячами и десятками тысяч, когда они выступают как круглые числа, например:
\begin{prfsample}
    \item \prfC{བརྒྱ་ཕྲག་}{\ul{br}gya-phrag}{<<сотня>>},
    \item \prfC{ཁྲི་ཕྲག་}{khri-phrag}{<<десяток тысяч>>}.
\end{prfsample}
С круглыми числами следующих разрядов в этой функции обычно употребляется морфема \prfB{ཚོ་}{tsho}, например:
\begin{prfsample}
    \item \prfC{འབུམ་ཚོ་}{'bum-tsho}{<<сотня тысяч>>}.
\end{prfsample}

В числе, состоящем из трёх и более разрядов, при отсутствии промежуточного разряда вместо него ставится соединительный союз \prfB{དང་}{dang}, например:
\begin{prfsample}
    \item \prfC{བརྒྱ་དང་བདུན་}{\ul{br}gya dang \ul{b}dun}{<<сто семь>> (букв. <<сто и семь>>)};
    \item \prfC{བཞི་སྟོང་དང་ལྔ་}{\ul{b}zhi-\ul{s}tong dang \ul{l}nga}{<<четыре тысячи пять>> (букв. <<четыре тысячи и пять>>)}.
\end{prfsample}

Числительные могут также обозначаться и цифрами, имеющими следующие начертания:
\begin{tabularx}{\textwidth}{XXXXXXXXXX}
    \prfA{༡} & \prfA{༢} & \prfA{༣} & \prfA{༤} & \prfA{༥} & \prfA{༦} & \prfA{༧} & \prfA{༨} & \prfA{༩} & \prfA{༠}\\
    1 & 2 & 3 & 4 & 5 & 6 & 7 & 8 & 9 & 0\\
\end{tabularx}

Прочие числа (от десяти и более) при данном способе их обозначения записываются так же, как и в языках, пользующихся арабскими цифрами, только разряды не разделяются точками; ср., например: 1.250.832 --- \prfA{༡༢༥༠༨༣༢}.

2. \emph{Дробные числительные} образуются при помощи морфемы \prfC{ཆ་}{cha}{<<часть, доля>>}. Число, которое пишется слева от \prfB{ཆ་}{cha} --- знаменатель, справа --- числитель, например:
\begin{prfsample}
    \item \prfC{ལྔ་ཆ་བཞི་}{\ul{l}nga-cha-\ul{b}zhi}{<<четыре пятых>>},
    \item \prfC{གསུམ་ཆ་གཉིས་}{\ul{g}sum-cha-\ul{g}nyi\ul{s}}{<<две третьих>>}.
\end{prfsample}
Если знаменатель единица, то он не пишется, например:
\begin{prfsample}
    \item \prfC{གསུམ་ཆ་}{\ul{g}sum-cha}{<<одна треть>>},
    \item \prfC{བདུན་ཆ་}{\ul{b}dun-cha}{<<одна седьмая>>}.
\end{prfsample}

Целое число плюс дробь в письменном языке можно выразить двояко:
\begin{description}
    \item а) дробь + соединительный союз \prfB{དང་}{dang} + целое число, например:
    \begin{prfsample}
        \item \prfC{ཕྱེད་དང་དྲུག་}{phyed dang drug}{<<шесть с половиной>>}, где	\prfC{ཕྱེད་}{phyed}{<<половина>>}, \prfC{དྲུག་}{drug}{<<шесть>>}
    \end{prfsample}
    (это построение представляет собой кальку с санскрита);
    \item б) целое число + соединительный союз + дробь, например:
    \begin{prfsample}
        \item \prfC{དྲུག་དང་ཕྱེད་ཀ་}{drug dang phye\ul{d}-ka}{<<шесть с половиной>>}
    \end{prfsample}
    (эта форма употребляется и в разговорной речи).
\end{description}

3. \emph{Неопределённые числительные} обозначают неопределённое количество:
\begin{prfsample}
    \item \prfC{མང་པོ་}{mang-po}{<<много>>},
    \item \prfC{ཉུང་བ་}{nyung-ba}{<<мало>>},
    \item \prfC{ཁ་ཤས་}{kha-sha\ul{s}}{<<несколько>>},
    \item \prfB{འགའ་}{'ga}, \prfC{འགའ་ཞིག་}{'ga-zhig}{<<некоторые, несколько>>}.
\end{prfsample}

4. \emph{Собирательные числительные} образуются присоединением к количественным числительным служебной морфемы \prfB{ཀ་}{ka}, например:
\begin{prfsample}
    \item \prfC{གཉིས་ཀ་}{\ul{g}nyi\ul{s}-ka}{<<оба, двое>>},
    \item \prfC{བདུན་ཀ་}{\ul{b}dun-ka}{<<семеро>>}.
\end{prfsample}

Сюда же можно отнести и числительные со служебной морфемой \prfB{པོ་}{po}:
\begin{prfsample}
    \item \prfC{གསུམ་པོ་}{\ul{g}sum-po}{<<из трёх состоящий, трое>>},
    \item \prfC{བཞི་པོ་}{\ul{b}zhi-po}{<<из четырёх состоящий, четверо>>}
\end{prfsample}
и т.д., например:
\begin{prfsample}
    \item \prfC{བུ་དྲུག་པོ་}{bu drug-po}{<<шестеро сыновей>>} (букв. <<сыновья из шести состоящие>>).
\end{prfsample}

5. \emph{Порядковые числительные} образуются путём присоединения к количественным числительным морфемы \prfB{པ་}{pa}. например:
\begin{prfsample}
    \item \prfC{བཅུ་གཅིག་པ་}{\ul{b}cu-\ul{g}sig-pa}{<<одиннадцатый>>}.
\end{prfsample}
Исключение составляет порядковое числительное <<первый>>, которое имеет особое написание, а именно: \prfB{དང་པོ་}{dang-po}.

В книгах, изданных ксилографическим способом, при нумерации страниц, частей, томов, а также в индексах и т.п. использовались знаки алфавита, причём номера от 1 до 30 соответственно обозначались знаками тибетского алфавита от \prfB{ཀ}{ka} до \prfB{ཨ}{a}. Номера свыше тридцати обозначались следующим образом:
\begin{description}
    \item 31 до 60 --- от \prfB{ཀི}{ki} до \prfB{ཨི}{i}
    \item 61 до 90 --- от \prfB{ཀུ}{ku} до \prfB{ཨུ}{u}
    \item 91 до 120 --- от \prfB{ཀེ}{ke} до \prfB{ཨེ}{e}
    \item 121 до 150 --- от \prfB{ཀོ}{ko} до \prfB{ཨོ}{o}
    \item 151 до 180 --- от	\prfB{ཀཱ}{kaa} до \prfB{ཨཱ}{aa}
    \item 181 до 210 --- от	\prfB{ཀཱི}{kii} до \prfB{ཨཱི}{ii}
    \item 211 до 240 --- от \prfB{ཀཱུ}{kuu} до \prfB{ཨཱུ}{uu}
    \item 241 до 270 --- от \prfB{ཀཱེ}{kee} до \prfB{ཨཱེ}{ee}
    \item 271 до 300 --- от \prfB{ཀཱོ}{koo} до \prfB{ཨཱོ}{oo}
\end{description}

\section{Наречие}

Наречие --- знаменательная часть речи, обозначающая признаки действий и состояний, а также признаки качеств.

Синтаксически тибетские наречия характеризуются тем, что: а) выступают в предложении в качестве обстоятельств; б) не сочетаются ни с какими служебными словами; в) не могут входить в атрибутивные конструкции.

Морфологически наречия в своем подавляющем большинстве являются лексикализовавшимися синтаксическими структурами. Большая часть этих структур представляет собой сочетание имени с падежными частицами (косвенной, орудной или исходной).

По своему значению наречия распадаются на две группы: обстоятельственные наречия (1); качественные наречия (2).

1. \emph{Обстоятельственные наречия} подразделяются на:

а) \emph{наречия места}, например:
\begin{prfsample}
    \item \prfC{ཉི་བར་}{nye-bar}{<<вблизи, возле, около>>},
    \item \prfC{ཐག་རིང་ན་}{thag-ring-na}{<<вдали>>},
    \item \prfC{དེ་ན་}{de-na}{<<там>>},
    \item \prfC{འདི་ན་}{'di-na}{<<здесь>>},
    \item \prfC{ནང་དུ་}{nang-du}{<<внутри>>},
    \item \prfC{ནང་ནས་}{nang-na\ul{s}}{<<изнутри>>},
    \item \prfC{མདུན་དུ་}{\ul{m}dun-du}{<<впереди>>},
    \item \prfC{རྒྱབ་དུ་}{\ul{r}gyab-du}{<<сзади>>},
    \item \prfC{འོག་ན་}{'og-na}{<<под, внизу>>},
    \item \prfC{འོག་ནས་}{'og-na\ul{s}}{<<внизу>>},
    \item \prfC{གོང་ནས་}{gong-na\ul{s}}{<<сверху>>},
    \item \prfC{གྱེན་དུ་}{gyen-du}{<<вверх>>};
\end{prfsample}

б) \emph{наречия времени}, например:
\begin{prfsample}
    \item \prfC{རྟག་པར་}{\ul{r}tag-par}{<<всегда>>},
    \item \prfC{སྔར་}{\ul{s}ngar}{<<ранее, прежде>>},
    \item \prfC{རེས་འགའ་}{re\ul{s}-'ga}{<<иногда>>},
    \item \prfC{ཉིན་རེ་}{nyin-re}{<<ежедневно>>},
    \item \prfC{ལོ་རེ་}{lo-re}{<<ежегодно>>};
\end{prfsample}

в) \emph{наречия совместности}, например:
\begin{prfsample}
    \item \prfB{ལྷན་ཅིག་}{lhan-cig}, \prfB{ཆབས་ཅིག་}{chab\ul{s}-cig}, \prfB{མཉམ་དུ་}{\ul{m}nyam-du} <<сообща, вместе, совместно>>
\end{prfsample}
(обычно этим наречиям предшествует соединительный союз \prfC{དང་}{dang}{<<и, с>>}).

2. \emph{Качественные наречия} подразделяются на:

а) \emph{наречия способа действия}, например:
\begin{prfsample}
    \item \prfC{གློ་བུར་དུ་}{\ul{g}lo-bur-du}{<<внезапно>>},
    \item \prfC{རིམ་གྱིས་}{rim-gyi\ul{s}}{<<по очереди>>},
    \item \prfC{སླར་}{\ul{s}lar}{<<снова, опять>>},
    \item \prfC{དངོས་སུ་}{\ul{d}ngo\ul{s}-su}{<<явно>>},
    \item \prfC{ཕག་ཏུ་}{phag-tu}{<<тайно, скрыто>>},
    \item \prfC{ལམ་སང་}{lam-sang}{<<немедленно>>};
\end{prfsample}

б) \emph{наречия меры и степени}, например:
\begin{prfsample}
    \item \prfC{ལན་གཅིག་}{lan-\ul{g}cig}{<<однажды>>},
    \item \prfC{ལན་གཉིས་}{lan-\ul{g}nyi\ul{s}}{<<дважды>>},
    \item \prfC{ལྡབ་གཉིས་}{\ul{l}dab-\ul{g}nyi\ul{s}}{<<втрое>>},
    \item \prfC{ཉུང་དུ་}{nyung-du}{<<немного>>},
    \item \prfC{ཧ་ཅང་}{ha-cang}{<<весьма>>},
    \item \prfC{ཤིན་ཏུ་}{shin-tu}{<<очень>>},
    \item \prfC{རབ་ཏུ་}{rap-tu}{<<крайне>>},
    \item \prfC{ཀུན་ཏུ་}{kun-tu}{<<совсем, совершенно>>},
    \item \prfC{ཏོག་ཙམ་}{tog-tsam}{<<мало, немного>>},
    \item \prfC{ཞེ་སྐྲག་}{zhe-\ul{s}krag}{<<крайне, чрезвычайно>>}.
\end{prfsample}



