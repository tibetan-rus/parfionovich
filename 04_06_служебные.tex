\section{Служебные слова}
\label{sec:ss}

Служебные слова, в отличие от знаменательных, не обладают номинативной функцией, т.е. не служат для обозначения предметов, явлений, признаков, процессов реального мира. Поэтому они не могут быть членами предложения и не могут употребляться отдельно от членов предложения или предложения в целом, а используются в предложении в качестве грамматических средств языка.

В силу отсутствия флексий\prfnote{Флексия --- все окончания, которые присоединяются к корню или суффиксу слова, как например, окончания при склонениях и спряжениях.} у тибетского слова и специфики тибетского глагола роль служебных слов в тибетском языке особенно велика.

В отличие от служебных слов других языков той же семьи, например китайского или бирманского, ряд служебных слов в тибетском языке существует в нескольких вариантах, каждый из которых обладает одними и теми же грамматическими значениями, а выбор того или иного варианта зависит только от орфографии знаменательного слова, к которому относится данное служебное слово.

\index{\prfB{ཅིག་}{cig}}\index{\prfB{ཞིག་}{zhig}}\index{\prfB{ཤིག་}{shig}}Другой особенностью тибетских служебных слов является их полифункциональность. Так, например, служебные слова-варианты \prfB{ཅིག་}{cig}, \prfB{ཞིག་}{zhig} и \prfB{ཤིག་}{shig} могут выступать и как частицы неопределённости и как показатели повелительного наклонения глагола. В первом случае они сочетаются с существительным (или группой существительного), во втором --- с глаголом.

Служебные слова тибетского языка: а) всегда стоят в правой позиции по отношению к слову или группе слов, к которым они относятся; б) при наличии однородных членов предложения все они оформляются только одним служебным словом, которое следует за последним однородным членом предложения.

В тибетском языке существуют различные виды служебных слов.

I. \emph{Падежные частицы} (см. \ref{page71}) следуют за словом или группой слов, к которым они относятся. Имеются следующие падежные частицы:

\index{\prfB{སུ་}{su}}\index{\prfB{ར་}{ra}}\index{\prfB{རུ་}{ru}}\index{\prfB{དུ་}{du}}\index{\prfB{ན་}{na}}\index{\prfB{ལ་}{la}}\index{\prfB{ཏུ་}{tu}}1. Косвенная падежная частица --- существует в семи вариантах: \prfB{སུ་}{su}, \prfB{ར་}{ra}, \prfB{རུ་}{ru}, \prfB{དུ་}{du}, \prfB{ན་}{na}, \prfB{ལ་}{la}, \prfB{ཏུ་}{tu}\footnote[40]{Национальные грамматики называют их общим термином \prfB{ལ་དོན་}{la-don} букв. <<обладающие значением \prfB{ལ}{la}>> или <<варианты частицы la>> Эти же грамматики, которые, исходя из санскрита, выделяют в тибетском языке восемь падежей, называют указанные частицы падежными частицами второго, четвёртого и седьмого падежей.}.	
Варианты косвенной падежной частицы несут несколько грамматических значений, которые присущи каждому из них. Они могут указывать:

а) направление действия или место его совершения, например:
\begin{prfsample}
	\item \prfC{ཤར་ཕྱོགས་\selA{སུ་}འགྲོ་}{shar-phyog\ul{s}-\selA{su} 'gro}{<<идти на восток>> (букв. <<восток \textbullseye\footnote[41]{Знаком \textbullseye{} условно обозначаем место грамматической частицы} идти>>)};
	\item \prfC{བུམ་པ\selA{ར}་བླུག་}{bum-pa\selA{r} \ul{b}lug}{<<наливать в бутылку>> (букв. <<бутылка \textbullseye{} наливать>>)};
	\item \prfC{ལམ་\selA{ལ་}བཞུད་}{lam-\selA{la} \ul{b}zhu\ul{d}}{<<проходить по дороге>> (букв. <<дорога \textbullseye{} проходить>>)};
	\item \prfC{སྟེན་\selA{ན་}གར་འཁྲབ་}{\ul{s}ten-\selA{na} gar-'khrab}{<<танцевать наверху>> (букв. <<верх \textbullseye{} танцевать>>)};
	\item \prfC{ར་མདའ་\selA{རུ་}སྦྲོན་}{ra-\ul{m}da'-\selA{ru} \ul{s}bron}{<<взывать о помощи>> (букв. <<помощь \textbullseye{} взывать>>)};
	\item \prfC{མལ་\selA{ན་}ཉལ་}{mal-\selA{na} nyal}{<<спать на ложе>> (букв. <<ложе \textbullseye{} спать>>)};
	\item \prfC{ཁྲི་\selA{ལ་}བཞུགས་}{khri-\selA{la} \ul{b}zhug\ul{s}}{<<восседать на троне>> (букв. <<трон \textbullseye{} восседать>>)}\footnote[42]{Национальные тибетские грамматики называют эти отношения вторым падежом (\prfB{ལས་སུ་བྱ་བ་}{la\ul{s}-su-bya-ba}), который можно соотнести со вторым (винительным) падежом санскрита.};
\end{prfsample}

б) для кого совершается действие или адресат действия в самом широком смысле этого слова, например:
\begin{prfsample}
	\item \prfC{སློང་མོ་བ་\selA{ལ་}ལྟོ་སྦྲད་}{\ul{s}long-mo-ba-\selA{la} \ul{l}to \ul{s}bra\ul{d}}{<<подавать нищему пищу>> (букв. <<нищий \textbullseye{} пища давать>>)};
	\item \prfC{རྨ་\selA{རུ་}སྨན་འདེབས་}{\ul{r}ma-\selA{ru} \ul{s}man 'deb\ul{s}}{<<класть лекарство на рану>> (букв. <<рана \textbullseye{} лекарство класть>>)};
	\item \prfC{ཤིང་\selA{ཏུ་}ལུད་བཀྲམ་}{shing-\selA{tu} lu\ul{d} \ul{b}kram}{<<унавоживать дерево>> (букв. <<дерево \textbullseye{} навоз разбрасывать>>)};
	\item \prfC{ལུས་\selA{སུ་}གོས་གྱོན་}{lu\ul{s}-\selA{su} go\ul{s} gyon}{<<надевать одежду на тело>> (букв. <<тело \textbullseye{} одежда надевать>>)}\footnote[43]{Национальные грамматики называют такого рода отношения четвертым падежом (\prfB{དགོས་ཆེད་}{\ul{d}go\ul{s}-che\ul{d}}), который можно соотнести с четвертым (винительным) падежом санскрита.};
\end{prfsample}

в) место, где находится предмет, а также обозначают субъект, обладающий предметом, например:
\begin{prfsample}
	\item \prfC{ནགས་\selA{སུ་}བཅན་གཟན་}{nag\ul{s}-\selA{su} \ul{b}can-\ul{g}zan}{<<звери в лесу>> (букв. <<лес \textbullseye{} зверь>>)};
	\item \prfC{ཁ་\selA{རུ་}སོ་}{kha-\selA{ru} so}{<<зубы во рту>> (букв. <<рот \textbullseye{} зубы>>)};
	\item \prfC{ཕོ་ཕྲང་\selA{ན་}རྒྱལ་པོ་}{pho-phrang-\selA{na} \ul{r}gyal-po}{<<король во дворце>> (букв. <<дворец \textbullseye{} король>>)};
	\item \prfC{ལུས་\selA{ལ་}ཆས་གོས་}{lu\ul{s}-\selA{la} cha\ul{s}-go\ul{s}}{<<одежда на теле>> (букв. <<тело \textbullseye{} одежда>>)}\footnote[44]{Национальные тибетские грамматики называют подобные отношения седьмым падежом (\prfB{གནས་གཞི་}{\ul{g}na\ul{s}-\ul{g}zhi}), который можно соотнести с седьмым (местным) падежом санскрита.};
	\item \prfC{མི་དམངས་ཚོ\selA{ར}་བཅིངས་བཀྲོལ་ཐོབ་}{mi-\ul{d}mang\ul{s}-tsho\selA{r} \ul{b}cing\ul{s}-\ul{b}kro\ul{l} thob}{<<народы обрели освобождение>> (букв. <<народы \textbullseye{} освобождение обретать>>)};
	\item \prfC{མ་\selA{ལ་}བུ་བཙས་}{ma-\selA{la} bu \ul{b}tsa\ul{s}}{<<мать родила сына>> (букв. <<мать \textbullseye{} сын родить>>)};
\end{prfsample}

г) время совершения действия, например:
\begin{prfsample}
	\item \prfC{ཆུ་ཚོད་དགུ་\selA{ལ་}སློབ་གྲྭ་ཚུགས་}{chu-tsho\ul{d} \ul{d}gu-\selA{la} \ul{s}lob-gr\ul{w}a-tshug\ul{s}}{<<занятия в школе начинаются в девять часов>> (букв. <<час девять \textbullseye{} школа начинать>>)};
	\item \prfC{མཚན་མོ\selA{ར}་}{\ul{m}tshan-mo\selA{r}}{<<в ночь, ночью>> (букв. <<ночь \textbullseye{}>>)};
\end{prfsample}

Косвенная падежная частица может следовать также за первым сказуемым предложения, указывая в этом случае на цель действия, выражаемого вторым сказуемым, например:
\begin{prfsample}
	\item \prfC{ལྟོ་ཟ་\selA{ལ་}འགྲོ་}{\ul{l}to za-\selA{la} 'gro}{<<идти есть пищу>> (букв. <<пища есть \textbullseye{} идти>>)};
	\item \prfC{ཤིང་གཅོད་\selA{དུ་}སྟ་རེ་དགོས་}{shing \ul{g}co\ul{d}-\selA{du} \ul{s}ta-re \ul{d}go\ul{s}}{<<необходим топор, чтобы рубить дерево>> (букв. <<дерево рубить \textbullseye{} топор необходим>>)};
\end{prfsample}

Таким образом, косвенной падежной частицей оформляются в предложении: а) косвенное дополнение; б) обстоятельства; в) подлежащее при глаголах, обозначающих наличие или приобретение (см. раздел \hyperref[sec:glagol]{<<Глагол>>}).

Употребление того или иного варианта косвенной падежной частицы (за исключением \prfB{ལ་}{la} и \prfB{ན་}{na}) зависит только от приписной графемы предыдущего слога. После приписной \prfB{ས}{sa} употребляется вариант \prfB{སུ་}{su}; после приписных \prfB{ང}{nga}, \prfB{ད}{da}, \prfB{ན}{na}, \prfB{མ}{ma}, \prfB{ར}{ra}, \prfB{ལ}{la} --- вариант  \prfB{དུ་}{du}; после приписных \prfB{ག}{ga} или \prfB{བ}{ba} --- вариант \prfB{ཏུ་}{tu}. Если предыдущий слог оканчивается на приписную \prfB{འ}{'a} или вообще не имеет приписной, то употребляются варианты \prfB{རུ་}{ru} или \prfB{ར་}{ra}. При этом вариант \prfB{ར་}{ra} пишется слитно со словом, к которому он относится, например:
\begin{prfsample}
	\item \prfC{ཁ་}{kha}{<<рот>>} --- \prfC{ཁ\selA{ར}་}{kha\selA{r}}{<<во роту>>}.	
\end{prfsample}
При наличии в слоге приписной \prfB{འ}{'a} частица \prfB{ར་}{ra} заменяет его, например:
\begin{prfsample}
	\item \prfC{ནམ་མཁའ་}{nam-\ul{m}kha'}{<<небо>>} --- \prfC{ནམ་མཁ\selA{ར}་}{nam-\ul{m}kha\selA{r}}{<<в небо>> или <<в небе>>}.
\end{prfsample}

Варианты \prfB{ལ་}{la} и \prfB{ན་}{na} употребляются независимо от орфографии предыдущего слова, т.е. могут заменять собой все остальные варианты. Хотя национальные грамматики не делают никакого различия между \prfB{ལ་}{la} и \prfB{ན་}{na}, практически \prfB{ལ་}{la} употребляется значительно чаще. \prfB{ན་}{na} обычно употребляется с глаголами состояния\footnote[45]{В живой речи (лхасский диалект) употребляются только частицы-варианты \prfB{ལ་}{la} и \prfB{ར་}{ra}.}.

\index{\prfB{ཀྱི་}{kyi}}\index{\prfB{གི་}{gi}}\index{\prfB{གྱི་}{gyi}}\index{\prfB{ཡི་}{yi}}\index{\prfB{འི་}{'i}}\label{04_06_ppch}2. Притяжательная падежная частица --- представлена вариантами: \prfB{ཀྱི་}{kyi}, \prfB{གི་}{gi}, \prfB{གྱི་}{gyi}, \prfB{ཡི་}{yi}, \prfB{འི་}{'i}. Все они выполняют одинаковую грамматическую функцию, а именно указывают на то, что слово (группа слов), к которому они примыкают, являются атрибутом (в самом широком смысле) последующего слова\footnote[46]{Такого рода отношения тибетские национальные грамматики называют шестым падежом (\prfB{འབྲེལ་བ་}{'bre\ul{l}-ba}), а падежные частицы --- падежными частицами шестого падежа.}.
Эти частицы могут указывать на:

а) материал, из которого сделан или состоит предмет, например:
\begin{prfsample}
	\item \prfC{ཤིང་\selA{གི་}ཁྲི་}{shing-\selA{gi} khri}{<<деревянное сидение>> (букв. <<дерево \textbullseye{} сидение>>)};
	\item \prfC{ལོ་མ\selA{འི}་སྤྱི་པོ་}{lo-ma\selA{'i} \ul{s}pyi po}{<<шалаш из листьев>> (букв. <<лист \textbullseye{} шалаш>>)};
	\item \prfC{ནོར་བུ\selA{འི}་ཁྲབ་}{nor-bu\selA{'i} khrab}{<<панцирь с драгоценными камнями>> (букв. <<драгоценный камень \textbullseye{} панцирь>>)};
\end{prfsample}

б) содержимое предмета, например:
\begin{prfsample}
	\item \prfC{སྡིག་པ\selA{འི}་ལུས་}{\ul{s}dig-pa\selA{'i} lu\ul{s}}{<<греховное деяние>> (букв. <<грех \textbullseye{} деяние>>)};
	\item \prfC{ཁྲིམས་བརྒྱད་\selA{ཀྱི་}དཔེ་}{khrim\ul{s} \ul{br}gya\ul{d}-\selA{kyi} \ul{d}pe}{<<книга о восьми заповедях>> (букв. <<заповедь восемь \textbullseye{} книга>>)};
	\item \prfC{གོས་\selA{ཀྱི་}སྒྲོམ་}{go\ul{s}-\selA{kyi} \ul{s}grom}{<<корзина с одеждой>> (букв. <<одежда \textbullseye{} корзина>>)};
\end{prfsample}

в) назначение предмета, например:
\begin{prfsample}
	\item \prfC{བག་མ་བསུ་བ\selA{འི}་ཤིང་རྟ་}{bag-ma \ul{b}su-ba\selA{'i} shing-\ul{r}ta}{<<экипаж для встречи невесты>> (букв. <<невеста встречание \textbullseye{} экипаж>>)};
	\item \prfC{ལོ་གཅིག་\selA{གི་}ཟ་སྐལ་}{lo \ul{g}cig-\selA{gi} za-\ul{s}ka\ul{l}}{<<пища на год>> (букв. <<год один \textbullseye{} пища>>)};
\end{prfsample}

г) происхождение или принадлежность предмета:
\begin{prfsample}
	\item \prfC{དཀན་\selA{ཀྱི་}ཡི་གེ་}{\ul{d}kan-\selA{kyi} yi-ge}{<<палатальные звуки>> (букв. <<нёбо \textbullseye{} буква>>)};
	\item \prfC{ཁོང་\selA{གི་}དཔེ་ཆ་}{khong-\selA{gi} \ul{d}pe-cha}{<<его книга>> (букв. <<он \textbullseye{} книга>>)};
	\item \prfC{བོད་\selA{ཀྱི་}རྒྱལ་པོ་}{bo\ul{d}-\selA{kyi} \ul{r}gya\ul{l}-po}{<<царь Тибета>> (букв. <<Тибет \textbullseye{} царь>>)};
\end{prfsample}

Употребление того или иного варианта притяжательной падежной частицы зависит только от приписной графемы предыдущего слога. Так, после приписных графем \prfB{ད}{da}, \prfB{བ}{ba}, \prfB{ས}{sa} употребляется вариант \prfB{ཀྱི་}{kyi}; после приписных графем \prfB{ག}{ga} или \prfB{ང}{nga} --- вариант \prfB{གི་}{gi}; после приписных графем \prfB{ན}{na}, \prfB{མ}{ma}, \prfB{ར}{ra}, \prfB{ལ}{la} --- вариант \prfB{གྱི་}{gyi}; после приписной графемы \prfB{འ}{'a} в открытом слоге могут употребляться варианты \prfB{ཡི་}{yi} и \prfB{འི་}{'i}. При этом вариант \prfB{འི་}{'i} пишется слитно со слогом, к которому он относится, и меняет его звучание (см. таб. \ref{tab:4}). Если же слово оканчивается на приписную \prfB{འ}{'a}, то последняя получает лишь огласовку \prfB{ ི}{i}, например:
\begin{prfsample}
	\item \prfC{ནམ་མཁའ་}{nam-mkha'}{<<небо>>} --- \prfC{ནམ་མཁ\selA{འི}་}{nam-mkha\selA{'i}}{<<небесный>>}.
\end{prfsample}

\index{\prfB{ཀྱིས་}{kyi\ul{s}}}\index{\prfB{གིས་}{gi\ul{s}}}\index{\prfB{གྱིས་}{gyi\ul{s}}}\index{\prfB{ཡིས་}{yi\ul{s}}}\index{\prfB{འིས་}{'i\ul{s}}}3. Орудная падежная частица --- существует в виде следующих вариантов:
\prfB{ཀྱིས་}{kyi\ul{s}}, \prfB{གིས་}{gi\ul{s}}, \prfB{གྱིས་}{gyi\ul{s}}, \prfB{ཡིས་}{yi\ul{s}}, \prfB{འིས་}{'i\ul{s}}\footnote[47]{Национальные тибетские грамматики называют их падежными частицами третьего падежа (\prfB{བྱེད་སྒྲ་}{bye\ul{d}-\ul{s}gra}).}.
Все они выполняют одинаковые грамматические функции, указывая на:

а) субъект действия (в предложении, содержащем переходный глагол), например:
\begin{prfsample}
	\item \prfC{ང\selA{ས}་དཔེ་ཆ་ཀློག་}{nga\selA{\ul{s}} \ul{d}pe-cha \ul{k}log}{<<я читаю книгу>> (букв. <<я \textbullseye{} книгу читаю>>)};
	\item \prfC{བདག་\selA{གིས་}ཐོས་}{\ul{b}dag-\selA{gis} tho\ul{s}}{<<я \textbullseye{} слышу>>};
\end{prfsample}

б) орудие действия в самом широком смысле этого слова, например:
\begin{prfsample}
	\item \prfC{སྟ་རེ\selA{ས}་གཅོད་}{\ul{s}ta-re\selA{\ul{s}} \ul{g}co\ul{d}}{<<рубить топором>> (букв. <<топор \textbullseye{} рубить>>)};
	\item \prfC{ལག་པ\selA{ས}་རྡུང་}{lag-pa\selA{\ul{s}} \ul{r}dung}{<<бить рукой>> (букв. <<рука \textbullseye{} бить>>)};
	\item \prfC{སྙན་\selA{གྱིས་}གསན་}{\ul{s}nyan-\selA{gyi\ul{s}} \ul{g}sa\ul{n}}{<<слушать ушами>> (букв. <<ухо \textbullseye{} слышать>>)};
	\item \prfC{ནད་\selA{ཀྱིས་}མནར་}{na\ul{d}-\selA{kyi\ul{s}} \ul{m}nar}{<<страдать из-за болезни>> (букв. <<болезнь \textbullseye{} страдать>>)};
	\item \prfC{རྫུ་འཕྲུལ་\selA{གྱིས་}ཆར་དྲག་བབས་}{\ul{r}dzu-'phru\ul{l}-\selA{gyi\ul{s}} char-drag bab\ul{s}}{<<магическими чарами вызвать ливень>> (букв. <<магические чары \textbullseye{} ливень разразился>>)};
	\item \prfC{མངུལ་\selA{གྱིས་}བཟོས་}{\ul{m}ngu\ul{l}-\selA{gyi\ul{s}} \ul{b}zo\ul{s}}{<<сделал из серебра>> (букв. <<серебро \textbullseye{} сделал>>)};
	\item \prfC{ཁྲག་\selA{གིས་}དམར་}{khrag-\selA{gi\ul{s}} \ul{d}mar}{<<окрасить кровью>> (букв. <<кровь \textbullseye{} окрасить>>)};
	\item \prfC{གསེར་གཡུ་བྱི་རུ་མུ་ཏིག་སོགས་\selA{ཀྱིས་}བཀང་}{\ul{g}ser \ul{g}.yu byi ru mu-tig sog\ul{s}-\selA{kyi\ul{s}} \ul{b}kang}{<<наполнил золотом, бирюзой, кораллами и жемчугом>> (букв. <<золото, бирюза, кораллы, жемчуг \textbullseye{} наполнить>>)}.
\end{prfsample}

В новом письменном языке варианты орудной падежной частицы также широко используются для образования наречных конструкций типа:
\begin{prfsample}
	\item \prfC{སྤྲོ་སེམས་ཆེན་པོ\selA{ས}་}{\ul{s}pro-sem\ul{s} chen-po\selA{\ul{s}}}{<<с огромной радостью>> (букв. <<радость огромный \textbullseye{} >>)};
	\item \prfC{ཧུར་བརྩེན་\selA{གྱིས་}}{hur-\ul{br}tsen-\selA{gyi\ul{s}}}{<<активно>> (букв. <<активность \textbullseye{} >>)}.
\end{prfsample}

Варианты орудной падежной частицы могут присоединяться к сказуемому придаточного предложения, указывая на причину действия главного предложения:
\begin{prfsample}
	\item \prfC[pp]{ཁ་འཆམ་མིན་\selA{གྱིས་}རྒྱལ་པོའི་དྲང་དུ་བབས་}{kha-'cham min \selA{gyi\ul{s}} \ul{r}gyal-po'i drang-du bab\ul{s}}{<<так как не было [среди них] согласия, [они] пришли к царю>> (букв. <<согласие не иметься \textbullseye{} царь к пришли>>)}.
\end{prfsample}

При наличии в одном высказывании двух членов предложения, оформленных орудной падежной частицей (например, субъект действия и орудие действия), на первом месте всегда ставится субъект действия, например:
\begin{prfsample}
	\item \prfC{ཁོང་\selA{གིས་}པིར་\selA{གྱིས་}འབྲིས་}{khong-\selA{gi\ul{s}} pir-\selA{gyi\ul{s}}-'bri\ul{s}}{<<он писал кисточкой>> (букв. <<он \textbullseye{} кисточка \textbullseye{} писал>>)}.
\end{prfsample}

Употребление того или иного варианта орудной падежной частицы связано только с приписными графемами предыдущего слога и полностью совпадает с употреблением вариантов притяжательной падежной частицы.

\index{\prfB{ལས་}{la\ul{s}}}\index{\prfB{ནས་}{na\ul{s}}}4. Исходные падежные частицы \prfB{ལས་}{la\ul{s}} и \prfB{ནས་}{na\ul{s}} --- выполняют следующие грамматические функции:

а) указывают на связь между предметами (явлениями), если одно вытекает из другого, например:
\begin{prfsample}
	\item \prfC{བ་\selA{ལས་}འོ་མ་}{ba \selA{la\ul{s}} 'o-ma}{<<молоко от коровы>> (букв. <<корова \textbullseye{} молоко>>)};
	\item \prfC{མེ་\selA{ནས་}དུ་བ་}{me \selA{na\ul{s}} du-ba}{<<дым от огня>> (букв. <<огонь \textbullseye{} дым>>)};
	\item \prfC{མ་\selA{ལས་}བུ་}{ma \selA{la\ul{s}} bu}{<<ребенок от матери>> (букв. <<мать \textbullseye{} ребенок>>)};
\end{prfsample}

б) указывают на отправной пункт действия, например:
\begin{prfsample}
	\item \prfC{རྟ་\selA{ལས་}ལྷུང་}{\ul{r}ta \selA{la\ul{s}} lhung}{<<падать с лошади>> (букв. <<лошадь \textbullseye{} падать>>)};
	\item \prfC{སྐས་\selA{ནས་}བབས་}{\ul{s}ka\ul{s} \selA{na\ul{s}} bab\ul{s}}{<<спуститься с лестницы>> (букв. <<лестница \textbullseye{} спуститься>>)}.
\end{prfsample}

Во всех случаях может быть употреблено как \prfB{ལས་}{la\ul{s}}, так и \prfB{ནས་}{na\ul{s}}\footnote[48]{Национальные грамматики относят такого рода связи к пятому падежу (\prfB{འབྱུང་ཁུངས་}{'byung-khung\ul{s}}).}.

\index{\prfB{ནས་}{na\ul{s}}}Кроме того, частица \prfB{ནས་}{na\ul{s}} употребляется:

а) для выделения одного предмета (явления) из группы однородных предметов (явлений), например:
\begin{prfsample}
	\item \prfC{བྱ་ཡི་ནང་\selA{ནས་}རྨ་བྱ་བཟུགས་མཛེས་}{bya-yi nang-\selA{na\ul{s}} \ul{r}ma-bya \ul{b}zug\ul{s} \ul{m}dze\ul{s}}{<<из птиц самая красивая павлин>>};
	\item \prfC{ཤིང་གི་ནང་\selA{ནས་}ཅན་དན་དྲི་ཞིམ་}{shing-gi nang-\selA{na\ul{s}} can-dan dri zhim}{<<из деревьев самое ароматное сандал>>};
\end{prfsample}

б) в конструкции <<от\ldots{} до\ldots{}>>, например:
\begin{prfsample}
	\item \prfC{གཅིག་\selA{ནས་}བཅུའི་བར་}{\ul{g}cig \selA{na\ul{s}} bcu'i bar}{<<от одного до десяти>>};
	\item \prfC{མགོ་\selA{ནས་}རྐང་པའི་བར་}{\ul{m}go \selA{na\ul{s}} \ul{r}kang-pa'i bar}{<<с головы до ног>>}.
\end{prfsample}
В этих случаях \prfB{ལས་}{la\ul{s}} употребить нельзя.

\index{\prfB{ལས་}{la\ul{s}}}Частица же \prfB{ལས་}{la\ul{s}} может участвовать в образовании сравнительной степени у прилагательных (см. разд. \hyperref[sec:prilagatelnoe:kat_step_srav]{<<Категории степеней сравнения>>}).

\index{\prfB{ཀྱང་}{kyang}}\index{\prfB{འང་}{'ang}}\index{\prfB{ཡང་}{yang}}II. \emph{Уступительно-усилительная частица} --- существует в трёх вариантах (\prfB{ཀྱང་}{kyang}, \prfB{འང་}{'ang}, \prfB{ཡང་}{yang}) и выполняет две основные грамматические функции:

а) после сказуемого выступает в роли уступительного союза <<хотя>>, <<хотя\ldots{} однако>>, например:
\begin{prfsample}
	\item \prfC{ཆུ་བཙོལ་\selA{ཡང་}མ་རྙེད་}{chu \ul{b}tso\ul{l} \selA{yang} ma \ul{r}nye\ul{d}}{<<хотя искал воду, но не нашёл>>};
	\item \prfC{རིག་གནས་བསླབས་\selA{ཀྱང་}}{rig-\ul{g}na\ul{s} \ul{bs}lab\ul{s} \selA{kyang}}{<<хотя изучал науки, однако\ldots{}>>};
\end{prfsample}

б) сочетаясь с другими членами предложения, выступает в роли усилительной частицы <<также, даже, же>>, например:
\begin{prfsample}
	\item \prfC{གཅིག་\selA{ཀྱང་}མ་སྦྱིན་}{\ul{g}cig \selA{kyang} ma \ul{s}byin}{<<даже одного не дам>>};
	\item \prfC{ང་རྒྱལ་\selA{ཡང་}ཆེ་}{nga-\ul{r}gyal \selA{yang} che}{<<чванство также велико>>};
	\item \prfC{བློ\selA{འང}་རྣོ་}{\ul{b}lo\selA{'ang} \ul{r}no}{<<ум также остер>>};
	\item \prfC{ཅིག་ཤོས་\selA{ཀྱང་}སྨྲས་}{cig-sho\ul{s} \selA{kyang} \ul{s}mra\ul{s}}{<<другой же сказал>>};
	\item \prfC{རང་གིས་\selA{ཀྱང་}མནའ་བོར་}{rang-gi\ul{s} \selA{kyang} \ul{m}na' bor}{<<сам же дал клятву>>}.
\end{prfsample}

Употребление вариантов этой частицы зависит от приписных графем предыдущего слога. После приписных графем \prfB{ག}{ga}, \prfB{ད}{da}, \prfB{བ}{ba}, \prfB{ས}{sa} употребляется вариант \prfB{ཀྱང་}{kyang}; после приписных графем \prfB{ང}{nga}, \prfB{ན}{na}, \prfB{མ}{ma}, \prfB{ར}{ra}, \prfB{ལ}{la} --- вариант \prfB{ཡང་}{yang}; после приписной графемы \prfB{འ}{'a} или в открытом слоге могут употребляться варианты \prfB{ཡང་}{yang} или \prfB{འང་}{'ang}, причём при употреблении варианта он пишется слитно со словом, к которому относится, например: \prfC{བློ\selA{འང}་རྣོ་}{\ul{b}lo\selA{'ang} \ul{r}no}{<<ум также остер>>}.

\index{\prfB{གམ་}{gam}}\index{\prfB{ངམ་}{ngam}}\index{\prfB{དམ་}{dam}}\index{\prfB{ནམ་}{nam}}\index{\prfB{བམ་}{bam}}\index{\prfB{མམ་}{mam}}\index{\prfB{འམ་}{'am}}\index{\prfB{རམ་}{ram}}\index{\prfB{ལམ་}{lam}}\index{\prfB{སམ་}{sam}}\index{\prfB{ཏམ་}{tam}}III. \label{sec:ss:soed_protiv}\emph{Соединительно-противительная частица} --- существует в одиннадцати вариантах:
\prfB{གམ་}{gam}, \prfB{ངམ་}{ngam}, \prfB{དམ་}{dam},
\prfB{ནམ་}{nam}, \prfB{བམ་}{bam}, \prfB{མམ་}{mam},
\prfB{འམ་}{'am}, \prfB{རམ་}{ram}, \prfB{ལམ་}{lam},
\prfB{སམ་}{sam}, \prfB{ཏམ་}{tam}\footnote[49]{Национальные тибетские грамматики называют их общим термином --- \prfC{འབྱེད་སྡུད་}{'bye\ul{d}-\ul{s}du\ul{d}}{<<разделяющие и собирающие>>}.}.
Эти частицы-варианты выполняют следующие функции:

а) соединяют однородные члены предложения например:
\begin{prfsample}
	\item \prfC{བདེ་སྐྱིད་\selA{དམ་}སྡུག་བསྔལ་ནི་ལས་ཀྱི་འབྲས་བུའོ་}{\ul{b}de-\ul{s}kyi\ul{d} \selA{dam} \ul{s}dug-\ul{bs}nga\ul{l} ni la\ul{s}-kyi 'bra\ul{s}-bu'o}{<<блаженство и страдание --- результат кармы>>};
	\item \prfC{ཆུ་སྐྱུར་\selA{རམ་}ནེ་ཙོ་ནི་བྱའོ་}{chu-\ul{s}kyur \selA{ram} ne-tso ni bya'o}{<<чайка и попугай --- птицы>>};
\end{prfsample}

б) выступают в качестве противительного союза <<или>>, например:
\begin{prfsample}
	\item \prfC{ལམ་པ\selA{འམ}་སྣ་འདྲེན་པ་}{lam-pa\selA{'am} \ul{s}na-'dren-pa}{<<путник или проводник>>};
	\item \prfC{མཐོང་\selA{ངམ་}ཐོས་}{\ul{m}thong \selA{ngam} tho\ul{s}}{<<видел или слышал>>};
\end{prfsample}

в) принимают участие образовании альтернативного вопроса, например:
\begin{prfsample}
	\item \prfC{ཁྱོད་ལ་ནོར་ཡོད་\selA{དམ་}མེད་}{khyo\ul{d} la nor yo\ul{d} \selA{dam} me\ul{d}}{<<он богат или нет?>>};
	\item \prfC{ཤེས་\selA{སམ་}མི་ཤེས་}{she\ul{s} \selA{sam} mi she\ul{s}}{<<знает или не знает?>>};
\end{prfsample}

г) выступают в качестве вопросительной частицы, сопоставимой с русской частице <<ли>>, например:
\begin{prfsample}
	\item \prfC{ཡོད་\selA{དམ་}}{yod \selA{dam}}{<<имеется ли?>>};
	\item \prfC{མཐོང་\selA{ངམ་}}{\ul{m}thong \selA{ngam}}{<<видел ли?>>};
\end{prfsample}

Употребление того или иного варианта частицы зависит от приписной графемы предыдущего слога: после приписной \prfB{ག}{ga} употребляется \prfB{གམ་}{gam}, после приписной \prfB{ད}{da} употребляется \prfB{དམ་}{dam} и т.п.

Если предыдущий слог не имеет приписной графемы, то употребляется вариант \prfB{འམ་}{'am}, который пишется слитно с предыдущим слогом, например: \prfC{ལྷོའམ་ནུབ་}{lho'am nub}{<<юг или запад>>}.

\index{\prfB{གོ་}{go}}\index{\prfB{ངོ་}{ngo}}\index{\prfB{དོ་}{do}}\index{\prfB{ནོ་}{no}}\index{\prfB{བོ་}{bo}}\index{\prfB{མོ་}{mo}}\index{\prfB{འོ་}{'o}}\index{\prfB{རོ་}{ro}}\index{\prfB{ལོ་}{lo}}\index{\prfB{སོ་}{so}}\index{\prfB{ཏོ་}{to}}IV. \emph{Конечная частица} --- существует в одиннадцати вариантах:
\prfB{གོ་}{go}, \prfB{ངོ་}{ngo}, \prfB{དོ་}{do},
\prfB{ནོ་}{no}, \prfB{བོ་}{bo}, \prfB{མོ་}{mo},
\prfB{འོ་}{'o}, \prfB{རོ་}{ro}, \prfB{ལོ་}{lo},
\prfB{སོ་}{so}, \prfB{ཏོ་}{to}.

Варианты конечной частицы ставятся в конце последнего предложения (если оно сложное) и указывают на то, что данное предложение в целом полностью закончено.

Употребление того или иного варианта частицы зависит только от конечной согласной предыдущего слога. При открытом слоге употребляется частица-вариант \prfB{འོ་}{'o}, которая пишется слитно с предыдущим слогом. В новом письменном языке конечная частица не употребляется.

\index{\prfB{ནི་}{ni}}V. \emph{Выделительная частица} \prfB{ནི་}{ni} --- выполняет следующие грамматические функции:

а) выделяет, подчеркивает слово (группу слов), особенно подлежащее, например:
\begin{prfsample}
	\item \prfC{ཟང་ཟིང་གི་སྦྱིན་པ་\selA{ནི་}ཆོས་ཀྱི་སྦྱིན་པ་ལས་གཞན་ནོ་}{zang-zing-gi \ul{s}byin-pa \selA{ni} cho\ul{s}-kyi \ul{s}byin-pa la\ul{s} \ul{g}zhan no}{<<даяния же материальные не равнозначны даяниям духовным>>};
	\item \prfC{འདི་\selA{ནི་}བཟོ་ལ་ལྷག་པར་མཁས་པ་ཡིན་}{'di \selA{ni} \ul{b}zo-la lhag-par \ul{m}kha\ul{s}-pa-yin}{<<именно этот искуснее в мастерстве>>};
	\item \prfC{བདག་གིས་ནི་ཁོང་སྐྱེངས་པར་བྱ་}{\ul{b}dag-gi\ul{s} \selA{ni} khong \ul{s}kyeng\ul{s}-par-bya}{<<я же его пристыжу>>};
\end{prfsample}

б) указывает на то, что последующее разъясняет предыдущее, например:
\begin{prfsample}
	\item \prfC{དུས་\selA{ནི་}འདས་པ་དང། མ་འོངས་པ་དང་ད་ལྟ་བ་}{du\ul{s} \selA{ni} 'da\ul{s}-pa dang/ ma-'ong\ul{s}-pa dang da-\ul{l}ta-ba}{<<времена: прошедшее, настоящее и будущее>>};
\end{prfsample}

Кроме того, частица \prfB{ནི་}{ni} часто употребляется в поэзии, как пустой слог, для того чтобы получить необходимое количество слогов в строке.

\index{\prfB{ཅིག་}{cig}}\index{\prfB{ཞིག་}{zhig}}\index{\prfB{ཤིག་}{shig}}\index{\prfB{དེ་}{de}}VI. \emph{Частица неопределённости} (существует в трёх вариантах: \prfB{ཅིག་}{cig}, \prfB{ཞིག་}{zhig} и \prfB{ཤིག་}{shig}) и \emph{частица определённости} (\prfB{དེ་}{de}).

В тибетском языке, когда речь о предмете идет впервые, то слово, называющее этот предмет, часто оформляется частицей неопределённости, которую иной раз можно переводить как <<некий, некая, некто>>, например:
\begin{prfsample}
	\item \prfC{ཡུལ་\selA{ཞིག་}ན་མི་དབུལ་པོ་ཞིག་ཡོད་}{yul \selA{zhig}-na mi \ul{d}bu\ul{l}-po zhig yo\ul{d}}{<<в некой местности жил бедняк>>};
	\item \prfC{\selA{དེ་}ན་ཆུ་བོ་གཏིང་ཟབ་པོ་ཞིང་ཡོད་}{\selA{de}-na chu-bo \ul{g}ting-zab-po zhing yo\ul{d}}{<<там имелась глубокая река>>};
\end{prfsample}

Частица неопределённости, видимо, происходит от числительного \prfC{གཅིག་}{\ul{g}cig}{<<один>>}. Она напоминает английский неопределённый артикль, но отличается от него тем, что: а) употребляется не столь регулярно, как артикль в английском языке; б) может употребляться и тогда, когда речь идет и о множестве однородных предметов, например:
\begin{prfsample}
	\item \prfC{མི་མང་པོ་\selA{ཞིག་}}{mi mang-po \selA{zhig}}{<<много неких людей>>}.
\end{prfsample}

\index{\prfB{དེ་}{de}}Когда речь о данном предмете идет повторно, то слово, выражающее этот предмет, может оформляться частицей определённости \prfB{དེ་}{de} --- грамматикализованным указательным местоимением \prfC{དེ་}{de}{<<тот, то>>}. Главная функция частицы определённости и состоит в указании на то, что речь об этом предмете уже шла, например:
\begin{prfsample}
	\item \prfC{དེས་བ་གླང་\selA{ཞིག་}བརྙས་ཏེ། ཉིན་པར་སྤྱད་ནས་བ་གླང་\selA{དེ་}ཁྲིད་}{de{s} ba-{g}lang \selA{zhig} {br}nya{s}-te/ nyin-par {s}pya{d}-na{s} ba-{g}lang \selA{de} khri{d}\ldots{}}{<<он одолжил вола и, пропахав день, привел вола\ldots{}>>}.
\end{prfsample}
На русский язык переводить эту частицу нет необходимости.

\index{\prfB{ནས་}{na\ul{s}}}VII. \emph{Послелоги} --- представляют собой грамматикализовавшиеся или грамматикализующиеся синтаксические образования, состоящие из существительного и косвенной падежной частицы или (значительно реже) существительного и исходной падежной частицы \prfB{ནས་}{na\ul{s}}. Поскольку первая часть этого синтаксического образования представляет собой существительное, то послелог может соединяться со словам (конструкцией), к которому он относится, при помощи вариантов притяжательной падежной частицы.

Предложные конструкции в тибетском языке главным образом выражают пространственные отношения (местонахождение предмета, его положение по отношению к другим предметам и т.п.). Эти отношения выражаются при помощи следующих послелогов:
\begin{prfsample}
	\item \prfC{རྒྱབ་ཏུ་}{\ul{r}gyab-tu}{<<сзади>>};
	\item \prfC{ནང་དུ་}{nang-du}{<<в, внутри>>};
	\item \prfC{རྗེས་སུ་}{\ul{r}je\ul{s}-su}{<<позади>>};
	\item \prfC{ཁྲོད་དུ་}{khro\ul{d}-du}{<<среди>>};
	\item \prfB{མདན་དུ་}{\ul{m}dan-du}, \prfB{དྲང་དུ་}{drang-du}, \prfC{གམ་ན་}{gam-na}{<<вблизи, около, перед, у>>};
	\item \prfC{སྟེན་ན་}{\ul{s}ten-na}{<<наверху>>};
	\item \prfC{ཐོག་ཏུ་}{thog-tu}{<<в, на>>};
	\item \prfC{བར་དུ་}{bar-du}{<<между>>};
	\item \prfC{འོག་ཏུ་}{'og-tu}{<<под, внизу>>};
	\item \prfC{འོག་ནས་}{'og-na\ul{s}}{<<из, под>>};
	\item \prfC{ཐད་དུ་}{tha\ul{d}-du}{<<к, в направлении, в области>>};
\end{prfsample}

Послелоги \prfB{སླད་དུ་}{\ul{s}la\ul{d}-du}, \prfC{ཆེད་དུ་}{che\ul{d}-du}{<<ради, для>>} указывают на цель, например:
\begin{prfsample}
	\item \prfC{བོད་གསར་འཛུགས་སྒྲེན་གྱི་\selA{ཆེད་དུ་}}{bo\ul{d}-\ul{g}sar 'dzug\ul{s}-\ul{s}gren-gyi \selA{che\ul{d}-du}}{<<для строительства нового Тибета>>}.
\end{prfsample}

Послелоги \prfB{རིང་ལ་}{ring-la} и \prfB{རྗེས་སུ་}{\ul{r}je\ul{s}-su} указывают на промежутки времени или процессы, длящиеся во времени, например:
\begin{prfsample}
	\item \prfC{ལོ་གཅིག་ཙམ་གྱི་\selA{རིང་ལ་}}{lo \ul{g}cig-tsam-gyi \selA{ring-la}}{<<почти за год>>};
	\item \prfC{ལོ་གསུམ་གྱི་\selA{རྗེས་སུ་}}{lo \ul{g}sum-gyi \selA{\ul{r}je\ul{s}-su}}{<<по истечении трёх лет>>}.
\end{prfsample}

Послелог \prfC{སྐོར་ལ་}{\ul{s}kor-la}{<<относительно, о>>} указывает на предмет речи, например:
\begin{prfsample}
	\item \prfC{གོ་སྒྲིག་བྱེད་པའི་\selA{སྐོར་ལ་}ཞིབ་སྡུར་བྱས་}{go-\ul{s}grig-bye\ul{d}-pa'i \selA{\ul{s}kor-la} zhib-\ul{s}dur-bya\ul{s}}{<<обсудили (вопрос) о планировании>>}.
\end{prfsample}

В новом письменном языке косвенная падежная частица, входящая в состав послелога, может иногда опускаться, например:
\begin{prfsample}
	\item \prfC{འཚམས་དྲི་གནང་བའི་\selA{སྐོར་}གསུངས་}{'tsham\ul{s}-dri-\ul{g}nang-ba'i \selA{\ul{s}kor} \ul{g}sung\ul{s}}{<<рассказал о сделанном визите>>};
	\item \prfC{ལོ་མང་པོའི་\selA{རིང་}}{lo mang-po'i \selA{ring}}{<<в течение многих лет>>}.
\end{prfsample}

\index{\prfB{སྒོ་ནས་}{\ul{s}go-na\ul{s}}}\index{\prfB{ཐོག་ནས་}{thog-na\ul{s}}}\index{\prfB{ངང་ནས་}{ngang-na\ul{s}}}VIII. \emph{Служебные слова, образующие наречные конструкции}. В новом письменном языке распространены наречные конструкции, образованные от существительного или существительного с определением при помощи служебных слов \prfB{སྒོ་ནས་}{\ul{s}go-na\ul{s}}, \prfB{ཐོག་ནས་}{thog-na\ul{s}}, \prfB{ངང་ནས་}{ngang-na\ul{s}}.
Эти служебные слова соединяются с существительным посредством одного из вариантов притяжательной падежной частицы, например:
\begin{prfsample}
	\item \prfC{སྲ་བརྟན་}{sra-\ul{br}tan}{<<твёрдость>>} --- \prfC{སྲ་བརྟན་ཀྱི་\selA{སྒོ་ནས་}}{sra-\ul{br}tan-kyi-\selA{\ul{s}go-na\ul{s}}}{<<твёрдо, решительно>>};
	\item \prfC{རྒྱལ་ཁ་}{\ul{r}gyal-kha}{<<победа>>} --- \prfC{རྒྱལ་ཁ་\selA{ཐོག་ནས་}}{\ul{r}gyal-kha-\selA{thog-na\ul{s}}}{<<победоносно>>};
	\item \prfC{ངར་སྒྲ་སྒྲོག་པ་}{ngar-\ul{s}gra-\ul{s}grog-pa}{<<рычание>>} --- \prfC{ངར་སྒྲ་སྒྲོག་པའི་\selA{ངང་ནས་}}{ngar-\ul{s}gra-\ul{s}grog-pa'i-\selA{ngang-na\ul{s}}}{<<рыча>>}.
\end{prfsample}
В старом письменном языке конструкции такого рода встречаются значительно реже, причём для их образования используется только служебное слово \prfB{སྒོ་ནས་}{\ul{s}go-na\ul{s}}.

\index{\prfB{སོགས་}{sog\ul{s}}}\index{\prfB{བཅས་}{\ul{b}ca\ul{s}}}IX. \emph{Служебные слова, завершающие перечисление}. В тибетском языке перечисление обычно завершается служебными словами \prfB{སོགས་}{sog\ul{s}} и \prfB{བཅས་}{\ul{b}ca\ul{s}}, например:
\begin{prfsample}
	\item \prfC{བཟོ་ལས་དང་ཞིང་ལས།འབྲག་ལས་\selA{སོགས་}}{\ul{b}zo-la\ul{s} dang zhing-la\ul{s}/ 'brag-la\ul{s} \selA{sog\ul{s}}}{<<промышленность, земледелие и скотоводство>>};
	\item \prfC{ཆབ་སྲིད་དང།དཕལ་འབྱོར་རིག་གཞུང་\selA{བཅས་}}{chab-\ul{s}ri\ul{d} dang/ \ul{d}pha\ul{l}-'byor rig-\ul{g}zhung \selA{\ul{b}ca\ul{s}}}{<<политика, экономика и культура>>}.
\end{prfsample}
Вместе с тем эти служебные слова могут указывать на то, что речь идет об однородных с упомянутым предметах, выступая в значении <<и тому подобные>>, <<и так далее>>, например:
\begin{prfsample}
	\item \prfC{གསེར་\selA{སོགས་}}{\ul{g}ser-\selA{sog\ul{s}}}{<<золото и подобные [металлы]>>}.
\end{prfsample}

\index{\prfB{ཨེ་}{e}}X. \emph{Вопросительная частица} \prfB{ཨེ་}{e} --- ставится перед сказуемым, соотносима с частицей русского языка <<ли>>, например:
\begin{prfsample}
	\item \prfC{\selA{ཨེ་}གོ་}{\selA{e}-go}{<<понял ли?>>};
	\item \prfC{\selA{ཨེ་}ཡོད་}{\selA{e}-yo\ul{d}}{<<имеешь ли?>>}.
\end{prfsample}

\index{\prfB{མ་}{ma}}\index{\prfB{མི་}{mi}}XI. \emph{Частица отрицания}. В тибетском языке имеются две частицы отрицания --- \prfB{མ་}{ma} и \prfB{མི་}{mi}. Эти частицы ставятся непосредственно перед сказуемым, выраженным глаголом, а если имеется модальный или вспомогательный глагол, то перед ними, например:
\begin{prfsample}
	\item \prfC{སྒྲག་\selA{མི་}དགོས་}{\ul{s}grag \selA{mi} \ul{d}go\ul{s}}{<<не следует бояться>>};
	\item \prfC{ཁོ་བལྟ་གི་\selA{མ་}རེད་}{kho \ul{bl}ta-gi-\selA{ma}-re\ul{d}}{<<он не увидит>>}.
\end{prfsample}

XII. \emph{Союзы}. В тибетском языке можно выделить следующие группы союзов:

\index{\prfB{དང་}{dang}}1. Соединительный союз \prfC{དང་}{dang}{<<и, с>>} соединяет как однородные члены предложения, так и предложения, входящие в состав сложносочинённого предложения, например:
\begin{prfsample}
	\item \prfC{ཉི་མ་\selA{དང་}ཟླ་བ་}{nyi-ma \selA{dang} zla-ba}{<<солнце и луна>>};
	\item \prfC{ཉམས་མྱོང་བཟང་པོ་རྣམས་མཐོང་དགོས་པ་\selA{དང}།ཉམས་མྱོང་རྣམས་ལ་སྤྱི་བསྡོམས་བྱེད་}{nyam\ul{s}-myong-\ul{b}zang-po-\ul{r}nam\ul{s} \ul{m}thong \ul{d}go\ul{s}-pa \selA{dang}/ nyam\ul{s}-myong-\ul{r}nam\ul{s}-la \ul{s}pyi \ul{bs}dom\ul{s}-bye\ul{d}}{<<необходимо видеть положительный опыт и осуществлять обобщение этого опыта>>}.
\end{prfsample}
Союз \prfB{དང་}{dang} часто соединяет предложения, между которыми существует причинная или временная связь, например:
\begin{prfsample}
	\item \prfC{སྨན་ཟོས་པ་\selA{དང}།ནད་སོས་སོ་}{\ul{s}man zo\ul{s}-pa \selA{dang}/ na\ul{d} so\ul{s} so}{<<принял лекарство и (поэтому) болезнь прошла>>};
	\item \prfC{ཉི་མ་ཤར་བ་\selA{དང}།སློབ་གྲྭ་ཚུགས་}{nyi-ma shar-ba \selA{dang}/ \ul{s}lob-gr\ul{w}a tshug\ul{s}}{<<солнце всходит и (тогда) начинаются занятия в школе>>}.
\end{prfsample}

\index{\prfB{མ་གཏོགས་}{ma-\ul{g}tog\ul{s}}}\index{\prfB{མ་ཟད་}{ma-za\ul{d}}}\index{\prfB{ཁར་}{khar}}2. Присоединительные союзы \prfB{མ་གཏོགས་}{ma-\ul{g}tog\ul{s}}, \prfB{མ་ཟད་}{ma-za\ul{d}}, \prfC{ཁར་}{khar}{<<не только, но; кроме того; за исключением того>>} --- указывают, что второе предложение содержит какую-то добавочную информацию к тому, что сказано в предыдущем предложении, например:
\begin{prfsample}
	\item \prfC{ཆུ་མཛོད་ཁག་བཅུ་དྲུག་བཟོས་ཡོད་པ་\selA{མ་ཟད་}}{chu-\ul{m}dzo\ul{d}-khag \ul{b}cu-drug \ul{b}zo\ul{s} yo\ul{d}-pa-\selA{ma-za\ul{d}}\ldots{}}{<<не только построили шестьдесят водохранилищ, но \ldots{}>>}.
\end{prfsample}
(в старом письменном языке союзы \prfB{མ་ཟད་}{ma-za\ul{d}} и \prfB{ཁར་}{khar} не употребляются).

\index{\prfB{ན་}{na}}\index{\prfC{གལ་ཏེ་་་་ན་}{ga\ul{l}-te}}3. Условные союзы \prfB{ན་}{na} или \prfC{གལ་ཏེ་་་་ན་}{ga\ul{l}-te\ldots{}na}{<<если>>}. Когда служебное слово \prfB{ན་}{na} выступает как условный союз, то сказуемым придаточного предложения\footnote[50]{В тибетском языке придаточное предложение всегда предшествует главному.} всегда выступает основа глагола в прошедшем времени. В данном случае это указывает не на то, что действие совершилось, а на то, что оно должно совершиться прежде действия, о котором говорится в главном предложении, например:
\begin{prfsample}
	\item \prfC{བདག་གིས་ཁྲིམས་དང་འགལ་ཞིག་བྱས་\selA{ན་}}{\ul{b}dag-gi\ul{s} khrim\ul{s} dang 'ga\ul{l}-zhig bya\ul{s}-\selA{na}\ldots{}}{<<если я нарушу закон\ldots{}>>};
	\item \prfC{\selA{གལ་ཏེ་}སྨོན་པ་བཞིན་དུ་མ་བྱས་\selA{ན་}}{\selA{ga\ul{l}-te} \ul{s}mon-pa \ul{b}zhin-du ma bya\ul{s}-\selA{na}\ldots{}}{<<если не выполнить просьбы\ldots{}>>}.
\end{prfsample}

\index{\prfB{མོད་}{mo\ul{d}}}\index{\prfB{མོད་ཀྱང་}{mo\ul{d}-kyang}}\index{\prfB{ན་ཡང་}{na-yang}}4. Уступительные союзы: \prfB{མོད་}{mo\ul{d}}, \prfB{མོད་ཀྱང་}{mo\ul{d}-kyang}, \prfC{ན་ཡང་}{na-yang}{<<хотя>>}, например:
\begin{prfsample}
	\item \prfC{ས་ལ་ལྷུང་\selA{ན་ཡང་}}{sa-la lhung \selA{na-yang}\ldots{}}{<<хотя упал на землю\ldots{}>>}.
\end{prfsample}
Здесь союз \prfB{ན་ཡང་}{na-yang} не следует путать с вариантом косвенной падежной частицы \prfB{ན་}{na} в сочетании с уступительно-усилительной частицей \prfB{ཡང་}{yang}, когда они встречаются рядом в одном предложении, например:
\begin{prfsample}
	\item \prfC{བཙོམ་ལྡན་འདས་སྔོན་འདས་པའི་དུས་\selA{ན་ཡང་}}{\ul{b}tsom-\ul{l}dan-'da\ul{s} \ul{s}ngon 'da\ul{s}-pa'i du\ul{s} \selA{na-yang}\ldots{}}{<<Победоносный во время своих прежних существований также\ldots{}>>}
\end{prfsample}

\index{\prfB{སླད་དུ་}{\ul{s}la\ul{d}-du}}\index{\prfB{ཆེད་དུ་}{che\ul{d}-du}}\index{\prfB{ཕིར་}{phir}}5. Союзы цели: \prfB{སླད་དུ་}{\ul{s}la\ul{d}-du}, \prfB{ཆེད་དུ་}{che\ul{d}-du}, \prfC{ཕིར་}{phir}{<<ради, для>>}.

6. Союзы причины. В старом письменном языке в качестве союза причины широко используются варианты орудной падежной частицы (см. выше).

\index{\prfB{སྟབས་}{\ul{s}tab\ul{s}}}\index{\prfB{བརྟེན་}{\ul{br}ten}}В новом письменном языке употребляются союзы \prfC{སྟབས་}{\ul{s}tab\ul{s}}{<<так как>>}, \prfC{བརྟེན་}{\ul{br}ten}{<<вследствие>>}.

\index{\prfB{སྐབས་}{\ul{s}kab\ul{s}}}\index{\prfB{དུས་}{du\ul{s}}}\index{\prfB{ཚེ་}{tshe}}7. Временные союзы. В функции временных союзов широко используются существительные \prfB{སྐབས་}{\ul{s}kab\ul{s}}, \prfB{དུས་}{du\ul{s}}, \prfC{ཚེ་}{tshe}{<<время>>} (обычно с вариантом косвенной падежной частицы), например:
\begin{prfsample}
	\item \prfC{ང་ལོ་བཅུ་ལོན་\selA{སྐབས་}}{nga lo \ul{b}cu lon \selA{\ul{s}kab\ul{s}}\ldots{}}{<<когда мне исполнилось десять лет\ldots{}>>};
	\item \prfC{བུ་ཕྲུག་ཚོ་སློབ་གྲྭ་ལ་འགྲོ་\selA{དུས་སུ་}}{bu-phrug-tsho \ul{s}lob-gr\ul{w}a-la 'gro \selA{du\ul{s}-su}\ldots{}}{<<когда дети шли в школу\ldots{}>>}.
\end{prfsample}

\index{\prfB{ཀྱི་}{kyi}}\index{\prfB{གི་}{gi}}\index{\prfB{གྱི་}{gyi}}\index{\prfB{ཡི་}{yi}}\index{\prfB{འི་}{'i}}В старом письменном языке для присоединения придаточного предложения к главному могут использоваться частицы-варианты \prfB{ཀྱི་}{kyi}, \prfB{གི་}{gi}, \prfB{གྱི་}{gyi}, \prfB{ཡི་}{yi}, \prfB{འི་}{'i}. Эти частицы употребляются в значении:

а) противительных союзов, например:
\begin{prfsample}
	\item \prfC{འདི་ནི་མི་ཡིན་\selA{གྱི་}ལྷ་མ་ཡིན་}{'di ni mi yin \selA{gyi} lha ma yin}{<<это человек, но не дух>>};
\end{prfsample}

б) союзов причины, например:
\begin{prfsample}
	\item \prfC{ཤ་འདི་ལྷག་པར་ཞིམ་\selA{གྱི་}ཕྱིན་ཅད་ཤ་འདི་ལྟ་བུ་རྟག་ཏུ་སྦྱོར་}{sha 'di lhag-par zhim \selA{gyi} phyin-ca\ul{d} sha 'di \ul{l}ta-bu \ul{r}tag-tu \ul{s}byor}{<<так как это мясо очень вкусное, всегда готовь такое мясо>>}.
\end{prfsample}

\index{\prfB{ཏེ་}{te}}\index{\prfB{སྟེ་}{\ul{s}te}}\index{\prfB{དེ་}{de}}В значении соединительного и противительного союзов в старом письменном языке используются также частицы-варианты \prfB{ཏེ་}{te}, \prfB{སྟེ་}{\ul{s}te}, \prfB{དེ་}{de}\footnote[51]{Главной функцией этих частиц является образование незавершенной формы глагола (см. раздел \hyperref[sec:glagol]{<<Глагол>>}). Они также могут выступать как показатель того, что предыдущее высказывание уточняется или расшифровывается последующим, например:
\begin{prfsample}
	\item \prfC{དཀོན་མཆོག་གསུམ་\selA{སྟེ}།སངས་རྒྱས་ཆོས་དང་དགེ་འདུན་}{\ul{d}kon-\ul{m}chog \ul{g}sum \selA{\ul{s}te}/ sang\ul{s}-\ul{r}gya\ul{s} cho\ul{s} dang \ul{d}ge-'dun}{<<три драгоценности суть: Будда, Дхарма, Сангха>>.}
\end{prfsample}
}, например:
\begin{prfsample}
	\item \prfC{ཉི་མ་ནམ་མཁར་གནས་\selA{ཏེ་}ཟླ་བའམ་དེ་བཞིན་ནོ་}{nyi-ma nam-mkhar gnas-\selA{te} zla-ba'am de-bzhin no}{<<солнце находится на небе, и луна тоже находится там же>>};
	\item \prfC{ཤར་གྱི་ཕྱོགས་སུ་དགའ་ལྡན་\selA{ཏེ་}དེ་བཞིན་ལྷོ་རུ་བསམ་ཡས་སོ་}{shar-gyi phyog\ul{s}-su \ul{d}ga'-\ul{l}dan \selA{te} de-\ul{b}zhin lho-ru \ul{b}sam-ya\ul{s} so}{<<на востоке находится [монастырь] Галдан, а на юге [монастырь] Самье>>}.
\end{prfsample}

\index{\prfB{འདྲ་བ་}{'dra-ba}}\index{\prfB{ལྟ་བུ་}{\ul{l}ta-bu}}\index{\prfB{ལྟར་}{\ul{l}tar}}\index{\prfB{བཞིན་}{\ul{b}zhin}}XIII. \emph{Служебные слова уподобления}. Имеется четыре слова уподобления:
\prfB{འདྲ་བ་}{'dra-ba}, \prfB{ལྟ་བུ་}{\ul{l}ta-bu}, \prfB{ལྟར་}{\ul{l}tar}, \prfB{བཞིན་}{\ul{b}zhin}.

Служебные слова \prfB{འདྲ་བ་}{'dra-ba}, \prfB{ལྟ་བུ་}{\ul{l}ta-bu} (<<подобный, подобие, похожий>>) уподобляют друг другу различные существительные, например:
\begin{prfsample}
	\item \prfC{བྱིས་པ་ཁྲ་ཕྲག་\selA{འདྲ་བ་}}{byi\ul{s}-pa khra-phrag \selA{'dra-ba}}{<<юноша, подобный соколенку>>};
	\item \prfC{ཆུང་མ་ལྷ་མོ་\selA{ལྟ་བུ་}}{chung-ma lha-mo \selA{\ul{l}ta-bu}}{<<супруга, подобная богине>>};
\end{prfsample}

Служебные слова \prfB{ལྟར་}{\ul{l}tar}, \prfB{བཞིན་}{\ul{b}zhin} (<<как, подобно>>) указывают на сходство действий, процессов, качеств, например:
\begin{prfsample}
	\item \prfC{ཁྲས་བྱེའུ་ཆུང་བཟུང་བ་\selA{ལྟར་}བཟུང་}{khra\ul{s} bye'u-chung \ul{b}zung-ba \selA{\ul{l}tar} \ul{b}zung}{<<схватил, как хватает сокол пташку>>};
	\item \prfC{མཚོ་དེ་ནི་མེ་ལོང་\selA{ལྟར་}གསལ་}{\ul{m}tsho de ni me-long \selA{\ul{l}tar} \ul{g}sa\ul{l}}{<<это озеро блестит, как зеркало>>};
	\item \prfC{བཞིན་དམར་པོ་ཁྲག་\selA{ལྟར་}སོང་}{\ul{b}zhin \ul{d}mar-po khrag \selA{\ul{l}tar} song}{<<лицо покраснело, как кровь>>};
	\item \prfC{བཤད་པ་\selA{ལྟར་}བྱས་}{\ul{b}sha\ul{d}-pa \selA{\ul{l}tar} bya\ul{s}}{<<сделал, как было сказано>>};
	\item \prfC{རང་གི་སྙིང་\selA{བཞིན་}གཅེས་}{rang-gi \ul{s}nying \selA{\ul{b}zhin} \ul{g}ce\ul{s}}{<<любил, как собственную душу>>}.
\end{prfsample}

Служебные слова \prfB{ལྟར་}{\ul{l}tar} и \prfB{བཞིན་}{\ul{b}zhin} часто
употребляются в значении <<согласно, на основании, по>>, например:
\begin{prfsample}
	\item \prfC{བཀའ་\selA{བཞིན་}བསྒྲུབས་}{\ul{b}ka' \selA{\ul{b}zhin} \ul{bs}grub\ul{s}}{<<исполнил согласно приказу>>};
	\item \prfC{ཚོད་རྩིས་བྱས་པ་\selA{ལྟར་}}{tsho\ul{d}-\ul{r}tsi\ul{s}-bya\ul{s}-pa \selA{\ul{l}tar}}{<<по предварительным подсчётам>>}.
\end{prfsample}
