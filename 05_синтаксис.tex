\chapter{Синтаксис}

\section{Словосочетание}

В тибетском языке можно выделить четыре основных типа словосочетаний: 1) копулятивные; 2) атрибутивные или определительные; 3) обстоятельственные (их можно рассматривать как разновидность определительных); 4) глагольно-объектные.

Связь между словами в словосочетаниях может выражаться при помощи: а) порядка слов, т.е. определенной, твердо установленной позиции одного слова относительно другого; б) порядка слов плюс соответствующее служебное слово.

1. \emph{Копулятивные словосочетания} образуются частями речи, равноценными семантически и грамматически, --- чаще всего существительными и местоимениями, реже --- прилагательными, глаголами и числительными.

Существительные, образующие копулятивные словосочетания, могут соединяться как при помощи соединительного союза \prfB{དང་}{dang}, так и без него, например:
\begin{prfsample}
	\item \prfC{ཆབ་སྲིད་དང་དཔལ་འབྱོར་}{chab-s\ul{r}i\ul{d} dang \ul{d}pa\ul{l}-'byor}{<<экономика и политика>>};
	\item \prfC{མེ་འཁོར། གྲུ་གཟིངས་}{me-'khor, gru-\ul{g}zing\ul{s} }{<<поезд [и] пароход>>}.
\end{prfsample}
Копулятивные словосочетания образуются и от существительных, соединенных соединительно-противительными частицами (см. страницу 132).

Копулятивные словосочетания, образуемые местоимениями, требуют соединительного союза, который может стоять как между местоимениями, так и после них, например:
\begin{prfsample}
	\item \prfB{ང་དང་ཁོང་}{nga dang khong} или
	\item \prfC{ང་ཁོང་དང་}{nga khong dang}{<<я и он>>}.
\end{prfsample}

Имена прилагательные и глаголы в копулятивных словосочетаниях разделяются служебными словами \prfB{ལ་}{la} и \prfB{ཅིང་}{cing}, \prfB{ཞིང་}{zhing}, \prfB{ཤིང་}{shing} (последние три --- частицы-варианты), например:
\begin{prfsample}
	\item \prfC{ཟབ་ཅིང་རྒྱ་ཆེ་}{zab-cing \ul{r}gya-che}{<<глубокий и широкий>>};
	\item \prfC{སྡུག་ཅིང་བྱམས་}{\ul{s}dug-cing byam\ul{s}}{<<любил и лелеял>>};
	\item \prfC{སོང་ལ་བསྟོས་}{song-la \ul{bs}to\ul{s}}{<<пришел и посмотрел>>}.
\end{prfsample}

Копулятивные словосочетания числительных служебными словами не разделяются, например:
\begin{prfsample}
	\item \prfC{བཞི་ལྔ་}{\ul{b}zhi \ul{l}nga}{<<четыре-пять>>}.
\end{prfsample}

2. Атрибутивные словосочетания наиболее распространены. Они выражают принадлежность предмета (например: \prfC{ངའི་དཔེ་ཆ་}{nga'i \ul{d}pe-cha}{<<моя книга>>}),
качественный признак предмета (например: \prfC{མེ་ཏོག་དམར་པོ་}{me-tog \ul{d}mar-po}{<<красный цветок>>}),
характеристику предмета по совершаемому им действию (например: \prfC{འགྲོ་བའི་མི་}{'gro-ba'i mi}{<<идущий человек>>}), а также многие другие семантические отношения.

Определяемым словом в этих словосочетаниях всегда является существительное. В качестве определений могут выступать знаменательные слова разных лексико-грамматических категорий, а также целые предложения. В зависимости от своей принадлежности к тому или иному классу слов определение может стоять в препозиции или постпозиции к определяемому. В препозиции к определяемому всегда находятся личные местоимения и существительные, которые принимают после себя служебное слово --- один из вариантов притяжательной падежной частицы (например:
\prfC{ཁོང་གི་ཕ་མ་}{khong-gi pha-ma}{<<его родители>>};
\prfC{ཤར་ཕྱོགས་ཀྱི་མི་དམངས་ཚོ་}{shar-phyog\ul{s}-kyi mi-\ul{d}mang\ul{s}-tsho}{<<народы Востока>>}). Если определениями одновременно выступают личное местоимение и существительное, то личное местоимение всегда стоит на первом месте (например:
\prfC{ངའི་གྲོགས་པའི་ཆུང་མ་}{nga'i grog\ul{s}-pa'i chung-ma}{<<жена моего друга>>}). Если существительное, выступающее в качестве определения, образует с определяемым устойчивое словосочетание, то оно может не принимать после себя служебное слово (например:
\prfC{དམངས་སྤྱོད་གྲུ་གཟིངས་}{\ul{d}mang\ul{s}-\ul{s}pyo\ul{d} gru-\ul{g}zing\ul{s}}{<<пассажирский пароход>>};
\prfC{སྨན་བཅོས་ཡོད་བྱད་}{\ul{s}man-\ul{b}co\ul{s} yo\ul{d}-bya\ul{d}}{<<медицинское оборудование>>}).

В постпозиции к определяемому всегда находятся имена прилагательные\footnote[52]{Исключение составляет имя прилагательное, которое, приняв притяжательную падежную частицу, может стать в препозиции к определяемому.} и числительные. Определение, находящееся в постпозиции, служебных слов не требует. Если имеется несколько постпозиционных определений, то к определяемому примыкает имя прилагательное, а цепочку определений замыкает числительное (например: \prfC{མི་ཡག་པོ་གསུམ་}{mi yag-po \ul{g}sum}{<<три хороших человека>>}).

Как в постпозиции, так и в препозиции к определяемому могут находиться имена-атрибутивы, отглагольные имена и указательные местоимения. При этом имена-атрибутивы и отглагольные имена, находясь в препозиции к определяемому, принимают притяжательную падежную частицу.

Если определение выражено целым предложением, то оно всегда стоит перед определяемым, присоединяясь к нему притяжательной падежной частицей, например: \prfC{ངས་བཤད་པའི་སྐད་ཆ་}{nga\ul{s} \ul{b}sha\ul{d}-pa'i \ul{s}ka\ul{d}-cha}{<<мной сказанные слова>>}.

3. \emph{Обстоятельственные словосочетания} выражают качественную, пространственную, временную и т.п. характеристику действия. Определяемый член в этих словосочетаниях --- глагол. Определяющий член обстоятельственного словосочетания может быть выражен наречием, а также существительным или прилагательным, обязательно принимающими после себя служебные слова. Существительное, приняв после себя косвенную падежную частицу, может выражать пространственную и временную характеристику действия (состояния), например:
\begin{prfsample}
	\item \prfC{ལྷ་ས་ལ་འགྲོ་}{lha-sa-la 'gro}{<<иду в Лхасу>>};
	\item \prfC{དབྱར་དུས་མེ་ཏོག་མང་པོ་ཡོད་}{\ul{d}byar-du\ul{s} me-tog mang-po yo\ul{d}}{<<летом имеется много цветов>>}.
\end{prfsample}
Приняв после себя орудную падежную частицу или служебные слова \prfB{སྒོ་ནས་}{\ul{s}go-na\ul{s}}, \prfB{ངང་ནས་}{ngang-na\ul{s}}, \prfB{ཐོག་ནས་}{thog-na\ul{s}}, существительное может выражать качественную характеристику действия, например:
\begin{prfsample}
	\item \prfC{ཧུར་བརྩེན་གྱིས་ལས་ཀ་ལས་}{hur-\ul{br}tsen-gyi\ul{s} la\ul{s}-ka la\ul{s}}{<<активно работать>>};
	\item \prfC{སྲ་བརྟན་ཀྱི་སྒོ་ནས་འཐབ་རྩོད་བྱེད་}{sra-\ul{br}tan-kyi \ul{s}go-na\ul{s} 'thab-\ul{r}tso\ul{d}-bye\ul{d}}{<<решительно бороться>>}.
\end{prfsample}

В отличие от атрибутивных словосочетаний, где определение непосредственно или со служебным словом примыкает к определяемому, между обстоятельственным членом и глаголом могут находиться другие члены предложения. Исключение составляют словосочетания, в которых обстоятельственные члены образованы от прилагательных прибавлением к ним варианта косвенной падежной частицы \prfB{ར་}{ra}, например:	
\begin{prfsample}
	\item \prfC{ཡག་པོ་}{yag-po}{<<хороший>>} --- \prfC{ཡག་པོར་}{yag-por}{<<хорошо>>};
	\item \prfC{མགྱོགས་པོ་}{\ul{m}gyog\ul{s}-po}{<<быстрый>>} --- \prfC{མགྱོགས་པོར་}{\ul{m}gyog\ul{s}-por}{<<быстро>>}.
\end{prfsample}
Такого рода обстоятельственные члены обязательно стоят перед глаголом, допуская между собой и глаголом только частицу отрицания. Если подобные обстоятельственные члены выступают определением к сложному глаголу, то они помещаются перед его глагольным компонентом, например:
\begin{prfsample}
	\item \prfC{}{\ul{s}lob-\ul{s}byong-bye\ul{d}}{<<учиться>>} --- \prfC{}{\ul{s}lob-\ul{s}byong-yag-por-bye\ul{d}}{<<хорошо учиться>>}.
\end{prfsample}
Никакие другие обстоятельственные члены сложный глагол разорвать не могут.

4. \emph{Глагольно-объектные словосочетания} образуются путем сочетания глагола с существительным или местоимением. В тибетском языке вычленяются два подтипа таких сочетаний:

а) сочетание глагола с прямым объектом. В этом случае имя всегда выступает в своей исходной форме и (если оно не инверсировано) стоит перед глаголом, например:
\begin{prfsample}
	\item \prfC{ཡི་གེ་འབྲི་}{yi-ge 'bri}{<<писать письмо>>};
	\item \prfC{ཁ་ལག་ཟ་}{kha-lag za}{<<есть ищу>>};
\end{prfsample}
6) сочетание глагола с косвенным объектом, куда можно отнести и объекты места совершения (или направления) действия. В этом случае объект обязательно принимает после себя один из вариантов косвенной падежной частицы, например:
\begin{prfsample}
	\item \prfC{ཕ་མ་ལ་འབྲིས་}{pha-ma-la 'bri\ul{s}}{<<писал родителям>>};
	\item \prfC{ཁོང་ལ་བཤད་}{khong-la \ul{b}sha\ul{d}}{<<сказал ему>>};
\end{prfsample}

\section{Предложение}

Смысловая и грамматическая связь между словами (группами слов) тибетского предложения, как и в словосочетании, выражается, как правило, порядком слов (групп слов) и служебными словами. В том случае, когда налицо простое нераспространенное предложение, где в качестве сказуемого выступает непереходный глагол или сказуемое выражено именем со связкой, связь между подлежащим и сказуемым осуществляется только порядком слов, например:
\begin{prfsample}
	\item \prfC{མི་འགྲོ་}{mi-'gro}{<<человек идет>>};
	\item \prfC{མེ་ཏོག་དམར་པོའོ་}{me-tog \ul{d}mar-po'o}{<<цветок красный>>}.
\end{prfsample}
Перестановка в этом случае приведет к бессмыслице.

\subsection{Члены предложения}

Поскольку порядок слов в тибетском предложении фиксирован относительно не строго (твердо фиксировано только место сказуемого, которое всегда находится на конце предложения), то в нем значительную роль играют служебные слова.

Формулу простого нераспространенного предложения можно записать как \emph{П + С}, формулу простого распространенного предложения --- \emph{Ов + Ом + П + Дк + Д + С}, где:
\emph{Ов} --- обстоятельство времени,
\emph{Ом} --- обстоятельство места,
\emph{П} --- подлежащее;
\emph{Дк} --- косвенное дополнение,
\emph{Д} --- прямое дополнение,
\emph{С} --- сказуемое.

В предложении выделяются главные и второстепенные члены.

\emph{Главные члены предложения}. К ним относятся сказуемое (1) и подлежащее (2).

1. Сказуемое --- выражается либо глаголом в его простой или сложной форме, либо именем (существительным или прилагательным) в сочетании с глаголом-связкой. В старом письменном языке глагол-связка может опускаться, например:
\prfC{འདི་ཉའོ་}{'di nya'o}{<<это рыба>>}.
Кроме того, в качестве сказуемого может выступать отглагольное имя со значением процесса действия --- чаще всего: а) в придаточных предложениях (см. подраздел <<Сложное предложение>>); б) в названиях книг, глав, заголовках статей, надписях
под иллюстрациями, например:
\begin{prfsample}
	\item \prfC{བོད་ཀྱི་གནམ་གཤིས་བརྟག་དཔྱོད་ལས་དོན་ལ་གྲུབ་འབྲས་ཐོབ་པ་}{bo\ul{d}-kyi \ul{g}nam-\ul{g}shi\ul{s} \ul{br}tag-\ul{d}pyo\ul{d}-la\ul{s}-don-la grub-'bra\ul{s} thob-pa}{<<метеослужба в Тибете добилась успехов>>}.
\end{prfsample}
Здесь в роли сказуемого выступают отглагольное имя \prfB{ཐོབ་པ་}{thob-pa}.

2. Подлежащее чаше всего выражается существительным, местоимением, реже --- именем-атрибутивом и отглагольным именем. Крайне редко в роли подлежащего могут выступать числительное или прилагательное, например:
\begin{prfsample}
	\item \prfC{གསུམ་ནི་གཅིག་གི་སྣང་ལ་གཉིས་བརྒྱབ་པ་རེད་}{\ul{g}sum ni \ul{g}cig-gi \ul{s}nang-la gnyi\ul{s} \ul{br}gyab-pa-re\ul{d}}{<<три суть сумма одного и двух>>};
	\item \prfC{སེར་པོ་ལས་ལི་ཝང་དཀར་གི་རེད་}{ser-po la\ul{s} li-wang \ul{d}kar-gi-re\ul{d}}{<<серый [цвет] светлее, чем желтый>>}.
\end{prfsample}
Подлежащее может быть выражено также целым предложением. Место подлежащего в составе предложения не является строго фиксированным, обычно подлежащее следует за обстоятельством места или времени, которые чаще всего находятся в начале предложения. Твердо фиксируется место подлежащего только относительно члена предложения, обозначающего орудие совершения действия, так как и то и другое оформляется одинаковой служебной частицей. В этом случае подлежащее всегда стоит на первом месте, например:
\begin{prfsample}
	\item \prfC{ངས་སྨྱུ་གུས་འབྲི་}{nga\ul{s} \ul{s}myu-gu\ul{s} 'bri}{<<я пишу ручкой>>}.
\end{prfsample}

В зависимости от характера глагола, выступающего в качестве сказуемого, имя, играющее роль подлежащего, выступает либо в исходной форме, либо должно принять определенную служебную частицу.

Если сказуемое выражено непереходным глаголом, возвратным глаголом или глаголом-связкой в сочетании с именем, то подлежащее ничем не оформляется, например:
\begin{prfsample}
	\item \prfC{རྟ་རྒྱུགས་}{\ul{r}ta \ul{r}gyug\ul{s}}{<<лошадь бежит>>};
	\item \prfC{ཐག་ཆོད་}{thag cho\ul{d}}{<<верёвка рвется>>};
	\item \prfC{སློབ་དཔོན་བོད་པ་རེད་}{\ul{s}lob \ul{d}pon bo\ul{d}-pa re\ul{d}}{<<учитель --- тибетец>>}.
\end{prfsample}

Если сказуемое выражено переходным глаголом, то подлежащее должно быть оформлено орудной частицей, например:
\begin{prfsample}
	\item \prfC{མིས་འབྲི་}{mi\ul{s} 'bri}{<<человек пишет>>};
	\item \prfC{ཁོང་གིས་གཅོད་}{khong-gi\ul{s} \ul{g}co\ul{d}}{<<он рубит>>}.
\end{prfsample}

Если сказуемое выражено глаголами наличия или приобретения, то подлежащее оформляется косвенными падежными частицами \prfB{ལ་}{la} или \prfB{ན་}{na}, например:
\begin{prfsample}
	\item \prfC{གྲོགས་པོར་གྲུབ་འབྲས་ཐོབ་}{grog\ul{s}-po\ul{r} grub-'bra\ul{s} thob}{<<друг добился успехов>>};
	\item \prfC{ཕ་མ་ལ་ཡོད་}{pha-ma-la yo\ul{d}}{<<родители имеют>>};
	\item \prfC{བ་ལ་འབྲངས་}{ba-la 'brang\ul{s}}{<<корова отелилась>>};
\end{prfsample}

\emph{Второстепенные члены предложения}. К ним относятся дополнения (прямое и косвенное), обстоятельства.

Прямое дополнение никогда не оформляется служебным словом и всегда помещается перед сказуемым, например:
\begin{prfsample}
	\item \prfC{མིས་ཤིང་གཅོད་}{mi\ul{s} shing \ul{g}co\ul{d}}{<<человек рубит дерево>>}.
\end{prfsample}
Однако, если требуется сделать на прямом дополнении логическое ударение, то оно может быть инверсировано при помощи указательного местоимения \prfB{དེ་}{de}; ср., например:
\begin{prfsample}
	\item \prfC{ངས་ཡི་གེ་འབྲི་}{nga\ul{s} yi-ge 'bri}{<<я пишу письмо>>} (обычный порядок слов);
	\item \prfC{ཡི་གེ་དེ་ངས་འབྲི་}{yi-ge de nga\ul{s} 'bri}{<<это письмо пишу я>>} (инверсия).
\end{prfsample}

Остальные второстепенные члены предложения --- различные обстоятельства, косвенное дополнение --- не имеют строго фиксированного места в предложении. Обстоятельства места и времени обычно помещаются в начале предложения, обстоятельства образа действия тяготеют к сказуемому. Обстоятельства места, времени и косвенное дополнение всегда принимают косвенную падежную частицу.
Обстоятельства образа действия обычно оформлены орудной падежной частицей или служебными словами \prfB{སྒོ་ནས་}{\ul{s}go-na\ul{s}}, \prfB{ཐོག་ནས་}{thog-na\ul{s}}, \prfB{ངང་ནས་}{ngang-na\ul{s}} (последние два употребляются только в новом письменном языке).

\subsection{Типы предложений}

Все предложения по своей структуре делятся на простые и сложные.

I. \emph{Простые предложения}. Можно выделить четыре основных типа простых предложений:

1. \emph{Повествовательные предложения} передают сообщение без какой-либо эмоциональной окраски. Их можно подразделить на утвердительные и отрицательные. Отрицание того или иного факта действительности в тибетском языке выражается постановкой отрицательных частиц \prfB{མ་}{ma} или \prfB{མི་}{mi} перед сказуемым, например:
\begin{prfsample}
	\item \prfC{སྐྱོ་མི་ཤེས་}{\ul{s}kyo mi she\ul{s}}{<<не знать печали>>};
	\item \prfC{ལས་ཀ་མ་བྱེད་}{la\ul{s}-ka ma bye\ul{d}}{<<не делать дела>>}.
\end{prfsample}
При именных связочных сказуемых отрицание ставится перед связкой, причем перед связкой
\prfB{རེད་}{re\ul{d}} ставится отрицание \prfB{མ་}{ma} (например: \prfC{ཁོ་ཚོ་བོད་པ་མ་རེད་}{kho-tsho bo\ul{d}-pa ma re\ul{d}}{<<они не тибетцы>>}),
а перед связкой	\prfB{འདུག་}{'dug} --- отрицание \prfB{མི་}{mi} (например:
\prfC{ལོ་ཐོག་ཡག་པོ་མི་འདུག་}{lo-thog yag-po mi 'dug}{<<урожай не хорош>>}). Если в качестве связки выступают \prfB{ཡིན་}{yin} или \prfB{ཡོད་}{yo\ul{d}}, то отрицание обычно выражается не путем постановки отрицательных частиц, а через негативные формы этих связок, т.е. соответственно \prfB{མིན་}{min} и \prfB{མེད་}{me\ul{d}},
например:
\begin{prfsample}
	\item \prfC{ང་ཚོ་དགེ་ཕྲུག་མིན་}{nga-tsho \ul{d}ge-phrug min}{<<мы не ученики>>};
	\item \prfC{སྡོས་སའི་ས་ཆ་སྐྱིད་པོ་མེད་}{\ul{s}do\ul{s}-sa'i sa-cha \ul{s}kyi\ul{d}-po me\ul{d}}{<<местожительство неудобное>>}.
\end{prfsample}

В новом письменном языке, где конечный глагол принимает связки
\prfB{ཡིན་}{yin}, \prfB{ཡོད་}{yo\ul{d}}, \prfB{རེད་}{re\ul{d}}, \prfB{འདུག་}{'dug},
отрицание ставится перед ними, например:
\begin{prfsample}
	\item \prfC{ཁོང་གིས་དཔེ་ཆ་ཀློག་གི་མི་འདུག་}{khong-gi\ul{s} \ul{d}pe-cha \ul{k}log-gi-mi-'dug}{<<он не читает книгу>>}.
\end{prfsample}

2. \emph{Вопросительные предложения} --- образуются на базе повествовательных. Можно выделить: а) общий вопрос; б) специальный вопрос.

Общий вопрос образуется посредством вопросительных частиц
\prfB{གམ་}{gam}, \prfB{ངམ་}{ngam}, \prfB{དམ་}{dam} (см.страницу 132), которые ставятся после сказуемого, и вопросительной частицы \prfB{ཨེ་}{e}, которая ставится перед сказуемым. Эти частицы соотносимы с русским <<ли>>, например:
\begin{prfsample}
	\item \prfC{མཐོང་ངམ་}{\ul{m}thong-ngam}{<<видел ли?>>};
	\item \prfC{ཨེ་གོ་}{e-go}{<<понял ли?>>}.
\end{prfsample}

Специальный вопрос образуется при помощи различных вопросительных слов, например:	
\begin{prfsample}
	\item \prfC{ཁོང་སུ་རེད་}{khong su re\ul{d}}{<<он кто?>>};
	\item \prfC{ཁྱེད་རང་ག་རེ་གནང་གི་ཡོད་}{khye\ul{d}-rang ga re \ul{g}nang-gi-yo\ul{d}}{<<Вы что делаете?>>}.
\end{prfsample}

В новом письменном языке употребляется заимствованная из разговорного вопросительная частица	\prfB{པས་}{pa\ul{s}}, которая ставится после связки, оформляющей основной глагол, например:
\begin{prfsample}
	\item \prfC{ཁྱེད་རང་བོད་ཡིག་འབྲི་གི་ཡོད་པས་}{khye\ul{d}-rang bo\ul{d}-yig 'bri-gi-yo\ul{d}-pa\ul{s}}{<<Вы сейчас пишете по-тибетски?>>};
	\item \prfC{ཁོང་ཤལ་སྨན་མཆོད་པ་རེད་པས་}{khong sha\ul{l}-\ul{s}man \ul{m}cho\ul{d}-pa-re\ul{d}-pa\ul{s}}{<<он принимал лекарство?>>}.
\end{prfsample}

В новом письменном языке общий вопрос может быть также образован (как и в разговорном, откуда этот способ был заимствован) путем повторения связки, оформляющей главный глагол, с отрицанием \prfB{མ་}{ma} или \prfB{མི་}{mi} перед ней (типа китайского {\chinfont 是不是}\prfnote{иероглифы <<быть (глагол-связка)>>, <<не>>, <<быть (глагол-связка)>>}), например:
\begin{prfsample}
	\item \prfC{ཁོང་གིས་སྐར་མ་ཆེན་པོ་གཅིག་མཐོང་པ་རེད་མ་རེད་}{khong-gi\ul{s} \ul{s}kar-ma chen-po \ul{g}cig \ul{m}thong-pa-re\ul{d} ma-re\ul{d}}{<<он видел большую звезду или нет?>>}.
\end{prfsample}

3-4. \emph{Повелительные и запретительные предложения} --- образуются при помощи повелительного или запретительного наклонения глагола (см. раздел <<Глагол>>).

5. \emph{Побудительные предложения} --- образуются при помощи вспомогательного глагола \prfC{འཇུག་}{'jug}{<<заставлять, побуждать>>}, который присоединяется к основному глаголу посредством одного из вариантов предложной частицы\footnote[53]{За исключением вариантов \prfB{ལ་}{la} и \prfB{ན་}{na}.}, например:
\begin{prfsample}
	\item \prfC{བྱེད་དུ་འཇུག་}{bye\ul{d}-du-'jug }{<<заставлять делать>>};
	\item \prfC{ཉལ་དུ་འཇུག་}{nyal-du-'jug }{<<заставлять спать>>}.
\end{prfsample}

II. \emph{Сложные предложения}. В тибетском языке, как и во многих других языках, сложные предложения можно разделить на две группы: сложносочиненные и сложноподчиненные.

1. \emph{Сложносочиненные предложения} чаще всего соединяются между собой союзом \prfC{དང་}{dang}{<<и>>}. Кроме того, сложносочиненные предложения могут соединяться также союзами \prfB{མ་གཏོགས་}{ma-\ul{g}tog\ul{s}}, \prfB{མ་ཟད་}{ma-zad}, \prfC{ཁར་}{khar}{<<а также, и>>}, которые указывают, что второе предложение содержит какую-то дополнительную информацию к тому, что сказано в предыдущем. При употреблении этих союзов сказуемое первого предложения всегда выступает в форме отглагольного имени. Если такое сказуемое обозначить сокращенно \emph{Си}, то в месте соединения сложносочиненных предложений будем иметь: \emph{Си dang \ldots{}}; \emph{Си ma-\ul{g}tog\ul{s} \ldots{}}; \emph{Си ma-zad \ldots{}}; \emph{Си 'i khar \ldots{}}.

2. \emph{Сложноподчиненные предложения}. В тибетском языке придаточное предложение всегда стоит перед главным, а подчинительные союзы всегда стоят после сказуемого придаточного предложения. Здесь можно выделить несколько основных типов придаточных предложений:

(а)	Условные придаточные предложения --- чаще всего вводятся союзом \prfB{ན་}{na}, который помещается после сказуемого придаточного предложения, или конструкциями:
\prfB{གལ་ཏེ་་་་ན་}{ga\ul{l}-te\ldots{}na} и реже --- \prfB{གལ་ཏེ་་་་ནས་}{ga\ul{l}-te\ldots{}na\ul{s}}.

Глагол в роли сказуемого условного придаточного предложения в тибетском языке всегда выступает в форме основы прошедшего времени, например:
\begin{prfsample}
	\item \prfC{གལ་ཏེ་འདི་འདྲ་བ་བྱས་ནས་བྲམ་ཟེ་མ་ཡིན་}{ga\ul{l}-te 'di-'dra-ba bya\ul{s}-na\ul{s} bram-ze ma yin}{<<если сделаю подобное, то не буду брамином>>}.
\end{prfsample}

(б)	Причинные придаточные предложения --- чаше всего вводятся вариантом орудной частицы \prfB{ས་}{sa}, которая присоединяется к сказуемому придаточного предложения, всегда выступающему в этих случаях в форме отглагольного имени, например:
\begin{prfsample}
	\item \prfC{ཆུང་མ་རེ་བ་ལྟར་མ་གྱུར་བས། མཁོན་དུ་བཟུང་}{chung-ma re-ba \ul{l}tar ma gyur-ba\ul{s}/ \ul{m}khon-du-\ul{b}zung }{<<так как не исполнилось желание жены, [она] разгневалась>>}.
\end{prfsample}

В новом письменном языке причинные предложения часто вводятся союзами \prfC{བསྟབ་}{\ul{bs}tab}{<<так как>>} и \prfC{བརྟེན་}{\ul{br}ten}{<<вследствие>>}.

(в)	Придаточные предложения цели --- вводятся союзами \prfB{ཆེད་དུ་}{che\ul{d}-du}, \prfB{ཕྱིར་དུ་}{phyir-du}, \prfB{སླེད་དུ་}{\ul{s}le\ul{d}-du}, например:
\begin{prfsample}
	\item \prfC{རྒྱལ་ཁབ་སྲུང་སྐྱོང་བྱེད་ཆེད་དུ་}{\ul{r}gya\ul{l}-khab srung-\ul{s}kyong bye\ul{d} che\ul{d}-du\ldots{}}{<<чтобы защитить государство\ldots{}>>}.
\end{prfsample}
В придаточных предложениях цели сказуемое также может выражаться отглагольным именем. В этом случае союзы цели присоединяются к сказуемому посредством варианта притяжатель-ной падежной частицы \prfB{འི་}{'i} по схеме: \emph{\ldots{} Си 'i che\ul{d}-du}.

(г)	Придаточные временные --- вводятся словами
\prfB{སྐབས་}{\ul{s}kab\ul{s}}, \prfB{དུས་}{du\ul{s}}, \prfB{ཚེ་}{tshe} (в значении <<время>>), обычно сочетающимися с косвенной падежной частицей. Как правило, при употреблении слов \prfB{སྐབས་}{\ul{s}kab\ul{s}} и \prfB{དུས་}{du\ul{s}} в качестве временных союзов сказуемое придаточного предложения выступает в форме отглагольного имени, например:
\begin{prfsample}
	\item \prfC{བཅོས་སྒྱུར་བྱེད་པའི་སྐབས་སུ་}{\ul{b}co\ul{s}-\ul{s}gyur-bye\ul{d}-pa'i \ul{s}kab\ul{s}-su\ldots{}}{<<во время проведения реформы\ldots{}>>}.
\end{prfsample}

\section{Прямая речь}

В тибетском языке прямая речь выделяется при помощи частиц	\prfB{ཅེས་}{ce\ul{s}}, \prfB{ཞེས་}{zhe\ul{s}}, \prfB{ཤེས་}{she\ul{s}}, являющихся вариантами одной служебной морфемы. Причем оформляется обычно только заключительное слово прямой речи, начало же ее определяется контекстом. После указанных частиц-вариантов, обозначающих конец прямой речи, следуют сказуемые:
\prfB{སྨྲས་པ་}{\ul{s}m\ul{r}a\ul{s}-pa} (прош. вр. от \prfC{སྨར་}{\ul{s}mar}{<<говорить>>}),
\prfB{གསུངས་པ་}{\ul{g}sung\ul{s}-pa} (прош. вр. от \prfC{གསུང་}{\ul{g}sung}{<<говорить>>}) или
\prfC{ཟེར་}{zer}{<<сказал>>}, например:
\begin{prfsample}
	\item \prfC{ཁྱོད་གྱིས་ངའི་གླང་སླར་བྱིན་ཅིག་ཅེས་སྨྲས་པ་}{khyo\ul{d}-gyi\ul{s} nga'i \ul{g}lang \ul{s}lar byin-cig ce\ul{s} \ul{s}m\ul{r}a\ul{s}-pa}{<<сказал: "Ты верни моего вола"{} >>}.
\end{prfsample}

Иногда сказуемое \prfB{སྨྲས་པ་}{\ul{s}m\ul{r}a\ul{s}-pa} ставится в начале прямой речи, а затем оно повторяется после указанных служебных частиц, например:
\begin{prfsample}
	\item \prfC{མི་གདུང་བས་སྨྲས་པ་་་་ཞེས་སྨྲས་པ་}{mi-\ul{g}dung-ba\ul{s} \ul{s}m\ul{r}a\ul{s}-pa\ldots{}zhe{s} \ul{s}m\ul{r}a\ul{s}-pa}{<<Мидунгва сказал: "\ldots{}"{} >>}.
\end{prfsample}

\section{Знаки препинания}

В тибетском языке слова пишутся слитно, но после каждого слога ставится точка (\prfB{ཚེག་}{tsheg}).

После первого из однородных членов предложения или первого предложения в составе сложного предложения употребляется вертикальная черточка (\prfB{ཤད་}{sha\ul{d}} или
\prfB{རྐྱང་ཤད་}{\ul{r}kyang-sha\ul{d}}).

В конце предложения ставится двойная вертикальная черточка (\prfB{ཉིས་ཤད་}{nyi\ul{s}-sha\ul{d}}). В стихах ее употребляют после каждой строки.

В конце глав или параграфов ставятся две двойные вертикальные черточки (\prfB{བཞི་ཤད་}{\ul{b}zhi-sha\ul{d}}).

В старом письменном языке текст не делится на абзацы.

В новом письменном языке введены абзацы, а также используются кавычки для обозначения прямой речи или цитат.