\section{Глагол}

\label{sec:glagol}\emph{Семантически} тибетский глагол характеризуется тем, что обозначает действие (процесс) в его отношении к лицам и предметам, которыми (или над которыми) это действие совершается.

\emph{Морфологически} тибетский глагол характеризуется тем, что: 1) имеет четыре основы\footnote[32]{Термин <<основа>> используется нами в соответствии с тибетологической традицией и не предполагает сочетания корня с аффиксом.}: настоящего, прошедшего, будущего времени и повелительного наклонения. Так, например, глагол <<сделать>> имеет основу настоящего времени \prfB{བྱེད་}{bye\ul{d}}, прошедшего времени \prfB{བྱས་}{bya\ul{s}}, будущего времени \prfB{བྱ་}{bya} и повелительного наклонения \prfB{བྱོས་}{byo\ul{s}}; 2) принимает в правой позиции определённые служебные морфемы, образуя при этом: а) так называемые незавершенные, или деепричастные, формы глагола, которые указывают на то, что действие (состояние), выраженное при при помощи данных форм, обязательно предполагает последующее связанное с ним или однородное действие (состояние); б) именные формы глагола.

\emph{Синтаксически} тибетский глагол характеризуется тем, что основа глагола: 1) выступает всегда сказуемым и помещается в конце предложения; 2) принимает в правой позиции вспомогательные глаголы, при помощи которых в новом письменном языке (как и разговорном) выражается время и наклонение; 3) принимает в левой позиции частицы отрицания, причём между частицей отрицания и глаголом не может быть вставлено какое-либо другое слово; 4) управляет подлежащим, т.е. в зависимости от значения основы глагола (переходное--непереходное) подлежащее принимает (или не принимает) грамматическое оформление (служебные слова).

\subsection{Основы глагола}

Тибетский глагол имеет четыре основы: основу настоящего времени, основу прошедшего времени, основу будущего времени и основу повелительного наклонения. Основы глагола представляют собой однослоги, вполне определённой структуры, что находит свое отражение главным образом в наличии или отсутствии тех или иных префиксальных и приписных графем. Подавляющее большинство основ настоящего времени не имеет префиксальной графемы или принимает префиксальную графему \prfB{འ}{'a}. Вместе с тем для основы настоящего времени характерно отсутствие приписной графемы \prfB{ས}{sa} или вторичной приписной графемы. Большая часть глагольных основ прошедшего времени не имеет префиксальной графемы или принимает префиксальную графему \prfB{བ}{ba}, однако, в отличие от основ настоящего времени, получает приписную \prfB{ས}{sa}, например:
\begin{prfsample}
	\item \prfC{ངུ་}{ngu}{<<плакать>>} $\rightarrow$ \prfC{ངུས་}{ngu\ul{s}}{<<плакал>>};
	\item \prfC{ཞུ་}{zhu}{<<просить>>} $\rightarrow$ \prfC{ཞུས་}{zhu\ul{s}}{<<просил>>}.
\end{prfsample}
Большая часть основ будущего времени не имеет префиксальной графемы или принимает префиксальную графему \prfB{བ}{ba}, однако, в отличие от основ прошедшего времени, основы будущего времени глагола не имеют приписной графемы \prfB{ས}{sa}. Подавляющее большинство основ повелительного наклонения не имеет префиксальной графемы, но принимает приписную графему \prfB{ས}{sa} или вторичную приписную.

Следует отметить, что не каждый глагол обладает четырьмя основами, полностью отличающимися друг от друга написанием и произношением. В ряде случаев написание основы будущего времени может совпадать с написанием основы настоящего или прошедшего времени и т.п. Так, из 700 глаголов, приведённых в <<Таблице изменения тибетских
глаголов>>\footnote[33]{См.: Б.В.Семичов, Ю.М.Парфионович, Б.Д.Дандарон, Тибетско-русский словарь, М., 1963, стр.549-581.}, только 270 имеют четыре различные основы. Реже всего совпадают написания основ настоящего и прошедшего времени (21 из 700) и чаще всего - написания основ настоящего и будущего времени (300 из 700),
Для иллюстрации приведем таблицу, характеризующую сочетаемость глагольных основ с префиксальными и приписными графемами (таб. \ref{tab:15}).

\begin{tabularx}{\textwidth}{p{0.2\textwidth}*{7}c}
	\caption{Сочетаемость глагольных основ}\label{tab:15}\\
	\toprule
	\multirow[m]{2}{*}{\parbox[c]{0.2\textwidth}{\small\centering Характер основы}} &
	\multirow[m]{2}{*}{\parbox[c]{0.1\textwidth}{\small\centering Без префиксальных графем}} &
	\multicolumn{5}{c}{\parbox[c]{0.25\textwidth}{\small\centering С префиксальными графемами}} &
	\multirow{2}{*}{\parbox[c]{0.2\textwidth}{\small\centering С приписной графемой \prfB{ས}{sa} или вторичной приписной}}\\
	\addlinespace
	& & \parbox{0.05\textwidth}{\prfB[p]{ག}{ga}} & \parbox{0.05\textwidth}{\prfB[p]{ད}{da}} & \parbox{0.05\textwidth}{\prfB[p]{བ}{ba}} & \parbox{0.05\textwidth}{\prfB[p]{མ}{ma}} & \parbox{0.05\textwidth}{\prfB[p]{འ}{'a}} & \\
	\midrule
	\makecell[l]{Насто\-я\-щее\\время} & 337 & 59 & 18 & 32 & 8 & 242 & 18\\
	\addlinespace
	\makecell[l]{Прошед\-шее\\время} & 107 & 46 & 12 & 365 & 12 & 73 & 500\\
	\addlinespace
	\makecell[l]{Буду\-щее\\время} & 146 & 85 & 50 & 294 & 10 & 126 & 3\\
	\addlinespace
	\makecell[l]{Повели\-тель\-ное\\наклонение} & 530 & 42 & 9 & 28 & 7 & 102 & 318\\
	\bottomrule
\end{tabularx}
{\footnotesize \emph{Примечание}. Таб. \ref{tab:15} составлена на материале упомянутой <<Таблицы изменения тибетских глаголов>> (см. сноску ЗЗ).}

Различные основы глагола отличаются не только префиксальными и приписными графемами. Иногда меняются и некоторые корневые графемы, представляющие основу слога, например:
\begin{prfsample}
	\item \prfB{ཁ}{kha} и \prfB{ག}{ga} переходят в \prfB{ཀ}{ka};
	\item \prfB{ཆ}{cha} и \prfB{ཇ}{ja} переходят в \prfB{ཅ}{ca} и \prfB{ཞ}{zha};
	\item \prfB{ཐ}{tha} и \prfB{ད}{da} переходят в \prfB{ཏ}{ta};
	\item \prfB{བ}{ba} переходит в \prfB{ཕ}{pha};
	\item \prfB{ཚ}{tsha} и \prfB{ཛ}{dza} соответственно переходят в \prfB{ཙ}{tsa} и \prfB{ཟ}{za}.
\end{prfsample}

Изменение корневых графем основ глагола проиллюстрируем на примере следующих десяти глаголов:

\begin{tabularx}{\textwidth}{p{0.18\textwidth}p{0.18\textwidth}p{0.18\textwidth}p{0.18\textwidth}p{0.18\textwidth}}
	\toprule
	\parbox{0.18\textwidth}{\small\centering Глагол} & \parbox{0.18\textwidth}{\small\centering Настоящее время} & \parbox{0.18\textwidth}{\small\centering Прошедшее время} & \parbox{0.18\textwidth}{\small\centering Будущее время} & \parbox{0.18\textwidth}{\small\centering Повелительное наклонение}\\
	\midrule
	\endhead
	связывать, вязать & \prfB{འཁྱིག་}{'khyig} & \prfB{བཀྱིགས་}{\ul{b}kyig\ul{s}} & \prfB{བཀྱིག་}{\ul{b}kyig} & \prfB{ཁྱིགས་}{khyig\ul{s}} \\
	\addlinespace
	сгибаться & \prfB{གུག་}{gug} & \prfB{བཀུག་}{\ul{b}kug} & \prfB{བཀུག་}{\ul{b}kug} & \prfB{གུག་}{gug} \\
	\addlinespace
	\makecell[l]{таить,\\скрывать} & \prfB{འཆབ་}{'chab} & \prfB{བཅབས་}{\ul{b}cab\ul{s}} & \prfB{བཅབ་}{\ul{b}cab} & \prfB{ཆོབ་}{chob} \\
	\addlinespace
	\makecell[l]{колоть,\\раскалывать} & \prfB{འཆེག་}{'cheg} & \prfB{བཤེགས་}{\ul{b}sheg\ul{s}} & \prfB{བཤེག་}{\ul{b}sheg} & \prfB{ཤོག་}{shog} \\
	\addlinespace
	сосать & \prfB{འཇིབ་}{'jib} & \prfB{བཞིབས་}{\ul{b}zhib\ul{s}} & \prfB{བཞིབ་}{\ul{b}zhib} & \prfB{འཇིབ་}{'jib} \\
	\addlinespace
	ткать & \prfB{འཐག་}{'thag} & \prfB{བཏགས་}{\ul{b}tag\ul{s}} & \prfB{བཏག་}{\ul{b}tag} & \prfB{ཐོགས་}{thog\ul{s}} \\
	\addlinespace
	рассыпаться & \prfB{ཐོར་}{thor} & \prfB{བཏོར་}{\ul{b}tor} & \prfB{གཏོར་}{\ul{g}tor} & \prfB{ཐོར་}{thor} \\
	\addlinespace
	извлекать, вынимать & \prfB{འདོན་}{'don} & \prfB{བཏོན་}{\ul{b}ton} & \prfB{གཏོན་}{\ul{g}ton} & \prfB{ཐོན་}{thon} \\
	\addlinespace
	\makecell[l]{дарить,\\подносить} & \prfB{འབུལ་}{'bul} & \prfB{ཕུལ་}{phul} & \prfB{དབུལ་}{\ul{d}bul} & \prfB{ཕུལ་}{phul} \\
	\addlinespace
	процеживать, просеивать & \prfB{འཚག་}{'tshag} & \prfB{བཙགས་}{\ul{b}tsag\ul{s}} & \prfB{བཙག་}{\ul{b}tsag} & \prfB{ཚོགས་}{tshog\ul{s}} \\
	\bottomrule
\end{tabularx}

\subsection{Типы глаголов}

В тибетском языке можно выделить три типа глаголов: глаголы-связки (I), модальные глаголы (II) и знаменательные глаголы (III).

I. \emph{Глаголы-связки} указывают на существование или наличие некоторого лица или предмета. Связки не имеют временных форм и не могут сочетаться со вспомогательными глаголами. В тибетском языке имеются две связки существования --- \prfB{ཡིན་}{yin} и \prfC{རེད་}{re\ul{d}}{<<есть, быть>>} и две связки наличия --- \prfB{ཡོད་}{yo\ul{d}} и \prfC{འདུག་}{'dug}{<<иметь, иметься>>}. В вежливой речи в качестве связки существования используется морфема \prfB{ལགས་}{lag\ul{s}}, в качестве связок наличия изредка употребляются морфемы \prfB{མཆིས་}{mchi\ul{s}}, \prfB{གདའ་}{\ul{g}da'}, \prfB{སྣང་}{\ul{s}nang}. Особенность связок существования \prfB{ཡིན་}{yin} и \prfB{རེད་}{re\ul{d}} и связок наличия \prfB{ཡོད་}{yo\ul{d}} и \prfB{འདུག་}{'dug} состоит в том, что \prfB{ཡིན་}{yin} и \prfB{ཡོད་}{yo\ul{d}} могут выражать большую степень уверенности говорящего в той информации, которую он сообщает.

Глаголы-связки выполняют две основные функции:

1) употребляются в тибетском языке в предикативных синтагмах, причём, если предикативный член выражен существительным, то используются связки существования, например:
\begin{prfsample}
	\item \prfC{འདི་དཔེ་ཆ་རེད་}{'di \ul{d}pe-cha re\ul{d}}{<<это --- книга>>};
	\item \prfC{ང་བོད་པ་ཡིན་}{nga bo\ul{d}-pa yin}{<<я --- тибетец>>};
\end{prfsample}
если же предикативный член выражен прилагательным, то обычно используются связки наличия, например:
\begin{prfsample}
	\item \prfC{སྟོན་ཐོག་ཡག་པོ་ཡོད་}{\ul{s}ton-thog yag-po-yo\ul{d}}{<<урожай хорош>>};
	\item \prfC{ལྟོབ་ཆས་ཞིམ་པོ་འདུག་}{\ul{l}tob-cha\ul{s} zhim-po-'dug}{<<пища вкусна>>};
\end{prfsample}

2) употребляются в новом письменном языке для выражения времени и лица у глагола главного предложения (см. разд. \hyperref[sec:glagol:kat_vremeni]{<<Категория времени>>}).

Отрицание ставится перед связками, причём со связкой \prfB{རེད་}{re\ul{d}} употребляется отрицание \prfB{མ་}{ma}, а со связкой \prfB{འདུག་}{'dug} --- отрицание \prfB{མི་}{mi}. Отрицательная форма от связок \prfB{ཡིན་}{yin} и \prfB{ཡོད་}{yo\ul{d}} обычно выражается не при помощи частиц отрицания \prfB{མ་}{ma} или \prfB{མི་}{mi}, а через негативную форму самих связок --- \prfC{མིན་}{min}{<<не есть, не быть>>} и \prfC{མེད་}{med}{<<не иметь(ся), отсутствовать>>}, например:
\begin{prfsample}
	\item \prfC{ང་བོད་པ་མིན་}{nga bo\ul{d}-pa min}{<<я не тибетец>>};
	\item \prfC{སྟོན་ཐོག་ཡག་པོ་མེད་}{\ul{s}ton-thog yag-po me\ul{d}}{<<урожай не хорош>>}.
\end{prfsample}

II. \emph{Модальные глаголы} обозначают не само действие, а отношение субъекта к действию, т.е. указывают на желание, намерение, возможность, необходимость и т.п. реализации данного действия. Модальные глаголы, в отличие от других знаменательных глаголов, имеют только одну основу. В предложении они всегда следуют за глаголом, обозначающим действие. Между основой глагола, обозначающего действие, и модальным глаголом кроме частицы отрицания не может стоять никакое служебное или знаменательное слово. Если глагол должен принять служебное слово, оно следует за модальным глаголом. Схематически сказуемое с модальным глаголом можно изобразить так: \emph{ГМг}, где: \emph{Г} --- основной глагол, \emph{Мг} --- модальный глагол. Отрицательная форма такого сказуемого выглядит так: \emph{ГОМг}, где \emph{О} --- отрицание.

Выделяются следующие модальные глаголы:
\begin{itemize}
	\item \prfC{དགོས་}{\ul{d}go\ul{s}}{<<быть необходимым>>} --- указывает, что субъект должен осуществить действие, обозначенное глаголом, например:
	\begin{prfsample}
		\item \prfC{ཚང་མར་འཆར་གཞི་དང་ལྡན་དགོས་}{tshang-mar 'char-gzhi dang \ul{l}dan \ul{d}go\ul{s}}{<<все должны иметь план>>};
	\end{prfsample}
	\item \prfC{ཐུབ་}{thub}{<<мочь, быть в состоянии>>} --- указывает, что субъект в состоянии осуществить действие, обозначенное глаголом, например:
	\begin{prfsample}
		\item \prfC{ཚང་མས་བོད་སྐད་བཤད་ཐུབ་}{tshang-ma\ul{s} bo\ul{d}-\ul{s}ka\ul{d} \ul{b}sha\ul{d} thub}{<<все могут говорить по-тибетски>>};
	\end{prfsample}
	\item \prfC{ཆོག་}{chog}{<<разрешать, допускать>>} --- указывает, что объективные условия позволяют субъекту осуществить действия, например:
	\begin{prfsample}
		\item \prfC{ཆུ་འཁོལ་མ་རེད། འཐུང་ཆོག་}{chu 'khol-ma re\ul{d}, 'thung chog}{<<вода кипяченая, можно пить>>};
	\end{prfsample}
	\item \prfC{འདོད་}{'do\ul{d}}{<<желать, хотеть>>} --- указывает, что субъект хочет осуществить данное действие, например:
	\begin{prfsample}
		\item \prfC{ཁོང་བོད་ལ་འགྲོ་འདོད་}{khong bo\ul{d}-la 'gro 'do\ul{d}}{<<он хочет ехать в Тибет>>};
	\end{prfsample}
	\item \prfB{ནུས་}{nu\ul{s}}, \prfC{ཕོད་}{pho\ul{d}}{<<быть в силах; дерзать, осмеливаться>>} --- указывают, что субъект в состоянии или осмеливается осуществить данное действие, например:
	\begin{prfsample}
		\item \prfC{ནུ་བོ་འགྲོ་མ་ནུས་}{nu-bo 'gro ma nu\ul{s}}{<<младший брат был не в состоянии идти>>};
		\item \prfC{རྒྱལ་པོའི་བཀའ་སུ་ཞིག་གིས་ལྡོག་ནུས་}{\ul{r}gyal-po'i \ul{b}ka' su-zhig-gi\ul{s} \ul{l}dog nu\ul{s}}{<<кто осмелится не подчиниться приказу царя>>};
	\end{prfsample}
	\item \prfC{ངེས་}{nge\ul{s}}{<<быть достоверным, точным>>} --- указывает, что субъект непременно совершит данное действие, например:
	\begin{prfsample}
		\item \prfC{ཁོང་ཚོ་འགྲོ་ངེས་ཡིན་}{khong-tsho 'gro nges yin}{<<они обязательно пойдут>>};		
	\end{prfsample}
	\item \prfB{རན་}{ran} и \prfB{གྲབས་}{grab\ul{s}} (новый письменный язык) --- указывают, что наступило время совершения действия, например:
	\begin{prfsample}
		\item \prfC{སོན་འདེབས་རན་}{son 'deb\ul{s} ran}{<<время сеять>>};
		\item \prfC{ང་ཚོ་སླེབ་གྲབས་ཡོད་}{nga-tsho \ul{s}leb grab\ul{s} yo\ul{d}}{<<мы сейчас придем>>};
	\end{prfsample}
\end{itemize}

В функции модальных глаголов могут выступать следующие знаменательные глаголы:
\begin{itemize}
	\item \prfC{ཤེས་}{she\ul{s}}{<<знать>>} --- указывает, что субъект может (умеет) осуществить данное действие, например:
	\begin{prfsample}
		\prfC{ཁོང་གིས་བོད་ཡིག་འབྲི་ཤེས་}{khong-gi\ul{s} bo\ul{d}-yig 'bri she\ul{s}}{<<он умеет писать по-тибетски>>};
	\end{prfsample}
	\item \prfC{རྩིས་}{{r}tsi{s}}{<<считать, вычислять>>} --- указывает на намерение субъекта осуществить действие, например:
	\begin{prfsample}
		\item \prfC{ངས་བྱེད་རྩིས་ཡོད་}{nga\ul{s} bye\ul{d} \ul{r}tsi\ul{s} yo\ul{d}}{<<я намереваюсь сделать [это]>>};
	\end{prfsample}
	\item \prfC{སླ་}{\ul{s}la}{<<быть лёгким>>}, \prfC{དཀའ་}{dka'}{<<быть трудным>>} --- указывают на легкость или трудность осуществления действия, например:
	\begin{prfsample}
		\item \prfC{ལས་ཀ་དེ་ལས་སླ་}{la\ul{s}-ka de la\ul{s} \ul{s}la}{<<это дело сделать легко>>}.
	\end{prfsample}
\end{itemize}

III. Знаменательные глаголы по своей структуре делятся на простые (1) и сложные (2).

1. Простые глаголы представляют собой однослоговые вербальные морфемы. По способу управления именем-подлежащим простые глаголы делятся на: переходные, непереходные и обозначающие наличие или приобретение.

Переходные глаголы обозначают действия, которые по своему характеру подразумевают наличие как деятеля --- подлежащего, так и объекта, на который направлено само действие, --- прямого дополнения. Такие глаголы требуют грамматического оформления подлежащего специальной служебной морфемой --- орудной падежной частицей (см. \hyperref[sec:ss]{<<Служебные слова>>}), например:
\begin{prfsample}
	\item \prfC{ཁོང་གིས་རྟག་བསད་}{khong-gi\ul{s} \ul{r}tag \ul{b}sa\ul{d}}{<<он убил тигра>>};
	\item \prfC{མིས་ཤིང་བཅད་}{mi\ul{s} shing \ul{b}ca\ul{d}}{<<человек рубил дерево>>}
\end{prfsample}
(\prfB{གིས་}{gi\ul{s}} и \prfB{ས}{sa} --- варианты орудной частицы). Прямое дополнение при переходных глаголах грамматического оформления не принимает.

Непереходные глаголы выражают действия, не вызывая при этом представления о лицах или предметах, на которые это действие переходит, например:
\begin{prfsample}
	\item \prfC{ཁོང་འགྲོ་}{khong 'gro}{<<он идет>>}.
\end{prfsample}
При непереходном глаголе подлежащее не оформляется каким-либо служебным словом. К непереходным в тибетском языке относятся также так называемые глаголы чувствования (<<любить>>, <<презирать>>, <<уважать>> и т.п.). Несмотря на наличие объекта действия у этих глаголов, они относятся к группе непереходных, поскольку подлежащее при этих глаголах также не оформляется служебными словами, объект же, в отличие от объекта при переходном глаголе, оформляется косвенной падежной частицей (обычно \prfB{ལ}{la} или \prfB{ར}{ra}), выступая в предложении косвенным дополнением, например:
\begin{prfsample}
	\item \prfC{སློབ་མ་རྣམས་སློབ་དཔོན་ལ་བལྟེ་བསྟང་བྱེད་}{\ul{s}lob-ma-\ul{r}nam{s} \ul{s}lob-\ul{d}pon-la \ul{bl}te-\ul{bs}tang bye\ul{d}}{<<ученики уважают учителя>>} (букв. <<ученики учитель \emph{la} уважают>>);
	\item \prfC{མ་བུ་ལ་སྙིབ་}{ma bu-la \ul{s}nyib}{<<мать пожалела сына>>} (букв. <<мать сын \emph{la} пожалела>>).
\end{prfsample}

С точки зрения грамматической характеристики к непереходным следует отнести и небольшую группу возвратных глаголов. От соответствующих им переходных глаголов помимо отсутствия грамматического оформления у подлежащего они отличаются также и фонетически: при сопоставлении переходных и соответствующих им возвратных глаголов обнаруживаем:

а) оппозицию непридыхательной и придыхательной фонемы (непридыхательная фонема у переходного глагола, придыхательная у соответствующего возвратного глагола), например:
\begin{prfsample}
	\item \prfC{བཅད་}{\mfa{tsje}\toneN}{<<рвать>>} --- \prfC{ཆད་}{\mfa{ts'je}\toneN}{<<рваться>>}
	\item \prfC{སྦེལ་}{\mfa{pel}\toneR}{<<увеличивать>>} --- \prfC{འཕེལ་}{\mfa{p'el}\toneR}{<<увеличиваться>>}
\end{prfsample}

б) различные тоны, например:
\begin{prfsample}
	\item \prfC{སློག་}{\mfa{lok}\toneR}{<<возвращать>>} --- \prfC{ལོག་}{\mfa{lok}\toneV}{<<возвращаться>>}
	\item \prfC{སྤར་}{\mfa{pa:}\toneR}{<<зажигать>>} --- \prfC{འབར་}{\mfa{pa:}\toneV}{<<загораться>>}
\end{prfsample}

в) различные тоны и оппозицию фонем:
\begin{prfsample}
	\item \prfC{བཀྲོལ་}{\mfa{tso:}\toneR}{<<освобождать>>} --- \prfC{གྲོལ་}{\mfa{tso:}\toneV}{<<освобождаться>>}
\end{prfsample}

В тибетском языке имеется небольшая группа глаголов,
обозначающих наличие или появление чего-либо у подлежащего, например:
\begin{prfsample}
	\item \prfC{ཡོད་}{yo\ul{d}}{<<иметь, иметься>>};
	\item \prfC{ཐོབ་}{thob}{<<получать>>};
	\item \prfB{འབྱུང་}{'byung} (прош.вр. \prfB{བྱུང་}{byung}) <<возникать, появляться>>;
	\item \prfB{བཙའ་}{\ul{b}tsa'} (прош.вр. \prfB{བཙས་}{\ul{b}tsa\ul{s}}) <<рожать>> (о людях);
	\item \prfB{འབྲང་}{'brang} (прош.вр. \prfB{འབྲངས་}{'brangs}) <<рожать>> (о животных).
\end{prfsample}
Грамматически эти глаголы выделяются из всех остальных глаголов только им свойственным управлением: управление именем-подлежащим осуществляется при помощи вариантов косвенной падежной частицы (обычно \prfB{ལ}{la} или \prfB{ར}{ra}), помещающихся справа от слова (группы слов), обозначающего подлежащее, например:
\begin{prfsample}
	\item \prfC{མི་དམངས་ཚོར་བཙིངས་བཀྲོལ་ཐོབ་}{mi-\ul{d}mang\ul{s}-tshor \ul{b}tsing\ul{s}-\ul{b}krol thob}{<<народы получили освобождение>>};
	\item \prfC{གྲོགས་པོ་ལ་མེ་མདའ་ཡོད་}{grog\ul{s}-po-la me-\ul{m}da' yo\ul{d}}{<<приятель имеет ружье>>};
	\item \prfC{མ་ལ་བུ་བཙས་}{ma-la bu \ul{b}tsa\ul{s}}{<<мать родила сына>>}\footnote[34]{Наличие такого управления обусловлено одной и закономерностей тибетского языка, состоящей в том, что слово (группа слов), обозначающее место, где имеется (существует), появляется или возникает объект, должно быть оформлено падежной частицей.}.
\end{prfsample}

В новом письменном языке в определённых речевых ситуациях глагол наличия \prfB{ཡོད་}{yo\ul{d}} заменяется синтаксическим образованием \prfB{ཡོད་པ་རེད་}{yo\ul{d}-pa-re\ul{d}} (полная форма от разговорного \prfB{ཡོད་རེད་}{yo\ul{d}-re\ul{d}}). Конструкция фразы с таким синтаксическим образованием, обозначающим наличие, очень напоминает английскую конструкцию \emph{there is} или \emph{there are}, например:
\begin{prfsample}
	\item \prfC{བཟོ་པ་དེ་ཚོར་ཉམས་མྱོང་མང་པོ་ཡོད་པ་རེད་}{\ul{b}zo-pa de-tshor nyams-myong mang-po yo\ul{d}-pa-re\ul{d}}{<<эти рабочие имеют богатый опыт>>};
	\item \prfC{རི་སྒང་ལ་རི་དྭགས་ཡོད་པ་རེད་}{ri \ul{s}gang-la ri-d\ul{w}ag\ul{s} yo\ul{d}-pa-re\ul{d}}{<<в горах имеется дичь>>}.
\end{prfsample}
Отрицательная форма такой конструкции образуется постановкой частицы отрицания перед связкой \prfB{རེད་}{re\ul{d}}, например:
\begin{prfsample}
	\item \prfC{ཁང་པ་དིའི་ནང་ལ་གློག་ཡོད་པ་མ་རེད་}{khang-pa di'i nang-la \ul{g}log yo\ul{d}-pa-ma-re\ul{d}}{<<в этом доме нет электричества>>}.
\end{prfsample}

2. Сложные глаголы. Кроме простых глаголов имеется значительное число сложных глаголов, которые состоят из имени и простого глагола. В зависимости от характера глагольного компонента сложные глаголы делятся на переходные и непереходные (ср. простые переходные и непереходные глаголы).

Между именным и глагольным компонентами сложного глагола могут иметь место два вида отношений: адвербиально-предикатные и объектно-предикатные.

При адвербиально-предикатных отношениях между компонентами сложного глагола именная часть обозначает, в каком качестве, состоянии, обличье существует, действует, воспринимается предмет. Связь между компонентами сложного глагола осуществляется посредством вариантов косвенной падежной частицы (\prfB{ཏུ}{tu}, \prfB{དུ}{du}, \prfB{ར}{ra}, \prfB{རུ}{ru} или \prfB{སུ}{su}\footnote[35]{Частица \prfB{ན}{na} в таких конструкциях не употребляется, частица \prfB{ལ}{la} употребляется в единичных случаях.} --- в зависимости от конечного согласного слога), которая следует справа от имени, например:
\begin{prfsample}
	\item \prfC{ཀླུ་ཤིག་ཏུ་སྐྱེ་}{\ul{k}lu-shig-tu-\ul{s}kye}{<<возрождаться драконом>>},
	\item \prfC{དངོས་པོར་འཛིན་}{\ul{d}ngo\ul{s}-por-'dzin}{<<воспринимать реальностью>>},
	\item \prfC{རྒྱལ་པོར་སྐོ་}{\ul{r}gyal-por-\ul{s}ko}{<<выбирать королем>>},
	\item \prfC{ཁྲོས་པར་རིག་}{khro\ul{s}-par-rig}{<<видеть гневным>>},
	\item \prfC{མེར་འབོར་}{mer-'bor}{<<возгораться огнём>>}.
\end{prfsample}

В образовании таких сложных глаголов наиболее продуктивно участвуют простые глаголы	\prfB{འགྱུར་}{'gyur} (прош.вр. \prfB{གྱུར་}{gyur}) <<превращаться, изменяться, становиться>> и \prfB{བྱེད་}{bye\ul{d}}(прош.вр. \prfB{བྱས་}{bya\ul{s}}, буд.вр. \prfB{བྱ་}{bya}) <<делать(ся), осуществляться(ся)>>, например:
\begin{prfsample}
	\item \prfC{ཕན་པར་འགྱུར་}{phan-par-'gyur}{<<становиться полезным>>},
	\item \prfC{ལོང་བར་འགྱུར་}{long-bar-'gyur}{<<делаться слепым, слепнуть>>},
	\item \prfC{སྟོང་བར་བྱེད་}{\ul{s}tong-bar-bye\ul{d}}{<<делать(ся) пустым, пустеть>>},
	\item \prfC{ངན་པར་བྱེད་}{ngan-par-bye\ul{d}}{<<делаться дурным, плохим>>}.
\end{prfsample}

Между компонентами сложных глаголов данного вида не может быть поставлено никакого слова кроме частицы отрицания.

В новом письменном языке широко распространены сложные глаголы с адвербиально-предикатной связью, образованные при помощи глагольных компонентов \prfC{གཏོང་}{{g}tong}{<<давать, посылать>>} и \prfC{འགྲོ་}{'gro}{<<идти>>}, лексическое значение которых стерлось. Значение сложного глагола вытекает из значения его именной части. Глагольный компонент \prfB{གཏོང་}{{g}tong} указывает лишь на переходность, а \prfB{འགྲོ་}{'gro} --- на непереходность соответствующего сложного глагола, ср., например:
\begin{prfsample}
	\item \prfC{ཁ་ཐོར་དུ་གཏོང་}{kha-thor-du-\ul{g}tong}{<<распылять>>} и \prfC{ཁ་ཐོར་དུ་འགྲོ་}{kha-thor-du-'gro}{<<распыляться>>};
	\item \prfC{གོང་འཕེལ་དུ་གཏོང་}{gong-'phel-du-\ul{g}tong}{<<увеличивать>>} и \prfC{གོང་འཕེལ་དུ་འགྲོ་}{gong-'phel-du-'gro}{<<увеличиваться>>}.
\end{prfsample}

Следует иметь в виду, что в ряде случаев косвенная падежная частица в подобных сложных глаголах выпадает, например:
\begin{prfsample}
	\item \prfC{ཁ་ཐོར་གཏོང་}{kha-thor-\ul{g}tong}{<<распылять>>},
	\item \prfC{གོང་འཕེལ་འགྲོ་}{gong-'phel-'gro}{<<увеличиваться>>}.
\end{prfsample}

При объектно-предикатных отношениях между составляющими компонентами сложного глагола глагол примыкает к имени без посредства каких-либо служебных слов, причём он семантически обесцвечивается и значение сложного глагола в целом вытекает из значения его именной части, например:
\begin{prfsample}
	\item \prfC{དགའ་བ་}{\ul{d}ga'-ba}{<<радость>>} + \prfC{སྐྱེ་}{\ul{s}kye}{<<родить>>} $\rightarrow$ \prfC{དགའ་བ་སྐྱེ་}{\ul{d}ga'-ba-\ul{s}kye}{<<радоваться>>};
	\item \prfC{ཐེ་ཚོམ་}{the-tshom}{<<сомнение>>} + \prfC{སྐྱེ་}{\ul{s}kye}{<<родить>>} $\rightarrow$ \prfC{ཐེ་ཚོམ་སྐྱེ་}{the-tshom-\ul{s}kye}{<<сомневаться>>}.
\end{prfsample}

Простых глаголов, участвующих в образовании сложных глаголов данного типа, немного, но они очень продуктивны. Весьма продуктивным является глагол \prfC{བྱེད་}{bye\ul{d}}{<<делать, делаться>>}, например:
\begin{prfsample}
	\item \prfC{སྒྲིག་འཛུགས་}{\ul{s}grig-'dzug\ul{s}}{<<организация>>} + \prfB{བྱེད་}{bye\ul{d}} $\rightarrow$ \prfC{སྒྲིག་འཛུགས་བྱེད་}{\ul{s}grig-'dzug\ul{s}-bye\ul{d}}{<<организовывать>>};
	\item \prfC{འགོ་ཁྲིད་}{'go-khri\ul{d}}{<<руководство>>} + \prfB{བྱེད་}{bye\ul{d}} $\rightarrow$ \prfC{འགོ་ཁྲིད་བྱེད་}{'go-khri\ul{d}-bye\ul{d}}{<<руководить>>}.
\end{prfsample}

В новом письменном языке также очень широко распространены сложные глаголы с глагольным компонентом \prfB{རྒྱག་}{\ul{r}gyag} (прош.вр. \prfB{བརྒྱབ་}{\ul{br}gyab}). Сложные глаголы, образованные при помощи \prfB{རྒྱག་}{\ul{r}gyag}, могут быть как переходными, так и непереходными, например:
\begin{prfsample}
	\item \prfC{ཁྲག་}{khrag}{<<кровь>>} + \prfB{རྒྱག་}{\ul{r}gyag} $\rightarrow$ \prfC{ཁྲག་རྒྱག་}{khrag-\ul{r}gyag}{<<кровоточить>>};
	\item \prfC{དྲི་མ་}{dri-ma}{<<запах>>} + \prfB{རྒྱག་}{\ul{r}gyag} $\rightarrow$ \prfC{དྲི་མ་རྒྱག་}{dri-ma-\ul{r}gyag}{<<пахнуть>>};
	\item \prfC{རྔུལ་ནག་}{\ul{r}ngu\ul{l}-nag}{<<пот>>} + \prfB{རྒྱག་}{\ul{r}gyag} $\rightarrow$ \prfC{རྔུལ་ནག་རྒྱག་}{\ul{r}ngu\ul{l}-nag-\ul{r}gyag}{<<потеть>>};
	\item \prfC{ལྡེ་མིག་}{\ul{l}de-mig}{<<ключ>>} + \prfB{རྒྱག་}{\ul{r}gyag} $\rightarrow$ \prfC{ལྡེ་མིག་རྒྱག་}{\ul{l}de-mig-\ul{r}gyag}{<<запирать>>};
	\item \prfC{ཐབ་ཅི་}{thab-ci}{<<пуговица>>} + \prfB{རྒྱག་}{\ul{r}gyag} $\rightarrow$ \prfC{ཐབ་ཅི་རྒྱག་}{thab-ci-\ul{r}gyag}{<<застегивать>>}.
\end{prfsample}

\subsection{Формы глагола}

В тибетском языке выделяются: незавершенные, или деепричастные, формы глагола (I) и именные формы глагола (II).

I. \emph{Незавершенные, или деепричастные, формы глагола}. Если два (и более) глагола обозначают действия (или состояния) одного субъекта, то возможны два варианта: 1) действие, выраженное первым глаголом, уточняет обстоятельстве действия, выраженного вторым глаголом, или служит необходимой предпосылкой для совершения действия, выраженного вторым глаголом; 2) оба глагола обозначают однородные действия. В обоих случаях первый глагол оформляется определённой служебной частицей, образующей незавершенную, или деепричастную, форму глагола. Незавершенная форма глагола указывает на то, что действие, выраженное этой формой, обязательно предполагает последующее связанное с ним или однородное действие (состояние).

Когда действие, выраженное первым глаголом, уточняет обстоятельства действия или служит необходимой предпосылкой для совершения действия, выраженного вторым глаголом, то первый глагол принимает справа служебную частицу, которая существует в трёх вариантах: \prfB{ཏེ་}{te},\prfB{སྟེ་}{\ul{s}te}, \prfB{དེ་}{de}. Употребление того или иного варианта зависит от орфографии данного глагола\footnote[36]{\prfB{ཏེ་}{te} употребляется после конечных \prfA{ན},\prfA{ར},\prfA{ལ},\prfA{ས}; \prfB{སྟེ་}{\ul{s}te} --- после конечных \prfA{ག},\prfA{ང},\prfA{བ},\prfA{མ},\prfA{འ}; \prfB{དེ་}{de} --- после конечной \prfA{ད}}, например:
\begin{prfsample}
	\item \prfC{བལྟོས་ཏེ་མཐོང་}{\ul{bl}to\ul{s}-te \ul{m}thong}{<<посмотрел и увидел>>};
	\item \prfC{ལག་པ་བརྒྱང་ཏེ་བཟུང་}{lag-pa \ul{br}gyang-te \ul{b}zung}{<<вытянул руку и схватил>>};
	\item \prfC{ཇ་བསྐོལ་ཏེ་འཐུང་}{ja \ul{bs}ko\ul{l}-te 'thung}{<<вскипятив чай, пил>>};
	\item \prfC{ངུས་ཏེ་སྨྲས་}{ngu\ul{s}-te \ul{s}m\ul{r}a\ul{s}}{<<плача, говорил>>};
	\item \prfC{མེ་ཤིང་ཚག་སྒྲ་དང་བཅས་ཏེ་འབར་}{me-shing tshag-\ul{s}gra dang \ul{b}ca\ul{s}-te 'bar}{<<хворост, потрескивая, горел>>}.
\end{prfsample}

Для указанных частиц-вариантов не имеет значения, какие действия выражаются глаголами: одновременные или разновременные. Если требуется несколько оттенить разновременность действий, то первый глагол оформляется служебной морфемой \prfB{ནས་}{na\ul{s}}\footnote[37]{Графически и фонетически эта морфема совпадает с исходной падежной частицей (см. разд. \hyperref[sec:ss]{<<Служебные слова>>})}, например:
\begin{prfsample}
	\item \prfC{ལྷ་ཁང་ལ་འོངས་ནས་འདུམ་སྐུའི་རྒྱབ་ལ་ཡིབ་}{lha-khang-la 'ong\ul{s}-na\ul{s} 'dum-\ul{s}ku'i \ul{r}gyab-la yib}{<<пришёл в храм и спрятался за статуей>>};
	\item \prfC{བུ་ལ་སྙིབ་བརྩེ་ནས་ཁ་ཟས་སྐྱེལ་དུ་བཏང་}{bu-la \ul{s}nyib-\ul{br}tse-na\ul{s} kha-za\ul{s} \ul{s}kye\ul{l}-du-\ul{b}tang}{<<пожалев сына, послала [ему] еду>>}.
\end{prfsample}

Когда два глагола обозначают однородные действия, то первый глагол принимает справа иную служебную частицу. Эта частица существует в трёх вариантах --- \prfB{ཅིང་}{cing}, \prfB{ཞིང་}{zhing}, \prfB{ཤིང་}{shing}, например:
\begin{prfsample}
	\item \prfC{སྡུག་ཅིང་བྱམས་}{\ul{s}dug-cing \ul{b}yams}{<<любил и лелеял>>};
	\item \prfC{ཐམས་ཅད་ཁོང་དུ་ཆུད་ཅིང་རྟོགས་པར་གྱར་}{tham\ul{s}-ca\ul{d} khong-du-chu\ul{d}-cing \ul{r}tog\ul{s}-par-gyar}{<<всё понял и изучил>>};
	\item \prfC{ཉེས་པ་འཐོལ་ཞིང་འབྱོད་}{nye\ul{s}-pa 'tho\ul{l}-zhing 'byo\ul{d}}{<<признался в злодеяниях и раскаялся>>}.
\end{prfsample}
Эти частицы-варианты очень часто употребляются с предикативными морфемами, обозначающими наличие какого-либо качества, например:
\begin{prfsample}
	\item \prfC{བཟང་ཞིང་ལེགས་}{\ul{b}zang-zhing leg\ul{s}}{<<красив и хорош>>};
	\item \prfC{དྭངས་ཤིང་གསལ་}{d\ul{w}ang\ul{s}-shing \ul{g}sal}{<<светел и ясен>>}.
\end{prfsample}

При глаголах, обозначающих однородные действия, незавершенная форма глагола может быть также образована при помощи служебной морфемы \prfB{ལ་}{la}, например:
\begin{prfsample}
	\item \prfC{ཡོན་ཏན་ཕུན་སུམ་ཚོངས་ལ།སྡིག་པ་ཡོངས་སུ་ཟད་}{yon-tan phun-sum-tshong\ul{s}-la/ \ul{s}dig-pa yong\ul{s}-su za\ul{d}}{<<соединил (в себе) все достоинства и избавился от грехов>>};
	\item \prfC{སོང་ལ་བལྡས་}{song la {bl}da{s}}{<<пришёл и посмотрел>>}.	
\end{prfsample}

Между глаголом и упомянутыми выше служебными частицами, примыкающими к нему, существует очень прочная связь, не допускающая разрыва их каким-либо знаменательным или служебным словом.

II. \emph{Именные формы глагола} подразделяются на два подтипа:

1. Герундивно-причастные формы. Эти именные формы образуются путём присоединения к основе глагола служебных морфем \prfB{པ་}{pa} или \prfB{བ་}{ba}. Отглагольное имя, которое образуется присоединением этих морфем к основе любого непереходного глагола, может обозначать либо процесс действия, либо имя деятеля. В зависимости от того, к какой временной основе присоединяются указанные морфемы, образуется отглагольное имя соответствующей временной отнесённости --- как при выражении процесса, так и при выражении деятеля. Например, от глаголов \prfC{འགྲོ་}{'gro}{<<идти>>} и \prfC{ཕྱིན་}{phyin}{<<шёл, ходил>> (прош.вр.)} образуются формы: \prfC{འགྲོ་བ་}{'gro-ba}{<<хождение>> или <<идущий>>} и \prfC{ཕྱིན་པ་}{phyin-pa}{<<ходивший>> или <<хождение>> (имевшее место в прошлом)}. Конкретное значение данного имени (т.е, выражение процесса или деятеля) вытекает из контекста. Так, например, в сочетании \prfC{འགྲོ་བའི་དུས་}{'gro-ba'i du\ul{s}}{<<время хождения>>} \prfC{འགྲོ་བ་}{'gro-ba}{(<<хождение>>)} --- процесс, а в сочетании \prfC{འགྲོ་བའི་མི་}{'gro-ba'i mi}{<<идущий человек>>} \prfC{འགྲོ་བ་}{'gro-ba}{(<<идущий>>)} --- деятель.

Отглагольное имя, образованное от основы любого переходного глагола при помощи тех же служебных морфем, может обозначать: а) имя деятеля, б) предмет, подвергающийся воздействию, в) процесс действия.

Так, от глагола \prfC{ཀློག་}{\ul{k}log}{<<читать>>} образуется именная форма \prfB{ཀློག་པ་}{\ul{k}log-pa} которая означает:
\begin{prfsample}
	\item в сочетании \prfC{ཀློག་པའི་མི་}{\ul{k}log-pa'i mi}{<<читающий человек>>} --- <<читающий>>,
	\item в сочетании \prfC{ཀློག་པའི་དཔེ་ཆ་}{\ul{k}log-pa'i \ul{d}pe-cha}{<<читаемая книга>>} --- <<читаемый>>,
	\item в сочетании \prfC{ཀློག་པའི་དུས་}{\ul{k}log-pa'i du\ul{s}}{<<время чтения>>} --- <<чтение>>.
\end{prfsample}

Путём присоединения к подобной именной форме косвенной падежной частицы \prfB{ར}{ra} образуется своеобразная деепричастная форма, например:
\begin{prfsample}
	\item \prfC{ཡོད་པར་}{yo\ul{d}-par}{<<имея>>} (\prfC{ཡོད་}{yo\ul{d}}{<<иметь>>}),
	\item \prfC{མི་འཛེམས་པར་}{mi-'dzem\ul{s}-par}{<<не боясь>>} (\prfC{འཛེམས་}{'dzem\ul{s}}{<<бояться>>}),
	\item \prfC{མི་བསད་པར་}{mi-\ul{b}sa\ul{d}-par}{<<не убив>>} (\prfC{གསོད་}{\ul{g}so\ul{d}}{<<убивать>>}).
\end{prfsample}
Такого рода деепричастные формы в тибетском языке употребляются довольно редко и в подавляющем большинстве случаев им предшествуют отрицания.

2. Причастные формы. Здесь выделяются два вида причастных форм:

1) образованные путём присоединения к основе настоящего времени морфем \prfB{བྱེད་}{bye\ul{d}}, \prfB{མཁན་}{\ul{m}khan}, \prfB{པ་པོ་}{pa-po} и обозначающие имя деятеля. Так, например, от глагола \prfC{འབྲི་}{'bri}{<<писать>>} могут быть образованы формы:
\begin{prfsample}
	\item \prfB{འབྲི་བྱེད་}{'bri-bye\ul{d}}, \prfB{འབྲི་མཁན་}{'bri-\ul{m}khan}, \prfC{འབྲི་པ་པོ་}{'bri-pa-po}{<<пишущий>>}.
\end{prfsample}

2) образованные путём прибавления к основе будущего времени переходного глагола морфемы  \prfB{བྱ་}{bya} (будущее время от глагола \prfC{བྱེད་}{bye\ul{d}}{<<делать>>}), обозначающие то, что должно быть сделано, например:
\begin{prfsample}
	\item \prfC{བསློབ་བྱ་}{\ul{bs}lob-bya}{<<то, что должно изучаться>>},
	\item \prfC{བསག་བྱ་}{\ul{b}sag-bya}{<<то, что должно накапливаться>>},
	\item \prfC{གསང་བྱ་}{\ul{g}sang-bya}{<<то, что должно быть скрываемым>>}.
\end{prfsample}
На русский язык эти (и подобные им) имена обычно можно соответственно переводить: <<изучаемое>>, <<накапливаемое>>, <<скрываемое>>. В новом письменном языке аналогичные имена образуются также прибавлением к основе будущего времени морфемы \prfB{རྒྱུ་}{\ul{r}gyu} (этот способ образования имен заимствован из разговорного).

\subsection{Категория времени}

\label{sec:glagol:kat_vremeni}Процессы, обозначаемые глаголами, могут совпадать с моментом речи (настоящее время), совершаться до момента речи (прошедшее время), мыслиться как процессы, осуществление которых произойдет после момента речи (будущее время).

Для старого письменного языка характерны два основных способа выражения времени совершения действия:

а) путём употребления различных временных основ глагола, например:
\begin{prfsample}
	\item \prfB{ཁོང་གིས་རིག་པ་}{khong-gi\ul{s}-rig-pa} $\rightarrow$
	\item \quad$\rightarrow$ \prfC{སློབ་}{\ul{s}lob}{<<он изучает науки>>}
	\item \quad$\rightarrow$ \prfC{བསླབས་}{\ul{bs}lab\ul{s}}{<<он изучал науки>>}
	\item \quad$\rightarrow$ \prfC{བསླབ་}{\ul{bs}lab}{<<он будет изучать науки>>}
\end{prfsample}
где	\prfB{སློབ་}{\ul{s}lob} --- основа настоящего времени, \prfB{བསླབས་}{\ul{bs}lab\ul{s}} --- основа прошедшего времени, \prfB{བསླབ་}{\ul{bs}lab} --- основа будущего времени;

б) посредством сложного глагола с адвербиально-предикативной связью компонентов, в котором в качестве именной части выступает герундивно-причастная форма основного
глагола, а вербальной морфемой --- \prfB{བྱེད་}{bye\ul{d}} (основа прошедшего времени \prfB{བྱས་}{bya\ul{s}}, основа будущего времени \prfB{བྱ་}{bya}) <<делать, делаться>> или \prfB{འགྱུར་}{'gyur} (основа прошедшего времени \prfB{གྱུར་}{gyur}) <<изменяться, превращаться>>.

Вербальные морфемы \prfB{བྱེད་}{bye\ul{d}} и \prfB{འགྱུར་}{'gyur} в подобного рода сложных глаголах семантически обесцветились и указывают лишь на временную отнесённость действия, например:
\begin{prfsample}
	\item \prfB{ངས་ཡི་གེ་}{nga\ul{s} yi-ge} $\rightarrow$
	\item \quad$\rightarrow$ \prfC{གཏོང་བར་བྱེད་}{\ul{g}tong-bar-bye\ul{d}}{<<я посылаю письмо>>}
	\item \quad$\rightarrow$ \prfC{བཏང་བར་བྱས་}{\ul{b}tang-bar-bya\ul{s}}{<<я посылал письмо>>}
	\item \quad$\rightarrow$ \prfC{གཏང་བར་བྱ་}{\ul{g}tang-bar-bya}{<<я пошлю письмо>>}
\end{prfsample}

Как мы видим, от различных временных основ (\prfB{གཏོང་}{\ul{g}tong}, \prfB{བཏང་}{\ul{b}tang}, \prfB{གཏང་}{\ul{g}tang}) глагола <<посылать>> образованы соответствующие герундивно-причастные формы (\prfB{གཏོང་བ་}{\ul{g}tong-ba}, \prfB{བཏང་བ་}{\ul{b}tang-ba}, \prfB{གཏང་བ་}{\ul{g}tang-ba}), а от них уже сложный глагол со значением основного глагола. В данном случае указание на время совершения действия содержится как в герундивно-причастных формах, так и в основах вербальной морфемы. Если же глагол, от которого образована герундивно-причастная форма, не имеет особых временных основ, то время выражается только через соответствующую основу вербальной морфемы. Так, глагол
\prfC{བཀུར་}{\ul{b}kur}{<<уважать>>} имеет только одну основу для всех времен. В сложном же глаголе, образованном от этого глагола, время выражается только через основы вербальной морфемы, например:

\begin{prfsample}
	\item \prfB{ཀུན་གྱིས་}{kun-gyi\ul{s}} $\rightarrow$
	\item \quad$\rightarrow$ \prfC{བཀུར་བར་བྱེད་}{\ul{b}kur-bar-bye\ul{d}}{<<все уважают>>}
	\item \quad$\rightarrow$ \prfC{བཀུར་བར་བྱས་}{\ul{b}kur-bar-bya\ul{s}}{<<все уважали>>}
	\item \quad$\rightarrow$ \prfC{བཀུར་བར་བྱ་}{\ul{b}kur-bar-bya}{<<все будут уважать>>}
\end{prfsample}

В новом письменном языке время глагола простого предложения и глагола главного предложения выражается главным образом при помощи связок (этот способ выражения времени заимствован из разговорного языка). Вместе с тем в новом письменном языке намечается тенденция при помощи связок обозначать также и лицо.

1. \emph{Настоящее время}. Внутри категории настоящего времени можно выделить конкретное настоящее время и расширенное настоящее время:

а) конкретное настоящее время выражается при помощи связок \prfB{ཡོད་}{yo\ul{d}} и \prfB{འདུག་}{'dug}, присоединяющихся к основе настоящего времени знаменательного глагола посредством одной из частиц: \prfB{ཀྱི་}{kyi}, \prfB{གི་}{gi} или \prfB{གྱི་}{gyi}, причём первому лицу соответствует связка \prfB{ཡོད་}{yo\ul{d}}, а второму и третьему лицу --- \prfB{འདུག་}{'dug} например:
\begin{prfsample}
	\item \prfC{ང་འགྲོ་གི་ཡོད་}{nga 'gro-gi-yod}{<<я иду>>}
	\item \prfC{ཁྱོད་འགྲོ་གི་འདུག་}{khyo\ul{d} 'gro-gi-'dug}{<<ты идешь>>}
	\item \prfC{ཁོང་འགྲོ་གི་འདུག་}{khong 'gro-gi-'dug}{<<он идет>>}
\end{prfsample}

Если действие совершается вторым или третьим лицом, но направлено оно на первое лицо, то также употребляется связка \prfB{ཡོད་}{yo\ul{d}}, например:
\begin{prfsample}
	\item \prfC{ཁོང་གིས་ངར་ཕྱག་རོགས་གནང་གི་ཡོད་}{khong-gi\ul{s} ngar phyag-rog\ul{s}-\ul{g}nang-gi-yo\ul{d}}{<<он помогает мне>>};
\end{prfsample}

б) расширенное настоящее время. Если действие совершается постоянно, регулярно, то это постоянство, регулярность выражается путём присоединения к основе знаменательного глагола-связки \prfB{ཡོད་}{yo\ul{d}} посредством одной из указанных частиц, например:
\begin{prfsample}
	\item \prfC{ཟམ་པ་ཏེ་ལྡེམ་ལྡེམ་བྱེད་གི་ཡོད་}{zam-pa te \ul{l}dem-\ul{l}dem-bye\ul{d}-gi-yo\ul{d}}{<<тот мост шатается>>};
	\item \prfC{ཁོང་རྟག་པར་ངུ་གི་ཡོད་}{khong-\ul{r}tag-par ngu-gi-yo\ul{d}}{<<он постоянно плачет>>}.
\end{prfsample}

2. \emph{Прошедшее время} выражается путём присоединения к герун\-див\-но-при\-част\-ной форме знаменательного глагола, образованной от основы прошедшего времени (если такая имеется) и глагола-связки \prfB{ཡིན་}{yin} или \prfB{རེད་}{re\ul{d}}\footnote[38]{Герундивно-причастная форма, образованная от основы настоящего времени, в сочетании с указанными связками может также обозначать настоящее время, например:
\begin{prfsample}
	\item \prfC{ང་འགྲོ་བ་ཡིན་}{nga 'gro-ba-yin}{<<я иду>>},
\end{prfsample}
но такие конструкции редки},
причём связка \prfB{ཡིན་}{yin} обычно употребляется с первым лицом, а связка \prfB{རེད་}{re\ul{d}} --- со вторым и третьим, например:
\begin{prfsample}
	\item \prfC{ང་ཕྱིན་པ་ཡིན་}{nga phyin-pa-yin}{<<я шёл>>};
	\item \prfC{ཁྱོད་ཕྱིན་པ་རེད་}{khyo\ul{d} phyin-pa-re\ul{d}}{<<ты шёл>>};
	\item \prfC{ཁོང་ཕྱིན་པ་རེད་}{khong phyin-pa-re\ul{d}}{<<он шёл>>};
\end{prfsample}

3. \emph{Будущее время} образуется путём присоединения к основе глагола будущего времени (если таковая имеется) связок \prfB{ཡིན་}{yin} и \prfB{རེད་}{re\ul{d}} посредством одной из следующих частиц: \prfB{ཀྱི་}{kyi}, \prfB{གི་}{gi}, \prfB{གྱི་}{gyi}, причём связка
\prfB{ཡིན་}{yin} обычно употребляется с первым лицом, а связка \prfB{རེད་}{re\ul{d}} --- со вторым и третьим, например:
\begin{prfsample}
	\item \prfC{ང་འགྲོ་གི་ཡིན་}{nga 'gro-gi-yin}{<<я пойду>>};
	\item \prfC{ཁྱོད་འགྲོ་གི་རེད་}{khyo\ul{d} 'gro-gi-re\ul{d}}{<<ты пойдешь>>};
	\item \prfC{ཁོང་འགྲོ་གི་རེད་}{khong 'gro-gi-re\ul{d}}{<<он пойдет>>};
\end{prfsample}

Будущее время может быть также выражено сочетанием служебной морфемы \prfB{རྒྱུ་}{\ul{r}gyu} с указанными выше связками. В этом случае в высказывание привносится оттенок долженствования, необходимости выполнения данного действия, например:
\begin{prfsample}
	\item\prfC{ང་འགྲོ་རྒྱུ་ཡིན་}{nga 'gro-\ul{r}gyu-yin}{<<я (обязательно) пойду>>};
	\item\prfC{ཁྱོད་འགྲོ་རྒྱུ་རེད་}{khyo\ul{d} 'gro-\ul{r}gyu-re\ul{d}}{<<ты (обязательно) пойдешь>>};
	\item\prfC{ཁོང་འགྲོ་རྒྱུ་རེད་}{khong 'gro-\ul{r}gyu-re\ul{d}}{<<он(обязательно) пойдет>>};
\end{prfsample}

Будущее время может быть выражено и при помощи сочетания глагольной основы с вербальной морфемой \prfC{ཡོང་}{yong}{(<<приходить>>)} в служебной функции, например:
\begin{prfsample}
	\item \prfC{ང་ལམ་སང་སླེབ་ཡོང་}{nga lam-sang \ul{s}leb-yong}{<<я немедленно прибуду>>}.
\end{prfsample}
Обычно сочетание глагольной основы с морфемой \prfB{ཡོང་}{yong} означает, что действие наступит при каких-то определённых условиях, например:
\begin{prfsample}
	\item \prfC{ལམ་དཀྱིལ་དུ་དོང་མ་རྐོ། མི་དམངས་ལ་གནོད་ཡོང་}{lam-\ul{d}kyil-du dong ma \ul{r}ko, mi-\ul{d}mang\ul{s}-la-\ul{g}no\ul{d}-yong}{<<не копай яму посреди дороги, людям навредишь>> (т.е. если будешь копать яму, то навредишь людям)};
	\item \prfC{རང་སྐྱོན་རང་ལ་འཁོར་ཡོང་}{rang \ul{s}kyon rang-la 'khor-yong}{<<сoбственные недостатки на тебя обратятся>> (т.е. если имеешь недостатки, то они на тебе скажутся)}.
\end{prfsample}

В отрицательных конструкциях отрицание ставится перед гла\-го\-лом-связ\-кой (если это связки \prfB{རེད་}{re\ul{d}} или \prfB{འདུག་}{'dug}), например:
\begin{prfsample}
	\item \prfC{ཁོང་འགྲོ་གི་མ་རེད་}{khong 'gro-gi-ma-re\ul{d}}{<<он не пойдет>>};
	\item \prfC{ཁོང་འགྲོ་གི་མི་འདུག་}{khong 'gro-gi-mi-'dug}{<<он не идет>>}.
\end{prfsample}
При связках	\prfB{ཡོད་}{yo\ul{d}} или \prfB{ཡིན་}{yin} в отрицательной конструкции употребляется их негативная форма (\prfB{མེད་}{me\ul{d}}, \prfB{མིན་}{min}), например:
\begin{prfsample}
	\item \prfC{ང་འགྲོ་རྒྱུ་མིན་}{nga 'gro-\ul{r}gyu-min}{<<я не пойду>>};
	\item \prfC{ང་འགྲོ་གི་མེད་}{nga 'gro-gi-me\ul{d}}{<<я не иду>>}.
\end{prfsample}

В новом письменном языке существует сложная временная форма (заимствована из разговорного языка), указывающая на то, что в высказывании логическое ударение акцентируется на факте действия. Если внимание обращается на факт наличия действия в прошлом, то к основе прошедшего времени знаменательного глагола присоединяется конструкция, состоящая из герундивно-причастной формы от глагола \prfB{ཡོད་}{yo\ul{d}} и связки \prfB{རེད་}{re\ul{d}}, т.е. \prfB{ཡོད་པ་རེད་}{yo\ul{d}-pa-re\ul{d}}. Если же обращается внимание на факт наличия действия в настоящем или будущем, то эта конструкция присоединяется к соответствующей основе знаменательного глагола посредством одной из частиц (\prfB{ཀྱི་}{kyi}, \prfB{གི་}{gi} или \prfB{གྱི་}{gyi}), например:
\begin{prfsample}
	\item \prfC{ཁོང་ཚོས་གྲོ་དང་ནས་སྲན་མ་གསུམ་བཏབ་ཡོད་པ་རེད་}{khong-tsho\ul{s} gro dang nas s\ul{r}an-ma \ul{g}sum \ul{b}tab-yo\ul{d}-pa-re\ul{d}}{<<они посеяли пшеницу, ячмень и горох>>};
	\item \prfC{གཙང་པོ་ཆུ་རྒྱ་མཚོའི་ནང་ལ་འགྲོ་གི་ཡོད་པ་རེད་}{gtsang-po-chu \ul{r}gya-\ul{m}tsho'i nang-la 'gro-gi-yo\ul{d}-pa-re\ul{d}}{<<река Цанпо впадает в море>>}.
\end{prfsample}

В тибетском языке существует группа служебных морфем, употребляющихся в служебной функции, которые, сочетаясь с глаголом, характеризуют обозначаемое им действие с точки зрения его протекания, причём в ряде случаев соотносят его определённым образом с моментом речи. Рассмотрим эти морфемы.

(1)	Служебные морфемы \prfB{བཞིན་}{\ul{b}zhin} и \prfB{མུས་}{mu\ul{s}}, а также морфемы-варианты \prfB{ཀྱིན་}{kyin}, \prfB{གིན་}{gin} и \prfB{གྱིན་}{gyin} указывают на длительность совершаемого действия, например:
\begin{prfsample}
	\item \prfC{ང་འགྲོ་བཞིན་ཡོད་}{nga 'gro-\ul{b}zhin-yo\ul{d}}{<<я иду>>};
	\item \prfC{ཁོང་འགྲོ་གྱིན་འདུག་}{khong 'gro-gyin-'dug}{<<он идет>>};
	\item \prfC{ཁིང་ལྷ་སར་བསྡད་མུས་རེད་}{khing lha-sar \ul{bs}dad-mu\ul{s}-re\ul{d}}{<<он проживает в Лхасе>>}.
\end{prfsample}
Эти служебные морфемы ставятся между основой глагола и глаголом-связкой. При наличии отрицания оно ставится перед глаголом-связкой.

Основа знаменательного глагола и служебная морфема \prfB{བཞིན་}{\ul{b}zhin} могут образовывать деепричастные формы, например:
\begin{prfsample}
	\item \prfC{བྱེད་བཞིན་པར་}{byed-\ul{b}zhin-par}{<<делая>>};
	\item \prfC{ངུ་བཞིན་པར་}{ngu-\ul{b}zhin-par}{<<плача>>}.
\end{prfsample}

(2)	Морфемы \prfB{བྱུང་}{byung} (основа прошедшего времени от глагола \prfC{འབྱུང་}{'byung}{<<возникать>>}) и \prfB{སོང་}{song} (основа прошедшего времени от глагола \prfC{འགྲོ་}{'gro}{<<идти>>}), присоединяясь к основе знаменательного глагола, указывают, что действие, совершившееся в прошлом, достигло своего результата, например:
\begin{prfsample}
	\item \prfC{མཐོང་བྱུང་}{\ul{m}thong-byung}{<<увидел>>};
	\item \prfC{བཙུངས་བྱུང་}{\ul{b}tsung\ul{s}-byung}{<<уколол>>};
	\item \prfC{བཟོས་བྱུང་}{\ul{b}zo\ul{s}-byung}{<<сделал>>};
	\item \prfC{བརླག་སོང་}{\ul{br}lag-song}{<<потерял>>};
	\item \prfC{བརྫེད་སོང་}{\ul{br}dze\ul{d}-song}{<<забыл>>};
	\item \prfC{བཏུབ་སོང་}{\ul{b}tub-song}{<<разрезал>>};
\end{prfsample}

(3)	Морфемы	\prfB{ཡོད་}{yo\ul{d}} и	\prfB{འདུག་}{'dug} --- связки наличия, присоединяясь к основе прошедшего времени, указывают на то, что хотя действие совершено в прошлом, но результат его сохранился до настоящего времени, например:
\begin{prfsample}
	\item \prfC{ཞིང་འབངས་ཀྱི་རྡོག་ཁྲེས་ཀྱང་ཆུང་དུ་ཕྱིན་ཡོད་}{zhing-'bang\ul{s}-kyi \ul{r}dog-khre\ul{s} kyang chung-du-phyin-yo\ul{d}}{<<повинности крестьянства также уменьшились>>}.
\end{prfsample}

(4)	Морфемы \prfC{ཆར་}{char}{<<завершать>>}, \prfB{གྲུབ་}{grub} (прошедшее время от глагола \prfC{འགྲུབ་}{'grub}{<<исполняться>>}), \prfC{ཟིན་}{zin}{<<быть выполненным>>} в сочетании с основой прошедшего времени знаменательного глагола указывают, что действие полностью завершено, например:
\begin{prfsample}
	\item \prfC{ཁ་ལག་ཟས་ཆར་པ་རེད་}{kha-lag za\ul{s}-char-pa-re\ul{d}}{<<пища съедена>>};
	\item \prfC{ཡི་གེ་བྲིས་གྲུབ་པ་རེད་}{yi-ge bri\ul{s}-grub-pa-re\ul{d}}{<<письмо написано>>};
	\item \prfC{རྨང་གཞི་བཏིང་ཟིན་པ་རེད་}{\ul{r}mang-\ul{g}zhi \ul{b}ting-zin-pa-re\ul{d}}{<<база заложена>>}.
\end{prfsample}

(5)	Морфема \prfB{མྱོང་}{myong} хотя и сочетается с основой глагола в настоящем времени, однако также относит действие в план прошедшего времени. Она указывает на то, что данное действие совершалось и закончено в прошлом, причём совершалось один или несколько раз и субъект имеет опыт этого
действия (сопоставима с китайской морфемой {\chinfont 過} \emph{го}), например:
\begin{prfsample}
	\item \prfC{བོད་ལ་འགྲོ་མྱོང་}{bo\ul{d}-la 'gro-myong}{<<хаживал в Тибет>>};
	\item \prfC{ཁ་ལག་དེ་ཟ་མ་མྱོང་}{kha lag de za ma myong}{<<этой пищи не едал>>};
	\item \prfC{བྱེད་མྱོང་}{bye\ul{d}-myong}{<<делывал>>}.
\end{prfsample}

\subsection{Категория наклонения}

Категория наклонения указывает на различия в отношении обозначаемых глаголом действий к их реальному осуществлению.

1. \emph{Изъявительное наклонение} обозначает, что действие происходит на самом деле, фактически осуществляется, например:
\begin{prfsample}
	\item \prfC{དཔེ་ཆ་ས་ལ་བཟར་སོང་}{\ul{d}pe-cha sa-la \ul{b}zar-song}{<<книга упала на землю>>}.
\end{prfsample}

2. \emph{Повелительное наклонение} выражает волю говорящего, направленное к другому лицу побуждение к совершению действия. Образуется сочетанием основы повелительного наклонения с одной из следующих морфем-вариантов: \prfB{ཅིག་}{cig}, \prfB{ཞིག་}{zhig}, \prfB{ཤིག་}{shig}, например:
\begin{prfsample}
	\item \prfC{གསུངས་ཤིག་}{\ul{g}sung\ul{s}-shig}{<<говори!>>};
	\item \prfC{ཚོངས་ཤིག་}{tshong\ul{s}-shig}{<<продавай!>>}.
\end{prfsample}

В вежливой речи повелительное наклонение образуется сочетанием основы повелительного наклонения с морфемой \prfB{དང་}{dang} например:
\begin{prfsample}
	\item \prfC{ལྟོས་དང་}{\ul{l}to\ul{s}-dang}{<<смотрите!>>};
	\item \prfC{སོང་དང་}{song-dang}{<<идите!>>}.
\end{prfsample}

3. \emph{Запретительное наклонение} выражает волю говорящего, направленное к другому лицу побуждение не совершать действия. Образуется сочетанием частицы отрицания с основой настоящего времени, например:
\begin{prfsample}
	\item \prfC{མ་བྱེད་}{ma-bye\ul{d}}{<<не делай!>>};
	\item \prfC{མ་རྐོ་}{ma-\ul{r}ko}{<<не копай!>>};
	\item \prfC{མ་འགྲོ་}{ma-'gro}{<<не ходи!>>}.
\end{prfsample}

4. \emph{Уступительное наклонение}. При этом наклонении имеется не одна, а две ситуации. Первая ситуация выражается предложением, где сказуемым выступает глагольная основа прошедшего времени\footnote[39]{Употребление глагольной основы прошедшего времени в данном случае отнюдь не означает, что действие, выражаемое этой основой, уже совершилось, а указывает на последовательность действий, т.е. сначала должно совершиться первое действие, а затем уже второе.}, за которой следует условный союз \prfC{ན་}{na}{<<если>>}. Это предложение представляет собой условие или предпосылку осуществления второй ситуации, которая выражена следующим предложением, например:
\begin{prfsample}
	\item \prfC[pp]{རི་དྭགས་རྒྱལ་པོ་བཀྲེན་ན། བླང་ཆེན་སྤྱི་བོ་མྱུར་དུ་འགེམས་}{ri-d\ul{w}ags \ul{r}gyal-po \ul{b}kren na, \ul{b}lang-chen \ul{s}pyi-bo myur-du 'gem\ul{s}}{<<если проголодается царь зверей, [он] быстро разобьет череп и слону>>}.
\end{prfsample}

\section{Междометие}

Междометие --- часть речи, служащая для выражения эмоций. Междометия выступают как своеобразные эквиваленты предложений или примыкают к предложениям, придавая им эмоциональную окраску. В тибетском языке можно выделить следующие группы междометий:

1. Выражающие печаль, горе, скорбь, страдание (<<Ах! Ох! Увы! Горе!>>):
\prfB{ཀྱེ་མ་}{kye-ma},
\prfB{ཀྱེ་ཧུད་}{kye-hu\ul{d}},
\prfB{ཀྱི་ཧུད་}{kyi-hu\ul{d}},
\prfB{འོད་དོད་}{'od-do\ul{d}},
\prfB{ཨ་ཏ་ད་}{a-ta-da},
\prfB{ཧ་ཧ་བ་}{ha-ha-ba}.

2. Выражающие удивление (<<Ого! Ну и ну!>>):
\prfB{ཨེ་མ་}{e-ma},
\prfB{ཨེ་མ་དོ་}{e-ma-do},
\prfB{ཨ་དོ་}{a-do},
\prfB{ཨ་དོ་ཨ་དོ་}{a-do-a-do}.

3. Выражающие одновременно и радость и удивление: \prfB{ཧ་ཧ་}{ha-ha}.

4. Выражающие одобрение и согласие (<<Хорошо! Прекрасно! Конечно!>>):
\prfB{ཨ་ཨ་}{a-a},
\prfB{ཡ་ཡ་}{ya-ya},
\prfB{ཨུ་ཧོ་}{u-ho},
\prfB{རིགས་སོ་}{rig\ul{s}-so},
\prfB{ཡིན་ནོ་}{yin-no},
\prfB{ལེགས་སོ་}{leg\ul{s}-so},
\prfB{བཞིན་ནོ་}{\ul{b}zhin-no},
\prfB{ངེས་སོ་}{nge\ul{s}-so},
\prfB{བདེན་ནོ་}{\ul{b}den-no}.

5. Выражающие порицание, неодобрение (<<Фи! Фу! Тьфу!>>):
\prfB{ཨ་ཁ་}{a-kha},
\prfB{ཨ་ཁ་ཁ་}{a-kha-kha},
\prfB{འ་ཁ་}{'a-kha},
\prfB{འ་ཁ་ཁ་}{'a-kha-kha}.

6. Выражающие ужас, отвращение:
\prfB{ཨ་ཚི་}{a-tshi},
\prfB{ཨ་ཚི་ཚི་}{a-tshi-tshi}.

7. Выражающие страдания от холода:
\prfB{ཨ་ཆུ་}{a-chu},
\prfB{ཨ་ཆུ་ཆུ་}{a-chu-chu}.

8. Выражающие страх, беспокойство:
\prfB{ཨ་ར་}{a-ra},
\prfB{ཨ་ན་}{a-na},
\prfB{ཨ་ར་ར་}{a-ra-ra},
\prfB{ཨ་ན་ན་}{a-na-na}.

9. Выражающие страдание от жары:
\prfB{ཨ་ཚ་}{a-tsha},
\prfB{ཨ་ཚ་ཚ་}{a-tsha-tsha}.

К междометиям можно отнести восклицания-обращения, как например: мистическое восклицание, обращаемое к Будде: \prfB{ༀ}{om} и обращения к человеку (<<Эй, послушай!>>):
\prfB{ཀྱེ་}{kye},
\prfB{ཀྱེ་ཀྱེ་}{kye-kye},
\prfB{ཀྱེ་ཧོ་}{kye-ho},
\prfB{ཀྭ་}{k\ul{w}a},
\prfB{ཀྭ་ཡེ་}{k\ul{w}a-ye},
\prfB{ཨ་ལེ་}{a-le},
\prfB{ཨ་རེ་}{a-re},
\prfB{ཧེ་}{he},
\prfB{ཧེ་ཧེ་}{he-he},
\prfB{ཝ་}{wa},
\prfB{ཝ་ཡི་}{wa-yi}.
