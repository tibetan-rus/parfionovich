\section{Имя прилагательное}

Имя прилагательное в тибетском --- часть речи, обозначающая качество предмета. Категория относительных прилагательных в тибетском языке отсутствует, соответствующее значение передаётся существительным, выступающим в роли определения, например:
\begin{prfsample}
    \item \prfC{ཤིང་གི་ཁྲི་}{shing-gi-khri}{<<деревянное сидение>>}, где \prfC{ཤིང་}{shing}{<<дерево>>}.
\end{prfsample}

Морфологически прилагательное характеризуется: а) наличием категорий степеней сравнения; б) своими средствами и способами словообразования.

Синтаксически прилагательное характеризуется тем, что:
а) выступает в предложении определением и именной частью сказуемого;
б) сочетается с существительным и стоит в правой позиции по отношении к нему\footnote[27]{В крайне редких случаях (вызванных требованиями стиля) и при отсутствии другого определения, прилагательное может стоять в левой позиции, обязательно принимая при этом частицу притяжательного падежа.};
в) сочетается с отрицаниями и наречиями типа <<очень>>, <<весьма>>.

\subsection{Структура и способы образования прилагательных}

Все прилагательные можно разделить на производные и сложные (простые прилагательные в тибетском языке отсутствуют).

1. \emph{Производные прилагательные} образуются сочетанием предикативных морфем, указывающих на качество субъекта (типа: \prfC{ལྕི་}{\ul{l}ci}{<<быть тяжёлым>>}, \prfC{མཐོ་}{\ul{m}tho}{<<быть высоким>>}, \prfC{མཛེས་}{\ul{m}dze\ul{s}}{<<быть красивым>>}), со служебными словообразующими морфемами \prfB{པོ་}{po} (наиболее продуктивна), \prfB{པ་}{pa}, \prfB{མ་}{ma}, \prfB{མོ་}{mo} и \prfB{བ་}{ba} (наименее продуктивна), например:
\begin{prfsample}
    \item \prfC{དཀར་པོ་}{\ul{d}kar-po}{<<белый>>},
    \item \prfC{རྙིང་པ་}{\ul{r}nying-pa}{<<старый>>},
    \item \prfC{གཙང་མ་}{\ul{g}tsang-ma}{<<чистый>>},
    \item \prfC{མངར་མོ་}{\ul{m}ngar-mo}{<<сладкий>>},
    \item \prfC{ཁ་བ་}{kha-ba}{<<горький>>}.
\end{prfsample}

2. \emph{Сложные прилагательные} образуются:

а) путём редупликации предикативных морфем, указывающих на качество предмета, например:
\begin{prfsample}
    \item \prfC{ཐུང་ཐུང་}{thung-thung}{<<короткий>>},
    \item \prfC{སྒོར་སྒོར་}{\ul{s}gor-\ul{s}gor}{<<круглый>>},
    \item \prfC{ཆུང་ཆུང་}{chung-chung}{<<маленький>>};
\end{prfsample}

б) посредством лексикализации словосочетаний, состоящих из прилагательного и существительного\footnote[28]{Сложные прилагательные данного вида встречаются только в новом письменном языке.}, например:
\begin{prfsample}
    \item \prfC{རྒྱ་}{\ul{r}gya}{<<ширина>>} + \prfC{ཆེན་པོ་}{chen-po}{<<большой>>} --- \prfC{རྒྱ་ཆེན་པོ་}{\ul{r}gya-chen-po}{<<широкий>>};
    \item \prfC{ལག་པ་}{lag-pa}{<<рука>>} + \prfC{མཁས་པོ་}{\ul{m}kha\ul{s}-po}{<<знающий>>} --- \prfC{ལག་པ་མཁས་པོ་}{lag-pa-\ul{m}kha\ul{s}-po}{<<искусный, умелый>>}.
\end{prfsample}
Образования подобного рода мы относим к разряду сложных прилагательных на том основании, что они могут выступать определением к существительному точно так же, как и производное прилагательное, например:
\begin{prfsample}
    \item \prfC{ལས་འགུལ་རྒྱ་ཆེན་པོ་}{la\ul{s}-'gu\ul{l} \ul{r}gya-chen-po}{<<широкое движение>>} (\prfC{ལས་འགུལ་}{la\ul{s}-'gu\ul{l}}{<<движение>>});
    \item \prfC{མི་ལག་པ་མཁས་པོ་}{mi lag-pa-\ul{m}kha\ul{s}-po}{<<умелый, искусный человек>>} (\prfC{མི་}{mi}{<<человек>>}).
\end{prfsample}

\subsection{Категория степеней сравнения прилагательных}\label{sec:prilagatelnoe:kat_step_srav}

1. \emph{Сравнительная степень} прилагательных выражается при помощи служебных морфем \prfB{ལས་}{la\ul{s}}, \prfB{བས་}{ba\ul{s}} или \prfB{པས་}{pa\ul{s}}\footnote[29]{\prfB{བས་}{ba\ul{s}} и \prfB{པས་}{pa\ul{s}} --- два варианта одной морфемы. Их употребление зависит от написания предыдущего слога. Если предыдущий слог оканчивается на приписные \prfA{ང}, \prfA{འ}, \prfA{ར}, \prfA{ལ} или не имеет приписных, то употребляется \prfB{བས་}{ba\ul{s}} в остальных случаях --- \prfB{པས་}{pa\ul{s}}.}, например:
\begin{prfsample}
    \item \prfB{འདི་ལས་དེ་བཟང་}{'di la\ul{s} de \ul{b}zang} или \prfC{འདི་བས་དེ་བཟང་}{'di ba\ul{s} de \ul{b}zang}{<<то лучше, чем это>>};
    \item \prfC{ཁང་པ་དེ་ལས་ཁང་པ་འདི་ཆེན་པོ་རེད་}{khang-pa de la\ul{s} khang-pa 'di chen-po re\ul{d}}{<<этот дом больше, чем тот>>}.
\end{prfsample}
В отличие от морфем \prfB{བས་}{ba\ul{s}} и \prfB{པས་}{pa\ul{s}}, морфема \prfB{ལས་}{la\ul{s}} может употребляться в живой речи\footnote[30]{В разговорном языке сравнительная степень может также выражаться синтетически. При этом второй слог производного прилагательного заменяется на конечный согласный первого слога, например:
\begin{prfsample}
    \item \prfC{ཡག་པོ་}{yag-po}{<<хорошо>>} --- \prfC{ཡག་ག་}{yag-ga}{<<лучше>>};
    \item \prfC{ཆུང་ཆུང་}{chung-chung}{<<маленький>>} --- \prfC{ཆུང་ང་}{chung-nga}{<<меньше>>}.
\end{prfsample}
Иногда эти формы встречаются и в письменном языке.}.

2. \emph{Превосходная степень} сравнения выражается при помощи служебной морфемы \prfB{ཤོས་}{sho\ul{s}}, которая присоединяется к знаменательной морфеме прилагательного, занимая место исходной служебной морфемы, например:
\begin{prfsample}
    \item \prfC{ཡག་པོ་}{yag-po}{<<хороший>>} --- \prfC{ཡག་ཤོས་}{yag-sho\ul{s}}{<<наилучший>>};
    \item \prfC{ཆུང་ཆུང་}{chung-chung}{<<маленький>>} --- \prfC{ཆུང་ཤོས་}{chung-sho\ul{s}}{<<наименьший>>};
    \item \prfC{ཞོན་ཞོན་}{zhon-zhon}{<<молодой>>} --- \prfC{ཞོན་ཤོས་}{zhon-sho\ul{s}}{<<самый молодой, моложе всех>>}.
\end{prfsample}

Превосходная степень может выражаться также лексически:

1) при помощи особых оборотов
\prfB{ཚང་མའི་ནང་ནས་}{tshang-ma'i nang na\ul{s}},
\prfB{གང་ཁའི་དཀྱིལ་ན་}{gang-kha'i \ul{d}kyi\ul{l} na} или
\prfC{ཐམས་ཅད་ལས་}{tham\ul{s}-ca\ul{d} la\ul{s}}{<<изо всех>>}, например:
\begin{prfsample}
    \item \prfC{ཐམས་ཅད་ལས་ཡག་པོ་རེད་}{tham\ul{s}-ca\ul{d} la\ul{s} yag-po re\ul{d}}{<<изо всех этот наилучший>> (букв. <<изо всех это хороший>>)};
\end{prfsample}

2)	при помощи слов \prfC{ཤིན་ཏུ་}{shin-tu}{<<очень>>},
\prfC{རབ་ཏུ་}{rab-tu}{<<весьма>>}, \prfB{ཞེ་སྐྲག་}{zhe-\ul{s}krag} или \prfC{ཞེ་དྲག་}{zhe-drag}{<<крайне, чрезвычайно>>} (в новом письменном языке), например:
\begin{prfsample}
    \item \prfC{ཤིན་ཏུ་གསལ་པོ་}{shin-tu \ul{g}sal-po}{<<очень ясный>>};
    \item \prfC{ཡག་པོ་ཞེ་དྲག་རེད་}{yag-po zhe-drag re\ul{d}}{<<очень хороший>>}.
\end{prfsample}

\subsection{Синтаксические характеристики прилагательных}

1. Прилагательное обычно выступает в предложении:

а) \emph{определением} в правой позиции, контактно примыкая к  опре\-де\-ля\-емо\-му-су\-ще\-стви\-тель\-но\-му, например:
\begin{prfsample}
    \item \prfC{མི་བཟང་པོ་}{mi \ul{b}zang-po}{<<хороший человек>>};
    \item \prfC{མེ་ཏོག་དཀར་པོ་}{me-tog \ul{d}kar-po}{<<белый цветок>>};
\end{prfsample}

б) \emph{именной частью сказуемого}\prfnote{Составным именным сказуемым называется сказуемое, которое состоит из именной части (существительное, прилагательное\dots{}) и глагола-связки (быть, казаться\dots{}).}, например:
\begin{prfsample}
    \item \prfC{སྡོས་སའི་ས་ཆ་སྐྱིད་པོ་རེད་}{\ul{s}do\ul{s}-sa'i sa-cha \ul{s}kyi\ul{d}-po re\ul{d}}{<<местожительство комфортабельно>>};
    \item \prfC{སྟོན་ཐོག་ཡག་པོ་རེད་}{ston-thog yag-po re\ul{d}}{<<урожай хорош>>}.
\end{prfsample}

2. Прилагательное сочетается с частицами отрицания \prfB{མ་}{ma} и \prfB{མི་}{mi}, например:
\begin{prfsample}
    \item \prfC{མི་སྡུག་པ་}{mi \ul{s}dug-pa}{<<некрасивый>>};
    \item \prfC{མི་ཡག་པོ་}{mi yag-po}{<<нехороший>>}.
\end{prfsample}

