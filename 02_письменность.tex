%TODO: заменить longtable на tabu
\chapter{Письменность}
\section{Тибетский алфавит и его графемы}
Тибетские национальные грамматики делят тибетские графемы\prfnote{Графема --- минимальная единица письменности: в алфавитных системах письма --- буква (или другое отражение фонемы), в неалфавитных системах письма --- слоговой знак, иероглиф, идеограмма и другие.} на две категории: \prfC{གསལ་བྱེད་}{сэльце}{(букв. 'объясняющие, делающие ясным')} --- корневые, или основные, графемы и	\prfC{དབྱངས་}{янг}{(букв. 'звучные')} --- некорневые графемы, или графемы-диакритики\footnote[9]{Все европейские грамматики тибетского языка называют первые графемы <<согласными буквами>>, а вторые --- <<гласными буквами>>. Такая терминология, по нашему мнению, неточна. Вернее было бы тибетские графемы противопоставлять по линии <<корневые (основные) --- некорневые (вспомогательные)>>, что отражает их действительные различия на уровне графики. Термины же <<согласные буквы>> и <<гласные буквы>> отражают раз-личие тибетских графем лишь в фонетическом плане, не указывая при этом на их графическую неоднородность. Однако в отдельных случаях, когда необходимо отразить фонетическую значимость графем, мы будем пользоваться и терминами <<согласная>> и <<гласная>>, имея в виду соответственно корневые и некорневые графемы.}.

\emph{Корневые графемы} составляют тибетский алфавит, насчитывающий 30 знаков. Каждая корневая графема тибетского языка, за исключением графем \prfA{འ} и \prfA{ཨ} (см. таб. \ref{tab:1}), записывает слог, состоящий из двух элементов (звуков): согласного и присущего гласного \emph{a}, который на письме не обозначается. Каждому слогу присущ тон: либо ровный высокого регистра (\toneR), либо восходящий (\toneV). Кроме этих двух основных тонов можно выделить два дополнительных тона: восходяще-падающий (\toneVN) и падающий (\toneN), которые возникают при произношении двух корневых графем, когда одна приписывается к другой (см. \hyperref[sec:gss]{<<Графическая структура слога и его произношение>>}).

Все корневые графемы тибетского алфавита располагаются в определенном порядке: слева направо по классам и сверху вниз по рядам (см. таб. \ref{tab:1}).

\begin{longtable}[H]{|Sc|Sc|Sc|Sc|Sc|}
	\caption{Корневые графемы тибетского языка}\label{tab:1}
	\\
		\hline
	\diagbox{Класс}{Ряд} & 1 & 2 & 3 & 4\\
	\hline
	1 & \prfB{ཀ}{ka} \toneR & \prfB{ཁ}{k'a} \toneR & \prfB{ག}{k'a} \toneV & \prfB{ང}{\.{n}a} \toneV \\
	2 & \prfB{ཅ}{tsja} \toneR & \prfB{ཆ}{ts'ja} \toneR & \prfB{ཇ}{ts'ja} \toneV & \prfB{ཉ}{nja} \toneV \\
	3 & \prfB{ཏ}{ta} \toneR & \prfB{ཐ}{t'a} \toneR & \prfB{ད}{t'a} \toneV & \prfB{ན}{na} \toneV \\
	4 & \prfB{པ}{pa} \toneR & \prfB{ཕ}{p'a} \toneR & \prfB{བ}{p'a} \toneV & \prfB{མ}{ma} \toneV \\
	5 & \prfB{ཙ}{tsa} \toneR & \prfB{ཚ}{ts'a} \toneR & \prfB{ཛ}{ts'a} \toneV & \prfB{ཝ}{wa} \toneV \\
	6 & \prfB{ཞ}{sja} \toneV & \prfB{ཟ}{sa} \toneV & \prfB{འ}{a} \toneV & \prfB{ཡ}{ja} \toneV \\
	7 & \prfB{ར}{ra} \toneV & \prfB{ལ}{la} \toneV & \prfB{ཤ}{sja} \toneR & \prfB{ས}{sa} \toneR \\
	8 & \prfB{ཧ}{ha} \toneR & \prfB{ཨ}{a} & & \\
	\hline
\end{longtable}

Тибетские национальные грамматики делят корневые графемы от \prfB{ཀ}{ka} до \prfB{ཧ}{ha} на пять категорий в зависимости от твердости \prfA{(དམ་པ་)} произношения, называя эти категории родовыми:

\begin{longtable}[H]{Sc|Sc}
	Родовая категория & Согласные\\
	Мужской род \prfA{(ཕོ་)} & \prfA{ཀ}, \prfA{ཅ}, \prfA{ཏ}, \prfA{པ}, \prfA{ཙ} (произносятся наиболее твердо)\\
	Средний род \prfA{(མ་ནིང་)} & \prfA{ཁ}, \prfA{ཆ}, \prfA{ཐ}, \prfA{ཕ}, \prfA{ཚ}\\
	Женский род \prfA{(མོ་)} & \prfA{ག}, \prfA{ཇ}, \prfA{ད}, \prfA{བ}, \prfA{ཛ}, \prfA{ཝ}, \prfA{ཞ}, \prfA{ཟ}, \prfA{འ}, \prfA{ཡ}\\
	Совершенно женский род \prfA{(ཤིན་ཏུ་མོ་)} & \prfA{ཤ}, \prfA{ས}, \prfA{ང}, \prfA{ཉ}, \prfA{ན}, \prfA{མ}\\
	Нейтральный род \prfA{(མོ་གཤམ་)} & \prfA{ར}, \prfA{ལ}, \prfA{ཧ} (произносятся наиболее неопределенно)
\end{longtable}

Приведенное выше деление согласных на соответствующие родовые категории связано с позицией корневых морфем в слоге (см. разд. \hyperref[sec:gss]{<<Графическая структура слога и его произношение>>}).

\emph{Некорневые графемы}, или \emph{графемы-диакритики}, представляют собой четыре огласовки: \prfB{ ི}{(i)}, \prfB{ ུ}{(u)}, \prfB{ ེ}{(e)}, \prfB{ ོ}{(o)}. Первая, третья и четвертая надписываются к корневой графеме, вторая --- подписывается к ней, при этом соответственно меняется качество гласного \textit{(а)}, присущего корневой графеме, например: \prfB{ཀ}{ka} {\unifont →} \prfB{ཀི}{ki}, \prfB{ཀུ}{ku}, \prfB{ཀེ}{ke},\prfB{ཀོ}{ko}. Корневая графема может сочетаться только с одной некорневой графемой, обозначающей огласовку.

\section{Графическая структура слога и его произношение}
\label{sec:gss}

На письме различаются простой и производный слоги. Простой слог состоит только из одной корневой графемы.

К такой корневой графеме могут присоединяться другие графемы, в результате чего образуется производный слог, причем данная корневая графема становится главным элементом слога --- основой, а остальные графемы --- составляющими слога.

По тибетской традиции усложнение производного слога идет в следующем порядке:
\begin{longtable}[H]{p{1cm}p{10cm}}
	{\unifont ①} --- & основа слога, в качестве которой может выступить любая из 30 корневых графем тибетского алфавита.\\
	{\unifont ②} --- & приписные графемы \prfA{(རྗེས་འཇུག་)}, в качестве которых могут выступать 10 следующих корневых графем: \prfA{ག}, \prfA{ང}, \prfA{ད}, \prfA{ན}, \prfA{མ}, \prfA{བ}, \prfA{འ}, \prfA{ར}, \prfA{ལ}, \prfA{ས}.\\
	{\unifont ③} --- & вторичная приписная графема \prfA{(ཡང་འཇུག་)}, в качестве которой выступает корневая графема \prfA{ས}.\\
	{\unifont ④} --- & графемы-диакритики \prfA{ ི}, \prfA{ ུ}, \prfA{ ེ}, \prfA{ ོ}.\\
	{\unifont ⑤} --- & подписные графемы \prfA{(འདོགས་ཅན་)}, в качестве которых выступают корневые графемы \prfA{ཝ}, \prfA{ཡ}, \prfA{ར}, \prfA{ལ}.\\
	{\unifont ⑥} --- & префиксальные графемы \prfA{(སྔོན་ཨཇུག་)}, в качестве которых выступают корневые графемы \prfA{ག}, \prfA{ད}, \prfA{བ}, \prfA{མ}, \prfA{འ}.\\
	{\unifont ⑦} --- & надписные графемы \prfA{(མགོ་ཅན་)}, в качестве ко¬торых выступают корневые графемы \prfA{ར}, \prfA{ལ}, \prfA{ས}.
\end{longtable}

Таким образом, из 30 корневых графем в качестве составляющих элементов слога могут выступать только 12 корневых графем ( \prfA{ག}, \prfA{ང}, \prfA{ད}, \prfA{ན}, \prfA{བ}, \prfA{མ}, \prfA{ཝ}, \prfA{འ}, \prfA{ཡ}, \prfA{ར}, \prfA{ས}, \prfA{ལ} ) \footnote[10]{Одна и та же корневая графема может выступатьв качестве разных составляющих элементов слога.}.

Из изложенного выше видно, что тибетский слог конструируется как по горизонтали (приписные и префиксальные графемы), так и по вертикали (графемы-диакритики, подписные и надписные графемы). Графически и схематически различные типы тибетских слогов от простого до самого сложного могут выглядеть следующим образом:

1) \prfA{ག་} {\unifont →} {\unifont ①}; 2) \prfA{གངས་} {\unifont →} {\unifont ①②③};
3) \prfA{གིང་} {\unifont →}
\begin{tikzpicture}[baseline=0pt]
	\draw (0pt,0pt) node {\unifont ①};
	\draw (0pt,14pt) node {\unifont ④};
	\draw (14pt,0pt) node {\unifont ②};
\end{tikzpicture};
4) \prfA{གྲུབ་} {\unifont →}
\begin{tikzpicture}[baseline=0pt]
	\draw (0pt,0pt) node {\unifont ①};
	\draw (0pt,-14pt) node {\unifont ⑤};
	\draw (0pt,-28pt) node {\unifont ④};
	\draw (14pt,0pt) node {\unifont ②};
\end{tikzpicture};

5) \prfA{སྒྲུབ་} {\unifont →}
\begin{tikzpicture}[baseline=0pt]
	\draw (0pt,0pt) node {\unifont ⑦};
	\draw (0pt,-14pt) node {\unifont ①};
	\draw (0pt,-28pt) node {\unifont ⑤};
	\draw (0pt,-42pt) node {\unifont ④};
	\draw (14pt,0pt) node {\unifont ②};
\end{tikzpicture};
6) \prfA{མགའ་} {\unifont →} {\unifont ⑥①②};
7) \prfA{བསྐྲེངས་} {\unifont →}
\begin{tikzpicture}[baseline=0pt]
	\draw (-14pt,0pt) node {\unifont ⑥};
	\draw (0pt,14pt) node {\unifont ④};
	\draw (0pt,0pt) node {\unifont ⑦};
	\draw (0pt,-14pt) node {\unifont ①};
	\draw (0pt,-28pt) node {\unifont ⑤};
	\draw (14pt,0pt) node {\unifont ②};
	\draw (28pt,0pt) node {\unifont ③};
\end{tikzpicture};

Следует отметить, что тибетские корневые графемы пишутся по горизонтали в строчку. Над строчкой могут выступать только графемы-диакритики. В слогах, где имеются надписная и приписная графемы, основная графема опускается ниже строки, а приписная графема следует за надписной, как это видно из примеров 5 и 7.

Необходимо также отметить, что не каждая корневая графема, выступающая в качестве основы слога, может вступать в сочетание с любой подписной, надписной или приписной графемой. Некоторые же основы слога не могут вступать в сочетания и с целой группой составляющих слога. Так, например, корневые графемы \prfA{ཁ}, \prfA{ཆ}, \prfA{ཐ}, \prfA{ཕ}, \prfA{ཚ}, \prfA{ཝ}, \prfA{ཞ}, \prfA{ཟ}, \prfA{ཡ}, \prfA{ར}, \prfA{ལ}, \prfA{ས}, \prfA{ཨ}, \prfA{ཤ}, выступающие основой слога, не могут вступать в сочетания с надписными графемами. Корневые графемы \prfA{ཝ}, \prfA{འ}, \prfA{ར}, \prfA{ལ}, выступая основой слога, не могут принимать префиксальных графем и т.д.

Огласовки также не могут сочетаться с любой основой слога, особенно если последняя входит в состав производного слога. Это видно из следующей таблицы (таб. \ref{tab:2}). 

\begin{longtable}[H]{|p{1cm}|p{3cm}|p{7cm}|}
	\caption{Слоги, не принимающие огласовку}
	\label{tab:2}\\
		\hline
	Огласовка & Характер слога & Слоги, не принимающие огласовку\\
	\hline
	\rule{0cm}{5mm}
	\multirow{4}{*}{\prfB{ ི}{(i)}} & простой & \prfA{ཁ}, \prfA{ང}, \prfA{ཕ}, \prfA{བ}, \prfA{ཝ}, \prfA{ཧ}\\
	& с надписными графемами & \prfA{རྐ}, \prfA{རྒ}, \prfA{རྔ}, \prfA{རྗ}, \prfA{རྣ}, \prfA{རྦ}|\quad  \prfA{ལྐ}, \prfA{ལྒ}, \prfA{ལྔ}, \prfA{ལྤ}, \prfA{ལྦ}, \prfA{ལྷ}|\quad \prfA{སྐ}, \prfA{སྒ}, \prfA{སྔ}, \prfA{སྣ}, \prfA{སྤ}, \prfA{སྦ}, \prfA{སྩ}\\
	\rule{0cm}{5mm}
	& с подписными графемами & \prfA{མྱ}|\quad \prfA{ཏྲ}, \prfA{པྲ}|\quad \prfA{ཀླ}, \prfA{བླ}, \prfA{ཟླ}, \prfA{སླ}\\
	\rule{0cm}{5mm}
	& с надписными и подписными графемами & \prfA{རྐྱ}, \prfA{རྒྱ}, \prfA{རྨྱ}|\quad \hl{??}\hyperref[tab:2:spec1]{$^*$},\prfA{རྩྭ}|\quad \prfA{ཕྱྭ}|\quad \prfA{གྲྭ},\prfA{སྨྲ}\\ 
	\hline
	\rule{0cm}{5mm}
	\multirow{4}{*}{\prfB{ ུ}{(u)}} & простой & \prfA{ཝ}\\
	& с надписными графемами & \prfA{རྙ}, \prfA{རྦ}|\quad \prfA{ལྒ}, \prfA{ལྔ}, \prfA{ལྗ}, \prfA{ལྤ}|\quad \prfA{སྩ}\\
	\rule{0cm}{5mm}
	& с подписными графемами & \prfA{པྱ}|\quad \prfA{ཏྲ}, \prfA{ཐྲ}, \prfA{པྲ}, \prfA{ཧྲ}\\
	\rule{0cm}{5mm}
	& с надписными и подписными графемами & \prfA{རྐྱ}, \prfA{རྨྱ}|\quad \hl{??}\hyperref[tab:2:spec1]{$^*$}, \prfA{རྩྭ}|\quad \prfA{ཕྱྭ}|\quad \prfA{སྦྱ}|\quad \prfA{སྨྲ}\\
	\hline
	\rule{0cm}{5mm}
	\multirow{4}{*}{\prfB{ ེ}{(e)}} & простой & \prfA{ཝ}, \prfA{འ}\\
	& с надписными графемами & \prfA{རྒ}, \prfA{རྣ}, \prfA{རྦ}|\quad \prfA{ལྐ}, \prfA{ལྒ}, \prfA{ལྔ}, \prfA{ལྗ}, \prfA{ལྤ}, \prfA{ལྦ}|\quad \prfA{སྔ}, \prfA{སྩ}\\
	\rule{0cm}{5mm}
	& с подписными графемами & \prfA{པྱ}, \prfA{མྱ}|\quad \prfA{ཏྲ}, \prfA{ཐྲ}, \prfA{པྲ}|\quad \prfA{ཀླ}, \prfA{བླ}, \prfA{ཟླ}, \prfA{སླ}\\
	\rule{0cm}{5mm}
	& с надписными и подписными графемами & \prfA{རྒྱ}, \prfA{རྨྱ}|\quad \hl{??}\hyperref[tab:2:spec1]{$^*$}, \prfA{རྩྭ}|\quad \prfA{ཕྱྭ}, \prfA{གྲྭ}|\quad \prfA{སྤྱ}, \prfA{སྦྱ}, \prfA{སྨྱ}\\
	\hline
	\rule{0cm}{5mm}
	\multirow{4}{*}{\prfB{ ོ}{(o)}} & простой & \prfA{ཝ}\\
	& с надписными графемами & \prfA{ལྒ}, \prfA{ལྔ}, \prfA{ལྤ}, \prfA{ལྦ}\\
	\rule{0cm}{5mm}
	& с подписными графемами & \prfA{དྲ}, \prfA{ཐྲ}, \prfA{མྲ}\\
	\rule{0cm}{5mm}
	& с надписными и подписными графемами & \prfA{རྨྱ}|\quad \hl{??}\hyperref[tab:2:spec1]{$^*$}, \prfA{རྩྭ}|\quad \prfA{ཕྱྭ}|\quad \prfA{གྲྭ}|\quad \prfA{སྦྲ}\\
	\hline
\end{longtable}
{\footnotesize{\label{tab:2:spec1}* В таблице приведен слог, которые невозможно отобразить современными средствами: \textit{rgrwa}. Возможно, это описка. Следует уточнить в других источниках.}}
 
Производный слог с подписным \prfA{ཝ} , например: \prfA{ཀྭ}, \prfA{འྭ} и т.д., вообще не может иметь огласовок.

Тибетский производный слог читается по корневой графеме, выступающей основой слога. Компоненты производного слога оказывают влияние на чтение основы слога, определяя произношение слога в целом. Влияние компонентов слога может заключаться в: а) изменении тона корневой (читаемой) графемы; б) озвончении ее; в) изменении качества гласного основы; г) появлении полугласного; д) изменении согласного звука основы; е) назализации согласного звука основы. Кроме этого, некоторые приписные графемы произносятся сами как конечный согласный слога или дают гортанную смычку. Некоторые префиксальные и надписные графемы не оказывают никакого влияния на звучание слога и играют только смыслоразличительную роль на письме. Все это наглядно иллюстрирует таб. \ref{tab:3}.


\begin{longtable}[H]{|c|*{8}{c|}}
	\caption{Роли компонентов слога}
	\label{tab:3}\\
		\hline
	\diagbox{{Компонент слога}}{{Роль компонента слога}} &
	\rotatebox{90}{Изменяет тон}
		&
		\rotatebox{90}{Озвончает качество гласного} &
		\rotatebox{90}{Изменяет качество гласного} &
		\rotatebox{90}{Вводит полугласный} &
		\rotatebox{90}{Изменяет согласный} &
		\rotatebox{90}{Назализирует согласный} &
		\rotatebox{90}{\parbox{4cm}{Читается как конечный согласный}} &
		\rotatebox{90}{\parbox{4cm}{Играет смыслоразличительную роль на письме}}\\
	\hline
	Приписные графемы & + & - & + & - & - & - & + & - \\
	\hline
	Подписные графемы & + & - & - & + & + & - & - & - \\
	\hline
	Префиксальные графемы & + & + & - & - & - & + & - & + \\
	\hline
	Надписные графемы & + & + & - & - & - & + & - & + \\
	\hline
\end{longtable}
\footnotesize{\emph{Примечание}. Знак + в таблице отнюдь не говорит, что вес компоненты слога данной группы (т.е. все приписные, все надписные и т.д.) играют в слоге указанную роль. Так, например, из 11 приписных только 4 меняют тон слога. Подробнее см. таб. \ref{tab:4}, \ref{tab:5}, \ref{tab:6}, \ref{tab:7}.}


Каждая группа составляющих производного слога ока¬зывает определенное влияние на его основу.
1.\emph{Приписные графемы} (за исключением \prfA{འ}\footnote[11]{\prfA{འ} служит для выделения основы слога в слоге, состоящем из двух корневых графем, не имеющих ни огласовки, ни подписного знака, например: в слоге \prfA{དགའ་} приписная \prfA{འ} указывает, что \prfA{ད} --- префиксальная графема, а \prfA{ག} --- основа; в слоге \prfA{དག་} отсутствие приписной \prfA{འ} указывает, что \prfA{ད} --- основа, а \prfA{ག} --- приписная. Сама приписная не оказывает влияния на звучание слога.}), входя в состав слога, оказывают то или иное влияние на его звучание. Приписная графема может: а) изменить ровный тон высокого регистра основы слога на падающий
( \toneR {\unifont →} \toneN ), восходящий тон на восходяще-падающий ( \toneV {\unifont →} \toneVN ); б) дать гортанную смычку ( \toneG ) удлинить или видоизменить гласный слога (см. таб. \ref{tab:4}).

2. \emph{Вторичная приписная графема}. За приписными \prfA{ག}, \prfA{ང}, \prfA{བ}, \prfA{མ} может следовать еще одна приписная --- вторичная приписная графема --- \prfB{ས}{sa}. Если \prfA{ས} следует за приписными \prfA{ག} или \prfA{བ}, то она играет только смыслоразличительную роль на письме, например: \prfC{ཐབས་}{t'\textschwa{}p\toneN}{'способ, метод'} --- \prfC{ཐབ་}{t'\textschwa{}p\toneN}{'печь, очаг'}. Если же вторичная приписная следует за приписными \prfA{ང}, \prfA{མ}, то ровный тон слога меняется на падающий, а восходящий --- на восходяще-падающий, например:  5; \prfC{ཐེང་}{t'e\textrtailn{}\toneR}{'хромой'} ---  \prfC{ཐེངས་}{t'e\textrtailn{}\toneN}{'раз', 'один раз'},\prfC{གང་}{k'a\textrtailn{}\toneV}{'где, куда'} --- \prfC{གངས་}{k'a\textrtailn{}\toneVN}{'снег'}.

\begin{longtable}[H]{|p{2cm}|p{7cm}|p{2cm}|}
	\caption{Приписные графемы}\label{tab:4}
	\\
	\hline
	Приписная графема & Её роль в слоге & Примеры\\
	\hline
	\prfA{ག} & \makecell{Меняет тон и\\ дает гортанную смычку\\
$
\left.\begin{aligned}
\toneR \rightarrow \toneN \\
\toneV \rightarrow \toneVN \\
\end{aligned}\right\rbrace + 
$ \toneG } & \makecell{\prfA{ཐག་} \textit{t'a\toneG}\toneN\\ \prfA{དག་} \textit{t'a\toneG}\toneVN}\\
\hline
\prfA{ས}, \prfA{ད} & \makecell{Меняет тон и гласный основы\\ слога, дает гортанную смычку\\
$
\left.\begin{aligned}
	\toneR \rightarrow \toneN \\
	\toneV \rightarrow \toneVN
\end{aligned}\right\rbrace + 
$ \makecell{a {\unifont →} \textepsilon \\u {\unifont →} y + \toneG\\o {\unifont →} \unifont{ɵ}} }  & 
\makecell{
	\prfA{ཆས་} \textit{ts'\unifont{ɵ}\toneG}\toneN\\
	\prfA{ཟས་} \textit{s\textepsilon'}\toneVN\\
	\prfA{བོད་} \textit{p'{\unifont ɵ}\toneG}\toneVN\\
	\prfA{ཕུད་} \textit{p'y\toneG}\toneVN
}\\
\hline
\prfA{བ} & \makecell{Меняет тон и гласный \textit{a} основы \\слога, читается \textit{p}\\
$
\left.\begin{aligned}
	\toneR \rightarrow \toneN \\
	\toneV \rightarrow \toneVN
\end{aligned}\right\rbrace + p
$ a {\unifont →} \textschwa} &
\makecell{
	\prfA{ཁབ་} \textit{k'\textschwa{}p}\toneN\\
	\prfA{གབ་} \textit{k'\textschwa{}p}\toneVN
}\\	
\hline
\prfA{ལ}, \prfA{འི} & \makecell{Меняет гласный основы \\слога и удлиняет его\\
$
\left.\begin{aligned}
	\toneR \rightarrow \toneN \\
	\toneV \rightarrow \toneVN
\end{aligned}\right\rbrace +
$ \makecell{a {\unifont →} e: \\u {\unifont →} y:\\o {\unifont →} \unifont{ɵ}:}} &
\makecell{
	\prfA{ཐལ་} \textit{t'e:}\toneR\\
	\prfA{ཁོལ་} \textit{k'{\unifont ɵ}:}\\
	\prfA{སའི་} \textit{se:}\toneR\\
	\prfA{ཁོའི་} \textit{k'{\unifont ɵ}:}\toneR
}\\	
\hline
\prfA{ན} & \makecell{Меняет гласный основы \\слога и читается как \~{n}\\
$
\left.\begin{aligned}
	a \rightarrow e \\
	u \rightarrow y \\
	o \rightarrow o\\
\end{aligned}\right\rbrace +
$ \~{n}} & 
\makecell{
	\prfA{ཕན་} \textit{p'en}\toneR\\
	\prfA{སོན་} \textit{s{\unifont ɵ}\~{n}}\toneR\\
	\prfA{གོན་} \textit{k'{\unifont ɵ}\~{n}}\toneV\\
	\prfA{ཀུན་} \textit{ky\~{n}}\toneR
}\\	
\hline
\prfA{ར} & Удлиняет гласный основы слога & 
\makecell{
	\prfA{ཐུར་} \textit{tu:}\toneR\\
	\prfA{མར་} \textit{ma:}\toneV
}\\	
\hline
\prfA{ང} & Читается как \textrtailn{}&
\makecell{
	\prfA{གང་} \textit{k'a\textrtailn}\toneR\\
	\prfA{དང་} \textit{t'a\textrtailn}\toneV
}\\	
\hline
\prfA{མ} & Читается как \~{m} &
\makecell{\prfA{དམ་} \textit{t'a\~{m}}\toneV}\\	
\hline
\end{longtable}
\footnotesize{\prfA{འི} --- вариант притяжательной падежной частицы (см. ), которая присоединятеся к основе открытого слога, выступая как суффикс.}

В древнеписьменном языке имелась еще одна вторичная приписная --- \prfA{ད}, которая могла следовать за приписными \prfA{ན}, \prfA{ར}, \prfA{ལ}. Хотя эта приписная давно уже на письме не употребляется, однако до сего времени сохранилось ее влияние на орфографию служебного слова, которое сочетается со словами, оканчивающимися на корневые графемы
\prfA{ན}, \prfA{ར}, \prfA{ལ} (см. разд. <<Служебные слова>>).

3. \emph{Префиксальные графемы}. Их наличие в слоге влияет на: а) тон и звонкость основы слога; б) звучание слога. Как влияют префиксальные графики на тон и звонкость основы слога, показано в таб. \ref{tab:5}.

\begin{longtable}[H]{|*{7}{Sc|}}
	\caption{Префиксальные графемы}\label{tab:5}\\
	\hline
	\multirow{2}{*}{Префикс} & \multicolumn{6}{c|}{Основы, с которыми сочетается префиксальная графема}\\
	%\hline
	& \multicolumn{3}{c|}{Остаются без изменения} & Озвончаются & \multicolumn{2}{c|}{\makecell{Восходящий тон\\меняют на ровный}}\\
	\hline
	\prfA{ག} & \prfA{ཅ}, \prfA{ཏ}, \prfA{ཙ} & & \prfA{ཤ}, \prfA{ས}, \prfA{ཞ}, \prfA{ཟ} & \prfA{ད} & \prfA{ཉ}, \prfA{ན} & \prfA{ཡ}\\
	\hline
	\prfA{ད} & \prfA{ཀ}, \prfA{པ} & & & \prfA{ག}, \prfA{བ} & \prfA{ང}, \prfA{མ} & \\
	\hline
	\prfA{བ} & \prfA{ཀ}, \prfA{ཅ}, \prfA{ཏ}, \prfA{ཙ} & & \prfA{ཤ}, \prfA{ས}, \prfA{ཞ}, \prfA{ཟ} & \prfA{ག}, \prfA{ད} & & \\
	\hline
	\prfA{མ} & & \prfA{ཁ}, \prfA{ཆ}, \prfA{ཐ}, \prfA{ཚ} & & \prfA{ག}, \prfA{ཇ}, \prfA{ད}, \prfA{ཛ} & \prfA{ང}, \prfA{ཉ}, \prfA{མ} & \\
	\hline
	\prfA{འ} & \prfA{ཁ}, \prfA{ཆ}, \prfA{ཐ}, \prfA{ཕ}, \prfA{ཚ} & & & \prfA{ག}, \prfA{ཇ}, \prfA{ད}, \prfA{བ}, \prfA{ཛ} & & \\
	\hline
\end{longtable}

На звучание слога оказывают влияние префиксы \prfA{མ} и \prfA{འ}, а также префикс \prfA{ད}, если он стоит перед основой слога \prfA{བ}.

Префиксы \prfA{མ} и \prfA{འ} придают основе слога назализацию, соответственна: а) \textrtailn{} --- перед основой \prfA{ག}; б) \~{n} --- перед основами \prfA{ཇ}, \prfA{ད}, \prfA{ཛ} и производными слогами \prfA{གྱ}, \prfA{གྲ}, \prfA{བྲ}; в) \~{m} --- перед основой \prfA{བ}, например: \prfB{མགའ་}{\textrtailn{}ka}\toneV, \prfB{འབང་}{\~{m}pa\textrtailn{}}\toneV, \prfB{མཛའ་}{\~{n}tsa}\toneV.

Префикс \prfA{ད} c основой слога \prfA{བ} дает следующее звучание: а) если гласный слога \textit{а}, то \textit{p'} переходит в \textit{w}, например: \prfB{དབའ་}{wa}\toneR, \prfB{དབང་}{wa\textrtailn}\toneR; б) если гласный основы не \textit{а}, то согласный не произносится вообще, например: \prfB{དབེན་}{e\~{n}}\toneR, \prfB{དབུལ་}{\"{u}:}\toneR, \prfB{དབྱ་}{ja}\toneR, \prfB{དབྱི་}{ji:}\toneR.

4. \emph{Подписные графемы}. \prfA{ཝ}, \prfA{ཡ}, \prfA{ར}, \prfA{ལ}, подписываясь под основой слога, меняют свое начертание (кроме \prfA{ལ}), и оказывают определенное влияние на произношение слога (см. таб. \ref{tab:6}).

\begin{longtable}[H]{|S{p{2cm}}|S{p{2cm}}|S{p{5cm}}|S{p{5cm}}|}
	\caption{Подписные графемы}\label{tab:6}\\
	\hline
	Подписная графема & Характер начертания & Какие сложные графемы * образует & Роль в слоге\\
	\hline
	\prfA{ཝ} & \prfA{ ྭ} & \prfA{ཀྭ}, \prfA{ཁྭ}, \prfA{གྭ}, \prfA{ཅྭ}, \prfA{ཉྭ}, \prfA{ཏྭ}, \prfA{དྭ}, \prfA{ཙྭ}, \prfA{ཚྭ}, \prfA{ཞྭ}, \prfA{ཟྭ}, \prfA{རྭ}, \prfA{ལྭ}, \prfA{ཤྭ}, \prfA{སྭ}, \prfA{ཧྭ}, \prfA{གྲྭ} & Играет только смыслоразличительную роль на письме, например: \prfC{ཤ་}{sja\toneR}{'мясо'} --- \prfC{ཤྭ་}{sja\toneR}{'олень'}; \prfC{རྩ་}{tsa\toneR}{'корень'} --- \prfC{རྩྭ་}{tsa\toneR}{'трава'}\\
	\hline
	\multirow{3}{*}{\prfA{ཡ}} & \multirow{3}{*}{\prfA{ ྱ}} & 1) \prfA{ཀྱ}, \prfA{ཁྱ}, \prfA{གྱ} & Читается как \textit{j}, например: \prfB{ཀྱ}{kja\toneR}, \prfB{ཁྱ}{k'ja\toneR}, \prfB{གྱ}{k'ja\toneV}\\
	& & 2) \prfA{པྱ}, \prfA{ཕྱ}, \prfA{བྱ} & Меняет чтение основы на \textit{tsj} и \textit{ts'j}, например: \prfB{པྱ}{tsja\toneR}, \prfB{ཕྱ}{ts'ja\toneR}, \prfB{བྱ}{ts'ja\toneR}\\
	& & 3) \prfA{མྱ} & Меняет чтение основы на \textit{nj}, например: \prfB{མྱ}{nja\toneR}\\
	\hline
	\multirow{3}{*}{\prfA{ར}} & \multirow{3}{*}{\prfA{ ྲ}} & 1) \prfA{ཀྲ}, \prfA{ཏྲ}*, \prfA{པྲ}, \prfA{ཁྲ}, \prfA{གྲ}, \prfA{དྲ}, \prfA{ཐྲ}*, \prfA{ཕྲ}, \prfA{བྲ} & Меняет чтение основы на \textit{t\v{s}} и \textit{t\v{s}'}, например: \prfA{ཀྲ}, \prfA{ཏྲ}, \prfA{པྲ} читаются \textit{t\v{s}a\toneR}; \prfA{ཁྲ}, \prfA{ཐྲ}, \prfA{ཕྲ} читаются \textit{t\v{s}'a\toneR}; \prfA{གྲ}, \prfA{དྲ}, \prfA{བྲ} читаются \textit{t\v{s}a\toneV}\\
	& & 2) \prfA{ཧྲ} & Меняет чтение основы на \textit{\v{s}}, например \prfB{ཧྲ}{\v{s}a\toneR}\\
	& & 3) \prfA{ནྲ}, \prfA{མྲ}, \prfA{ཤྲ}, \prfA{སྲ} & Играет только смыслоразличительную роль на письме\\
	\hline
	\multirow{2}{*}{\prfA{ལ}} & \multirow{2}{*}{\prfA{ ླ}} & 1) \prfA{ཀླ}, \prfA{གླ}, \prfA{བླ}, \prfA{རླ}, \prfA{སླ} & У \prfA{ག}, \prfA{བ}, \prfA{ར} меняет восходящий тон на ровный высокого регистра,  а ткаже меняет чтение основы на \textit{la}, например: \prfB{ཀླ}{la}, \prfB{གླ}{la\toneR}\\
	& & 2) \prfA{ཟླ} & Меняет чтение основы на \textit{da}, например: \prfB{ཟླ}{da}\\
	\hline
\end{longtable}
\footnotesize{* Сложные графемы, помеченные звездочкой, употреблялись только в древнетибетском.}

5. \emph{Надписные графемы}. Их роль видна из следующей таблицы (таб. \ref{tab:7}).

\begin{longtable}[H]{|S{p{2cm}}|S{p{2cm}}|S{p{5cm}}|S{p{5cm}}|}
	\caption{Надписные графемы}\label{tab:7}\\
	\hline
	Надписная графема* & Характер начертания & Какие сложные графемы образует & Роль в слоге\\
	\hline
	\multirow{3}{*}{\prfA{ར}} & \multirow{3}{*}{\prfA{ར}} & 1) \prfA{རྐ}, \prfA{རྟ}, \prfA{རྩ} & Играет только смыслоразличительную роль на письме\\
	& & 2) \prfA{རྒ},\prfA{རྗ},\prfA{རྡ},\prfA{རྦ},\prfA{རྫ} & Озвончает основу, например: \prfB{རྒ}{ka\toneV}, \prfB{རྡ}{ta\toneV}\\
	& & 3) \prfA{རྔ}, \prfA{རྙ}, \prfA{རྣ}, \prfA{རྨ} & Меняет восходящий тон на ровный высокого регистра, например: \prfB{རྔ}{\textrtailn{}a\toneR}\\
	\hline
	\multirow{4}{*}{\prfA{ལ}} & \multirow{4}{*}{\prfA{ལ}} & 1) \prfA{ལྐ}, \prfA{ལྕ}, \prfA{ལྟ}, \prfA{ལྤ} & Играет только смыслоразличительную роль на письме, например: \prfC{རྟ་}{ta\toneR}{'лошадь'} --- \prfC{ལྟ་}{ta\toneR}{'глядеть'}\\
	& & 2) \prfA{ལྒ}, \prfA{ལྗ}, \prfA{ལྡ}, \prfA{ལྦ} & Озвончает и назализует основу, например: \prfB{ལྒ}{\textrtailn{}ka\toneV}; \prfB{ལྗ}{\^{n}sja\toneV}; \prfB{ལྡ}{\~{n}sa\toneV}; \prfB{ལྦ}{\~{m}pa\toneV}\\
	& & 3) \prfA{ལྔ} & Меняет восходящий тон на ровный высокого регистра, например: \prfB{ལྔ}{\textrtailn{}a\toneR}\\
	& & 4) \prfA{ལྷ} & Меняет чтение, например: \prfB{ལྷ}{lha\toneR}\\
	\hline
	\multirow{3}{*}{\prfA{ས}} & \multirow{3}{*}{\prfA{ས}} & 1) \prfA{སྐ}, \prfA{སྟ}, \prfA{སྤ}, \prfA{སྩ} & Играет только смыслоразличительную роль на письме, например: \prfC{རྐང་}{ka\textrtailn{}}{'костный мозг'} --- \prfC{སྐང་}{ka\textrtailn{}}{'наполнить, удовлетворять'}\\
	& & 2) \prfA{སྒ}, \prfA{སྡ}, \prfA{སྦ} & Озвончает основу, например: \prfB{སྒ}{ka\toneV}, \prfB{སྡ}{ta\toneV}\\
	& & 3) \prfA{སྔ}, \prfA{སྙ}, \prfA{སྣ}, \prfA{སྨ} & Меняет восходящий тон на ровный высокого регистра, например: \prfB{སྔ}{\textrtailn{}a\toneR}\\
	\hline
\end{longtable}

\section{Тон и звучание слогов в двусложном слове}

Приведенные выше правила чтения слогов полностью применимы только в отношении отдельного слога, т.е. когда слог выступает как самостоятельная лексическая или грамматическая единица. Подавляющее же большинство тибетских слов --- двусложные. В двусложных словах тоны и произношение слогов влияют друг на друга, в результате чего в них происходят изменения. Ниже приводятся наиболее характерные из этих изменений.

1. В двусложном слове изменяются тоны его составляющих (см. таб. \ref{tab:8}).

\begin{longtable}[H]{|S{p{3cm}}|S{p{3cm}}|S{p{3cm}}|}
	\caption{Таблица смены тонов}\label{tab:8}\\
	\hline
	Тон первого слога & Тон второго слога & Тоны слогов в слове\\ \hline
	\toneR & \toneV & \toneR \toneR\\ \hline
	\toneR & \toneVN & \toneR \toneN\\ \hline
	\toneV & \toneV & \toneV \toneR\\ \hline
	\toneN & \toneR & \toneR \toneR\\ \hline
	\toneN & \toneN & \toneR \toneN\\ \hline
	\toneN & \toneVN & \toneR \toneN\\ \hline
	\toneVN & \toneR & \toneV \toneR\\ \hline
	\toneVN & \toneN & \toneV \toneN\\ \hline
	\toneVN & \toneVN & \toneV \toneN\\ \hline
\end{longtable}

2. Согласные, которые записываются первыми пятью корневыми графемами третьего ряда тибетского алфавита (\prfA{ག}, \prfA{ཇ}, \prfA{ད}, \prfA{བ}, \prfA{ཛ}), выступая в качестве второго слога в слове, озвончаются, например: \prfB{ཡི}{ji\toneV} + \prfB{གེ}{k'e\toneV} {\unifont →} \prfC{ཡི་གེ་}{ji\toneV{}ke\toneR}{'буква, письмена'}.

3. В первом слоге слова приписная \prfA{ག} произносится \textit{k}, например: \prfC{ངག་མ་}{nak\toneV{}ma\toneR}{'речь'}.

4. Если \prfB{བ}{p'a\toneV}, \prfB{བེ}{p'e\toneV} или \prfB{བོ}{p'o\toneV} являются вторым слогом двуслога, то \textit{р'} переходит в \textit{w}, например: \prfB{གཏའ་བ་}{ta\toneR{}wa\toneR}, \prfB{དཔའ་བོ་}{pa\toneR{}wo\toneR}, \prfB{པེ་བེ་}{pe\toneR{}we\toneR}, но \prfB{ནོན་བུ་}{no\~{n}\toneV{}p'y\toneR}.

5.	Если слоги слова содержат гласные звуки \textit{i} и \textit{a} или \textit{y} и \textit{a}, то в составе слова происходит трансформация: \textit{a} + \textit{i} > \textit{{\unifont ɘ}} + \textit{i}; \textit{i} + \textit{a} > \textit{i} + \textit{{\unifont ɘ}}; \textit{у} + \textit{a} > \textit{у} + \textit{{\unifont ɘ}}; \textit{a} + \textit{у} > \textit{{\unifont ɘ}} + \textit{у}, например: \prfB{ཟི་མ་}{si\toneV{}m{\unifont ɘ}\toneR}, \prfB{མ་གི་}{m{\unifont ɘ}\toneV{}ki\toneR}, \prfB{ཁ་ཆུ་}{k'{\unifont ɘ}\toneR{}ts'u\toneR}.

6. Если приписная первого слога \prfA{ང}, \prfA{ར} или \prfA{ལ}, а вторым слогом являются \prfA{བ} или \prfA{པ}, то вместо \textit{wa} или \textit{pa} соответственно произносят \textrtailn{}, \textit{ra} или \textit{la}. В этом случае приписные \prfA{ར} и \prfA{ལ}гласный не удлиняют, например:
\prfB{ཟོར་བ་}{so\toneV{}ra\toneR},
\prfB{ཡོལ་བ་}{j\={o}\toneV{}ra\toneR},
\prfB{ཡོང་བ་}{jo\textrtailn{}\toneV{}na\toneR},
\prfB{ཐར་པ་}{t'a\toneR{}ra\toneR}.

7.	Если слово построено путем удвоения слогов, то при наличии надписной или подписной корневая графема второго слога не произносится, а произносится надписная или подписная, например:
\prfB{ལྷོད་ལྷོད་}{lh\"{o}\toneR{}lo\toneR}, \prfB{ཧྲང་ཧྲང་}{\u{s}a\textrtailn{}\toneR{}ra\textrtailn{}\toneR}
;
если надписных и подписных нет, а имеется приписная \prfA{བ} или \prfA{ག}, то во втором слоге она не произносится, например:
\prfB{ལེབ་ལེབ་}{lep\toneV{}le\toneN},
\prfB{ཞིབ་ཞིབ་}{sjip\toneV{}sji\toneG{}\toneN},
\prfB{རྡོག་རྡོག་}{tok\toneV{}to\toneR}.

8. Если первый слог слова открытый, а второй слог начинается с префиксальных графем \prfA{བ}, \prfA{མ} или \prfA{འ}, , то эти префиксальные становятся читаемыми и соответственно произносятся \textit{р}, \textit{m} и \textit{n}, например:
\prfB{ཡ་མཚན་}{jam\toneV{}ts'{\unifont ɛ}n\toneR},
\prfB{ལྔ་བརྒྱ་}{{\unifont ŋ}ap\toneR{}kja\toneR},
\prfB{མེ་འཁོར་}{men\toneV{}k'o:\toneR}.

9. Если первый слог слова открытый, а второй слог имеет надписную \prfA{ར}, то эта надписная становится читаемой, например: \prfB{རྡོ་རྗེ་}{tor\toneV{}tse\toneR}. По такому же правилу иногда произносятся слова с надписными \prfA{ལ} во втором слоге (но не \prfA{ས}).

\section{Транслитерация}

При издании текстов в европейских странах часто не представляется возможным воспроизвести эти тексты в традиционной графике из-за трудности и дороговизны набора. Вместе с тем нельзя давать эти тексты и в транскрипции, так как несовпадение графики тибетского слова с его звучанием, большое количество омонимов (но не омографов) сильно затрудняет перевод транскрипции обратно в графику. Так, например, \textit{lu{\unifont ŋ}\toneR} означает и <<река>> (\prfA{ཀླུང་}) и <<ветер>> (\prfA{རླུང་}); \textit{ka\toneR{}wa\toneR} означает и <<столб>> (\prfA{ཀ་བ་}) и <<густой>> (\prfA{སྐ་བ་}). Поэтому при издании тибетских текстов прибегают к транслитерации, т.е. к транскрибированию всех графем, входящих в состав слога. Такая запись обычно не передает действительного звучания производного слога, но позволяет свободно осуществлять обратный перевод его в традиционную графику.

Существует ряд способов записи тибетских корневых графем латинскими литерами, предложенных тибетологами различных стран.

Впервые такая запись была произведена в латино-тибетском словаре, составленном капуцинами Джузеппе да Асколи, Франческо Мария да Тоурс, Доменико да Фано в период 1708-1713 гг.\footnote[12]{К настоящему времени этот словарь не сохранился. Парижская национальная библиотека имеет рукопись с извлечениями из этого словаря, содержащую 2500 слов.}

Затем передача тибетских слов латинскими литерами была осуществлена капуцином А. Георги в его учебнике тибетского языка\footnote[13]{См.: A.Georgi, Alphabetum tangutanum sive tibetanum, Romae, 1762.}, составленном для нужд католических миссионеров.

Однако впервые способ записи тибетских графем латинскими литерами как систему предложил основатель тибетологии --- венгр Чома де Кёрёши в своей грамматике тибетского языка\footnote[14]{См.: A.Cs.Körös, A grammar of Tibetan Language, Calcutta, 1834.}.

Специально вопросу о транскрипции посвящена работа Т. Уайли, вышедшая в 1959 г.\footnote[15]{См.: T.Wylie, A standart System of Tibetan Transcription, Narvard, 1959.}, а в 1964 г. появилась работа Е.Рихтера <<Основы фонетики лхасского диалекта>>\footnote[16]{E.Richter, Grundlagen des phonetik des Lhasa dialektes, Berlin, 1964.}.

Существует около 30 систем записи тибетских корневых графем латинскими буквами, большинство из которых в значительной части совпадает друг с другом. Наиболее близки системы, предложенные Чома де Кёрёши и Т. Уайли. Наиболее удобна для транслитерации, пожалуй, система Т. Уайли, не содержащая диакритических знаков. Ниже приводится шесть систем записи корневых графем тибетского алфавита латинскими буквами (см. \ref{tab:9}).

\tabulinesep=1mm

\begin{longtabu} to \linewidth {|X[1,c] | X[1,c] | X[1,c] |X[1,c] |X[1,c] |X[1,c] |X[1,c] |}
	\everyrow {\tabucline{-}}
	\caption{Системы транслитерации корневых графем}\label{tab:9}\\
	Корневая графема & Чома де Кёрёши & Х.А.Ешке & С.Ч.Дас & Ж.Бако & Т.Уайли & Е.Рихтер\\
	%1
	\prfA{ཀ} & \textit{ka} & \textit{ka} & \textit{ka} & \textit{ka} & \textit{ka} & \textit{ka}\\
	\prfA{ཁ} & \textit{kha} & \textit{k'a} & \textit{kha} & \textit{kha} & \textit{kha} & \textit{kha}\\
	\prfA{ག} & \textit{ga} & \textit{ga} & \textit{ga} & \textit{ga} & \textit{ga} & \textit{ga}\\
	\prfA{ང} & \textit{nga} & \textit{\.{n}a} & \textit{\^{n}a} & \textit{\.{n}a} & \textit{nga} & \textit{\.{n}a}\\
	%2
	\prfA{ཅ} & \textit{cha} & \textit{ca} & \textit{ca} & \textit{\u{c}a} & \textit{ca} & \textit{t\u{s}a}\\
	\prfA{ཆ} & \textit{chha} & \textit{cha} & \textit{cha} & \textit{\u{c}ha} & \textit{cha} & \textit{t\u{s}ha}\\
	\prfA{ཇ} & \textit{ja} & \textit{ja} & \textit{ja} & \textit{\u{j}a} & \textit{ja} & \textit{d\u{z}a}\\
	\prfA{ཉ} & \textit{nya} & \textit{nya} & \textit{\~{n}a} & \textit{\~{n}a} & \textit{nya} & \textit{\~{n}a}\\
	%3
	\prfA{ཏ} & \textit{ta} & \textit{ta} & \textit{ta} & \textit{ta} & \textit{ta} & \textit{ta}\\
	\prfA{ཐ} & \textit{tha} & \textit{t'a} & \textit{tha} & \textit{tha} & \textit{tha} & \textit{tha}\\
	\prfA{ད} & \textit{da} & \textit{da} & \textit{da} & \textit{da} & \textit{da} & \textit{da}\\
	\prfA{ན} & \textit{na} & \textit{na} & \textit{na} & \textit{na} & \textit{na} & \textit{na}\\
	%4
	\prfA{པ} & \textit{pa} & \textit{pa} & \textit{pa} & \textit{pa} & \textit{pa} & \textit{pa}\\
	\prfA{ཕ} & \textit{pha} & \textit{p'a} & \textit{pha} & \textit{pha} & \textit{pha} & \textit{pha}\\
	\prfA{བ} & \textit{ba} & \textit{ba} & \textit{ba} & \textit{ba} & \textit{ba} & \textit{ba}\\
	\prfA{མ} & \textit{ma} & \textit{ma} & \textit{ma} & \textit{ma} & \textit{ma} & \textit{ma}\\
	%5
	\prfA{ཙ} & \textit{tsa} & \textit{tsa} & \textit{tsa} & \textit{ca} & \textit{tsa} & \textit{tsa}\\
	\prfA{ཚ} & \textit{ts'ha} & \textit{ts'a} & \textit{tsha} & \textit{cha} & \textit{tsha} & \textit{tsha}\\
	\prfA{ཛ} & \textit{dsa} & \textit{dza} & \textit{dsa} & \textit{ja} & \textit{dza} & \textit{dza}\\
	\prfA{ཝ} & \textit{wa} & \textit{wa} & \textit{ua} & \textit{va} & \textit{wa} & \textit{wa}\\
	%6
	\prfA{ཞ} & \textit{zha} & \textit{\'{z}a} & \textit{sha} & \textit{\u{z}a} & \textit{zha} & \textit{\u{z}a}\\
	\prfA{ཟ} & \textit{za} & \textit{za} & \textit{za} & \textit{za} & \textit{za} & \textit{za}\\
	\prfA{འ} & \textit{ha} & \textit{'a} & \textit{ha} & \textit{'a} & \textit{'a} & \textit{'a}\\
	\prfA{ཡ} & \textit{ya} & \textit{ya} & \textit{ya} & \textit{ya} & \textit{ya} & \textit{ya}\\
	%7
	\prfA{ར} & \textit{ra} & \textit{ra} & \textit{ra} & \textit{ra} & \textit{ra} & \textit{ra}\\
	\prfA{ལ} & \textit{la} & \textit{la} & \textit{la} & \textit{la} & \textit{la} & \textit{la}\\
	\prfA{ལ} & \textit{la} & \textit{la} & \textit{la} & \textit{la} & \textit{la} & \textit{la}\\
	\prfA{ཤ} & \textit{sha} & \textit{sa} & \textit{ca} & \textit{sa} & \textit{sha} & \textit{sa}\\
	%8
	\prfA{ས} & \textit{sa} & \textit{sa} & \textit{sa} & \textit{sa} & \textit{sa} & \textit{sa}\\
	\prfA{ཧ} & \textit{ha} & \textit{ha} & \textit{ha} & \textit{ha} & \textit{ha} & \textit{ha}\\
	\prfA{ཨ} & \textit{a} & \textit{'a} & \textit{a} & \textit{a} & \textit{a} & \textit{a}\\
\end{longtabu}

При транслитерации тибетского слога запись производится в строчку слева направо.

Графемы --- составляющие слога (приписные, надписные и т.п.) передаются только согласной буквой.

Буквы, записывающие нечитаемые графемы, иногда подчеркиваются.

Если в слоге имеется подписная графема, то присущий гласный основы слога пишется после нее, например:

\prfA{སྒྲུབ་} ---
\begin{tikzpicture}[baseline=0pt]
	\draw (0pt,0pt) node {\unifont ⑦};
	\draw (0pt,-14pt) node {\unifont ①};
	\draw (0pt,-28pt) node {\unifont ⑤};
	\draw (0pt,-42pt) node {\unifont ④};
	\draw (14pt,0pt) node {\unifont ②};
\end{tikzpicture}
{\unifont →} \textit{\ul{s}grub} -- {\unifont ⑦①⑤④②}

\prfA{བསྐྲེངས་} ---
\begin{tikzpicture}[baseline=0pt]
	\draw (0pt,0pt) node {\unifont ⑥};
	\draw (14pt,14pt) node {\unifont ④};
	\draw (14pt,0pt) node {\unifont ⑦};
	\draw (14pt,-14pt) node {\unifont ①};
	\draw (14pt,-28pt) node {\unifont ⑤};
	\draw (28pt,0pt) node {\unifont ②};
	\draw (42pt,0pt) node {\unifont ③};
\end{tikzpicture}
{\unifont →} \textit{\ul{bs}kreng\ul{s}} -- {\unifont ⑥⑦①⑤④②③}

В последнее время предложены также различные системы транслитерации тибетских графем русскими буквами. Приводим одну из них: В.В. Семичова -- К. Седлачека (см. таб. \ref{tab:10}).

\begin{longtabu} to \linewidth {|X[1,c] X[1,c]|X[1,c] X[1,c]|X[1,c] X[1,c]|X[1,c] X[1,c]|}
	\everyrow {\tabucline{-}}
	\caption{Транслитерация русскими буквами}\label{tab:10}\\
	Тибетские графемы & Транс\-ли\-те\-ра\-ция & Тибетские графемы & Транс\-ли\-те\-ра\-ция &
	Тибетские графемы & Транс\-ли\-те\-ра\-ция & Тибетские графемы & Транс\-ли\-те\-ра\-ция\\
	\prfA{ཀ} & \textit{ка} & \prfA{ཁ} & \textit{кха} & \prfA{ག} & \textit{га} & \prfA{ང} & \textit{\.{н}а}\\
	\prfA{ཅ} & \textit{ча} & \prfA{ཆ} & \textit{чха} & \prfA{ཇ} & \textit{джа} & \prfA{ཉ} & \textit{ня}\\
	\prfA{ཏ} & \textit{та} & \prfA{ཐ} & \textit{тха} & \prfA{ད} & \textit{да} & \prfA{ན} & \textit{на}\\
	\prfA{པ} & \textit{па} & \prfA{ཕ} & \textit{пха} & \prfA{བ} & \textit{ба} & \prfA{མ} & \textit{ма} \\
	\prfA{ཙ} & \textit{ца} & \prfA{ཚ} & \textit{цха} & \prfA{ཛ} & \textit{дза} & \prfA{ཝ} & \textit{ва}\\
	\prfA{ཞ} & \textit{жа} & \prfA{ཟ} & \textit{за} & \prfA{འ} & \textit{'} & \prfA{ཡ} & \textit{я}\\
	\prfA{ར} & \textit{ра} & \prfA{ལ} & \textit{ла} & \prfA{ཤ} & \textit{ша} & \prfA{ས} & \textit{са}\\
	\prfA{ཧ} & \textit{ха} & \prfA{ཨ} & \textit{а} & & & & \\
\end{longtabu}

\section{Порядок расположения словарных статей в тибетско-тибетских и тибетско-европейских словарях}

Согласно тибетской лексикографической традиции, принятой и в тибетско-европейских словарях со времени выхода в свет <<Тибетско-русского словаря>> Я. Шмидта (СПб., 1843), все слова располагаются в алфавитном порядке корневых (читаемых) графем, а словарные статьи на одну корневую графему --- по степени усложнения производного слога.

Корневая графема выделяется по следующим формальным признакам:

1) графема, имеющая огласовку, надписную или подписную графему, является корневой. Так, например, в слоге \prfB{མཚོ་}{\ul{m}tsho}\footnote[17]{Здесь и ниже за тибетской графикой следует не транскрипция, а транслитерация по системе Т.Уайли (T. Wylie). Непроизносимые буквы подчеркиваются линейкой, слоги, входящие в состав одного слова, соединяются дефисами.} корневая графема --- \prfB{ཚ}{tsha}; в слоге \prfB{བསྡམ་}{\ul{bs}dam} корневая --- \prfB{ད}{da}; в слоге	\prfB{ངེས་}{nge\ul{s}} корневая - \prfB{ང}{nga}; в слоге \prfB{བསླད་}{\ul{bs}la\ul{d}} корневая --- \prfB{ས}{sa} (в данном случае \prfB{ལ}{la} не может быть корневой, поскольку к ней не надписывается \prfB{ས}{sa});

2)	если слог не имеет указанных выше компонентов, то: а) в слоге, состоящем из двух графем, первая будет корневой, например:
\prfB{ནང་}{nang} --- корневая \prfB{ན}{na};
\prfB{ཞག་}{zhag} --- корневая \prfB{ཞ}{zha};
\prfB{དག་}{dag} ---	корневая \prfB{ད}{da};
б) в слоге, состоящем из трех графем, средняя --- корневая, например:
\prfB{དཔལ}{\ul{d}pa\ul{l}} --- корневая \prfB{པ}{pa};
\prfB{མནར་}{\ul{m}nar} --- корневая \prfB{ན}{na};
\prfB{གདས་}{\ul{g}da\ul{s}} - корневая \prfB{ད}{da}.
Исключение из этого правила составляет случай, когда третьей графемой является \prfB{ས}{sa} следующая за графемами \prfB{ག}{ga}, \prfB{ང}{nga}, \prfB{བ}{ba} или \prfB{མ}{ma}. В этом случае \prfB{ས}{sa} выступает как вторичная приписная, а корневой будет первая графема, например:
\prfB{ཐགས་}{thag\ul{s}} --- корневая \prfB{ཐ}{tha};
\prfB{གངས་}{gang\ul{s}} --- корневая \prfB{ག}{ga};
\prfB{ཆབས་}{chab\ul{s}} --- корневая \prfB{ཆ}{cha};
\prfB{ཁམས་}{kham\ul{s}} --- корневая \prfB{ཁ}{kha}.

Усложнение слога, степенью которого определяется место того или иного слова в словаре, происходит в следующем порядке.

Слоги первой степени сложности представляют собой сочетания корневой графемы с приписной или с приписной в сочетании со вторичной приписной графемами. При этом место слова в словаре определяется алфавитным порядком приписной графемы по формуле:
\begin{equation*}
a(b_{1} + b_{1}c + b_{2} + b_{2}c + \dots{}b_{10}c)
\end{equation*}
где <<$a$>> --- корневая: <<$b_{1-10}$>> --- приписные и <<$c$>> — вторичная приписная.

Слоги второй степени сложности представляют собой сочетания корневой графемы с предыдущими компонентами и четырьмя огласовками по формуле:
\begin{equation*}
	a(d_{1} + d_{1}b_{1} + d_{1}b_{1}c + \dots{}d_{1}b_{10}c + d_{2} + d_{2}b_{1} + \dots{}\dots{}d_{4}b_{10}c)
\end{equation*}
где <<$d_{1-4}$>> --- четыре огласовки.

Слоги третей степени сложности представляют собой сочетания корневой с подписными графемами в алфавитном порядке, при этом каждая подписная графема последовательно сочетается с предыдущими компонентами.

Слоги четвертой степени сложности представляют собой сочетания корневой графемы с приписными в их алфавитном порядке.

Следующая и последняя степень сложности образуется сочетанием корневой с надписными графемами в их алфавитном порядке.
