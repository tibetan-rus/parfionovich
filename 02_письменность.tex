\chapter{Письменность}
\section{Тибетский алфавит и его графемы}
Тибетские национальные грамматики делят тибетские графемы\prfnote{Графема --- минимальная единица письменности: в алфавитных системах письма --- буква (или другое отражение фонемы), в неалфавитных системах письма --- слоговой знак, иероглиф, идеограмма и другие.} на две категории: \prfC{གསལ་བྱེད་}{сэльце}{(букв. 'объясняющие, делающие ясным')} --- корневые, или основные, графемы и	\prfC{དབྱངས་}{янг}{(букв. 'звучные')} --- некорневые графемы, или графемы-диакритики\footnote[9]{Все европейские грамматики тибетского языка называют первые графемы <<согласными буквами>>, а вторые --- <<гласными буквами>>. Такая терминология, по нашему мнению, неточна. Вернее было бы тибетские графемы противопоставлять по линии <<корневые (основные) --- некорневые (вспомогательные)>>, что отражает их действительные различия на уровне графики. Термины же <<согласные буквы>> и <<гласные буквы>> отражают раз-личие тибетских графем лишь в фонетическом плане, не указывая при этом на их графическую неоднородность. Однако в отдельных случаях, когда необходимо отразить фонетическую значимость графем, мы будем пользоваться и терминами <<согласная>> и <<гласная>>, имея в виду соответственно корневые и некорневые графемы.}.

\emph{Корневые графемы} составляют тибетский алфавит, насчитывающий 30 знаков. Каждая корневая графема тибетского языка, за исключением графем \prfA{འ} и \prfA{ཨ} (см. таб. \ref{tab:1}), записывает слог, состоящий из двух элементов (звуков): согласного и присущего гласного \emph{a}, который на письме не обозначается. Каждому слогу присущ тон: либо ровный высокого регистра (\toneR), либо восходящий (\toneV). Кроме этих двух основных тонов можно выделить два дополнительных тона: восходяще-падающий (\toneVN) и падающий (\toneN), которые возникают при произношении двух корневых графем, когда одна приписывается к другой (см. \hyperref[sec:gss]{<<Графическая структура слога и его произношение>>}).

Все корневые графемы тибетского алфавита располагаются в определенном порядке: слева направо по классам и сверху вниз по рядам (см. таб. \ref{tab:1}).

\begin{longtable}[H]{|Sc|Sc|Sc|Sc|Sc|}
	\caption{Корневые графемы тибетского языка}\\
		\hline
	\diagbox{Класс}{Ряд} & 1 & 2 & 3 & 4\\
	\hline
	1 & \prfB{ཀ}{ka} \toneR & \prfB{ཁ}{k'a} \toneR & \prfB{ག}{k'a} \toneV & \prfB{ང}{\.{n}a} \toneV \\
	2 & \prfB{ཅ}{tsja} \toneR & \prfB{ཆ}{ts'ja} \toneR & \prfB{ཇ}{ts'ja} \toneV & \prfB{ཉ}{nja} \toneV \\
	3 & \prfB{ཏ}{ta} \toneR & \prfB{ཐ}{t'a} \toneR & \prfB{ད}{t'a} \toneV & \prfB{ན}{na} \toneV \\
	4 & \prfB{པ}{pa} \toneR & \prfB{ཕ}{p'a} \toneR & \prfB{བ}{p'a} \toneV & \prfB{མ}{ma} \toneV \\
	5 & \prfB{ཙ}{tsa} \toneR & \prfB{ཚ}{ts'a} \toneR & \prfB{ཛ}{ts'a} \toneV & \prfB{ཝ}{wa} \toneV \\
	6 & \prfB{ཞ}{sja} \toneV & \prfB{ཟ}{sa} \toneV & \prfB{འ}{a} \toneV & \prfB{ཡ}{ja} \toneV \\
	7 & \prfB{ར}{ra} \toneV & \prfB{ལ}{la} \toneV & \prfB{ཤ}{sja} \toneR & \prfB{ས}{sa} \toneR \\
	8 & \prfB{ཧ}{ha} \toneR & \prfB{ཨ}{a} & & \\
	\hline
	\label{tab:1}
\end{longtable}

Тибетские национальные грамматики делят корневые графемы от \prfB{ཀ}{ka} до \prfB{ཧ}{ha} на пять категорий в зависимости от твердости \prfA{(དམ་པ་)} произношения, называя эти категории родовыми:

\begin{longtable}[H]{Sc|Sc}
	Родовая категория & Согласные\\
	Мужской род \prfA{(ཕོ་)} & \prfA{ཀ}, \prfA{ཅ}, \prfA{ཏ}, \prfA{པ}, \prfA{ཙ} (произносятся наиболее твердо)\\
	Средний род \prfA{(མ་ནིང་)} & \prfA{ཁ}, \prfA{ཆ}, \prfA{ཐ}, \prfA{ཕ}, \prfA{ཚ}\\
	Женский род \prfA{(མོ་)} & \prfA{ག}, \prfA{ཇ}, \prfA{ད}, \prfA{བ}, \prfA{ཛ}, \prfA{ཝ}, \prfA{ཞ}, \prfA{ཟ}, \prfA{འ}, \prfA{ཡ}\\
	Совершенно женский род \prfA{(ཤིན་ཏུ་མོ་)} & \prfA{ཤ}, \prfA{ས}, \prfA{ང}, \prfA{ཉ}, \prfA{ན}, \prfA{མ}\\
	Нейтральный род \prfA{(མོ་གཤམ་)} & \prfA{ར}, \prfA{ལ}, \prfA{ཧ} (произносятся наиболее неопределенно)
\end{longtable}

Приведенное выше деление согласных на соответствующие родовые категории связано с позицией корневых морфем в слоге (см. разд. \hyperref[sec:gss]{<<Графическая структура слога и его произношение>>}).

\emph{Некорневые графемы}, или \emph{графемы-диакритики}, представляют собой четыре огласовки: \prfB{ ི}{(i)}, \prfB{ ུ}{(u)}, \prfB{ ེ}{(e)}, \prfB{ ོ}{(o)}. Первая, третья и четвертая надписываются к корневой графеме, вторая --- подписывается к ней, при этом соответственно меняется качество гласного \textit{(а)}, присущего корневой графеме, например: \prfB{ཀ}{ka} {\unifont →} \prfB{ཀི}{ki}, \prfB{ཀུ}{ku}, \prfB{ཀེ}{ke},\prfB{ཀོ}{ko}. Корневая графема может сочетаться только с одной некорневой графемой, обозначающей огласовку.

\section{Графическая структура слога и его произношение}
\label{sec:gss}

На письме различаются простой и производный слоги. Простой слог состоит только из одной корневой графемы.

К такой корневой графеме могут присоединяться другие графемы, в результате чего образуется производный слог, причем данная корневая графема становится главным элементом слога --- основой, а остальные графемы --- составляющими слога.

По тибетской традиции усложнение производного слога идет в следующем порядке:
\begin{longtable}[H]{p{1cm}p{10cm}}
	{\unifont ①} --- & основа слога, в качестве которой может выступить любая из 30 корневых графем тибетского алфавита.\\
	{\unifont ②} --- & приписные графемы \prfA{(རྗེས་འཇུག་)}, в качестве которых могут выступать 10 следующих корневых графем: \prfA{ག}, \prfA{ང}, \prfA{ད}, \prfA{ན}, \prfA{མ}, \prfA{བ}, \prfA{འ}, \prfA{ར}, \prfA{ལ}, \prfA{ས}.\\
	{\unifont ③} --- & вторичная приписная графема \prfA{(ཡང་འཇུག་)}, в качестве которой выступает корневая графема \prfA{ས}.\\
	{\unifont ④} --- & графемы-диакритики \prfA{ ི}, \prfA{ ུ}, \prfA{ ེ}, \prfA{ ོ}.\\
	{\unifont ⑤} --- & подписные графемы \prfA{(འདོགས་ཅན་)}, в качестве которых выступают корневые графемы \prfA{ཝ}, \prfA{ཡ}, \prfA{ར}, \prfA{ལ}.\\
	{\unifont ⑥} --- & префиксальные графемы \prfA{(སྔོན་ཨཇུག་)}, в качестве которых выступают корневые графемы \prfA{ག}, \prfA{ད}, \prfA{བ}, \prfA{མ}, \prfA{འ}.\\
	{\unifont ⑦} --- & надписные графемы \prfA{(མགོ་ཅན་)}, в качестве ко¬торых выступают корневые графемы \prfA{ར}, \prfA{ལ}, \prfA{ས}.
\end{longtable}

Таким образом, из 30 корневых графем в качестве составляющих элементов слога могут выступать только 12 корневых графем ( \prfA{ག}, \prfA{ང}, \prfA{ད}, \prfA{ན}, \prfA{བ}, \prfA{མ}, \prfA{ཝ}, \prfA{འ}, \prfA{ཡ}, \prfA{ར}, \prfA{ས}, \prfA{ལ} ) \footnote[10]{Одна и та же корневая графема может выступатьв качестве разных составляющих элементов слога.}.

Из изложенного выше видно, что тибетский слог конструируется как по горизонтали (приписные и префиксальные графемы), так и по вертикали (графемы-диакритики, подписные и надписные графемы). Графически и схематически различные типы тибетских слогов от простого до самого сложного могут выглядеть следующим образом:

1) \prfA{ག་} {\unifont →} {\unifont ①}; 2) \prfA{གངས་} {\unifont →} {\unifont ①②③};
3) \prfA{གིང་} {\unifont →}
\begin{tikzpicture}[baseline=0pt]
	\draw (0pt,0pt) node {\unifont ①};
	\draw (0pt,14pt) node {\unifont ④};
	\draw (14pt,0pt) node {\unifont ②};
\end{tikzpicture};
4) \prfA{གྲུབ་} {\unifont →}
\begin{tikzpicture}[baseline=0pt]
	\draw (0pt,0pt) node {\unifont ①};
	\draw (0pt,-14pt) node {\unifont ⑤};
	\draw (0pt,-28pt) node {\unifont ④};
	\draw (14pt,0pt) node {\unifont ②};
\end{tikzpicture};

5) \prfA{སྒྲུབ་} {\unifont →}
\begin{tikzpicture}[baseline=0pt]
	\draw (0pt,0pt) node {\unifont ⑦};
	\draw (0pt,-14pt) node {\unifont ①};
	\draw (0pt,-28pt) node {\unifont ⑤};
	\draw (0pt,-42pt) node {\unifont ④};
	\draw (14pt,0pt) node {\unifont ②};
\end{tikzpicture};
6) \prfA{མགའ་} {\unifont →} {\unifont ⑥①②};
7) \prfA{བསྐྲེངས་} {\unifont →}
\begin{tikzpicture}[baseline=0pt]
	\draw (-14pt,0pt) node {\unifont ⑥};
	\draw (0pt,14pt) node {\unifont ④};
	\draw (0pt,0pt) node {\unifont ⑦};
	\draw (0pt,-14pt) node {\unifont ①};
	\draw (0pt,-28pt) node {\unifont ⑤};
	\draw (14pt,0pt) node {\unifont ②};
	\draw (28pt,0pt) node {\unifont ③};
\end{tikzpicture};

Следует отметить, что тибетские корневые графемы пишутся по горизонтали в строчку. Над строчкой могут выступать только графемы-диакритики. В слогах, где имеются надписная и приписная графемы, основная графема опускается ниже строки, а приписная графема следует за надписной, как это видно из примеров 5 и 7.

Необходимо также отметить, что не каждая корневая графема, выступающая в качестве основы слога, может вступать в сочетание с любой подписной, надписной или приписной графемой. Некоторые же основы слога не могут вступать в сочетания и с целой группой составляющих слога. Так, например, корневые графемы \prfA{ཁ}, \prfA{ཆ}, \prfA{ཐ}, \prfA{ཕ}, \prfA{ཚ}, \prfA{ཝ}, \prfA{ཞ}, \prfA{ཟ}, \prfA{ཡ}, \prfA{ར}, \prfA{ལ}, \prfA{ས}, \prfA{ཨ}, \prfA{ཤ}, выступающие основой слога, не могут вступать в сочетания с надписными графемами. Корневые графемы \prfA{ཝ}, \prfA{འ}, \prfA{ར}, \prfA{ལ}, выступая основой слога, не могут принимать префиксальных графем и т.д.

Огласовки также не могут сочетаться с любой основой слога, особенно если последняя входит в состав производного слога. Это видно из следующей таблицы (таб. \ref{tab:2}). 

\begin{longtable}[H]{|p{1cm}|p{3cm}|p{7cm}|}
	\caption{Слоги, не принимающие огласовку}
	\label{tab:2}\\
		\hline
	Огла-совка & Характер слога & Слоги, не принимающие огласовку\\
	\hline
	\rule{0cm}{5mm}
	\multirow{4}{*}{\prfB{ ི}{(i)}} & простой & \prfA{ཁ}, \prfA{ང}, \prfA{ཕ}, \prfA{བ}, \prfA{ཝ}, \prfA{ཧ}\\
	& с надписными графемами & \prfA{རྐ}, \prfA{རྒ}, \prfA{རྔ}, \prfA{རྗ}, \prfA{རྣ}, \prfA{རྦ}|\quad  \prfA{ལྐ}, \prfA{ལྒ}, \prfA{ལྔ}, \prfA{ལྤ}, \prfA{ལྦ}, \prfA{ལྷ}|\quad \prfA{སྐ}, \prfA{སྒ}, \prfA{སྔ}, \prfA{སྣ}, \prfA{སྤ}, \prfA{སྦ}, \prfA{སྩ}\\
	\rule{0cm}{5mm}
	& с подписными графемами & \prfA{མྱ}|\quad \prfA{ཏྲ}, \prfA{པྲ}|\quad \prfA{ཀླ}, \prfA{བླ}, \prfA{ཟླ}, \prfA{སླ}\\
	\rule{0cm}{5mm}
	& с надписными и подписными графемами & \prfA{རྐྱ}, \prfA{རྒྱ}, \prfA{རྨྱ}|\quad \hl{??}\hyperref[tab:2:spec1]{$^*$},\prfA{རྩྭ}|\quad \prfA{ཕྱྭ}|\quad \prfA{གྲྭ},\prfA{སྨྲ}\\ 
	\hline
	\rule{0cm}{5mm}
	\multirow{4}{*}{\prfB{ ུ}{(u)}} & простой & \prfA{ཝ}\\
	& с надписными графемами & \prfA{རྙ}, \prfA{རྦ}|\quad \prfA{ལྒ}, \prfA{ལྔ}, \prfA{ལྗ}, \prfA{ལྤ}|\quad \prfA{སྩ}\\
	\rule{0cm}{5mm}
	& с подписными графемами & \prfA{པྱ}|\quad \prfA{ཏྲ}, \prfA{ཐྲ}, \prfA{པྲ}, \prfA{ཧྲ}\\
	\rule{0cm}{5mm}
	& с надписными и подписными графемами & \prfA{རྐྱ}, \prfA{རྨྱ}|\quad \hl{??}\hyperref[tab:2:spec1]{$^*$}, \prfA{རྩྭ}|\quad \prfA{ཕྱྭ}|\quad \prfA{སྦྱ}|\quad \prfA{སྨྲ}\\
	\hline
	\rule{0cm}{5mm}
	\multirow{4}{*}{\prfB{ ེ}{(e)}} & простой & \prfA{ཝ}, \prfA{འ}\\
	& с надписными графемами & \prfA{རྒ}, \prfA{རྣ}, \prfA{རྦ}|\quad \prfA{ལྐ}, \prfA{ལྒ}, \prfA{ལྔ}, \prfA{ལྗ}, \prfA{ལྤ}, \prfA{ལྦ}|\quad \prfA{སྔ}, \prfA{སྩ}\\
	\rule{0cm}{5mm}
	& с подписными графемами & \prfA{པྱ}, \prfA{མྱ}|\quad \prfA{ཏྲ}, \prfA{ཐྲ}, \prfA{པྲ}|\quad \prfA{ཀླ}, \prfA{བླ}, \prfA{ཟླ}, \prfA{སླ}\\
	\rule{0cm}{5mm}
	& с надписными и подписными графемами & \prfA{རྒྱ}, \prfA{རྨྱ}|\quad \hl{??}\hyperref[tab:2:spec1]{$^*$}, \prfA{རྩྭ}|\quad \prfA{ཕྱྭ}, \prfA{གྲྭ}|\quad \prfA{སྤྱ}, \prfA{སྦྱ}, \prfA{སྨྱ}\\
	\hline
	\rule{0cm}{5mm}
	\multirow{4}{*}{\prfB{ ོ}{(o)}} & простой & \prfA{ཝ}\\
	& с надписными графемами & \prfA{ལྒ}, \prfA{ལྔ}, \prfA{ལྤ}, \prfA{ལྦ}\\
	\rule{0cm}{5mm}
	& с подписными графемами & \prfA{དྲ}, \prfA{ཐྲ}, \prfA{མྲ}\\
	\rule{0cm}{5mm}
	& с надписными и подписными графемами & \prfA{རྨྱ}|\quad \hl{??}\hyperref[tab:2:spec1]{$^*$}, \prfA{རྩྭ}|\quad \prfA{ཕྱྭ}|\quad \prfA{གྲྭ}|\quad \prfA{སྦྲ}\\
	\hline
\end{longtable}
{\footnotesize{\label{tab:2:spec1}* В таблице приведен слог, которые невозможно отобразить современными средствами: \textit{rgrwa}. Возможно, это описка. Следует уточнить в других источниках.}}
 
Производный слог с подписным \prfA{ཝ} , например: \prfA{ཀྭ}, \prfA{འྭ} и т.д., вообще не может иметь огласовок.

Тибетский производный слог читается по корневой графеме, выступающей основой слога. Компоненты производного слога оказывают влияние на чтение основы слога, определяя произношение слога в целом. Влияние компонентов слога может заключаться в: а) изменении тона корневой (читаемой) графемы; б) озвончении ее; в) изменении качества гласного основы; г) появлении полугласного; д) изменении согласного звука основы; е) назализации согласного звука основы. Кроме этого, некоторые приписные графемы произносятся сами как конечный согласный слога или дают гортанную смычку. Некоторые префиксальные и надписные графемы не оказывают никакого влияния на звучание слога и играют только смыслоразличительную роль на письме. Все это наглядно иллюстрирует таб. \ref{tab:3}.


\begin{longtable}[H]{|c|*{8}{c|}}
	\caption{Роли компонентов слога}
	\label{tab:3}\\
		\hline
	\diagbox{{Компонент слога}}{{Роль компонента слога}} &
	\rotatebox{90}{Изменяет тон}
		&
		\rotatebox{90}{Озвончает качество гласного} &
		\rotatebox{90}{Изменяет качество гласного} &
		\rotatebox{90}{Вводит полугласный} &
		\rotatebox{90}{Изменяет согласный} &
		\rotatebox{90}{Назализирует согласный} &
		\rotatebox{90}{\parbox{4cm}{Читается как конечный согласный}} &
		\rotatebox{90}{\parbox{4cm}{Играет смыслоразличительную роль на письме}}\\
	\hline
	Приписные графемы & + & - & + & - & - & - & + & - \\
	\hline
	Подписные графемы & + & - & - & + & + & - & - & - \\
	\hline
	Префиксальные графемы & + & + & - & - & - & + & - & + \\
	\hline
	Надписные графемы & + & + & - & - & - & + & - & + \\
	\hline
\end{longtable}
\footnotesize{\emph{Примечание}. Знак + в таблице отнюдь не говорит, что вес компоненты слога данной группы (т.е. все приписные, все надписные и т.д.) играют в слоге указанную роль. Так, например, из 11 приписных только 4 меняют тон слога. Подробнее см. таб. \ref{tab:4}, \ref{tab:5}, \ref{tab:6}, \ref{tab:7}.}


Каждая группа составляющих производного слога ока¬зывает определенное влияние на его основу.
1.\emph{Приписные графемы} (за исключением \prfA{འ}\footnote[11]{\prfA{འ} служит для выделения основы слога в слоге, состоящем из двух корневых графем, не имеющих ни огласовки, ни подписного знака, например: в слоге \prfA{དགའ་} приписная \prfA{འ} указывает, что \prfA{ད} --- префиксальная графема, а \prfA{ག} --- основа; в слоге \prfA{དག་} отсутствие приписной \prfA{འ} указывает, что \prfA{ད} --- основа, а \prfA{ག} --- приписная. Сама приписная не оказывает влияния на звучание слога.}), входя в состав слога, оказывают то или иное влияние на его звучание. Приписная графема может: а) изменить ровный тон высокого регистра основы слога на падающий
( \toneR {\unifont →} \toneN ), восходящий тон на восходяще-падающий ( \toneV {\unifont →} \toneVN ); б) дать гортанную смычку ( \toneG ) удлинить или видоизменить гласный слога (см. таб. \ref{tab:4}).

2. \emph{Вторичная приписная графема}. За приписными \prfA{ག}, \prfA{ང}, \prfA{བ}, \prfA{མ} может следовать еще одна приписная --- вторичная приписная графема --- \prfB{ས}{sa}. Если \prfA{ས} следует за приписными \prfA{ག} или \prfA{བ}, то она играет только смыслоразличительную роль на письме, например: \prfC{ཐབས་}{t'\textschwa{}p\toneN}{'способ, метод'} --- \prfC{ཐབ་}{t'\textschwa{}p\toneN}{'печь, очаг'}. Если же вторичная приписная следует за приписными \prfA{ང}, \prfA{མ}, то ровный тон слога меняется на падающий, а восходящий --- на восходяще-падающий, например:  5; \prfC{ཐེང་}{t'e\textrtailn{}\toneR}{'хромой'} ---  \prfC{ཐེངས་}{t'e\textrtailn{}\toneN}{'раз', 'один раз'},\prfC{གང་}{k'a\textrtailn{}\toneV}{'где, куда'} --- \prfC{གངས་}{k'a\textrtailn{}\toneVN}{'снег'}.

\begin{longtable}[H]{|p{2cm}|p{7cm}|p{2cm}|}
	\caption{Приписные графемы}\label{tab:4}
	\\
	\hline
	Приписная графема & Её роль в слоге & Примеры\\
	\hline
	\prfA{ག} & \makecell{Меняет тон и\\ дает гортанную смычку\\
$
\left.\begin{aligned}
\toneR \rightarrow \toneN \\
\toneV \rightarrow \toneVN \\
\end{aligned}\right\rbrace + 
$ \toneG } & \makecell{\prfA{ཐག་} \textit{t'a\toneG}\toneN\\ \prfA{དག་} \textit{t'a\toneG}\toneVN}\\
\hline
\prfA{ས}, \prfA{ད} & \makecell{Меняет тон и гласный основы\\ слога, дает гортанную смычку\\
$
\left.\begin{aligned}
	\toneR \rightarrow \toneN \\
	\toneV \rightarrow \toneVN
\end{aligned}\right\rbrace + 
$ \makecell{a {\unifont →} \textepsilon \\u {\unifont →} y + \toneG\\o {\unifont →} \unifont{ɵ}} }  & 
\makecell{
	\prfA{ཆས་} \textit{ts'\unifont{ɵ}\toneG}\toneN\\
	\prfA{ཟས་} \textit{s\textepsilon'}\toneVN\\
	\prfA{བོད་} \textit{p'{\unifont ɵ}\toneG}\toneVN\\
	\prfA{ཕུད་} \textit{p'y\toneG}\toneVN
}\\
\hline
\prfA{བ} & \makecell{Меняет тон и гласный \textit{a} основы \\слога, читается \textit{p}\\
$
\left.\begin{aligned}
	\toneR \rightarrow \toneN \\
	\toneV \rightarrow \toneVN
\end{aligned}\right\rbrace + p
$ a {\unifont →} \textschwa} &
\makecell{
	\prfA{ཁབ་} \textit{k'\textschwa{}p}\toneN\\
	\prfA{གབ་} \textit{k'\textschwa{}p}\toneVN
}\\	
\hline
\prfA{ལ}, \prfA{འི} & \makecell{Меняет гласный основы \\слога и удлиняет его\\
$
\left.\begin{aligned}
	\toneR \rightarrow \toneN \\
	\toneV \rightarrow \toneVN
\end{aligned}\right\rbrace +
$ \makecell{a {\unifont →} e: \\u {\unifont →} y:\\o {\unifont →} \unifont{ɵ}:}} &
\makecell{
	\prfA{ཐལ་} \textit{t'e:}\toneR\\
	\prfA{ཁོལ་} \textit{k'{\unifont ɵ}:}\\
	\prfA{སའི་} \textit{se:}\toneR\\
	\prfA{ཁོའི་} \textit{k'{\unifont ɵ}:}\toneR
}\\	
\hline
\prfA{ན} & \makecell{Меняет гласный основы \\слога и читается как \~{n}\\
$
\left.\begin{aligned}
	a \rightarrow e \\
	u \rightarrow y \\
	o \rightarrow o\\
\end{aligned}\right\rbrace +
$ \~{n}} & 
\makecell{
	\prfA{ཕན་} \textit{p'en}\toneR\\
	\prfA{སོན་} \textit{s{\unifont ɵ}\~{n}}\toneR\\
	\prfA{གོན་} \textit{k'{\unifont ɵ}\~{n}}\toneV\\
	\prfA{ཀུན་} \textit{ky\~{n}}\toneR
}\\	
\hline
\prfA{ར} & Удлиняет гласный основы слога & 
\makecell{
	\prfA{ཐུར་} \textit{tu:}\toneR\\
	\prfA{མར་} \textit{ma:}\toneV
}\\	
\hline
\prfA{ང} & Читается как \textrtailn{}&
\makecell{
	\prfA{གང་} \textit{k'a\textrtailn}\toneR\\
	\prfA{དང་} \textit{t'a\textrtailn}\toneV
}\\	
\hline
\prfA{མ} & Читается как \~{m} &
\makecell{\prfA{དམ་} \textit{t'a\~{m}}\toneV}\\	
\hline
\end{longtable}
\footnotesize{\prfA{འི} --- вариант притяжательной падежной частицы (см. ), которая присоединятеся к основе открытого слога, выступая как суффикс.}

В древнеписьменном языке имелась еще одна вторичная приписная --- \prfA{ད}, которая могла следовать за приписными \prfA{ན}, \prfA{ར}, \prfA{ལ}. Хотя эта приписная давно уже на письме не употребляется, однако до сего времени сохранилось ее влияние на орфографию служебного слова, которое сочетается со словами, оканчивающимися на корневые графемы
\prfA{ན}, \prfA{ར}, \prfA{ལ} (см. разд. <<Служебные слова>>).

3. \emph{Префиксальные графемы}. Их наличие в сло¬ге влияет на: а) тон и звонкость основы слога-; б) звучание слога. Как влияют префиксальные графики на тон и звон¬кость основы слога, показано в табл. 5.
Таблица 5
Пре- Основы, с которыми сочетается префиксальная графема
 ИКС остаются без изменения	озвончаются восходящий тон
меняют на ровный
з, 5, 	VW	щ
	 ,*3	ъ,* 	
 	3	1
	Р.*.*й*		
и,	'' 1V '
На звучание слога оказывают влияние префиксы Эи и <2* , а также префикс   , если он стоит перед основой
слога .
Префиксы и придают основе слога назализацию, соответственна: а) д - перед основой ; б) л — перед основами   Л.  и производными слогами	;
в) т - перед основой с;. например:	 ка ,
«•ЯК, "paijk,
Префикс I c основой слога дает следующее зву¬чание: а) если гласный слога а, то р9 переходит в w, напри¬мер:	wa],	wagli б) если гласный основы
не а, то согласный не произносится вообще, например:
enl, 2 61 “•"! .	/®Ь  *9, /*• Ч•
4.	Подписные графемы *» »*»* . под¬писываясь под основой слога, меняют свое начертание (кро¬ме СЧЛ ), и оказывают определенное влияние на произноше¬ние слога (см. табл. 6).
Таблица 6
Подпис- Харак- Какие сложные графе- Роль в слоге ная гра- тер на-	мы • образует
фема черта-
	НИЯ	
Играет только смысло- , различительную роль на Л|	!	4	,	письме, например:  
 	4 va   5 \$	*/°'! ‘МЯ00'" Ч в/*1
*9	/ •'i J	‘оленье*;   tea] ‘корень’
-   *а"| ‘трава’
			 Продолжение пабл.б
Подпис- Харак- Какие сложные графе-	Роль в слоге
нал гра- тер на- мы образует фема черта-
	ния			
1)	Q, (9, Э	Читается как /, например:
,	*>л
!	2>	ъм.в,	Меняет чтение основы	на
tsj и ts’j, например: Sfr
3)		Меняет чтение основы на
п/, например:
1)	  ,55 Меняет чтение основы на
*	Ъ'ЧЧ ' ** *• наприме1>:	Щ,
I / *4	ч1 читаются *Хо1;
*	:	читаются
м	а   5 читаются
	fraA 1	1	
2)	С,	Меняет чтение основы	на
\$, например:   'Sol
3)	З’Аз’з	Играет только смысло¬
различительную роль на письме
»	3,a,S,S,S	1) У Д) «Л м«н«т
восходящий тон на ров-ный высокого регистра,
I
•	а также меняет чтение
CM	основы на /а, например:
Si la, Ш
2)	 	2) меняет чтение основы
на da, например:   da
5.	Надписные графемы. Их роль видна из сле-дующей таблицы (табл. 7).
Таблица 7
Надпис- Харак- Какие сложные гра-
ная гра- тер на- фемы образует	Роль в слоге
фема * черта-
	ния	
1)	Играет только смыслоразли-
чительную роль на письме
Л	
 	5--J	Озвончает основу, например:
£) kaK   la А
3) Ssj     у Ъ{	Меняет восходящий тон на ров¬
ный высокого регистра, напри¬мер: 2L
Продолжение яабл.7
П-	Г		 	i 	 ■— >
Надпис- Харак- Какие сложные гра-
ная гра- тер на- фемы образует	Роль в слоге
фе ма черта-
		 ния		
Озвончает и назализует ос но-
w сц У	ву, например: 3  4kai; < ntejai;
Меняет восходящий тон на ров-
3)	 	ный высокого регистра, например:
*7-1
4)	Й	Меняет чтение, например:
Ihol
Играет только бмыслооазличи-
 4	1)	    2,   тельную роль на письме, например:
!	*01)1 'костный мозг* —
кси) 1 'наполнять, удовлетворять9
Озвончает основу, например:
2)	Ц f   Д	4«/Ц ЮЛ.
I Меняет восходящий тон на
3)	ровный	высокого регистра,
например:   jA
Тон и звучание слогов в двусложном слове Приведенные выше правила чтения слогов полностью применимы только в отношении отдельного слога, т.е. когда слог выступает как самостоятельная лексическая или грамматическая единица. Подавляющее же большинство тибет¬ских слов --- двусложные. В двусложных словах тоны и произношение слогов влияют друг на друга, в результате чего в них происходят изменения. Ниже приводятся наиболее характерные из этих изменений.
1.	В двусложном слове изменяются тоны его состав¬ляющих (см. табл. 8).
Таблица 8
Тон первого	Тон второго Тоны слогов в
слога	слога	слове
1	7	П
1	з	1	N
	А	А	А 1
N	1	1	1
N	N	1	Ч
N	3	1	Ч
'N	1	 1
Л	N	 Ч
*1 I Л 1 А А
2.	Согласные, которые записываются первыми пятью корневыми графемами третьего ряда тибетского алфавита
(	  2    I Ef ), выступая в качестве второго слога в сло¬ве, озвончаются, например: "СЛа /'»Л +	к’еЛ—*-04'
ji/Це‘буква, письмена*.
3.	В первом слоге слова приписная произносится Jkj-например:	лвЛ то1*речь*.
4.	Если CJ p’a/l.Z  p*eA или p’o/i являются вторым слогом двуслога, то р* переходит в ш, например:
  toluol»	  palwol,	Р*~1и>в1, НО =л'
  nonjip'y).
5.	Если слоги слова содержат гласные звуки *' и о или у и а, то в составе слова происходит трансформация: о + i > 9 + i; i + a > i + 9; у + a > у + 9; a + у > 9 + yt например:
'a'su si A me!,	me ki   A’el ’ul.
6.	Если приписная первого слога -   или Оч, а вторым слогом являются   или  4 , то вместо U)a ИЛИ ра соответственно произносят р, га или /о. В этом случае при¬писные   и гласный не удлиняют, например: ВЛ'   soAra~], ХАЛ Ob' Q joAla EJ4PV с jotjAnal,   * 4 t*ara
7.	Если слово построено путем удвоения слогов, то
при наличии надписной или подписной корневая графема второго слога не произносится, а произносится надписная или подписная, например:	/Aol lo 1,    -5 
sajlragl; если надписных и подписных нет, а имеется при¬писная   или ЧД , то во втором слоге она не произносит- ся, например:	UAQ UpHe\,	ejlpAsji7']
Х4!   юкЬоЛ.
8.	Если первый слог слова открытый, а второй слог
начинается с префиксальных графем t З4 или ©, , то эти префиксальные становятся читаемыми и соответствен¬но произносятся р, т и п, например:	jamA
ts'«1, \%' 4   gap! kjal,	54 *	men А к*of].
9.	Если первый слог слова открытый, а второй слог
имеет надписную , то эта надписная становится читае-
 . -»
мой, например:	torAtsc |.По такому же правилу иног¬
да произносятся слова с надписными СХ\ во втором слоге (но не *sl ).

\section{Транслитерация}

При издании текстов в европейских странах часто не представляется возможным воспроизвести эти тексты в традиционной графике из-за трудности и дороговизны набора. Вместе с тем нельзя давать эти тексты и в транскрипции, так как несовпадение графики тибетского слова с его звучанием, большое количество омонимов (но не омографов) сильно затрудняет перевод транскрипции обратно в графику. Так, например, lun 1 означает и <<река>> (	) и <<ветер>> (	)» telwolозначает и «столб* ( JTj'Q ) и «гуотой* (	). Поэтому при издании тибетских текстов прибегают к транслитерации, т.е. к транскрибированию всех графем, входящих в состав слога. Такая запись обычно не передает действительного звучания производного слога, но позволяет свободно осуществлять обратный перевод его в традиционную графику.

Существует ряд способов записи тибетских корневых графем латинскими литерами, предложенных тибетологами различных стран.

Впервые такая запись была произведена в латино-тибетском словаре, составленном капуцинами Джузеппе да Асколи, Франческо Мария да Тоурс, Доменико да Фано в период 1708-1713 гг. .
Затем передача тибетских слов латинскими литерами была осуществлена капуцином А. Георги в его учебнике тибетского языка   , составленном для нужд католических миссионеров.
Однако впервые способ записи тибетских графем ла-тинскими литерами как систему предложил основатель тибетологии — венгр Чома де Кёрбши в своей грамматике тибетского языка .

Специально вопросу о транскрипции посвящена работа Т. Уайли, вышедшая в 1959 г. , а в 1964 г. появилась работа Е.Рихтера <<Основы фонетики лхасского диалекта>>.

Существует около 30 систем записи тибетских корне¬вых графем латинскими буквами, большинство из которых в значительной части совпадает друг с другом. Наиболее близки системы, предложенные Чома де КбрЗши и Т, Уайли. Наиболее удобна для транслитерации, пожалуй, система Т. Уайли, не содержащая диакритических знаков. Ниже при-водится шесть систем записи корневых графем тибетского алфавита латинскими буквами (см. \ref{tab:9}).

Таблица 9
Корневая Чома де Х.А.Ешке С.Ч.Дас Ж.Бако Т. Уайли Рихтер графема Кёрёши
I	ка	 ка	ка	ка	ка	ка
 	I	kha	k9a	kha	kha	kha	kha
	go	go	ga	ga	go	go
nga	no	no	ha	nga	ha
 	eha	со	ea	ca	ca	tsa
chha cha	cha cha	eha tsha
jo	jo	jo	Jo	jo	dza
9)	nya	nya	no	no	nya	Ha
 	ta	to	to	to	to	to
 	tha	t9a	tha	tha	tha	tha
I da	da	da	da	da	da
 	na	na	na	na	na	na
 4	pe	po	pa	pa	pa	pa
:2<	pha	p9a	pha	pha	pha	pha
 	I ba	ba	ba	ba	ba	ba
Ima	то	та	та	та	та
tsa	tsa	tsa	ca	tsa	tsa
 	ts9ha	ts9a	tsha	cha	tsha	tsha
		I		  д	■ ■ ■	■ —	- 1	*	-	1	■ ■	- 1	— ■	■ ■«
 	dsa	dza	dsa	j	dza	dza
\%	u>a	ua	и a	va	u a	wa
Продолжение табл.9
Корневая 4ома де Х.А. Ешке С.Ч.Дас Ж.Бако Т. Уайли Е. Рихтер графема	Кёр'ёши
Д	га	га	ха	га	га	га
ha	ва	ha	ра	ра	*а
уа	уа	уа	уа	уа	уа
Зч го	го	 га  га  га I га	
СЯ\	la	la	la	la	la	1а
 	sha	so	еа	за	sho	за
за	за	за	за	за	за
5	ha	ha	ha	ha	ha	ha
a	pa	a	a	a	a

При транслитерации тибетского слога запись производится в строчку слева направо.
Графемы --- составляющие слога (приписные, надписные и т.п.) передаются только согласной буквой.
Буквы, записывающие нечитаемые графемы, иногда подчеркиваются.
Если в слоге имеется подписная графема, то прису¬щий гласный основы слога пишется после нее, например:
—0 Ш—-до» - 0Щ0Ш@
й Ш 0 0
В
-®0@GD—
5 Ш
В
В последнее время предложены также различные систе¬мы транслитерации тибетских графем русскими буквами. Приводим одну из них: В.В. Семичова - К. Седлачека (см. \ref{tab:10}).

Таблица 10
		 		 	 	-
Тибет- Транс- Тибет- Транс- Тибет- Транс. Тибет- Транс- ские литера-	ские	литера-	ские	литера-	ские	литера-
графе- ция	графе-	ция	графе-	ция	графе-	ция
мы	мы	мы	мы
’Л \%а	(* \%ха	ха	*4 на
3\$ ча	чха   джа	у	нл
J	яа	тха	* да	 на
 	па	 	пха	q	ба	3	на
3\$ Ча	Н**	£ дза	ва
жа	  за	9	л
 	ра	0*ч	ла	 	ша	 	са
£	ха	а

\section{Порядок расположения словарных статей в тибетско-тибетских и тибетско-европейских словарях}

Согласно тибетской лексикографической традиции, принятой и в тибетско-европейских словарях со времени выхода в свет <<Тибетско-русского словаря>> Я. Шмидта (СПб., 1843), все слова располагаются в алфавитном порядке корневых (читаемых) графем, а словарные статьи на одну корневую графему --- по степени усложнения производного слога.

Корневая графема выделяется по следующим формаль¬ным признакам:
1)	графема, имеющая огласовку, надписную или под¬писную графему, является корневой. Так, например, в слоге mtsho  корневая графема - Зь tsha; в слоге fzsdam корневая —   da; в слоге	nges кор-невая - nga; в слоге	k ai корневая — sa (в данном случае од 1а не может быть корневой, поскольку к ней не надписывается sa;
2)	если слог не имеет указанных выше компонентов,
то-: а) в слоге, состоящем из двух графем, первая будет корневой, например:	nang - корневая   па; <3 
zhag - корневая   zha;	<*а8	“ корневая   da;
б) в слоге, состоящем из трех графем, средняя - корневая, например:	dpai - корневая pa;	mnar -
корневая па;	gdas - корневая   da. Исклю¬
чение из этого правила составляет случай, когда третьей графемой является sa9 следующая за графемами   ga,
ngat Q Ьа или 34 та. В этом случае *4 sa выступает как вторичная приписная, а корневой будет первая графема, например: 51"   4 thags - корневая	tha;
gangs — корневая ga;	chaps - корневая cha;
( *  4 khams - корневая   kha.
Усложнение слога, степенью которого определяется место того или иного слова в словаре, происходит в следую¬щем порядке.
Слоги первой степени сложности представляют собой сочетания корневой графемы с приписной или с приписной в сочетании со вторичной приписной графемами. При этом место слова в словаре определяется алфавитным порядком приписной графемы по формуле-:
a(bj + bjC + b2 + b2c + . • ,Ь10с) где «а» - корневая: «b1-10» — приписные и «с» — вторичная приписная.
Слоги второй степени сложности представляют собой сочетания корневой графемы с предыдущими компонентами и четырьмя огласовками по формуле:
a(dj 4* dj b j djbjC + ,, ,djb QC -t- d2 “f" d2b   *f ••• 4 10  где	— четыре огласовки.
Слоги третей степени сложности представляют собой сочетания корневой с подписными графемами в алфавитном порядке, при этом каждая подписная графема последова¬тельно сочетается с предыдущими компонентами.
Слоги четвертой степени сложности представляют собой сочетания корневой графемы с приписными в их алфа¬витном порядке.
Следующая и последняя степень сложности образуется сочетанием корневой с надписными графемами в их ал¬фавитном порядке. 
<Ч	й)	11 *1.8, ,3 Играет только смыслоразличи- ;
1тельную роль на письме, напри
мер:   lal ‘лошадь* - ю 1 ‘глядеть*.
Надписные графемы Л    4 помимо изменения то¬на играют также смыслоразличительную роль на письме, например:     звучат одинаково rja, но имеют различные зна¬чения - соответственно: 'старый, прежний', 'пять' и 'барабан' *;
Л	<
МЪ и читаются каждый njiy, но значат соответственно / *
'душа, сердце' и 'старый, изношенный*.
