\chapter{Морфология}

\section{Классификация морфем в тибетском языке}

Морфемы\prfnote{Морфема --- наименьшая единица языка, имеющая некоторый смысл} в тибетском языке в подавляющем большинстве односложны\footnote[19]{Иноязычные (главным образом санскритские и китайские) фонетические заимствования, как правило, являются одноморфемными. В большинстве своем это многосложные морфемы, которые, однако, здесь не рассматриваются.}. Имеются также и двусложные морфемы, однако их число сравнительно невелико. Приведем примеры.

\begin{tabularx}{\textwidth}{p{0.4\textwidth}p{0.4\textwidth}}
    \caption{Односложные морфемы}\\
    \toprule

    (1) \prfC{ང་}{nga}{'я'} & (6) \prfC{མིག་}{mig}{'глаз'}\\
    (2) \prfC{བཅུ་}{\ul{b}cu}{'десять'} & (7) \prfC{གཟིག་}{\ul{g}zig}{'леопард'}\\
    (3) \prfC{རི་}{ri}{'гора'} & (8) \prfC{འགྲོ་}{'gro}{'идти'}\\
    (4) \prfC{མི་}{mi}{'человек'} & (9) \prfC{ཟ་}{za}{'есть'}\\
    (5) \prfC{རྟ་}{\ul{r}ta}{'лошадь'} & (10) \prfC{འབྲི་}{'bri}{'писать'}\\
    \bottomrule
\end{tabularx}

\begin{tabularx}{\textwidth}{p{0.4\textwidth}p{0.4\textwidth}}
    \caption{Двусложные морфемы}\\
    \toprule

    (11) \prfC{ཡི་གེ་}{yi-ge}{'буква'} & (13) \prfC{ཁོ་བོ་}{kho-bo}{'я'}\\
    (12) \prfC{བུ་ལོན་}{bu-lon}{'заём'} & (14) \prfC{ཅ་ནེ་}{ca-ne}{'чашка'}\\
    \bottomrule
\end{tabularx}

По особенностям употребления все морфемы тибетского языка подразделяются на \emph{свободные} и \emph{связанные}. К свободным относятся морфемы, которые употребляются самостоятельно и соответствуют одноморфемному слову (все эти морфемы могут также входить в состав сложного многоморфемного слова). К связанным относятся морфемы, которые не могут быть употреблены на уровне слова (они входят в состав либо производных, либо сложных многоморфемных слов). Все приведённые выше односложные и двусложные морфемы (1-14) являются свободными.

По функции все морфемы можно разделить на две группы: \emph{знаменательные} и \emph{служебные}.

Знаменательные морфемы по своему значению подразделяются на два больших класса: номинативные и предикативные. Номинативные морфемы в свою очередь делятся на предметные и местоименные, а предикативные --- на вербальные, качественные и количественные. Вербальные, местоименные и количественные морфемы являются морфемами свободными и функционируют как самостоятельные слова. Большинство предметных морфем самостоятельно, на уровне слова, употребляться не может. Однако часть их употребляется на уровне слова (примеры 3-7).

Служебные морфемы не обладают номинативной функцией и не употребляются самостоятельно. Их можно разделить на служебные морфемы морфологического уровня и служебные морфемы синтаксического уровня.

К служебным морфемам \emph{морфологического уровня} относятся словообразующие и формообразующие морфемы.

Словообразующие морфемы, присоединяясь к знаменательным морфемам, могут выполнять различные функции: а) образуют самостоятельные слова, не внося дополнительного лексического значения; б) образуют слова с новым лексическим значением, вытекающим из семантики знаменательной морфемы; в) образуют слова иной лексико-грамматической категории.

К формообразующим морфемам относятся: а) форманты множественности; б) форманты незавершенной формы глагола; п) форманты повелительной формы глагола.

Служебные морфемы \emph{синтаксического уровня} используются в предложении в качестве грамматических средств языка. К ним относятся прежде всего различного рода частицы: предложные, усилительные, вопросительно-противительные, разделительные, заключительные, а также частицы сравнения, отрицания и др.

Все эти служебные морфемы вместе с рядом знаменательных морфем, выступающих в служебной функции, входят в категорию служебных слов.

\section{Словообразование}

Все слова тибетского языка можно подразделить на простые, производные и сложные.

1. \emph{Простые слова}. К ним относятся слова одноморфемного состава, односложные или многосложные (главным образом двусложные), например:

\begin{tabularx}{\textwidth}{p{0.4\textwidth}p{0.4\textwidth}}
    \toprule
    \prfC{ཆུ་}{chu}{'вода'} & \prfC{ཉན་}{nyan}{'слушать'}\\
    \prfC{མགོ་}{\ul{m}go}{'голова'} & \prfC{མཐུང་}{\ul{m}thung}{'пить'}\\
    \prfC{འགྲམ་}{'gram}{'берег'} & \prfC{ག་ཤ་}{ga-sha}{'ожерелье'}\\
    \prfC{རྩྭ་}{\ul{r}ts\ul{w}a}{'трава'} & \prfC{ཁས་ས་}{kha\ul{s}-sa}{'вчера'}\\
    \prfC{ཉ་}{nya}{'рыба'} & \prfC{ང་རོ་}{nga-ro}{'звук; шум'}\\
    \bottomrule
\end{tabularx}

Односложные одноморфемные слова весьма многочисленны и представляют собой основной строительный материал для образования производных и сложных слов.

2. \emph{Производные слова} по своему составу преимущественно двуморфемны и значительно реже трёхморфемны. Производное слово обычно состоит из знаменательной морфемы и служебной морфемы или знаменательной морфемы в служебной функции. Па первом месте всегда стоит знаменательная морфема. Приведем примеры производных слов:

\begin{tabularx}{\textwidth}{p{0.4\textwidth}p{0.4\textwidth}}
    \toprule
    \prfC{ཟམ་པ་}{zam-pa}{'мост'} & \prfC{ཚོང་ཁང་}{tshong-khang}{'магазин'}\\
    \prfC{གྲུ་མ་}{gru-ma}{'угол'} & \prfC{སྨན་ཁང་}{\ul{s}man-khang}{'аптека'}\\
    \prfC{ངག་མ་}{ngag-ma}{'речь'} & \prfC{ཤིང་མཁན་}{shing-\ul{m}khan}{'плотник'}\\
    \prfC{ལུག་གུ་}{lug-gu}{'ягненок'} & \prfC{ཁྲིམས་རྭ་}{khrim-r\ul{w}a}{'суд'}\\
    \prfC{ཁྱི་གུ་}{khyi-gu}{'щенок'} & \prfC{གྲུ་བོ་}{gru-bo}{'лодка'}\\
    \prfC{དགར་བ་}{\ul{d}gar-ba}{'кузнец'} & \prfC{དམར་པོ་}{\ul{d}mar-po}{'красный'}\\
    \bottomrule
\end{tabularx}

Здесь морфемы \prfB{པ}{pa}, \prfB{མ}{ma}, \prfB{བ}{ba}, \prfB{པོ}{po}, \prfB{བོ}{bo}, \prfB{གུ}{gu} --- служебные, а морфемы \prfB{ཁང}{khang}, \prfB{རྭ}{rwa}, \prfB{མཁན}{mkhan} --- знаменательные в служебной функции.

К такого же рода производным словам относится большая группа имен, образуемых путём присоединения к существительному служебных морфем \prfB{ཅན་}{can}, \prfB{བཅས་}{bcas} или глаголов \prfB{ལྡན་}{ldan}, \prfB{ཡོད་}{yod}, \prfB{མངའ་}{mnga'}, выступающих в служебной функции --- в роли словообразовательных морфем, с общим исходным значением как для морфем, так и для глаголов <<иметь, обладать; наличествовать>>, например:	\prfC{ཡོན་ཏན་ཅན་}{yon-tan-can}{'достойный'}, \prfC{ཆོས་བཅས་}{chos-\ul{b}ca\ul{s}}{'благочестивый'},
\prfC{ནོར་ལྡན་}{nor-\ul{l}dan}{'богатый'} и т.п.

К производным словам можно также отнести и группу слов, образуемых сочетанием частицы отрицания \prfB{མི་}{mi} или \prfB{མ་}{ma} с предикативной морфемой, например: \prfC{མི་འགྱུར་}{mi-gyur}{'неизменный'}, \prfC{མ་དད་པ་}{ma-da\ul{d}-pa}{'неверующий'}.

К производным словам относятся также слова, образованные редупликацией основ, например: \prfC{ཆུང་ཆུང་}{chung-chung}{'маленький'}.

3. \emph{Сложные слова}. Словосложение играет главную роль в системе словообразования в тибетском языке. Сложные слова представляют собой неделимые сочетания, состоящие в подавляющем большинстве случаев из двух знаменательных морфем, связь между которыми вскрывается лишь этимологически. Можно выделить следующие типы связей между морфемами, составляющими сложное слово:

1) атрибутивная связь. В сложных словах с атрибутивной связью в качестве главного составляющего всегда выступает предметная морфема. Главный составляющий занимает конечную позицию, если в качестве зависимого составляющего выступает предметная или вербальная морфема, и начальную позицию, если зависимый компонент представляет качественную морфему, например: 
\prfC{རྩྭ་ཐང་}{\ul{r}ts\ul{w}a-thang}{'пастбище'} --- \prfC{རྩྭ་}{\ul{r}ts\ul{w}a}{'трава'} и \prfC{ཐང་}{thang}{'равнина'};
\prfC{རྒྱུགས་ཆུ་}{\ul{r}gyug\ul{s}-chu}{'поток'} --- \prfC{རྒྱུགས་}{\ul{r}gyug\ul{s}}{'бежать'} и \prfC{ཆུ་}{chu}{'вода'};
\prfC{ལག་དམར་}{lag-\ul{d}mar}{'палач'} --- \prfC{ལག་}{lag}{'рука'} и \prfC{དམར་}{\ul{d}mar}{'красный'};

2) субъектно-предикативная связь. Сложные слова с этим типом связи компонентов состоят из предметной и вербальной морфемы, причём предметная морфема всегда стоит в начальной позиции, например:
\prfC{མི་རྒྱུག་}{mi-\ul{r}gyug}{'состязание в беге'} --- \prfC{མི་}{mi}{'человек'} и \prfC{རྒྱུག་}{\ul{r}gyug}{'бежать'};
\prfC{མི་བཟོས་}{mi-\ul{b}zo\ul{s}}{'искусственный'} --- \prfC{མི་}{mi}{'человек'} и \prfC{བཟོས་}{\ul{b}zo\ul{s}}{сделать''};
\prfC{ས་ཡོམ་}{sa-yom}{'землетрясение'} --- \prfC{ས་}{sa}{'земля'} и \prfC{ཡོམ་}{yom}{'колебаться, дрожать'};

3) объектно-предикативная связь. Сложные слова этого типа связи состоят из предметной и вербальной морфемы, причём предметная стоит в начальной позиции, например: \prfC{ཤ་འཛིན་}{sha-'dzin}{'вилка'} --- \prfC{ཤ་}{sha}{'мясо'} и \prfC{འཛིན་}{'dzin}{'держать'};

4) адвербиально-предикативная связь. Сложные слова данного типа связи представляют собой лексикализовавшееся словосочетание, состояшее из наречия и глагола (вербальной морфемы), всегда следующего за наречием, например, словосочетание \prfB{སྔོན་དུ་འགྲོ་བ་}{\ul{s}ngon-du 'gro-ba} букв. 'идущий впереди'	лексикализовалось в \prfC{སྔོན་འགྲོ་}{\ul{s}ngon-'gro}{'застрельщик, инициатор'};

5) копулятивная связь. Сложные слова данного типа связи состоят из двух морфем одного класса, например:

а) двух предметных морфем: \prfC{རྐང་ལག་}{\ul{r}kang-lag}{'конечности'} -- \prfC{རྐང་}{\ul{r}kang}{'нога'} и \prfC{ལག་}{lag}{'рука'};

6)	двух предикативных морфем: \prfC{ཚོགས་འདུ་}{tshog\ul{s}-'du}{'собрание, митинг'} -- \prfC{ཚོགས་}{tshog\ul{s}}{'собираться'} и \prfC{འདུ་}{'du}{'собираться'};

\section{Части речи}

Национальные тибетские грамматики вообще не поднимают вопроса о выделении лексико-грамматических категорий слов. Они подразделяют все слова на знаменательные (\prfB{མིང་}{ming}) и служебные частицы, функции которых и описывают.

Авторы европейских грамматик тибетского языка, как, например,
Я.Шмидт\footnote[20]{\emph{Я.Шмидт}, Грамматика тибетского языка, СПб., 1839.},
Х.А.Ешке\footnote[21]{\emph{Н.А. Jaeschke}, Tibetan grammar, Berlin, 1929.},
Ч.Бэлл\footnote[22]{\emph{Ch. Bell}, Grammar of colloquial Tibetan, 2, Calcutta, 1834.},
Ю.Н.Рерих\footnote[23]{\emph{Ю.Н.Рерих}, Тибетский язык, M.: УРСС, 2001.},
выделяют части речи, исходя только из значения слова.

Существует ряд методов выделения грамматических классов слов.

Метод, который мы предлагаем, состоит в следующем. Прежде всего вся лексика тибетского языка делится на два слоя: слова, которые могут входить в атрибутивную конструкцию и обслуживать её (I); слова, которые в атрибутивную конструкцию входить не могут (II). Затем в зависимости от того, какую позицию могут занимать слова в атрибутивной конструкции и вне её, выделяем классы слов.

\emph{I слой слов}. В тибетском языке определения могут располагаться как в левой, так и в правой позициях по отношению к определяемому. В левой позиции после каждого из определений следует грамматическая частица, в правой позиции связь определения с определяемым осуществляется простым примыканием. Обозначив определяемое через Оп, грамматическую частицу через А, определения через О$_{1}$, О$_{2}$, О$_{3}$, О$_{4}$ и т.д., мы получим следующую полную схему атрибутивной конструкции:
\begin{center}
    О$_{1}$АО$_{2}$А\dots ОпО$_{4}$О$_{5}$\dots
\end{center}

Слова, выступающие в роли определения, занимают в атрибутивной конструкции вполне определённые позиции. В соответствии с позицией их можно разделить на несколько групп.

1. Слова, которые стоят только в левой позиции. Они, как правило, принимают после себя грамматическую частицу. При этом слова, составляющие препозиционную цепочку определений, в свою очередь подразделяются на две подгруппы:
а) занимающие в этой цепочке только начальную, первую позицию, например: \prfC{ངའི་ནང་གི་ཞིང་ཁ་}{nga'i-nang-gi zhing-kha}{'пашня моей семьи'} (\prfC{ང་}{nga}{'я'}, \prfC{ནང་}{nang}{'семья'}, \prfC{ཞིང་ཁ་}{zhing-kha}{'пашня'}; \prfB{འི་}{'i}, \prfB{གི་}{gi} --- варианты одной грамматической частицы). Слова этой подгруппы мы относим к классу \emph{местоимений}\footnote[24]{Указательные местоимения \prfC{འདི་}{'di}{'этот'} и \prfC{དེ་}{de}{'тот'} могут также стоять и в постпозиции по отношению к определяемому, например: \prfB{འདི་ལོ་}{'di lo} = \prfB{ལོ་འདི་}{lo 'di} 'этот год'.};
б) занимающие любую позицию в подобной цепочке. Слова данной подгруппы относятся к классу \emph{существительных}. В том случае, когда определение, выраженное существительным, и определяемое могут рассматриваться как одно понятие или представляют собой устойчивое, привычное словосочетание, они могут не разделяться грамматической частицей, например:
\prfC{ཆུ་བེད་བཟོ་སྒྲུན་}{chu-be\ul{d} \ul{b}zo-\ul{s}grun}{'ирригационное сооружение'};
\prfC{བོད་མི་ དམངས་}{bo\ul{d} mi-\ul{d}mang\ul{s}}{'тибетский народ'}. Личные и вопросительные местоимения, выступая определением, обязательно требуют после себя грамматической частицы.

2. Слова, которые стоят только в правой позиции. Они контактно или дистантно примыкают к определяемому. При наличии двух и более определений в роли контактного определения всегда выступают слова с общим значением признаков предмета. Последние составляют класс \emph{прилагательных}. Дистантным определением выступают слова, обозначающие число или количество. Эти слова составляют класс \emph{числительных}.

3. Слова, которые могут стоять как в левой позиции, оформляясь грамматической частицей, так и в правой позиции, примыкая к определяемому, причём и та и другая позиции в одинаковой степени допустимы. Эти слова обозначают предмет по его качеству, например:
\prfC{ནོར་ཅན་}{nor-can}{'богатый'}, \prfC{ཆོས་ལྡན་}{cho\ul{s}-\ul{l}dan}{'благочестивый'}.
Мы относим их к классу \emph{имен-атрибутивов}.

И наконец имеется класс слов, которые могут соединять определение и определяемое или относиться ко всей атрибутивной конструкции и целом. Это различного рода \emph{служебные слова}.

\emph{II слой слов} составляют слова, которые не могут входить в атрибутивную конструкцию. Среди этих слов легко вычленяются три категории, обладающие общими признаками.

Первая категории --- слова, которые не связываются грамматически со словами предложения и не служат членами предложения, а примыкают ко всему предложению в целом, выражая отношение говорящего к высказываемому, в частности, придавая предложению эмоциональную окраску. Эти слова стоят всегда в начале предложения и относятся к классу \emph{междометий}.

Вторая категория --- слова, которые всегда стоят на конце предложения и являются в предложении сказуемым. Это --- \emph{глаголы}.

Третья категория --- слова, которые не имеют в предложении твёрдо фиксированного места и лишь не могут занимать позицию на конце предложения. Это --- \emph{наречия}.

\begin{center}
* * *
\end{center}

Таким образом, в тибетском языке можно выделить следующие классы, или лексико-грамматические категории, слов: имя существительное, имя прилагательное, имя-атрибутив, местоимение, числительное, наречие, глагол, междометия и служебные слова. Положение слова в составе предложения и его функции в нём зависят только от категориальной принадлежности данного слова.