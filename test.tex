% !TEX program = xelatex
\documentclass[a4paper,12pt]{scrbook}
\usepackage{fontspec}
\usepackage{xltxtra}
\usepackage[english,russian]{babel}
\usepackage{nccfoots}
\usepackage{color}

\setmainfont{CMU Serif} % основной шрифт текста

\newfontfamily\tibfont{Tibetan Machine Uni}
\newfontfamily\unifont{Arial Unicode MS}

\newcommand{\prfA}[1]{{\tibfont #1}}
\newcommand{\prfB}[2]{{\tibfont #1} \textit{#2}}
\newcommand{\prfC}[3]{{\tibfont #1} \textit{#2} \textrm{#3}}

%сноски с собственной нумерацией
\newcounter{PrfFNCounter}[page]
\newcommand{\prfnote}[1]{\stepcounter{PrfFNCounter}\Footnote{\Asbuk{PrfFNCounter}}{#1}}

%
\newcommand{\ul}[1]{\underline{#1}}
\newcommand{\hl}[1]{\colorbox{yellow}{#1}}
\author{Парфионович Юрий Михайлович}
\title{Тибетский письменный язык (тест)}

\begin{document}
\maketitle
here> \prfA{སྐྱེས་མི་འགག་ནམ་མཁའི་ངོ་བོ་ཉིད།} <here\par
\prfB{སྐྱེས་མི་འགག་ནམ་མཁའི་ངོ་བོ་ཉིད། }{test and test}\par
\prfC{སྐྱེས་མི་འགག་ནམ་མཁའི་ངོ་བོ་ཉིད། }{test and test}{и это тест}\par
\prfC{སྐྱེས་མི་འགག་ནམ་མཁའི་ངོ་བོ་ཉིད། }{test and test}{'еще это тест'}\par

Это тест двойных сносок\footnote[2]{Сноска с номером 2}, моих сносок\prfnote{Моя сноска с буквой} и сносок\footnote{Это номер по умолчанию}

Проверка \ul{подчеркивания}, выделения \hl{цветом}, диакритический знак \.a\b{b}

Ровный высокого регистра \toneR\par
Восходящий \toneV\par
Нисходящий \toneN\par
Восходяще-падающий \toneVN\par

<>
\begin{tikzpicture}
    \draw (1.0cm,1.0cm) node[fill=white] {\unifont ①};
    \draw (1.0cm,1.0cm+14pt) node[fill=white] {\unifont ①};
\end{tikzpicture}
<>
\begin{tikzpicture}
    \draw (0,12pt) -- (6pt,12pt) -- (6pt,0pt);
\end{tikzpicture}
<>
\begin{tikzpicture}
    \draw (0,0pt) -- (6pt,5pt);
    \draw (6pt,0pt) -- (6pt,10pt);
\end{tikzpicture}
<>
\begin{tikzpicture}
    \draw (0,10pt) -- (6pt,5pt);
    \draw (6pt,0pt) -- (6pt,10pt);
\end{tikzpicture}
<>
\begin{tikzpicture}
    \draw (0pt,5pt) -- (3pt,10pt) -- (6pt,5pt);
    \draw (6pt,0pt) -- (6pt,10pt);
\end{tikzpicture}
<>
\clearpage
Страница 1. Моя 1) сноска \prfnote{Моя сноска с буквой} 2) сноска\prfnote{Моя сноска с буквой}
\clearpage
Страница 2. Моя 1) сноска \prfnote{Моя сноска с буквой} 2) сноска\prfnote{Моя сноска с буквой}
\end{document}